%% Generated by Sphinx.
\def\sphinxdocclass{report}
\documentclass[letterpaper,10pt,openany,onesideH,english]{sphinxmanual}
\ifdefined\pdfpxdimen
   \let\sphinxpxdimen\pdfpxdimen\else\newdimen\sphinxpxdimen
\fi \sphinxpxdimen=.75bp\relax

\PassOptionsToPackage{warn}{textcomp}
\usepackage[utf8]{inputenc}
\ifdefined\DeclareUnicodeCharacter
 \ifdefined\DeclareUnicodeCharacterAsOptional
  \DeclareUnicodeCharacter{"00A0}{\nobreakspace}
  \DeclareUnicodeCharacter{"2500}{\sphinxunichar{2500}}
  \DeclareUnicodeCharacter{"2502}{\sphinxunichar{2502}}
  \DeclareUnicodeCharacter{"2514}{\sphinxunichar{2514}}
  \DeclareUnicodeCharacter{"251C}{\sphinxunichar{251C}}
  \DeclareUnicodeCharacter{"2572}{\textbackslash}
 \else
  \DeclareUnicodeCharacter{00A0}{\nobreakspace}
  \DeclareUnicodeCharacter{2500}{\sphinxunichar{2500}}
  \DeclareUnicodeCharacter{2502}{\sphinxunichar{2502}}
  \DeclareUnicodeCharacter{2514}{\sphinxunichar{2514}}
  \DeclareUnicodeCharacter{251C}{\sphinxunichar{251C}}
  \DeclareUnicodeCharacter{2572}{\textbackslash}
 \fi
\fi
\usepackage{cmap}
\usepackage[T1]{fontenc}
\usepackage{amsmath,amssymb,amstext}
\usepackage{babel}
\usepackage{times}
\usepackage[Bjarne]{fncychap}
\usepackage{sphinx}

\usepackage{geometry}

% Include hyperref last.
\usepackage{hyperref}
% Fix anchor placement for figures with captions.
\usepackage{hypcap}% it must be loaded after hyperref.
% Set up styles of URL: it should be placed after hyperref.
\urlstyle{same}

\addto\captionsenglish{\renewcommand{\figurename}{Fig.}}
\addto\captionsenglish{\renewcommand{\tablename}{Table}}
\addto\captionsenglish{\renewcommand{\literalblockname}{Listing}}

\addto\captionsenglish{\renewcommand{\literalblockcontinuedname}{continued from previous page}}
\addto\captionsenglish{\renewcommand{\literalblockcontinuesname}{continues on next page}}

\addto\extrasenglish{\def\pageautorefname{page}}

\setcounter{tocdepth}{1}



\title{DTC Documentation}
\date{Sep 05, 2018}
\release{1.2}
\author{SOEP}
\newcommand{\sphinxlogo}{\vbox{}}
\renewcommand{\releasename}{Release}
\makeindex

\begin{document}

\maketitle
\sphinxtableofcontents
\phantomsection\label{\detokenize{index::doc}}



\chapter{Contents of SOEPcore}
\label{\detokenize{Contents of SOEPcore/index:contents-of-soepcore}}\label{\detokenize{Contents of SOEPcore/index::doc}}
The SOEP started in 1984 as a longitudinal survey of private households in the Federal Republic of Germany. The central aim then and now is to collect representative micro-data to measure stability and change in living conditions by following a micro-economic approach enriched with variables from sociology and political science (influenced by the “Social Indicator” movement). Therefore the central survey instruments are a household questionnaire, which is responded by the head of a household and an individual questionnaire, which each household member is intended to answer. Furthermore beginning with 1997, there are wave-specific \$LELA files (Lebenslauf - engl. life course) containing the biography information as collected in the respective year.

\sphinxstylestrong{Life History}

\begin{figure}[H]
\centering

\noindent\sphinxincludegraphics{{lebenslaufperspektive_eng}.PNG}
\end{figure}

The SOEP questionnaires are designed in such a way that people in a SOEP household can be analysed from birth to adulthood.
In addition to the {\hyperref[\detokenize{Contents of SOEPcore/index:youth-questionnaire}]{\sphinxcrossref{\DUrole{std,std-ref}{Youth Questionnaire}}}}, which was conducted for the first time in 2000/01, a series of questionnaires for certain cohorts of children living in SOEP households has been introduced since 2003. These are filled in every year since their year of introduction by mothers (in exceptional cases by fathers) with children of the appropriate age. In 2003 a questionnaire was developed for the mothers of newborn children (0-1 years) {\hyperref[\detokenize{Contents of SOEPcore/index:mother-child-questionnaire-a}]{\sphinxcrossref{\DUrole{std,std-ref}{Mother-Child Questionnaire A (Age 0-1)}}}}. The following instruments were developed in such a way that this starting cohort (born 2002/ 2003) can be followed up in its development and analysed longitudinally. This was followed in 2005 by a questionnaire for the mothers of 2-3-year-old children {\hyperref[\detokenize{Contents of SOEPcore/index:mother-child-questionnaire-b}]{\sphinxcrossref{\DUrole{std,std-ref}{Mother-Child Questionnaire B (Age 2-3)}}}} and in 2008 by a questionnaire for 5-6-year-olds {\hyperref[\detokenize{Contents of SOEPcore/index:mother-child-questionnaire-c}]{\sphinxcrossref{\DUrole{std,std-ref}{Mother-Child Questionnaire C (Age 5-6)}}}}. In 2010, the questionnaire for 7-8-year-old children {\hyperref[\detokenize{Contents of SOEPcore/index:parents-d}]{\sphinxcrossref{\DUrole{std,std-ref}{Parents Questionnaire D (Age 7-8)}}}}, completed by both mothers and fathers, was launched. In 2012, the questionnaire for 9-10-year-old children {\hyperref[\detokenize{Contents of SOEPcore/index:mother-child-questionnaire-e}]{\sphinxcrossref{\DUrole{std,std-ref}{Mother-Child Questionnaire E (Age 9-10)}}}} was the last questionnaire to be answered by the mothers. This was followed by two youth instruments in which the children, aged 12 {\hyperref[\detokenize{Contents of SOEPcore/index:pupils-questionnaire}]{\sphinxcrossref{\DUrole{std,std-ref}{Pupils Questionnaire}}}} and 14 {\hyperref[\detokenize{Contents of SOEPcore/index:early-youth-questionnaire}]{\sphinxcrossref{\DUrole{std,std-ref}{Early Youth Questionnaire}}}} respectively, answered questions about their own life situation for the first time. These were introduced in 2014 and 2016 respectively, so that in 2018 the first cohort went through the complete battery of age-specific instruments for the first time and then, as an adult, will answer annually thematically changing topics of the long-term SOEP study.
As soon as the age of 18 is reached, each person in a SOEP household receives the {\hyperref[\detokenize{Contents of SOEPcore/index:individual-questionnaire}]{\sphinxcrossref{\DUrole{std,std-ref}{Individual Questionnaire}}}}, the head of the household additionally receives the {\hyperref[\detokenize{Contents of SOEPcore/index:household-questionnaire}]{\sphinxcrossref{\DUrole{std,std-ref}{Household Questionnaire}}}}. As soon as a person dies, regardless of whether this person is part of a SOEP household, the {\hyperref[\detokenize{Contents of SOEPcore/index:deceased-persons-questionnaire}]{\sphinxcrossref{\DUrole{std,std-ref}{Deceased Persons Questionnaire}}}} is handed over to the person providing the information.


\section{SOEP Topics}
\label{\detokenize{Contents of SOEPcore/index:soep-topics}}
A rather stable set of core questions is asked every year covering the most essential areas of interest of the SOEP:


\subsection{Attitudes, Values and Personality}
\label{\detokenize{Contents of SOEPcore/index:attitudes-values-and-personality}}
The character of a person offers a variety of analysis possibilities. Information about the personality of the respondents, their political orientation, concerns, satisfaction, willingness to take risks and much more can be found in the “Attitudes, Values, and Personality” section.

\begin{figure}[H]
\centering
\sphinxhref{https://paneldata.org/soep-core/topics/\#topic\_dp}{\sphinxincludegraphics{{attitudes}.png}}\end{figure}


\subsection{Demography and Population}
\label{\detokenize{Contents of SOEPcore/index:demography-and-population}}
In this topic you find various information about the birth dates, no matter if interviewer, children, siblings or parents. Furthermore, there is data on places and history of births in households. The household sizes and relationships between the different persons in a household are also listed, as are the     sexes of all persons involved.

\begin{figure}[H]
\centering
\sphinxhref{https://paneldata.org/soep-core/topics/\#topic\_dp}{\sphinxincludegraphics{{demography}.png}}\end{figure}


\subsection{Education and Qualification}
\label{\detokenize{Contents of SOEPcore/index:education-and-qualification}}
Education is one of the cornerstones of our society today, and the information that can be obtained through the SOEP is numerous. Whether school achievement, vocational training or academic success in this section is everything about the education of people. The school history, reasons for lack of further training, educational goals and so on. Furthermore, basic skills of children can be found here to, whether they are able to speak in whole sentences or use scissors, for example.

\begin{figure}[H]
\centering
\sphinxhref{https://paneldata.org/soep-core/topics/\#topic\_dp}{\sphinxincludegraphics{{education}.png}}\end{figure}


\subsection{Family and Social Networks}
\label{\detokenize{Contents of SOEPcore/index:family-and-social-networks}}
As a household study, the SOEP determines rich information about family and social contacts and how these relationships change at different stages of life. The whole cycle of life with its wonderful and sad facets and a wide range of information is shown in this section: Pregnancy - birth - parenthood - kinship - circle of friends - marriage - divorce - death. And of course many more data can be found here.

\begin{figure}[H]
\centering
\sphinxhref{https://paneldata.org/soep-core/topics/\#topic\_dp}{\sphinxincludegraphics{{family}.png}}\end{figure}


\subsection{Home, Amenities and Contributions of Private HH}
\label{\detokenize{Contents of SOEPcore/index:home-amenities-and-contributions-of-private-hh}}
In this section you will find information about the household and everything that has to do with everyday life. What kind of home do you live in? Are you an owner or a tenant? Which expenses do you have on things like personal hygiene, the car or holidays? Who’s taking care of the kids? All this and much more information about living, its costs or the living environment can be seen here.

\begin{figure}[H]
\centering
\sphinxhref{https://paneldata.org/soep-core/topics/\#topic\_dp}{\sphinxincludegraphics{{home}.png}}\end{figure}


\subsection{Health and Care}
\label{\detokenize{Contents of SOEPcore/index:health-and-care}}
On the subject of health, numerous personal data such as the number of doctoral visits and habits like sport or alcohol consumption are recorded. There are also information on health insurance, health status and grip strength. However, health information from other people such as children or deceased persons are also displayed.

\begin{figure}[H]
\centering
\sphinxhref{https://paneldata.org/soep-core/topics/\#topic\_dp}{\sphinxincludegraphics{{health}.png}}\end{figure}


\subsection{Integration, Migration, Transnationalization}
\label{\detokenize{Contents of SOEPcore/index:integration-migration-transnationalization}}
Migration and establishment processes are changing society. With its large number of migration samples and specific migration questions, the SOEP can cover these research topics comprehensively. The area “Integration, Migration, Transnationalization” offers you analysis possibilities on migration history, discrimination, interethnic contacts, education, cultural integration, transnational relations, identification with Germany and the intention to stay.
\sphinxhref{https://paneldata.org/soep-core/topics/\#topic\_dp}{\sphinxincludegraphics{{integration}.png}}

\subsection{Income, Taxes and Social Security}
\label{\detokenize{Contents of SOEPcore/index:income-taxes-and-social-security}}
Income and finances are an essential part of our everyday life. How much money is earned and how much is spent. Child benefit, pensions, inheritance or salary, but also taxes and debts belong to this topic. No less interesting is the information on other assets such as real estate or property, plant and equipment.

\begin{figure}[H]
\centering
\sphinxhref{https://paneldata.org/soep-core/topics/\#topic\_dp}{\sphinxincludegraphics{{income}.png}}\end{figure}


\subsection{Survey Methodology}
\label{\detokenize{Contents of SOEPcore/index:survey-methodology}}
In the “Survey Methodology” section you will find many relevant variables on imputation, weighting, field work in SOEP core, identifiers, the interviewers’ working methods,survey methods and also information about our respondents’ exit from the survey.
\sphinxhref{https://paneldata.org/soep-core/topics/\#topic\_dp}{\sphinxincludegraphics{{survey}.png}}

\subsection{Time Use and Environmental Behavior}
\label{\detokenize{Contents of SOEPcore/index:time-use-and-environmental-behavior}}
Time is a valuable resource for every human being. Information on how a person plans their time, what obligations they have at what time and how they spend their free time can be found in the “Time Use and Environmental Behavior” section. This section also provides comprehensive information on environmental awareness. Which transport infrastructure is used, which energy resources are used to what extent and what is the position on the subject of renewable energies?

\begin{figure}[H]
\centering
\sphinxhref{https://paneldata.org/soep-core/topics/\#topic\_dp}{\sphinxincludegraphics{{time_use}.png}}\end{figure}


\subsection{Work and Employment}
\label{\detokenize{Contents of SOEPcore/index:work-and-employment}}
Information about the topic profession can be found in this section. From the very first job and further training, to job changes and parenthood, to part-time jobs and unemployment. However, not only objective information such as hours of work, but also perceptions of the working environment and feelings about work are shown.

\begin{figure}[H]
\centering
\sphinxhref{https://paneldata.org/soep-core/topics/\#topic\_dp}{\sphinxincludegraphics{{work}.png}}\end{figure}


\section{SOEP Questionnaires}
\label{\detokenize{Contents of SOEPcore/index:soep-questionnaires}}
\begin{figure}[H]
\centering

\noindent\sphinxincludegraphics{{overview}.png}
\end{figure}


\subsection{Household Questionnaire}
\label{\detokenize{Contents of SOEPcore/index:household-questionnaire}}\label{\detokenize{Contents of SOEPcore/index:id1}}
\sphinxstylestrong{Availability:} Since 1984

\sphinxstylestrong{Respondent:} Head of household


\subsubsection{Stable content}
\label{\detokenize{Contents of SOEPcore/index:stable-content}}
The household questionnaire has been a standard instrument since the beginning of the SOEP. Because the SOEP has a panel character, important questions have to be answered each year anew by the respondents. In order to enable analyses over time, the household questionnaire therefore has a large number of question modules which are asked every year. The following question modules are part of the core program of the household questionnaire.


\begin{savenotes}\sphinxatlongtablestart\begin{longtable}{|\X{4}{17}|\X{5}{17}|\X{2}{17}|\X{6}{17}|}
\hline
\sphinxstyletheadfamily 
Topics
&\sphinxstyletheadfamily 
Module
&\sphinxstyletheadfamily 
No. Vars
&\sphinxstyletheadfamily 
Variables
\\
\hline
\endfirsthead

\multicolumn{4}{c}%
{\makebox[0pt]{\sphinxtablecontinued{\tablename\ \thetable{} -- continued from previous page}}}\\
\hline
\sphinxstyletheadfamily 
Topics
&\sphinxstyletheadfamily 
Module
&\sphinxstyletheadfamily 
No. Vars
&\sphinxstyletheadfamily 
Variables
\\
\hline
\endhead

\hline
\multicolumn{4}{r}{\makebox[0pt][r]{\sphinxtablecontinued{Continued on next page}}}\\
\endfoot

\endlastfoot

\sphinxstylestrong{Household, Amenities and Contribution of private Households}
&&&\\
\hline&
Change of living situation
&
3
&
\sphinxhref{https://paneldata.org/soep-long/data/hl/hlf0523}{hlf0523} \& \sphinxhref{https://paneldata.org/soep-long/data/hl/hlf0106}{hlf0106} \& \sphinxhref{https://paneldata.org/soep-long/data/hl/hlf0107}{hlf0107}
\\
\hline&
Neighbourhood
&
1
&
\sphinxhref{https://paneldata.org/soep-long/data/hl/hlf0153}{hlf0153}
\\
\hline&
House type
&
4
&
\sphinxhref{https://paneldata.org/soep-long/data/hl/hlf0154}{hlf0154} \& \sphinxhref{https://paneldata.org/soep-long/data/hl/hlf0016}{hlf0016} \&  \sphinxhref{https://paneldata.org/soep-long/data/hl/hlf0155}{hlf0155} \& hlf0596
\\
\hline&
Size and condition of the house
&
4
&
\sphinxhref{https://paneldata.org/soep-long/data/hl/hlf0018}{hlf0018} \& \sphinxhref{https://paneldata.org/soep-long/data/hl/hlf0019}{hlf0019} \& \sphinxhref{https://paneldata.org/soep-long/data/hl/hlf0071}{hlf0071} \& \sphinxhref{https://paneldata.org/soep-long/data/hl/hcf0011}{hcf0011}
\\
\hline&
Apartment equipment
&
13
&
\sphinxhref{https://paneldata.org/soep-long/data/hl/hlf0023}{hlf0023} - \sphinxhref{https://paneldata.org/soep-long/data/hl/hlf0037}{hlf0037} \& \sphinxhref{https://paneldata.org/soep-long/data/hl/hlf0529}{hlf0529} - \sphinxhref{https://paneldata.org/soep-long/data/hl/hlf0023}{hlf0531}
\\
\hline&
Apartment status
&
4
&
\sphinxhref{https://paneldata.org/soep-long/data/hl/hlf0001}{hlf0001} \& \sphinxhref{https://paneldata.org/soep-long/data/hl/hlf0006}{hlf0006} \& \sphinxhref{https://paneldata.org/soep-long/data/hl/hlf0006}{hlf0007} \& \sphinxhref{https://paneldata.org/soep-long/data/hl/hlf0009}{hlf0009} \& \sphinxhref{https://paneldata.org/soep-long/data/hl/hlf0015}{hlf0015}
\\
\hline&
Loans, mortgages, building-society loans
&
2
&
\sphinxhref{https://paneldata.org/soep-long/data/hl/hlf0087}{hlf0087} \& \sphinxhref{https://paneldata.org/soep-long/data/hl/hlf0088}{hlf0088}
\\
\hline&
Hereditary lease interest
&
2
&
\sphinxhref{https://paneldata.org/soep-long/data/hl/hlf0087}{hlf0597} \& hlf0598
\\
\hline&
Modernization costs
&
2
&
hlf0599 \& hlf0600
\\
\hline&
Owner costs
&
2
&
hlf0601 - \sphinxhref{https://paneldata.org/soep-long/data/hl/hlf0090}{hlf0605  \& {}`hlf0090} \& \sphinxhref{https://paneldata.org/soep-long/data/hl/hlf0084}{hlf0084}
\\
\hline&
Photovoltaic and solar thermal system
&
6
&
\sphinxhref{https://paneldata.org/soep-long/data/hl/hlf0532}{hlf0532} \& \sphinxhref{https://paneldata.org/soep-long/data/hl/hlf0535}{hlf0535} - \sphinxhref{https://paneldata.org/soep-long/data/hl/hlf0539}{hlf0539}
\\
\hline&
Owner burden
&
1
&
hlf0606
\\
\hline&
Social housing/leased flat at a reduced rate
&
2
&
\sphinxhref{https://paneldata.org/soep-long/data/hl/hlf0007}{hcf0007} \& \sphinxhref{https://paneldata.org/soep-long/data/hl/hlf0073}{hlf0073}
\\
\hline&
Apartment owner
&
1
&
\sphinxhref{https://paneldata.org/soep-long/data/hl/hlf0013}{hlf0013}
\\
\hline&
Rental and ancillary costs
&
9
&
\sphinxhref{https://paneldata.org/soep-long/data/hl/hlf0069}{hlf0069} \& \sphinxhref{https://paneldata.org/soep-long/data/hl/hlf0074}{hlf0074} \& \sphinxhref{https://paneldata.org/soep-long/data/hl/hlf0078}{hlf0078} \& \sphinxhref{https://paneldata.org/soep-long/data/hl/hlf0079}{hlf0079} \& \sphinxhref{https://paneldata.org/soep-long/data/hl/hlf0081}{hlf0081} \& \sphinxhref{https://paneldata.org/soep-long/data/hl/hlf0082}{hlf0082} \& hlf0607 \& hlf0608 \& hlf0610
\\
\hline&
Tenant burden
&
1
&
hlf0611
\\
\hline&
Cleaning or household assistance
&
2
&
\sphinxhref{https://paneldata.org/soep-long/data/hl/hlf0261}{hlf0261} \& \sphinxhref{https://paneldata.org/soep-long/data/hl/hlf0262}{hlf0262}
\\
\hline&
Persons in need of care
&
22
&
\sphinxhref{https://paneldata.org/soep-long/data/hl/hlf0291}{hlf0291} \& \sphinxhref{https://paneldata.org/soep-long/data/hl/hlf0631}{hlf0631} \sphinxhref{https://paneldata.org/soep-long/data/hl/hlf0292}{hlf0292} \& \sphinxhref{https://paneldata.org/soep-long/data/hl/hlf0300}{hlf0300} - \sphinxhref{https://paneldata.org/soep-long/data/hl/hlf0304}{hlf0304} \& \sphinxhref{https://paneldata.org/soep-long/data/hl/hlf0315}{hlf0315} \& \sphinxhref{https://paneldata.org/soep-long/data/hl/hlf0317}{hlf0317} \& \sphinxhref{https://paneldata.org/soep-long/data/hl/hlf0319}{hlf0319} - \sphinxhref{https://paneldata.org/soep-long/data/hl/hlf0322}{hlf0322} \& \sphinxhref{https://paneldata.org/soep-long/data/hl/hlf0331}{hlf0331} \& \sphinxhref{https://paneldata.org/soep-long/data/hl/hlf0332}{hlf0332} \& \sphinxhref{https://paneldata.org/soep-long/data/hl/hlf0369}{hlf0369} \& \sphinxhref{https://paneldata.org/soep-long/data/hl/hlf0370}{hlf0370} \& \sphinxhref{https://paneldata.org/soep-long/data/hl/hlf0446}{hlf0446} - \sphinxhref{https://paneldata.org/soep-long/data/hl/hlf0448}{hlf0448} \& \sphinxhref{https://paneldata.org/soep-long/data/hl/hlf0595}{hlf0595}
\\
\hline&
Name and birth of children
&
1
&
\sphinxhref{https://paneldata.org/soep-long/data/hl/hlk0044}{hlk0044}
\\
\hline&
School attendance for child
&
2
&
\sphinxhref{https://paneldata.org/soep-long/data/kidl/ks\_gen}{ks\_gen} \& ks\_spe
\\
\hline&
Caring Situation for child
&
6
&
ks\_asc\_r \& \sphinxhref{https://paneldata.org/soep-long/data/kidl/kc\_relat}{kc\_relaz} \& \sphinxhref{https://paneldata.org/soep-long/data/kidl/kc\_frnd}{kc\_frdn} \& \sphinxhref{https://paneldata.org/soep-long/data/kidl/kc\_paid}{kc\_paid} \& \sphinxhref{https://paneldata.org/soep-long/data/kidl/kc\_mindr}{kc\_mindr} \& \sphinxhref{https://paneldata.org/soep-long/data/kidl/kc\_none}{kc\_none}
\\
\hline
\sphinxstylestrong{Income, Taxes and Social Security}
&&&\\
\hline&
Income and expenses from rental/lease
&
7
&
\sphinxhref{https://paneldata.org/soep-long/data/hl/hlc0007}{hlc0007} - \sphinxhref{https://paneldata.org/soep-long/data/hl/hlc0009}{hlc0009}  \& \sphinxhref{https://paneldata.org/soep-long/data/hl/hlc0111}{hlc0111} \& \sphinxhref{https://paneldata.org/soep-long/data/hl/hlc0112}{hlc0112}  \& hlc0176 \& hlc0177
\\
\hline&
Repayments for loans
&
2
&
\sphinxhref{https://paneldata.org/soep-long/data/hl/hlc0113}{hlc0113} \& \sphinxhref{https://paneldata.org/soep-long/data/hl/hlc0114}{hlc0114}
\\
\hline&
Credit burden
&
1
&
\sphinxhref{https://paneldata.org/soep-long/data/hl/hlc0115}{hlc0115}
\\
\hline&
Inheritance, present, lottery prize
&
2
&
hlc0178 - hlc0183
\\
\hline&
Investments
&
14
&
\sphinxhref{https://paneldata.org/soep-long/data/hl/hlc0104}{hlc0104} - \sphinxhref{https://paneldata.org/soep-long/data/hl/hlc0108}{hlc0108} \& \sphinxhref{https://paneldata.org/soep-long/data/hl/hlc0093}{hlc0093} - \sphinxhref{https://paneldata.org/soep-long/data/hl/hlc0098}{hlc0098} \& \sphinxhref{https://paneldata.org/soep-long/data/hl/hlc0013}{hlc0013} \& \sphinxhref{https://paneldata.org/soep-long/data/hl/hlc0014}{hlc0014} - hlc0184
\\
\hline&
Income/expanses household
&
43
&
\sphinxhref{https://paneldata.org/soep-long/data/hl/hlc0005}{hlc0005} \& \sphinxhref{https://paneldata.org/soep-long/data/hl/hlc0006}{hlc0006} \& \sphinxhref{https://paneldata.org/soep-long/data/hl/hlc0039}{hlc0039} \& \sphinxhref{https://paneldata.org/soep-long/data/hl/hlc0041}{hlc0041} - \sphinxhref{https://paneldata.org/soep-long/data/hl/hlc0047}{hlc0047} \& \sphinxhref{https://paneldata.org/soep-long/data/hl/hlc0049}{hlc0049} - \sphinxhref{https://paneldata.org/soep-long/data/hl/hlc0055}{hlc0055} - \sphinxhref{https://paneldata.org/soep-long/data/hl/hlc0057}{hlc0057} \&  \sphinxhref{https://paneldata.org/soep-long/data/hl/hlc0059}{hlc0059} \& \sphinxhref{https://paneldata.org/soep-long/data/hl/hlc0061}{hlc0061} - \sphinxhref{https://paneldata.org/soep-long/data/hl/hlc0065}{hlc0065} \& \sphinxhref{https://paneldata.org/soep-long/data/hl/hlc0067}{hlc0067} \& \sphinxhref{https://paneldata.org/soep-long/data/hl/hlc0068}{hlc0068} \& \sphinxhref{https://paneldata.org/soep-long/data/hl/hlc0070}{hlc0070} \& \sphinxhref{https://paneldata.org/soep-long/data/hl/hlc0071}{hlc0071} \& \sphinxhref{https://paneldata.org/soep-long/data/hl/hlc0077}{hlc0077} - \sphinxhref{https://paneldata.org/soep-long/data/hl/hlc0085}{hlc0085} \& \sphinxhref{https://paneldata.org/soep-long/data/hl/hlc0090}{hlc0090} \& \sphinxhref{https://paneldata.org/soep-long/data/hl/hlc0121}{hlc0121} - \sphinxhref{https://paneldata.org/soep-long/data/hl/hlc0125}{hlc0125}
\\
\hline&
Saving
&
3
&
\sphinxhref{https://paneldata.org/soep-long/data/hl/hlc0172}{hlc0172} - \sphinxhref{https://paneldata.org/soep-long/data/hl/hlc0174}{hlc0174}
\\
\hline
\end{longtable}\sphinxatlongtableend\end{savenotes}

\sphinxcode{\sphinxupquote{Download Stable Content Household (csv)}}


\subsubsection{Replication Calendar Household Questionnaire}
\label{\detokenize{Contents of SOEPcore/index:replication-calendar-household-questionnaire}}
Besides the topics that are asked every year in the household questionnaire, there are some topics modules that are collected irregularly. Many questions do not have to be asked every year as short-term changes are unlikely. In order to be able to react to current social changes, new topics on the household questionnaire are added, which are not surveyed every year and are therefore not part of the standard questions of the household questionnaire. You can find a selection of irregular but recurring topics in the replication calendar:


\begin{savenotes}\sphinxatlongtablestart\begin{longtable}{|\X{5}{23}|\X{5}{23}|\X{5}{23}|\X{2}{23}|\X{6}{23}|}
\hline
\sphinxstyletheadfamily 
Topics
&\sphinxstyletheadfamily 
Module
&\sphinxstyletheadfamily 
Replication
&\sphinxstyletheadfamily 
No. Vars
&\sphinxstyletheadfamily 
Variables
\\
\hline
\endfirsthead

\multicolumn{5}{c}%
{\makebox[0pt]{\sphinxtablecontinued{\tablename\ \thetable{} -- continued from previous page}}}\\
\hline
\sphinxstyletheadfamily 
Topics
&\sphinxstyletheadfamily 
Module
&\sphinxstyletheadfamily 
Replication
&\sphinxstyletheadfamily 
No. Vars
&\sphinxstyletheadfamily 
Variables
\\
\hline
\endhead

\hline
\multicolumn{5}{r}{\makebox[0pt][r]{\sphinxtablecontinued{Continued on next page}}}\\
\endfoot

\endlastfoot

\sphinxstylestrong{Home, Amenities and Contributions of private households}
&&&&\\
\hline&
Participation (financial reasons)
&
2001, 2003, 2005, 2007, 2011, 2013, 2015
&
24
&
\sphinxhref{https://paneldata.org/soep-long/data/hl/hlf0174}{hlf0174}\textendash{} \sphinxhref{https://paneldata.org/soep-long/data/hl/hlf0175}{hlf0175} \&  \sphinxhref{https://paneldata.org/soep-long/data/hl/hlf0439}{hlf0439}\textendash{} \sphinxhref{https://paneldata.org/soep-ong/data/hl/hlf0444}{hlf0444} \&  \sphinxhref{https://paneldata.org/soep-long/data/hl/hlf0178}{hlf0178}\textendash{} \sphinxhref{https://paneldata.org/soep-long/data/hl/hlf195}{hlf195}
\\
\hline&
Material deprivation
&
2016
&
24
&
\sphinxhref{https://paneldata.org/soep-long/data/hl/hlf0178}{hlf0178}-\sphinxhref{https://paneldata.org/soep-long/data/hl/hlf0181}{hlf0181} \&  \sphinxhref{https://paneldata.org/soep-long/data/hl/hlf0186}{hlf0186}-\sphinxhref{https://paneldata.org/soep-long/data/hl/hlf0195}{hlf0195} \&  hlf0613 - hlf0622
\\
\hline&
Household equipment since last year
&
1998, 1990, 1992, 1994, 1996, 1998, 2000, 2002, 2004, 2006, 2008, 2010
&
32
&
\sphinxhref{https://paneldata.org/soep-long/data/hl/hlf0163}{hlf0163}-\sphinxhref{https://paneldata.org/soep-long/data/hl/hlf0167}{hlf0167},  \sphinxhref{https://paneldata.org/soep-long/data/hl/hlf0209}{hlf0209},  \sphinxhref{https://paneldata.org/soep-long/data/hl/hlf0212}{hlf0212},  \sphinxhref{https://paneldata.org/soep-long/data/hl/hlf0214}{hlf0214}-\sphinxhref{https://paneldata.org/soep-long/data/hl/hlf0215}{hlf0215},  \sphinxhref{https://paneldata.org/soep-long/data/hl/hlf0217}{hlf0217}-\sphinxhref{https://paneldata.org/soep-long/data/hl/hlf0218}{hlf0218},  \sphinxhref{https://paneldata.org/soep-long/data/hl/hlf0159}{hlf0159},  \sphinxhref{https://paneldata.org/soep-long/data/hl/hlf0223}{hlf0223},  \sphinxhref{https://paneldata.org/soep-long/data/hl/hlf0228}{hlf0228}-\sphinxhref{https://paneldata.org/soep-long/data/hl/hlf0229}{hlf0229},  \sphinxhref{https://paneldata.org/soep-long/data/hl/hlf0231}{hlf0231},  \sphinxhref{https://paneldata.org/soep-long/data/hl/hlc0116}{hlc0116}-\sphinxhref{https://paneldata.org/soep-long/data/hl/hlc0118}{hlc0118},  \sphinxhref{https://paneldata.org/soep-long/data/hl/hlf0233}{hlf0233},  \sphinxhref{https://paneldata.org/soep-long/data/hl/hlf0236}{hlf0236}-\sphinxhref{https://paneldata.org/soep-long/data/hl/hlf0237}{hlf0237},  \sphinxhref{https://paneldata.org/soep-long/data/hl/hlf0169}{hlf0169}-\sphinxhref{https://paneldata.org/soep-long/data/hl/hlf0170}{hlf0170},  \sphinxhref{https://paneldata.org/soep-long/data/hl/hlf0239}{hlf0239}-\sphinxhref{https://paneldata.org/soep-long/data/hl/hlf0242}{hlf0242},  \sphinxhref{https://paneldata.org/soep-long/data/hl/hlf0244}{hlf0244}-\sphinxhref{https://paneldata.org/soep-long/data/hl/hlf0245}{hlf0245},  \sphinxhref{https://paneldata.org/soep-long/data/hl/hlf0247}{hlf0247}-\sphinxhref{https://paneldata.org/soep-long/data/hl/hlf0248}{hlf0248}
\\
\hline&
Amount of Books in household
&
2001, 2006, 2011, 2016
&
1
&
\sphinxhref{https://paneldata.org/soep-long/data/hl/hlf0197}{hlf0197}
\\
\hline&
Pets
&
1996, 2006, 2011, 2016
&
7
&
\sphinxhref{https://paneldata.org/soep-long/data/hl/hlf0254}{hlf0254}-\sphinxhref{https://paneldata.org/soep-long/data/hl/hlf0259}{hlf0259},  \sphinxhref{https://paneldata.org/soep-long/data/hl/hlf0196}{hlf0196}
\\
\hline&
Reasons for Moving and Comparison
&
1985-2013, 2015
&
27
&
\sphinxhref{https://paneldata.org/soep-long/data/hl/hlf0109}{hlf0109}-\sphinxhref{https://paneldata.org/soep-long/data/hl/hlf0132}{hlf0132},  \sphinxhref{https://paneldata.org/soep-long/data/hl/hlf0524}{hlf0524}-\sphinxhref{https://paneldata.org/soep-long/data/hl/hlf0526}{hlf0526}
\\
\hline&
Second Residence
&
2011,2016
&
3
&
\sphinxhref{https://paneldata.org/soep-long/data/hl/hlf0156}{hlf0156}-\sphinxhref{https://paneldata.org/soep-long/data/hl/hlf0158}{hlf0158}
\\
\hline&
Living Environment
&
1986, 1994, 1999, 2004, 2009, 2014
&
22
&
\sphinxhref{https://paneldata.org/soep-long/data/hl/hlf0135}{hlf0135}-\sphinxhref{https://paneldata.org/soep-long/data/hl/hlf0152}{hlf0152},  \sphinxhref{https://paneldata.org/soep-long/data/hl/hlj0004}{hlj0004},  \sphinxhref{https://paneldata.org/soep-long/data/hl/hld0001}{hld0001}-\sphinxhref{https://paneldata.org/soep-long/data/hl/hld0003}{hld0003}
\\
\hline&
Lunch, School
&
1997, 2002, 2005, 2007, 2011, 2013, 2015
&
1
&
\sphinxhref{https://paneldata.org/soep-long/data/kidl/ks\_lunch}{ks\_lunch}
\\
\hline&
Sponsors and Costs, School
&
1987, 1995, 1997, 2002, 2005, 2007, 2011, Mig 2013
&
7
&
\sphinxhref{https://paneldata.org/soep-long/data/kidl/kd\_publ}{kd\_publ},  \sphinxhref{https://paneldata.org/soep-long/data/kidl/kd\_indep}{kd\_indep},  \sphinxhref{https://paneldata.org/soep-long/data/kidl/kd\_priv}{kd\_priv},  \sphinxhref{https://paneldata.org/soep-long/data/kidl/kd\_comp}{kd\_comp},  \sphinxhref{https://paneldata.org/soep-long/data/kidl/kd\_comm}{kd\_comm},  \sphinxhref{https://paneldata.org/soep-long/data/kidl/ks\_amtp}{ks\_amtp},  \sphinxhref{https://paneldata.org/soep-long/data/kidl/ks\_cost}{ks\_cost}
\\
\hline&
Lunch, Childcare
&
1997, 2002, 2005, 2007, 2011, 2013, 2015
&
1
&
\sphinxhref{https://paneldata.org/soep-long/data/kidl/kd\_lunch}{kd\_lunch}
\\
\hline&
Sponsors and Costs, Childcare
&
2011, 2013, 2015
&
7
&
\sphinxhref{https://paneldata.org/soep-long/data/kidl/kd\_publ}{kd\_publ},  \sphinxhref{https://paneldata.org/soep-long/data/kidl/kd\_indep}{kd\_indep},  \sphinxhref{https://paneldata.org/soep-long/data/kidl/kd\_priv}{kd\_priv},  \sphinxhref{https://paneldata.org/soep-long/data/kidl/kd\_comp}{kd\_comp},  \sphinxhref{https://paneldata.org/soep-long/data/kidl/kd\_comm}{kd\_comm},  \sphinxhref{https://paneldata.org/soep-long/data/kidl/kc\_amtp}{kc\_amtp},  \sphinxhref{https://paneldata.org/soep-long/data/kidl/kc\_cost}{kc\_cost}
\\
\hline&
Rely on care times
&
2002
&
1
&
\sphinxhref{https://paneldata.org/soep-core/data/kidlong/kd\_rely}{kd\_rely}
\\
\hline&
Leisure Activities and Costs, Children
&
2006, 2008, 2010, 2012, 2014, 2016
&
19
&
\sphinxhref{https://paneldata.org/soep-long/data/kidl/ka06\_spo}{ka06\_spo},  \sphinxhref{https://paneldata.org/soep-long/data/kidl/ka06\_mus}{ka06\_mus},  \sphinxhref{https://paneldata.org/soep-long/data/kidl/ka06\_art}{ka06\_art},  \sphinxhref{https://paneldata.org/soep-long/data/kidl/ka06\_oth}{ka06\_oth},  \sphinxhref{https://paneldata.org/soep-long/data/kidl/ka06\_non}{ka06\_non},  \sphinxhref{https://paneldata.org/soep-long/data/kidl/ka16\_ssp}{ka16\_ssp},  \sphinxhref{https://paneldata.org/soep-long/data/kidl/ka16\_smu}{ka16\_smu},  \sphinxhref{https://paneldata.org/soep-long/data/kidl/ka16\_sar}{ka16\_sar},  \sphinxhref{https://paneldata.org/soep-long/data/kidl/ka16\_sth}{ka16\_sth},  \sphinxhref{https://paneldata.org/soep-long/data/kidl/ka16\_sot}{ka16\_sot},  \sphinxhref{https://paneldata.org/soep-long/data/kidl/ka16\_spo}{ka16\_spo},  \sphinxhref{https://paneldata.org/soep-long/data/kidl/ka16\_mus}{ka16\_mus},  \sphinxhref{https://paneldata.org/soep-long/data/kidl/ka16\_art}{ka16\_art},  \sphinxhref{https://paneldata.org/soep-long/data/kidl/ka16\_org}{ka16\_org},  \sphinxhref{https://paneldata.org/soep-long/data/kidl/ka16\_yth}{ka16\_yth},  \sphinxhref{https://paneldata.org/soep-long/data/kidl/ka16\_ctr}{ka16\_ctr},  \sphinxhref{https://paneldata.org/soep-long/data/kidl/ka16\_non}{ka16\_non},  \sphinxhref{https://paneldata.org/soep-long/data/kidl/kk\_amtp}{kk\_amtp},  \sphinxhref{https://paneldata.org/soep-long/data/kidl/kk\_cost}{kk\_cost}
\\
\hline&
Independent Income of The Children
&
2016
&&\\
\hline&
Price Comparison Apartment for Rent
&
1984-2014
&
1
&
\sphinxhref{https://paneldata.org/soep-long/data/hl/hlf0094}{hlf0094}
\\
\hline
\sphinxstylestrong{Health and Care}
&&&&\\
\hline&
Satisfaction With Availability Of Care
&
2002
&
1
&
\sphinxhref{https://paneldata.org/soep-long/data/hl/hlf0318}{hlf0318}
\\
\hline
\sphinxstylestrong{Income, Taxes and Social Security}
&&&&\\
\hline&
Saving
&
2010,2016
&
7
&
\sphinxhref{https://paneldata.org/soep-long/data/hl/hlc0024}{hlc0024}-\sphinxhref{https://paneldata.org/soep-long/data/hl/hlc0030}{hlc0030}
\\
\hline&
Expenditures on Food
&
1998, 2000, 2001, 2003, 2005, 2007, 2009, 2011, 2016
&
2
&
\sphinxhref{https://paneldata.org/soep-long/data/hl/hlf0435}{hlf0435}-\sphinxhref{https://paneldata.org/soep-long/data/hl/hlf0436}{hlf0436}
\\
\hline&
Alimony
&
2010
&
4
&
\sphinxhref{https://paneldata.org/soep-long/data/hl/hlc0091}{hlc0091}-\sphinxhref{https://paneldata.org/soep-long/data/hl/hlc0092}{hlc0092},  \sphinxhref{https://paneldata.org/soep-long/data/hl/hld0004}{hld0004}-\sphinxhref{https://paneldata.org/soep-long/data/hl/hld0005}{hld0005}
\\
\hline&
Consumption Module
&
2010
&
122
&
\sphinxhref{https://paneldata.org/soep-long/data/hl/hlf0163}{hlf0163}-\sphinxhref{https://paneldata.org/soep-long/data/hl/hlf0172}{hlf0172},  \sphinxhref{https://paneldata.org/soep-long/data/hl/hlf0209}{hlf0209}-\sphinxhref{https://paneldata.org/soep-long/data/hl/hlf0252}{hlf0252},  \sphinxhref{https://paneldata.org/soep-long/data/hl/hlf0159}{hlf0159},  \sphinxhref{https://paneldata.org/soep-long/data/hl/hlc0166}{hlc0166}-\sphinxhref{https://paneldata.org/soep-long/data/hl/hlc0168}{hlc0168},  \sphinxhref{https://paneldata.org/soep-long/data/hl/hlf0371}{hlf0371}-\sphinxhref{https://paneldata.org/soep-long/data/hl/hlf0434}{hlf0434}
\\
\hline&
Good/Low Income
&
1992, 1997, 2007
&
4
&
\sphinxhref{https://paneldata.org/soep-long/data/hl/hcc0005}{hcc0005}-\sphinxhref{https://paneldata.org/soep-long/data/hl/hcc0010}{hcc0010}
\\
\hline
\sphinxstylestrong{Time Use and Environmental Behaviour}
&&&&\\
\hline&
Traffic and Energy
&
1998, 2003, 2015
&
124
&
\sphinxhref{https://paneldata.org/soep-long/data/hl/hlf0540}{hlf0540}-\sphinxhref{https://paneldata.org/soep-long/data/hl/hlf0591}{hlf0591},  \sphinxhref{https://paneldata.org/soep-long/data/hl/hli0005}{hli0005},  \sphinxhref{https://paneldata.org/soep-long/data/hl/hli0077}{hli0077}-\sphinxhref{https://paneldata.org/soep-long/data/hl/hli0142}{hld0142}
\\
\hline
\end{longtable}\sphinxatlongtableend\end{savenotes}

\sphinxcode{\sphinxupquote{Download Replication Household (csv)}}


\subsection{Individual Questionnaire}
\label{\detokenize{Contents of SOEPcore/index:individual-questionnaire}}\label{\detokenize{Contents of SOEPcore/index:id11}}
\sphinxstylestrong{Availability:} Since 1984

\sphinxstylestrong{Respondent:} Persons over 18 years in the household


\subsubsection{Stable content}
\label{\detokenize{Contents of SOEPcore/index:id12}}
The individual questionnaire has been a standard instrument since the beginning of the SOEP. Because the SOEP has a panel character, important questions have to be answered each year anew by the respondents. In order to enable analyses over time, the individual questionnaire therefore has a large number of question modules which are asked every year. The following question modules are part of the core program of the individual questionnaire.


\begin{savenotes}\sphinxatlongtablestart\begin{longtable}{|\X{5}{18}|\X{5}{18}|\X{2}{18}|\X{6}{18}|}
\hline
\sphinxstyletheadfamily 
Topics
&\sphinxstyletheadfamily 
Module
&\sphinxstyletheadfamily 
No. Vars
&\sphinxstyletheadfamily 
Variables
\\
\hline
\endfirsthead

\multicolumn{4}{c}%
{\makebox[0pt]{\sphinxtablecontinued{\tablename\ \thetable{} -- continued from previous page}}}\\
\hline
\sphinxstyletheadfamily 
Topics
&\sphinxstyletheadfamily 
Module
&\sphinxstyletheadfamily 
No. Vars
&\sphinxstyletheadfamily 
Variables
\\
\hline
\endhead

\hline
\multicolumn{4}{r}{\makebox[0pt][r]{\sphinxtablecontinued{Continued on next page}}}\\
\endfoot

\endlastfoot

\sphinxstylestrong{Attitudes, Values and Personality}
&&&\\
\hline&
Satisfaction with various aspects
&
11
&
\sphinxhref{https://paneldata.org/soep-long/data/pl/plh0171}{plh0171} - \sphinxhref{https://paneldata.org/soep-long/data/pl/plh0181}{plh0181}
\\
\hline&
Mood
&
4
&
\sphinxhref{https://paneldata.org/soep-long/data/pl/plh0184}{plh0184}-  \sphinxhref{https://paneldata.org/soep-long/data/pl/plh0187}{plh0187}
\\
\hline&
Flourishing
&
1
&
\sphinxhref{https://paneldata.org/soep-long/data/pl/plh0334}{plh0334}
\\
\hline&
Risk Aversion
&
1
&
\sphinxhref{https://paneldata.org/soep-long/data/pl/plh0204}{plh0204}
\\
\hline&
Political orientation
&
4
&
\sphinxhref{https://paneldata.org/soep-long/data/pl/plh0007}{plh0007} ,  \sphinxhref{https://paneldata.org/soep-long/data/pl/plh0011}{plh0011} - \sphinxhref{https://paneldata.org/soep-long/data/pl/plh0013}{plh0013}
\\
\hline&
Worries
&
13
&
\sphinxhref{https://paneldata.org/soep-long/data/pl/plh0032}{plh0032} , \sphinxhref{https://paneldata.org/soep-long/data/pl/plh0033}{plh0033} , \sphinxhref{https://paneldata.org/soep-long/data/pl/plh0035}{plh0035} - \sphinxhref{https://paneldata.org/soep-long/data/pl/plh0038}{plh0038} , \sphinxhref{https://paneldata.org/soep-long/data/pl/plh0040}{plh0040} , \sphinxhref{https://paneldata.org/soep-long/data/pl/plh0042}{plh0042} , \sphinxhref{https://paneldata.org/soep-long/data/pl/plh0043}{plh0043} ,  \sphinxhref{https://paneldata.org/soep-long/data/pl/plh0046}{plh0046} ,  \sphinxhref{https://paneldata.org/soep-long/data/pl/plh0047}{plh0047} , \sphinxhref{https://paneldata.org/soep-long/data/pl/plh0335}{plh0335} , \sphinxhref{https://paneldata.org/soep-long/data/pl/plh0336}{plh0336}
\\
\hline&
Life satisfaction
&
1
&
\sphinxhref{https://paneldata.org/soep-long/data/pl/plh0182}{plh0182}
\\
\hline
\sphinxstylestrong{Demography and Population}
&&&\\
\hline&
Origin
&
6
&
\sphinxhref{https://paneldata.org/soep-long/data/pl/plj0014}{plj0014} ,  \sphinxhref{https://paneldata.org/soep-long/data/pl/plj0022}{plj0022} - \sphinxhref{https://paneldata.org/soep-long/data/pl/plj0025}{plj0025}, \sphinxhref{https://paneldata.org/soep-long/data/pl/plj0175}{plj0175}
\\
\hline
\sphinxstylestrong{Education and Qualification}
&&&\\
\hline&
Apprenticeship
&
1
&
\sphinxhref{https://paneldata.org/soep-long/data/pl/plg0012}{plg0012} -  \sphinxhref{https://paneldata.org/soep-long/data/pl/plg0015}{plg0015} ,   \sphinxhref{https://paneldata.org/soep-long/data/pl/plg0264}{plg0264} ,  \sphinxhref{https://paneldata.org/soep-long/data/pl/plg0265}{plg0265}
\\
\hline&
Acquired qualification
&
12
&
\sphinxhref{https://paneldata.org/soep-long/data/pl/plg0072}{plg0072}-  \sphinxhref{https://paneldata.org/soep-long/data/pl/plg0079}{plg0079} ,  \sphinxhref{https://paneldata.org/soep-long/data/pl/plg0284}{plg0284} ,  \sphinxhref{https://paneldata.org/soep-long/data/pl/plg0268}{plg0268}, p\_degree, p\_field
\\
\hline&
Advanced training
&
3
&
\sphinxhref{https://paneldata.org/soep-long/data/pl/plg0269}{plg0269} - \sphinxhref{https://paneldata.org/soep-long/data/pl/plg0271}{plg0271}
\\
\hline
\sphinxstylestrong{Family and Social Networks}
&&&\\
\hline&
Family situation
&
4
&
\sphinxhref{https://paneldata.org/soep-long/data/pl/pld0131}{pld0131} - \sphinxhref{https://paneldata.org/soep-long/data/pl/pld0133}{pld0133}, \sphinxhref{https://paneldata.org/soep-long/data/pl/plk0001}{plk0001}
\\
\hline&
Family changes
&
42
&
\sphinxhref{https://paneldata.org/soep-long/data/pl/pld0012}{pld0012}  - \sphinxhref{https://paneldata.org/soep-long/data/pl/pld0014}{pld0014} ,  \sphinxhref{https://paneldata.org/soep-long/data/pl/pld0038}{pld0038} -  \sphinxhref{https://paneldata.org/soep-long/data/pl/pld0040}{pld0040} ,  \sphinxhref{https://paneldata.org/soep-long/data/pl/pld0134}{pld0134} -  \sphinxhref{https://paneldata.org/soep-long/data/pl/pld0156}{pld0156} ,  \sphinxhref{https://paneldata.org/soep-long/data/pl/pld0158}{pld0158} -  \sphinxhref{https://paneldata.org/soep-long/data/pl/pld0171}{pld0171}
\\
\hline
\sphinxstylestrong{Helath and Care}
&&&\\
\hline&
State of health
&
1
&
\sphinxhref{https://paneldata.org/soep-long/data/pl/ple0008}{ple0008}
\\
\hline&
Disability or severe disability
&
2
&
\sphinxhref{https://paneldata.org/soep-long/data/pl/ple0040}{ple0040}- \sphinxhref{https://paneldata.org/soep-long/data/pl/ple0041}{ple0041}
\\
\hline&
Visits to the doctor
&
2
&
\sphinxhref{https://paneldata.org/soep-long/data/pl/ple0072}{ple0072} , \sphinxhref{https://paneldata.org/soep-long/data/pl/ple0073}{ple0073}
\\
\hline&
Hospital stays
&
3
&
\sphinxhref{https://paneldata.org/soep-long/data/pl/ple0053}{ple0053} , \sphinxhref{https://paneldata.org/soep-long/data/pl/ple0055}{ple0055} , \sphinxhref{https://paneldata.org/soep-long/data/pl/ple0056}{ple0056}
\\
\hline&
Sickness notifications to employer
&
10
&
\sphinxhref{https://paneldata.org/soep-long/data/pl/ple0044}{ple0044} ,  \sphinxhref{https://paneldata.org/soep-long/data/pl/ple0046}{ple0046} ,  \sphinxhref{https://paneldata.org/soep-long/data/pl/ple0048}{ple0048} - \sphinxhref{https://paneldata.org/soep-long/data/pl/ple0052}{ple0052} ,  \sphinxhref{https://paneldata.org/soep-long/data/pl/ple0174}{ple0174} , \sphinxhref{https://paneldata.org/soep-long/data/pl/ple0175}{ple0175} ,  \sphinxhref{https://paneldata.org/soep-long/data/pl/plb0024}{plb0024}
\\
\hline&
Health insurance
&
4
&
\sphinxhref{https://paneldata.org/soep-long/data/pl/ple0097}{ple0097} ,  \sphinxhref{https://paneldata.org/soep-long/data/pl/ple0099}{ple0099} , \sphinxhref{https://paneldata.org/soep-long/data/pl/ple0104}{ple0104} , \sphinxhref{https://paneldata.org/soep-long/data/pl/ple0160}{ple0160}
\\
\hline
\sphinxstylestrong{Income, Taxes and Social Security}
&&&\\
\hline&
Employment earnings and collective wage agreements
&
6
&
\sphinxhref{https://paneldata.org/soep-long/data/pl/plc0013}{plc0013} ,  \sphinxhref{https://paneldata.org/soep-long/data/pl/plc0014}{plc0014} , \sphinxhref{https://paneldata.org/soep-long/data/pl/plc0506}{plc0506} -  \sphinxhref{https://paneldata.org/soep-long/data/pl/plc0509}{plc0509}
\\
\hline&
Additional questions for employees
&
13
&
\sphinxhref{https://paneldata.org/soep-long/data/pl/plc0042}{plc0042} -  \sphinxhref{https://paneldata.org/soep-long/data/pl/plc0054}{plc0054}
\\
\hline&
Additional questions for retirees/pensioners
&
20
&
\sphinxhref{https://paneldata.org/soep-long/data/pl/plc0223}{plc0223} ,  \sphinxhref{https://paneldata.org/soep-long/data/pl/plc0236}{plc0236} ,  \sphinxhref{https://paneldata.org/soep-long/data/pl/plc0238}{plc0238} ,  \sphinxhref{https://paneldata.org/soep-long/data/pl/plc0240}{plc0240} , \sphinxhref{https://paneldata.org/soep-long/data/pl/plc0242}{plc0242} , \sphinxhref{https://paneldata.org/soep-long/data/pl/plc0243}{plc0243} , \sphinxhref{https://paneldata.org/soep-long/data/pl/plc0245}{plc0245} , \sphinxhref{https://paneldata.org/soep-long/data/pl/plc0247}{plc0247} , \sphinxhref{https://paneldata.org/soep-long/data/pl/plc0249}{plc0249} , \sphinxhref{https://paneldata.org/soep-long/data/pl/plc0251}{plc0251} , \sphinxhref{https://paneldata.org/soep-long/data/pl/plc0278}{plc0278} , \sphinxhref{https://paneldata.org/soep-long/data/pl/plc0279}{plc0279} , \sphinxhref{https://paneldata.org/soep-long/data/pl/plc0281}{plc0281} , \sphinxhref{https://paneldata.org/soep-long/data/pl/plc0283}{plc0283} , \sphinxhref{https://paneldata.org/soep-long/data/pl/plc0285}{plc0285} ,  \sphinxhref{https://paneldata.org/soep-long/data/pl/plc0286}{plc0286} , \sphinxhref{https://paneldata.org/soep-long/data/pl/plc0288}{plc0288} ,  \sphinxhref{https://paneldata.org/soep-long/data/pl/plc0290}{plc0290} ,  \sphinxhref{https://paneldata.org/soep-long/data/pl/plc0516}{plc0516} , \sphinxhref{https://paneldata.org/soep-long/data/pl/plc0517}{plc0517}
\\
\hline&
Transfer payments
&
21
&
\sphinxhref{https://paneldata.org/soep-long/data/pl/plj0131}{plj0131} - \sphinxhref{https://paneldata.org/soep-long/data/pl/plj0151}{plj0151}
\\
\hline
\sphinxstylestrong{Time Use and Environmental Behaviour}
&&&\\
\hline&
Calendar
&
12
&
\sphinxhref{https://paneldata.org/soep-long/data/pl2/pab0001}{pab0001}- \sphinxhref{https://paneldata.org/soep-long/data/pl2/pab0008}{pab0008}, \sphinxhref{https://paneldata.org/soep-long/data/pl2/pab0010}{pab010}- \sphinxhref{https://paneldata.org/soep-long/data/pl2/pab0013}{pab0013}
\\
\hline&
Use of time
&
60
&
\sphinxhref{https://paneldata.org/soep-long/data/pl/pli0001}{pli0001}-  \sphinxhref{https://paneldata.org/soep-long/data/pl/pli0060}{pli0060}
\\
\hline
\sphinxstylestrong{Work and Employment / Income, Taxes and Social Security}
&&&\\
\hline&
Secondary occupations
&
9
&
\sphinxhref{https://paneldata.org/soep-long/data/pl/plb0392}{plb0392} - \sphinxhref{https://paneldata.org/soep-long/data/pl/plb0396}{plb0396} ,  \sphinxhref{https://paneldata.org/soep-long/data/pl/plb0573}{plb0573} ,  \sphinxhref{https://paneldata.org/soep-long/data/pl/plc0062}{plc0062} ,  \sphinxhref{https://paneldata.org/soep-long/data/pl2/p\_isco88n}{p\_isco88n} , \sphinxhref{https://paneldata.org/soep-long/data/pl2/p\_isco08n}{p\_isco08n}
\\
\hline&
Income
&
62
&
\sphinxhref{https://paneldata.org/soep-long/data/pl/plc0015}{plc0015} -  \sphinxhref{https://paneldata.org/soep-long/data/pl/plc0017}{plc0017}  ,  \sphinxhref{https://paneldata.org/soep-long/data/pl/plc0064}{plc0064} ,   \sphinxhref{https://paneldata.org/soep-long/data/pl/plc0065}{plc0065} ,  \sphinxhref{https://paneldata.org/soep-long/data/pl/plc0073}{plc0073} - \sphinxhref{https://paneldata.org/soep-long/data/pl/plc0075}{plc0075} ,  \sphinxhref{https://paneldata.org/soep-long/data/pl/plc0116}{plc0116} ,  \sphinxhref{https://paneldata.org/soep-long/data/pl/plc0117}{plc0117} ,  \sphinxhref{https://paneldata.org/soep-long/data/pl/plc0126}{plc0126} , \sphinxhref{https://paneldata.org/soep-long/data/pl/plc0130}{plc0130} - \sphinxhref{https://paneldata.org/soep-long/data/pl/plc0132}{plc0132} , \sphinxhref{https://paneldata.org/soep-long/data/pl/plc0135}{plc0135} -  \sphinxhref{https://paneldata.org/soep-long/data/pl/plc0139}{plc0139} , \sphinxhref{https://paneldata.org/soep-long/data/pl/plc0152}{plc0152} - \sphinxhref{https://paneldata.org/soep-long/data/pl/plc0155}{plc0155} , \sphinxhref{https://paneldata.org/soep-long/data/pl/plc0168}{plc0168} - \sphinxhref{https://paneldata.org/soep-long/data/pl/plc0171}{plc0171} ,  \sphinxhref{https://paneldata.org/soep-long/data/pl/plc0177}{plc0177} ,  \sphinxhref{https://paneldata.org/soep-long/data/pl/plc0178}{plc0178} ,  \sphinxhref{https://paneldata.org/soep-long/data/pl/plc0181}{plc0181} - \sphinxhref{https://paneldata.org/soep-long/data/pl/plc0184}{plc0184} , \sphinxhref{https://paneldata.org/soep-long/data/pl/plc0188}{plc0188} -  \sphinxhref{https://paneldata.org/soep-long/data/pl/plc0190}{plc0190} ,  \sphinxhref{https://paneldata.org/soep-long/data/pl/plc0198}{plc0198} , \sphinxhref{https://paneldata.org/soep-long/data/pl/plc0202}{plc0202} - \sphinxhref{https://paneldata.org/soep-long/data/pl/plc0205}{plc0205} , \sphinxhref{https://paneldata.org/soep-long/data/pl/plc0232}{plc0232} -  \sphinxhref{https://paneldata.org/soep-long/data/pl/plc0235}{plc0235} , \sphinxhref{https://paneldata.org/soep-long/data/pl/plc0273}{plc0273}  - \sphinxhref{https://paneldata.org/soep-long/data/pl/plc0276}{plc0276} ,  \sphinxhref{https://paneldata.org/soep-long/data/pl/plc0488}{plc0488} - \sphinxhref{https://paneldata.org/soep-long/data/pl/plc0490}{plc0490} ,  \sphinxhref{https://paneldata.org/soep-long/data/pl/plc0494}{plc0494} - \sphinxhref{https://paneldata.org/soep-long/data/pl/plc0496}{plc0496} ,  \sphinxhref{https://paneldata.org/soep-long/data/pl/plc0513}{plc0513} , \sphinxhref{https://paneldata.org/soep-long/data/pl/plc0514}{plc0514}, \sphinxhref{https://paneldata.org/soep-long/data/pl/plc0515}{plc0515} , \sphinxhref{https://paneldata.org/soep-long/data/pl/plb0471}{plb0471} , \sphinxhref{https://paneldata.org/soep-long/data/pl/plb0474}{plb0474},  \sphinxhref{https://paneldata.org/soep-long/data/pl/plb0477}{plb0477}
\\
\hline
\sphinxstylestrong{Work and Employment}
&&&\\
\hline&
Work, last 7 days
&
1
&
\sphinxhref{https://paneldata.org/soep-long/data/pl/plb0018}{plb0018}
\\
\hline&
Maternity/ Parental leave
&
1
&
\sphinxhref{https://paneldata.org/soep-long/data/pl/plb0019}{plb0019} , \sphinxhref{https://paneldata.org/soep-long/data/pl/plb0020}{plb0020}
\\
\hline&
Care period (Pflegezeit)
&
1
&
\sphinxhref{https://paneldata.org/soep-long/data/pl/plb0020}{plb0020}
\\
\hline&
Registered unemployed
&
1
&
\sphinxhref{https://paneldata.org/soep-long/data/pl/plb0021}{plb0021}
\\
\hline&
Quitting a profession
&
4
&
\sphinxhref{https://paneldata.org/soep-long/data/pl/plb0282}{plb0282} ,  \sphinxhref{https://paneldata.org/soep-long/data/pl/plb0298}{plb0298}-  \sphinxhref{https://paneldata.org/soep-long/data/pl/plb0305}{plb0305} ,  \sphinxhref{https://paneldata.org/soep-long/data/pl/plc0040}{plc0040}, \sphinxhref{https://paneldata.org/soep-long/data/pl/plc0041}{plc0041}
\\
\hline&
Employment status
&
1
&
\sphinxhref{https://paneldata.org/soep-long/data/pl/plb0022}{plb0022}
\\
\hline&
Start of the job
&
9
&
\sphinxhref{https://paneldata.org/soep-long/data/pl/plb0417}{plb0417} - \sphinxhref{https://paneldata.org/soep-long/data/pl/plb0424}{plb0424} ,  \sphinxhref{https://paneldata.org/soep-long/data/pl/plb0240}{plb0240}
\\
\hline&
Change of job
&
9
&
\sphinxhref{https://paneldata.org/soep-long/data/pl/plb0031}{plb0031} - \sphinxhref{https://paneldata.org/soep-long/data/pl/plb0034}{plb0034} , \sphinxhref{https://paneldata.org/soep-long/data/pl/plb0478}{plb0478}- \sphinxhref{https://paneldata.org/soep-long/data/pl/plb0480}{plb0480} , \sphinxhref{https://paneldata.org/soep-long/data/pl/plb0284}{plb0284} , \sphinxhref{https://paneldata.org/soep-long/data/pl/plb0295}{plb0295}
\\
\hline&
Job search
&
2
&
\sphinxhref{https://paneldata.org/soep-long/data/pl/plb0362}{plb0362} , \sphinxhref{https://paneldata.org/soep-long/data/pl/plb0358}{plb0358}
\\
\hline&
Practised profession
&
4
&
\sphinxhref{https://paneldata.org/soep-long/data/pl/plb0072}{plb0072} , \sphinxhref{https://paneldata.org/soep-long/data/pl/plb0073}{plb0073} ,  \sphinxhref{https://paneldata.org/soep-long/data/pl2/p\_nace}{p\_nace} , \sphinxhref{https://paneldata.org/soep-long/data/pl2/p\_isco08}{p\_isco08}
\\
\hline&
Current employment
&
13
&
\sphinxhref{https://paneldata.org/soep-long/data/pl/plb0035}{plb0035} - \sphinxhref{https://paneldata.org/soep-long/data/pl/plb0037}{plb0037} ,  \sphinxhref{https://paneldata.org/soep-long/data/pl/plb0040}{plb0040} ,  \sphinxhref{https://paneldata.org/soep-long/data/pl/plb0041}{plb0041} ,  \sphinxhref{https://paneldata.org/soep-long/data/pl/plb0049}{plb0049} ,  \sphinxhref{https://paneldata.org/soep-long/data/pl/plb0058}{plb0058} ,  \sphinxhref{https://paneldata.org/soep-long/data/pl/plb0063}{plb0063}- \sphinxhref{https://paneldata.org/soep-long/data/pl/plb0065}{plb0065} ,  \sphinxhref{https://paneldata.org/soep-long/data/pl/plb0568}{plb0568} ,  \sphinxhref{https://paneldata.org/soep-long/data/pl/plb0570}{plb0570} ,  \sphinxhref{https://paneldata.org/soep-long/data/pl/plb0586}{plb0586}
\\
\hline&
Working hours
&
10
&
\sphinxhref{https://paneldata.org/soep-long/data/pl/plb0180}{plb0180} - \sphinxhref{https://paneldata.org/soep-long/data/pl/plb0182}{plb0182},  \sphinxhref{https://paneldata.org/soep-long/data/pl/plb0185}{plb0185} -  \sphinxhref{https://paneldata.org/soep-long/data/pl/plb0188}{plb0188} ,  \sphinxhref{https://paneldata.org/soep-long/data/pl/plb0241}{plb0241} , \sphinxhref{https://paneldata.org/soep-long/data/pl/plb0209}{plb0209} ,  \sphinxhref{https://paneldata.org/soep-long/data/pl/plb0210}{plb0210}
\\
\hline&
Overtime
&
10
&
\sphinxhref{https://paneldata.org/soep-long/data/pl/plb0193}{plb0193} - \sphinxhref{https://paneldata.org/soep-long/data/pl/plb0198}{plb0198} , \sphinxhref{https://paneldata.org/soep-long/data/pl/plb0483}{plb0483} , \sphinxhref{https://paneldata.org/soep-long/data/pl/plb0484}{plb0484} , \sphinxhref{https://paneldata.org/soep-long/data/pl/plb0220}{plb0220} , \sphinxhref{https://paneldata.org/soep-long/data/pl/plb0198}{plb0605}
\\
\hline
\end{longtable}\sphinxatlongtableend\end{savenotes}

\sphinxcode{\sphinxupquote{Download Stable Content Individual (csv)}}


\subsubsection{Replication Calendar Individual Questionnaire}
\label{\detokenize{Contents of SOEPcore/index:replication-calendar-individual-questionnaire}}
Besides the topics that are asked every year in the individual questionnaire, there are some topics modules that are collected irregularly. Many questions do not have to be asked every year as short-term changes are unlikely. In order to be able to react to current social changes, new topics on the individual questionnaire are added, which are not surveyed every year and are therefore not part of the standard questions of the individual questionnaire. You can find a selection of irregular but recurring topics in the replication calendar:


\begin{savenotes}\sphinxatlongtablestart\begin{longtable}{|\X{5}{23}|\X{5}{23}|\X{5}{23}|\X{2}{23}|\X{6}{23}|}
\hline
\sphinxstyletheadfamily 
Topics
&\sphinxstyletheadfamily 
Module
&\sphinxstyletheadfamily 
Replication
&\sphinxstyletheadfamily 
No. Vars
&\sphinxstyletheadfamily 
Variables
\\
\hline
\endfirsthead

\multicolumn{5}{c}%
{\makebox[0pt]{\sphinxtablecontinued{\tablename\ \thetable{} -- continued from previous page}}}\\
\hline
\sphinxstyletheadfamily 
Topics
&\sphinxstyletheadfamily 
Module
&\sphinxstyletheadfamily 
Replication
&\sphinxstyletheadfamily 
No. Vars
&\sphinxstyletheadfamily 
Variables
\\
\hline
\endhead

\hline
\multicolumn{5}{r}{\makebox[0pt][r]{\sphinxtablecontinued{Continued on next page}}}\\
\endfoot

\endlastfoot

\sphinxstylestrong{Attitudes, Values and Personality}
&&&&\\
\hline&
Well-being
&
1990 (only Ost), 1994, 1999
&
13
&
\sphinxhref{https://paneldata.org/soep-long/data/pl/plh0091}{plh0091} - \sphinxhref{https://paneldata.org/soep-long/data/pl/plh0103}{plh0103}
\\
\hline&
Optimism
&
1994, 2005, 2009, 2014
&
1
&
\sphinxhref{https://paneldata.org/soep-long/data/pl/plh0244}{plh0244}
\\
\hline&
Religious Affiliation
&
1990, 1997, 2003, 2007, 2011, 2015
&
1
&
\sphinxhref{https://paneldata.org/soep-long/data/pl/plh0258}{plh0258}
\\
\hline&
Organisational and community membership
&
1985, 1989, 1993, 1998, 2001, 2003, 2007, 2011, 2015
&
5
&
\sphinxhref{https://paneldata.org/soep-long/data/pl/plh0263}{plh0263} -\sphinxhref{https://paneldata.org/soep-long/data/pl/plh0103}{plh0267}
\\
\hline&
Personality traits (Big Five)
&
2005, 2009, 2013, 2017
&
16
&
\sphinxhref{https://paneldata.org/soep-long/data/pl/plh0212}{plh0212} -\sphinxhref{https://paneldata.org/soep-long/data/pl/plh0226}{plh0226}, \sphinxhref{https://paneldata.org/soep-long/data/pl/plh0255}{plh0255}
\\
\hline&
Anomy
&
1992,1993, 1995, 1996, 1997, 2008, 2013
&
4
&
\sphinxhref{https://paneldata.org/soep-long/data/pl/plh0212}{plh0188}-\sphinxhref{https://paneldata.org/soep-long/data/pl/plh0191}{plh0191}
\\
\hline&
Depressive Traits
&
2016
&
4
&
plh0339 \textendash{} plh0342
\\
\hline&
Goals in life (Kluckhohn)
&
1990, 1992, 1995, 2004, 2008, 2012, 2016
&
9
&
\sphinxhref{https://paneldata.org/soep-long/data/pl/plh0105}{plh0105} \textendash{}  \sphinxhref{https://paneldata.org/soep-long/data/pl/plh0112}{plh0112}, plh0343
\\
\hline&
Money and account balance
&
2016
&
3
&
plh0344 \textendash{}  plh0346
\\
\hline&
Control beliefs
&
2005, 2010, 2015
&
10
&
\sphinxhref{https://paneldata.org/soep-long/data/pl/plh0247}{plh0247}  \textendash{}  \sphinxhref{https://paneldata.org/soep-long/data/pl/plh0252}{plh0252} ,   \sphinxhref{https://paneldata.org/soep-long/data/pl/plh0245}{plh0245} ,   \sphinxhref{https://paneldata.org/soep-long/data/pl/plh0246}{plh0246}
\\
\hline&
Reciprocity
&
2005, 2010, 2015
&
11
&
\sphinxhref{https://paneldata.org/soep-long/data/pl/plh0206}{plh0206}  - \sphinxhref{https://paneldata.org/soep-long/data/pl/plh0211}{plh0211},  \sphinxhref{https://paneldata.org/soep-long/data/pl/plh0142}{plh0142}-  \sphinxhref{https://paneldata.org/soep-long/data/pl/plh0146}{plh0146}
\\
\hline&
Trust and fairness
&
2003, 2008, 2013
&
8
&
\sphinxhref{https://paneldata.org/soep-long/data/pl/plh0192}{plh0192} -  \sphinxhref{https://paneldata.org/soep-long/data/pl/plh0196}{plh0196} ,  \sphinxhref{https://paneldata.org/soep-long/data/pl/pld0043}{pld0043} - \sphinxhref{https://paneldata.org/soep-long/data/pl/pld0045}{pld0045}
\\
\hline&
Narcissism
&
2018
&&\\
\hline&
Loneliness
&
2013,2017
&
3
&
\sphinxhref{https://paneldata.org/soep-long/data/pl/plh0269}{plh0269} - \sphinxhref{https://paneldata.org/soep-long/data/pl/plh0271}{plh0271}
\\
\hline&
Impulsivity, patience
&
2008,2013
&
3
&
\sphinxhref{https://paneldata.org/soep-long/data/pl/plh0204}{plh0204}  , \sphinxhref{https://paneldata.org/soep-long/data/pl/plh0253}{plh0253}  , \sphinxhref{https://paneldata.org/soep-long/data/pl/plh0254}{plh0254}
\\
\hline&
Risk Aversion (long)
&
2004,2009
&
6
&
\sphinxhref{https://paneldata.org/soep-long/data/pl/plh0197}{plh0197} -  \sphinxhref{https://paneldata.org/soep-long/data/pl/plh0202}{plh0202}
\\
\hline&
Lottery question
&
2004,2009
&
1
&
\sphinxhref{https://paneldata.org/soep-long/data/pl/plh0203}{plh0203}
\\
\hline&
Policy objectives (Inglehart Index)
&
1984, 1985, 1986, 1996, 2006, 2016
&
4
&
\sphinxhref{https://paneldata.org/soep-long/data/pl/plh0054}{plh0054},  \sphinxhref{https://paneldata.org/soep-long/data/pl/plh0056}{plh0056},  \sphinxhref{https://paneldata.org/soep-long/data/pl/plh0058}{plh0058},  \sphinxhref{https://paneldata.org/soep-long/data/pl/plh0061}{plh0061}
\\
\hline&
Attitudes towards refugees
&
2016
&
11
&
plj0433 \textendash{} plj0443
\\
\hline&
Bundestag election
&
2014
&
1
&
\sphinxhref{https://paneldata.org/soep-long/data/pl/plh0333}{plh0333}
\\
\hline&
Political Tendency, Left-Right
&
2005, 2009, 2014
&
1
&
\sphinxhref{https://paneldata.org/soep-long/data/pl/plh0004}{plh0004}
\\
\hline&
Social responsibility
&
1987, 1992, 1997, 2002, 2017
&
11
&
\sphinxhref{https://paneldata.org/soep-long/data/pl/plh0016}{plh0016} \textendash{}  \sphinxhref{https://paneldata.org/soep-long/data/pl/plh0026}{plh0026}
\\
\hline&
Donations
&
2010,2015
&
2
&
\sphinxhref{https://paneldata.org/soep-long/data/pl/plh0129}{plh0129} ,  \sphinxhref{https://paneldata.org/soep-long/data/pl/plh0130}{plh0130}
\\
\hline&
Donation of blood
&
2010,2015
&
3
&
\sphinxhref{https://paneldata.org/soep-long/data/pl/plh0131}{plh0131} - \sphinxhref{https://paneldata.org/soep-long/data/pl/plh0133}{plh0133}
\\
\hline&
Donations of goods
&
2010
&
8
&
\sphinxhref{https://paneldata.org/soep-long/data/pl/plj0108}{plj0108} - \sphinxhref{https://paneldata.org/soep-long/data/pl/plj0115}{plj0115}
\\
\hline&
10000 Euro Question
&
2010,2017
&
3
&
\sphinxhref{https://paneldata.org/soep-long/data/pl/plh0134}{plh0134}- \sphinxhref{https://paneldata.org/soep-long/data/pl/plh0136}{plh0136}
\\
\hline&
Income justice, general
&
2005
&
12
&
\sphinxhref{https://paneldata.org/soep-long/data/pl/plh0116}{plh0116}-  \sphinxhref{https://paneldata.org/soep-long/data/pl/plh0127}{plh0127}
\\
\hline
\sphinxstylestrong{Family and Social Networks}
&&&&\\
\hline&
Family Network
&
1991, 1996, 2001, 2006, 2011, 2016
&
43
&
\sphinxhref{https://paneldata.org/soep-long/data/pl/pld0020}{pld0020} -  \sphinxhref{https://paneldata.org/soep-long/data/pl/pld0036}{pld0036} \&  \sphinxhref{https://paneldata.org/soep-long/data/pl/pld0107}{pld0107} - \sphinxhref{https://paneldata.org/soep-long/data/pl/pld0118}{pld0118} \& \sphinxhref{https://paneldata.org/soep-long/data/pl/plj0117}{plj0117} -  \sphinxhref{https://paneldata.org/soep-long/data/pl/plj0130}{plj0130}
\\
\hline&
Networks, trusted person
&
1991, 1996, 2001, 2006, 2011, 2016
&
29
&
\sphinxhref{https://paneldata.org/soep-long/data/pl/pld0089}{pld0089} -  \sphinxhref{https://paneldata.org/soep-long/data/pl/pld0088}{pld0088} \&  \sphinxhref{https://paneldata.org/soep-long/data/pl/plf0049}{plf0049} - \sphinxhref{https://paneldata.org/soep-long/data/pl/plf0050}{plf0050}
\\
\hline&
Networks, sociodemography
&
2006, 2011, 2016
&
18
&
\sphinxhref{https://paneldata.org/soep-long/data/pl/pld0089}{pld0089} -  \sphinxhref{https://paneldata.org/soep-long/data/pl/pld0106}{pld0106}
\\
\hline&
2003, 2008, 2011, 2013, 2015, 2017, 2018
&&
1
&
\sphinxhref{https://paneldata.org/soep-long/data/pl/pld0047}{pld0047}
\\
\hline&
LGBT Status
&
2016
&
1
&
pld0298
\\
\hline&
Gender Attitudes
&
2018
&
8
&\\
\hline
\sphinxstylestrong{Health and Care}
&&&&\\
\hline&
Ilness
&
2009, 2011, 2013, 2015, 2017
&
14
&
\sphinxhref{https://paneldata.org/soep-long/data/pl/ple0011}{ple0011} \textendash{}  \sphinxhref{https://paneldata.org/soep-long/data/pl/ple0024}{ple0024}
\\
\hline&
Stress and exhaustion (SF-12)
&
2002 - 2018 (every two years)
&
10
&
\sphinxhref{https://paneldata.org/soep-long/data/pl/ple0026}{ple0026} \textendash{}  \sphinxhref{https://paneldata.org/soep-long/data/pl/ple0036}{ple0036}
\\
\hline&
Disabilities in everyday life (SF-12)
&
1997-2002, 2004-2018 (every two years)
&
2
&
\sphinxhref{https://paneldata.org/soep-long/data/pl/ple0004}{ple0004} \textendash{}  \sphinxhref{https://paneldata.org/soep-long/data/pl/ple0005}{ple0005}
\\
\hline&
Height and Weight
&
2002 - 2018 (every two years)
&
2
&
\sphinxhref{https://paneldata.org/soep-long/data/pl/ple0006}{ple0006} \textendash{}  \sphinxhref{https://paneldata.org/soep-long/data/pl/ple0007}{ple0007}
\\
\hline&
Chronicall Illness
&
1984-1989, 1991, 2009, 2010, 2012, 2014, 2016, 2018
&
1
&
\sphinxhref{https://paneldata.org/soep-long/data/pl/ple0036}{ple0036}
\\
\hline&
Health restrictions
&
2011, 2012, 2013, 2015, 2017
&
2
&
\sphinxhref{https://paneldata.org/soep-long/data/pl/ple0009}{ple0009} \&  \sphinxhref{https://paneldata.org/soep-long/data/pl/ple0162}{ple0162}
\\
\hline&
Health insurance debts
&
2017
&
1
&\\
\hline&
Smoking
&
1998, 1999, 2001, 2002-2018 (every two years)
&
6
&
\sphinxhref{https://paneldata.org/soep-long/data/pl/ple0081}{ple0081}  \& \sphinxhref{https://paneldata.org/soep-long/data/pl/ple0086}{ple0086} -  \sphinxhref{https://paneldata.org/soep-long/data/pl/ple0088}{ple0088} \&  ple0176
\\
\hline&
Alcoholic beverages
&
2006, 2008, 2010, 2016
&
2
&
\sphinxhref{https://paneldata.org/soep-long/data/pl/ple0177}{ple0177} \textendash{}  \sphinxhref{https://paneldata.org/soep-long/data/pl/ple0178}{ple0178}
\\
\hline&
Nutritional awareness
&
2004-2016 (every two years)
&
4
&
ple0179 \textendash{} ple0182
\\
\hline&
Assisted or curative care
&
1999-2011
&
1
&
\sphinxhref{https://paneldata.org/soep-long/data/pl/ple0121}{ple0121}
\\
\hline&
Additional private insurance
&
2011-2014, 2016, 2018
&
8
&
\sphinxhref{https://paneldata.org/soep-long/data/pl/ple0127}{ple0127} \textendash{}  \sphinxhref{https://paneldata.org/soep-long/data/pl/ple0134}{ple0134}
\\
\hline&
Private supplementary care insurance
&
2016,2018
&
3
&
ple0183 \textendash{}  ple0185
\\
\hline&
Insurance status
&
2018
&&\\
\hline&
Type of disability
&
2001, 2002, 2004, 2006, 2008, 2010, 2015
&
2
&
\sphinxhref{https://paneldata.org/soep-long/data/pl/ple0140}{ple0140} \textendash{}  \sphinxhref{https://paneldata.org/soep-long/data/pl/ple0141}{ple0141}
\\
\hline&
Individual health service
&
2016,2018
&
1
&
ple0186
\\
\hline
\sphinxstylestrong{Income,Taxes and Social Security}
&&&&\\
\hline&
Labour income, hourly wage
&
2017
&&\\
\hline&
Earnings Work October 2014
&
2015
&
2
&
\sphinxhref{https://paneldata.org/soep-long/data/pl/plb0584}{plb0584} , \sphinxhref{https://paneldata.org/soep-long/data/pl/plb0585}{plb0585}
\\
\hline&
Balance sheet of assets
&
1988, 2002, 2007, 2012, 2017
&
67
&
\sphinxhref{https://paneldata.org/soep-long/data/pl/plc0315}{plc0315} - \sphinxhref{https://paneldata.org/soep-long/data/pl/plc0319}{plc0319} ,  \sphinxhref{https://paneldata.org/soep-long/data/pl/plc0328}{plc0328} -  \sphinxhref{https://paneldata.org/soep-long/data/pl/plc0374}{plc0374} ,  \sphinxhref{https://paneldata.org/soep-long/data/pl/plc0411}{plc0411} - \sphinxhref{https://paneldata.org/soep-long/data/pl/plc0425}{plc0425}
\\
\hline&
Inheritance
&
2001,2017
&
33
&
\sphinxhref{https://paneldata.org/soep-long/data/pl/plc0375}{plc0375} - \sphinxhref{https://paneldata.org/soep-long/data/pl/plc0407}{plc0407}
\\
\hline&
Transfer payments, income
&
2009, 2010, 2011
&
21
&
\sphinxhref{https://paneldata.org/soep-long/data/pl/plj0152}{plj0152}- \sphinxhref{https://paneldata.org/soep-long/data/pl/plj0172}{plj0172}
\\
\hline&
Social security
&
1987, 1992, 1997, 2007, 2012, 2017
&
7
&
\sphinxhref{https://paneldata.org/soep-long/data/pl/plc0008}{plc0008},  \sphinxhref{https://paneldata.org/soep-long/data/pl/plc0009}{plc0009},  \sphinxhref{https://paneldata.org/soep-long/data/pl/plc0111}{plc0111}- \sphinxhref{https://paneldata.org/soep-long/data/pl/plc0115}{plc0115}
\\
\hline&
Entitlements, statutory
&
2013
&
4
&
\sphinxhref{https://paneldata.org/soep-long/data/pl/plc0008}{plc0432}- \sphinxhref{https://paneldata.org/soep-long/data/pl/plc0435}{plc0435}
\\
\hline&
Riester
&
2004, 2006, 2007, 2010, 2012, 2013, 2015, 2017
&
3
&
\sphinxhref{https://paneldata.org/soep-long/data/pl/plc0430}{plc0430} ,  \sphinxhref{https://paneldata.org/soep-long/data/pl/plc0431}{plc0431} ,  \sphinxhref{https://paneldata.org/soep-long/data/pl/plc0313}{plc0313}
\\
\hline&
Riester payments
&
2013
&
3
&
\sphinxhref{https://paneldata.org/soep-long/data/pl/plc0437}{plc0437} - \sphinxhref{https://paneldata.org/soep-long/data/pl/plc0439}{plc0439}
\\
\hline&
Entitlements, company
&
2013
&
5
&
\sphinxhref{https://paneldata.org/soep-long/data/pl/plc0441}{plc0441} - \sphinxhref{https://paneldata.org/soep-long/data/pl/plc0445}{plc0445}
\\
\hline
\sphinxstylestrong{Immigration, Migration, Transnationalization}
&&&&\\
\hline&
Integration Indicators
&
1997, 1999, 2001, 2003, 2010, 2012, 2014, 2016, 2018
&
2
&
\sphinxhref{https://paneldata.org/soep-long/data/pl/plj0078}{plj0078} \& \sphinxhref{https://paneldata.org/soep-long/data/pl/plj0080}{plj0080}
\\
\hline&
Native tongue
&
2007-2011, 2013, 2015, 2017
&
8
&
\sphinxhref{https://paneldata.org/soep-long/data/pl/plj0009}{plj0009} \&  plj0691 -  plj0693 \& \sphinxhref{https://paneldata.org/soep-long/data/pl/plj0071}{plj0071} -  \sphinxhref{https://paneldata.org/soep-long/data/pl/plj0073}{plj0073} \&  \sphinxhref{https://paneldata.org/soep-long/data/pl/plj0077}{plj0077}
\\
\hline&
Contact, at home and abroad
&
1997-2017 (every two yeras)
&
4
&
\sphinxhref{https://paneldata.org/soep-long/data/pl/plj0060}{plj0060} -  \sphinxhref{https://paneldata.org/soep-long/data/pl/plj0063}{plj0063}
\\
\hline&
Citizenship Application
&
1998-2018 (every two years)
&
1
&
\sphinxhref{https://paneldata.org/soep-long/data/pl/plj0021}{plj0021}
\\
\hline&
Residence status, citizenship
&
2018
&
2
&\\
\hline&
Intention to stay
&
1997-2011, 2013, 2015, 2017
&
4
&
\sphinxhref{https://paneldata.org/soep-long/data/pl/plj0085}{plj0085} -  \sphinxhref{https://paneldata.org/soep-long/data/pl/plj0088}{plj0088}
\\
\hline&
Disadvantages due to origin (short)
&
1997-2011, 2013, 2017
&
1
&
\sphinxhref{https://paneldata.org/soep-long/data/pl/plj0048}{plj0048}
\\
\hline&
Disadvantages due to origin (area)
&
2015
&
15
&
\sphinxhref{https://paneldata.org/soep-long/data/pl/plj0048}{plj0048} \&  \sphinxhref{https://paneldata.org/soep-long/data/pl/plj0327}{plj0327} -  \sphinxhref{https://paneldata.org/soep-long/data/pl/plj0340}{plj0340}
\\
\hline&
Linguistic usage, newspaper
&
1996-2012 (every two years)
&
1
&
\sphinxhref{https://paneldata.org/soep-long/data/pl/plj0070}{plj0070}
\\
\hline&
Linguistic usage, media
&
2014, 2016, 2018
&
1
&
\sphinxhref{https://paneldata.org/soep-long/data/pl/plj0226}{plj0226}
\\
\hline&
Visit country of origin last 2 years
&
1996-2018 (every two years)
&
2
&
\sphinxhref{https://paneldata.org/soep-long/data/pl/plj0322}{plj0322} \&  \sphinxhref{https://paneldata.org/soep-long/data/pl/plj0323}{plj0323}
\\
\hline&
Sense of home
&
1996-2014 (every two years)
&
2
&
\sphinxhref{https://paneldata.org/soep-long/data/pl/plj0083}{plj0083} \&  \sphinxhref{https://paneldata.org/soep-long/data/pl/plj0340}{plj0340}
\\
\hline&
Regional attachment
&
2009,2014
&
3
&
\sphinxhref{https://paneldata.org/soep-long/data/pl/plj0343}{plj0343} -  \sphinxhref{https://paneldata.org/soep-long/data/pl/plj0345}{plj0345}
\\
\hline&
Contacts and thoughts abroad
&
2009,2014
&
6
&
\sphinxhref{https://paneldata.org/soep-long/data/pl/plj0104}{plj0104} \&  \sphinxhref{https://paneldata.org/soep-long/data/pl/plj0105}{plj0105} \&  \sphinxhref{https://paneldata.org/soep-long/data/pl/plj0089}{plj0089} - \sphinxhref{https://paneldata.org/soep-long/data/pl/plj0071}{plj0092}
\\
\hline&
Circle of Friends, Share of Migrants
&
2013,2018
&
1
&
\sphinxhref{https://paneldata.org/soep-long/data/pl/plj0191}{plj0191}
\\
\hline&
Foreign language skills
&
2013
&
1
&
plm0135 \& \sphinxhref{https://paneldata.org/soep-long/data/pl/plj0187}{plj0187}
\\
\hline
\sphinxstylestrong{Time Use and Environmental Behaviour}
&&&&\\
\hline&
Leisure activities (short)
&
1984-1986, 1988, 1992, 1994,1996, 1997, 1999,  2001, 2005, 2007, 2009, 2011, 2015, 2017
&
9
&
\sphinxhref{https://paneldata.org/soep-long/data/pl/pli0090}{pli0090} \textendash{}  \sphinxhref{https://paneldata.org/soep-long/data/pl/pli0098}{pli0098}
\\
\hline&
Leisure activities (long)
&
1990, 1995, 1998, 2003, 2008, 2013
&
19
&
\sphinxhref{https://paneldata.org/soep-long/data/pl/pli0079}{pli0079} \textendash{}  \sphinxhref{https://paneldata.org/soep-long/data/pl/pli0092}{pli0092} \& \sphinxhref{https://paneldata.org/soep-long/data/pl/pli0096}{pli0096} \textendash{} \sphinxhref{https://paneldata.org/soep-long/data/pl/pli0098}{pli0098} \& \sphinxhref{https://paneldata.org/soep-long/data/pl/pli0096}{pli0165} \& \sphinxhref{https://paneldata.org/soep-long/data/pl/pli0168}{pli0168}
\\
\hline&
Computer usage
&
1997,1999, 2000, 2001
&
8
&
\sphinxhref{https://paneldata.org/soep-long/data/pl/pli0066}{pli0066} \textendash{}  \sphinxhref{https://paneldata.org/soep-long/data/pl/pli0073}{pli0073}
\\
\hline&
Traffic behavior
&
1993,1998, 2003
&
77
&
\sphinxhref{https://paneldata.org/soep-long/data/pl/pli0101}{pli0101}  \textendash{}  \sphinxhref{https://paneldata.org/soep-long/data/pl/pli0160}{pli0160} \&   \sphinxhref{https://paneldata.org/soep-long/data/pl/plb0016}{plb0016} \&   \sphinxhref{https://paneldata.org/soep-long/data/pl/plb0145}{plb0145} \&  \sphinxhref{https://paneldata.org/soep-long/data/pl/plb0147}{plb0147} \textendash{}  \sphinxhref{https://paneldata.org/soep-long/data/pl/pli0156}{pli0156} \&  \sphinxhref{https://paneldata.org/soep-long/data/pl/plb0158}{plb0158} \&  \sphinxhref{https://paneldata.org/soep-long/data/pl/plb0159}{plb0159} \&  \sphinxhref{https://paneldata.org/soep-long/data/pl/plb0175}{plb0175} \&  \sphinxhref{https://paneldata.org/soep-long/data/pl/plb0591}{plb0591}
\\
\hline
\sphinxstylestrong{Work and Employment}
&&&&\\
\hline&
Illegal Employment
&
2015,2016
&
2
&
\sphinxhref{https://paneldata.org/soep-long/data/pl/plb0571}{plb0571} , \sphinxhref{https://paneldata.org/soep-long/data/pl/plb0572}{plb0572}
\\
\hline&
Intensity of work
&
2015, 2016, 2017
&
2
&
\sphinxhref{https://paneldata.org/soep-long/data/pl/plb0593}{plb0593},  \sphinxhref{https://paneldata.org/soep-long/data/pl/plb0594}{plb0594}
\\
\hline&
Work equipment
&
2015, 2016, 2017
&
6
&
\sphinxhref{https://paneldata.org/soep-long/data/pl/plb0595}{plb0595}-  \sphinxhref{https://paneldata.org/soep-long/data/pl/plb0600}{plb0600}
\\
\hline&
Work breaks
&
2015, 2016, 2017
&
3
&
\sphinxhref{https://paneldata.org/soep-long/data/pl/plb0601}{plb0601}- \sphinxhref{https://paneldata.org/soep-long/data/pl/plb0603}{plb0603}
\\
\hline&
Employment October 2014
&
2015
&
1
&
\sphinxhref{https://paneldata.org/soep-long/data/pl/plb0574}{plb0574}
\\
\hline&
Work breaks October 2014
&
2015
&
4
&
\sphinxhref{https://paneldata.org/soep-long/data/pl/plb0575}{plb0575} - \sphinxhref{https://paneldata.org/soep-long/data/pl/plb0578}{plb0578}
\\
\hline&
Working time October 2014
&
2015
&
3
&
\sphinxhref{https://paneldata.org/soep-long/data/pl/plb0579}{plb0579} - \sphinxhref{https://paneldata.org/soep-long/data/pl/plb0581}{plb0581}
\\
\hline&
Overtime October 2014
&
2015
&
2
&
\sphinxhref{https://paneldata.org/soep-long/data/pl/plb0582}{plb0582} ,  \sphinxhref{https://paneldata.org/soep-long/data/pl/plb0583}{plb0583}
\\
\hline&
Lifelong learning
&
2014
&
1
&
\sphinxhref{https://paneldata.org/soep-long/data/pl/plg0266}{plg0266}
\\
\hline&
Continuing education, initiative
&
1989, 1993, 2000, 2004, 2008, 2014
&
2
&
\sphinxhref{https://paneldata.org/soep-long/data/pl/plg0273}{plg0273} ,  \sphinxhref{https://paneldata.org/soep-long/data/pl/plg0274}{plg0274}
\\
\hline&
Further training, financing
&
1989, 1993, 2000, 2004, 2008, 2014, 2015, 2017
&
7
&
\sphinxhref{https://paneldata.org/soep-long/data/pl/plg0285}{plg0285} - \sphinxhref{https://paneldata.org/soep-long/data/pl/plg0291}{plg0291}
\\
\hline&
Further education, organizer
&&
1
&\\
\hline&
Continuing education, reasons for failure
&
1989, 1993, 2000, 2004, 2014
&
5
&
\sphinxhref{https://paneldata.org/soep-long/data/pl/plg0277}{plg0277} \textendash{}  \sphinxhref{https://paneldata.org/soep-long/data/pl/plg0281}{plg0281}
\\
\hline&
Further education, course details and motives
&
1989, 1993, 2000, 2004, 2008
&
60
&
\sphinxhref{https://paneldata.org/soep-long/data/pl/plg0108}{plg0108} -  \sphinxhref{https://paneldata.org/soep-long/data/pl/plg0122}{plg0122} ,  \sphinxhref{https://paneldata.org/soep-long/data/pl/plg0129}{plg0129} -  \sphinxhref{https://paneldata.org/soep-long/data/pl/plg0149}{plg0149} ,  \sphinxhref{https://paneldata.org/soep-long/data/pl/plg0152}{plg0152} ,  \sphinxhref{https://paneldata.org/soep-long/data/pl/plg0154}{plg0154} ,  \sphinxhref{https://paneldata.org/soep-long/data/pl/plg0164}{plg0164} ,  \sphinxhref{https://paneldata.org/soep-long/data/pl/plg0165}{plg0165} ,  \sphinxhref{https://paneldata.org/soep-long/data/pl/plg0169}{plg0169} ,  \sphinxhref{https://paneldata.org/soep-long/data/pl/plg0171}{plg0171} ,  \sphinxhref{https://paneldata.org/soep-long/data/pl/plg0172}{plg0172} ,  \sphinxhref{https://paneldata.org/soep-long/data/pl/plg0174}{plg0174} -  \sphinxhref{https://paneldata.org/soep-long/data/pl/plg0177}{plg0177} ,  \sphinxhref{https://paneldata.org/soep-long/data/pl/plg0182}{plg0182} - \sphinxhref{https://paneldata.org/soep-long/data/pl/plg0186}{plg0186}
\\
\hline&
Benefits from employer, additional benefits
&
2008, 2010, 2012, 2014-2017
&
14
&
\sphinxhref{https://paneldata.org/soep-long/data/pl/plc0026}{plc0026} - \sphinxhref{https://paneldata.org/soep-long/data/pl/plc0039}{plc0039}
\\
\hline&
Benefits from employer, company car
&
2016,2017
&
1
&
plc0532
\\
\hline&
Work time regulation
&
2003, 2005, 2007, 2009, 2011, 2014-2017
&
1
&
\sphinxhref{https://paneldata.org/soep-long/data/pl/plb0211}{plb0211}
\\
\hline&
Standby duty
&
2011, 2014-2017
&
4
&
\sphinxhref{https://paneldata.org/soep-long/data/pl/plb0212}{plb0212}- \sphinxhref{https://paneldata.org/soep-long/data/pl/plb0215}{plb0215}
\\
\hline&
Work time recording
&&&\\
\hline&
Overtime, compensation
&
1984-2014
&
1
&
\sphinxhref{https://paneldata.org/soep-long/data/pl/plb0195}{plb0195}
\\
\hline&
Start of working hours
&
2002, 2004, 2006, 2008, 2012, 2015, 2017
&
3
&
\sphinxhref{https://paneldata.org/soep-long/data/pl/plb0180}{plb0180}- \sphinxhref{https://paneldata.org/soep-long/data/pl/plb0182}{plb0182}
\\
\hline&
Evening - Weekend work
&
2005, 2007, 2009, 2011, 2013, 2015, 2017
&
4
&
\sphinxhref{https://paneldata.org/soep-long/data/pl/plb0216}{plb0216}- \sphinxhref{https://paneldata.org/soep-long/data/pl/plb0219}{plb0219}
\\
\hline&
Contract to Provide Specific Services (Werkvertrag)
&
2013,2015
&
1
&
\sphinxhref{https://paneldata.org/soep-long/data/pl/plb0482}{plb0482}
\\
\hline&
Wage Tax Classification
&
1991, 1993, 2004, 2016
&
1
&
\sphinxhref{https://paneldata.org/soep-long/data/pl/plc0091}{plc0091}
\\
\hline&
Exercised profession, training
&
1984-2014, 2016
&
4
&
\sphinxhref{https://paneldata.org/soep-long/data/pl/plb0076}{plb0076}- \sphinxhref{https://paneldata.org/soep-long/data/pl/plb0079}{plb0079}
\\
\hline&
Commuter Module
&
1991-2013, 2015, 2017
&
6
&
\sphinxhref{https://paneldata.org/soep-long/data/pl/plb0589}{plb0589}-  \sphinxhref{https://paneldata.org/soep-long/data/pl/plb0592}{plb0592},   \sphinxhref{https://paneldata.org/soep-long/data/pl/plb0158}{plb0158},  \sphinxhref{https://paneldata.org/soep-long/data/pl/plb0159}{plb0159}
\\
\hline&
Vacation claim
&
2000, 2005, 2010
&
8
&
\sphinxhref{https://paneldata.org/soep-long/data/pl/plb0269}{plb0269}- \sphinxhref{https://paneldata.org/soep-long/data/pl/plb0276}{plb0276}
\\
\hline&
Work from home
&
1997, 1999,  2002, 2009, 2014
&
3
&
\sphinxhref{https://paneldata.org/soep-long/data/pl/plb0095}{plb0095}- \sphinxhref{https://paneldata.org/soep-long/data/pl/plb0097}{plb0097}
\\
\hline&
Short-time allowance (Kurzarbeitergeld)
&
1984-2001, 2003-2005, 2010, 2011,
&
2
&
\sphinxhref{https://paneldata.org/soep-long/data/pl/plc0057}{plc0057} ,  \sphinxhref{https://paneldata.org/soep-long/data/pl/plc0058}{plc0058}
\\
\hline&
Performance evaluation in the company
&
2004, 2008, 2011, 2016
&
5
&
\sphinxhref{https://paneldata.org/soep-long/data/pl/plb0099}{plb0098}- \sphinxhref{https://paneldata.org/soep-long/data/pl/plb0102}{plb0102}
\\
\hline&
Leading position
&
2007, 2009, 2011, 2013, 2015, 2017
&
3
&
\sphinxhref{https://paneldata.org/soep-long/data/pl/plb0067}{plb0067}-  \sphinxhref{https://paneldata.org/soep-long/data/pl/plb0069}{plb0069}
\\
\hline&
Wage justice
&
2005, 2007, 2009, 2011, 2013, 2015, 2017
&
6
&
\sphinxhref{https://paneldata.org/soep-long/data/pl/plh0138}{plh0138}-  \sphinxhref{https://paneldata.org/soep-long/data/pl/plh0141}{plh0141} ,  \sphinxhref{https://paneldata.org/soep-long/data/pl/plh0337}{plh0337} ,  \sphinxhref{https://paneldata.org/soep-long/data/pl/plh0338}{plh0338}
\\
\hline&
Workload (Effort-Reward-Inbalance)
&
2001, 2006, 2011, 2016
&
26
&
\sphinxhref{https://paneldata.org/soep-long/data/pl/plb0112}{plb0112}- \sphinxhref{https://paneldata.org/soep-long/data/pl/plb0137}{plb0137}
\\
\hline&
Professional expectations, long
&
1985, 1987, 1989, 1991, 1993, 1994, 1996, 1998, 2000, 2005, 2009, 2013
&
11
&
\sphinxhref{https://paneldata.org/soep-long/data/pl/plb0432}{plb0432}- \sphinxhref{https://paneldata.org/soep-long/data/pl/plb0442}{plb0442}
\\
\hline&
Professional expectations, short
&
2015
&
5
&
\sphinxhref{https://paneldata.org/soep-long/data/pl/plb0433}{plb0433},  \sphinxhref{https://paneldata.org/soep-long/data/pl/plb0437}{plb0437},  \sphinxhref{https://paneldata.org/soep-long/data/pl/plb0440}{plb0440},  \sphinxhref{https://paneldata.org/soep-long/data/pl/plb0588}{plb0588}
\\
\hline&
Employee organization (Betriebsrat)
&
2001,2006, 2011, 2016
&
1
&
\sphinxhref{https://paneldata.org/soep-long/data/pl/plb0050}{plb0050}
\\
\hline&
Occupational qualification, use
&
1985-2007, 2009
&
1
&
\sphinxhref{https://paneldata.org/soep-long/data/pl/plb0357}{plb0357}
\\
\hline&
Self-employment, reasons
&
2010,2015
&
6
&
\sphinxhref{https://paneldata.org/soep-long/data/pl/plb0333}{plb0333}- \sphinxhref{https://paneldata.org/soep-long/data/pl/plb0338}{plb0338}
\\
\hline&
Postcode of the place of work
&
2016
&&\\
\hline&
Occupational expectations, non-workers
&
1999, 2001, 2003, 2005, 2007, 2009, 2011, 2013, 2015
&
3
&
\sphinxhref{https://paneldata.org/soep-long/data/pl/plb0427}{plb0427}- \sphinxhref{https://paneldata.org/soep-long/data/pl/plb0429}{plb0429}
\\
\hline&
Job search, preferences
&
1994, 1996, 1997, 1998, 1999, 2000, 2001, 2003, 2005, 2007, 2009, 2011, 2013, 2015, 2017
&
1
&
\sphinxhref{https://paneldata.org/soep-long/data/pl/plb0426}{plb0426}
\\
\hline&
Job search, motives
&
1994, 1995, 1996, 1997, 1998, 1999, 2000, 2001, 2003, 2005, 2007, 2009, 2011, 2013, 2017
&
1
&
\sphinxhref{https://paneldata.org/soep-long/data/pl/plb0111}{plb0111}
\\
\hline
\end{longtable}\sphinxatlongtableend\end{savenotes}

\sphinxcode{\sphinxupquote{Download Replication Individual (csv)}}


\subsection{Biography Questionnaire}
\label{\detokenize{Contents of SOEPcore/index:biography-questionnaire}}\label{\detokenize{Contents of SOEPcore/index:id24}}
\sphinxstylestrong{Availability:} Since 1987

\sphinxstylestrong{Respondent:} Supplementary, one-time data on the personal questionnaire of all persons aged 18 and over in the HH.

\sphinxstylestrong{Content:}
\begin{itemize}
\item {} 
Nationality

\item {} 
Origin

\item {} 
Childhood

\item {} 
Parents

\item {} 
Life course since the age of 15

\item {} 
Education

\item {} 
Occupation

\item {} 
Partnership/ Marriage

\item {} 
Information on children

\item {} 
Siblings

\end{itemize}


\subsection{Mother-Child Instruments}
\label{\detokenize{Contents of SOEPcore/index:mother-child-instruments}}

\begin{savenotes}\sphinxattablestart
\centering
\begin{tabular}[t]{|*{6}{\X{1}{6}|}}
\hline
\sphinxstyletheadfamily 
Topic
&\sphinxstyletheadfamily 
Mother-Child A (Age 0-1)
&\sphinxstyletheadfamily 
Mother-Child B (Age 2-3)
&\sphinxstyletheadfamily 
Mother-Child C (Age 5-6)
&\sphinxstyletheadfamily 
Parents D (Age 7-8)
&\sphinxstyletheadfamily 
Mother-Child E (Age 9-10)
\\
\hline
Pregnancy \& Birth
&
x
&&&&\\
\hline
Nursing
&
x
&
x
&&&\\
\hline
Health
&
x
&
x
&
x
&&
x
\\
\hline
Height \& Weight
&&
x
&
x
&&\\
\hline
Vineland Adaptive Behavior Scale
&&
x
&&&\\
\hline
Strength and Difficulties Questionnaire
&&&
x
&&
x
\\
\hline
Childcare
&
x
&
x
&
x
&
x
&
x
\\
\hline
Linguistic Usage
&&
x
&&&
x
\\
\hline
Temper
&
x
&
x
&&&\\
\hline
Big 5 Personality Traits
&&
x
&
x
&&
x
\\
\hline
School \& Homework
&&&&\begin{enumerate}
\setcounter{enumi}{23}
\item {} 
\end{enumerate}
&
x
\\
\hline
Educational Aspirations
&&&&
x
&
x
\\
\hline
Parenting Goals
&&&&
x
&\\
\hline
Styles of Parenting
&&&&
x
&\\
\hline
Mother Role/ Parent Role
&
x
&&&
x
&\\
\hline
Leisure and Activities (with Child)
&&
x
&
x
&&
x
\\
\hline
Friends
&&&&&
x
\\
\hline
Pocket Money
&&&&&
x
\\
\hline
\end{tabular}
\par
\sphinxattableend\end{savenotes}


\subsubsection{Mother-Child Questionnaire A (Age 0-1)}
\label{\detokenize{Contents of SOEPcore/index:mother-child-questionnaire-a-age-0-1}}\label{\detokenize{Contents of SOEPcore/index:mother-child-questionnaire-a}}
Mothers of newborn children primarily answer questions about the course of pregnancy, birth, breastfeeding and the health of the newborn child. It also asks to what extent the mother feels that her life circumstances have changed after the birth of the child, how the care of the child is regulated and how the temperament of the baby (as a precursor of the personality) is perceived by mothers.

\sphinxstylestrong{Availability:} Since 2003

\sphinxstylestrong{Respondent:} Mother in household (child age 0-1)

\sphinxstylestrong{Content:}
\begin{itemize}
\item {} 
Course of pregnancy

\item {} 
Childbirth

\item {} 
Health screening

\item {} 
Well-being

\item {} 
Childcare

\item {} 
Life circumstances

\end{itemize}


\subsubsection{Mother-Child Questionnaire B (Age 2-3)}
\label{\detokenize{Contents of SOEPcore/index:mother-child-questionnaire-b-age-2-3}}\label{\detokenize{Contents of SOEPcore/index:mother-child-questionnaire-b}}
Mothers of 2-3-year-old children also answer some questions about their child’s health and how long they have been breastfeeding. In addition, the child’s care situation is asked, again the temperament as well as a short scale for recording the personality (agreeableness, extraversion, openness and conscientiousness of the Big Five; McCrae and Costa 1987). In addition, the use of language in the family and activities carried out with the children (e.g. going to the playground, reading or telling stories, visiting other families with children) are recorded. Mothers also assess their children’s adaptive behaviour in the dimensions of communication, everyday skills, social relationships and motor skills. The acquisition is based on a translated version of the Vineland Adpative Behavior Scale, which was reduced to 20 items for the SOEP. This scale thus investigates the stage of development of the infant in everyday life.

\sphinxstylestrong{Availability:} Since 2005

\sphinxstylestrong{Respondent:} Mother in household (child age 2-3)

\sphinxstylestrong{Content:}
\begin{itemize}
\item {} 
Personality of the child

\item {} 
Well-being

\item {} 
Childcare

\item {} 
Language skills

\item {} 
Development

\item {} 
Abilities

\end{itemize}


\subsubsection{Mother-Child Questionnaire C (Age 5-6)}
\label{\detokenize{Contents of SOEPcore/index:mother-child-questionnaire-c-age-5-6}}\label{\detokenize{Contents of SOEPcore/index:mother-child-questionnaire-c}}
The subsequent age-specific survey is carried out as soon as the children turn six years old in the survey year. Among the topics it resembles the surveys conducted in previous years: health, care situation, a more comprehensive battery of items on the personality (from this age neuroticisum is also collected) and activities that are carried out with the child. In addition, there is the Strength and Difficulties Questionnaire (SDQ), which is a shortened version of the German version of the SDQ to 17 items and is a very frequently used instrument for the mental health of children and young people.

\sphinxstylestrong{Availability:} Since 2008

\sphinxstylestrong{Respondent:} Mother in household (child age 5-6)

\sphinxstylestrong{Content:}
\begin{itemize}
\item {} 
Personality of the child

\item {} 
Activities with children

\item {} 
Well-being

\item {} 
Childcare

\end{itemize}


\subsubsection{Parents Questionnaire D (Age 7-8)}
\label{\detokenize{Contents of SOEPcore/index:parents-questionnaire-d-age-7-8}}\label{\detokenize{Contents of SOEPcore/index:parents-d}}
The questionnaire, which was developed for 7-8-year-old children, is the only age-specific instrument to be completed by both parents, as long as they live together in the same household. In this age range, questions about school attendance (time of school enrolment) and idealistic and realistic educational aspirations become relevant for the first time. However, the focus of this instrument is on the educational goals, parenting styles and the role of both parents. The educational objectives can be differentiated between conformity and autonomy. Educational styles are asked by answering 18 items, which can be divided into six scales: Emotional warmth, inconsistent education, monitoring, negative communication, psychological control, strict control. The items were taken from the pairfam study, as were the 10 items for recording the role of parents. The parental role can be divided into three scales (autonomy, hostile attributes, willingness to make sacrifices).

\sphinxstylestrong{Availability:} Since 2012

\sphinxstylestrong{Respondent:} Partents in household (child age 7-8)

\sphinxstylestrong{Content:}
\begin{itemize}
\item {} 
Expectations for school achievments

\item {} 
Expectations of parental educational goals

\item {} 
Upbringing

\item {} 
Parental role

\item {} 
Childcare

\end{itemize}


\subsubsection{Mother-Child Questionnaire E (Age 9-10)}
\label{\detokenize{Contents of SOEPcore/index:mother-child-questionnaire-e-age-9-10}}\label{\detokenize{Contents of SOEPcore/index:mother-child-questionnaire-e}}
In addition to the items on health and the care situation recorded in almost all age groups, 9-10-year-old children are asked for more detailed information on the school situation. Here, too, the idealistic and realistic educational aspirations of the mothers for their child are recorded, but also the last grades of the three main subjects, as well as the child’s homework supervision and school motivation. Since friends and leisure activities are gaining in importance in this age group, questions are also asked on these topics. Whether and how much pocket money the child receives will be asked for the first time in this age group.

\sphinxstylestrong{Availability:} Since 2012

\sphinxstylestrong{Respondent:} Mother in household (child age 9-10)

\sphinxstylestrong{Content:}
\begin{itemize}
\item {} 
Expectations (school achievments, parental educational goals)

\item {} 
Education

\item {} 
parental commitment

\item {} 
Leisure activities for children

\item {} 
Family environment

\item {} 
Social behavior child

\item {} 
Personality Child

\item {} 
Health Child

\item {} 
Supervision

\item {} 
Pocket money

\end{itemize}


\subsection{Youth Instruments}
\label{\detokenize{Contents of SOEPcore/index:youth-instruments}}

\begin{savenotes}\sphinxattablestart
\centering
\begin{tabulary}{\linewidth}[t]{|T|T|T|T|}
\hline
\sphinxstyletheadfamily 
Topic
&\sphinxstyletheadfamily 
Pupils Questionnaire (Age 11-12)
&\sphinxstyletheadfamily 
Early Youth Questionnaire (Age 13-14)
&\sphinxstyletheadfamily 
Youth Questionnaire (Age16-17)
\\
\hline
Child’s State of Health
&&
x
&
x
\\
\hline
Height \& Weight
&
x
&
x
&
x
\\
\hline
Life Satisfaction
&
x
&
x
&
x
\\
\hline
Strength and Difficulties Questionnaire
&
x
&
x
&\\
\hline
Linguistic Usage
&
x
&
x
&\\
\hline
Big 5 Personality Traits
&
x
&
x
&
x
\\
\hline
Willingness to take Risks
&
x
&
x
&
x
\\
\hline
Locus of Control
&&
x
&
x
\\
\hline
Trust
&&&
x
\\
\hline
Time Preference
&&&
x
\\
\hline
School Attendance \& Homework
&
x
&
x
&
x
\\
\hline
Parents’ Interest in School Performance
&
x
&&
x
\\
\hline
Educational Aspirations
&
x
&
x
&
x
\\
\hline
Cultural Capital
&
x
&&\\
\hline
Relationship between Family Members
&
x
&
x
&
x
\\
\hline
Parentingl Behaviour
&&&
x
\\
\hline
Importance of personal environment
&&
x
&
x
\\
\hline
Hobbies
&
x
&
x
&
x
\\
\hline
Friends
&
x
&
x
&
x
\\
\hline
Pocket Money
&
x
&
x
&
x
\\
\hline
Saving
&&
x
&
x
\\
\hline
Political Interest
&&
x
&
x
\\
\hline
Housing Situation
&&&
x
\\
\hline
Jobs and Money
&&&
x
\\
\hline
Education and Career Plans
&&&
x
\\
\hline
Future
&&&
x
\\
\hline
Childhood and Parental Home
&&&
x
\\
\hline
Attitudes and Opinions
&&&
x
\\
\hline
\end{tabulary}
\par
\sphinxattableend\end{savenotes}


\subsubsection{Pupils Questionnaire}
\label{\detokenize{Contents of SOEPcore/index:pupils-questionnaire}}\label{\detokenize{Contents of SOEPcore/index:id25}}
In the year in which the children turn twelve, they answer questions about their situation for the first time. Here the focus is once again on the school situation: the start and end of school are asked differentiated according to the days of the week, the type of school attended, the number of pupils in the class and how many of them do not come from Germany, whether one feels discriminated against by the teacher and the last grades in math, German and English. It also determines how much time the student spends on homework, where he or she does the homework and who helps him or her with the homework and learning. The children are asked about their idealistic and realistic graduation aspiration. Since friends play an important role as caregivers at this age, they and various family members are asked what role they play in the support and how often there are disputes. Also asked about the number of close friendships and how often the parents interfere in the choice of friends. The educational aspirations of the three best friends and a maximum of three older siblings (if any) are asked. The cultural capital and learning environment of the pupils are assessed on the basis of various questions (e.g. availability of literature, instruments, art at home; a desk and a room for oneself). Furthermore, the type and frequency of leisure activities is again asked. The student answers whether and how much pocket money he or she receives and for the first time gives information about his or her own personality, willingness to take risks and life satisfaction. The use of the language in the family (only German or other languages) and with whom the meals are usually taken is also asked.

\sphinxstylestrong{Availability:} Since 2014

\sphinxstylestrong{Respondent:} 11-12-year-olds in the household

\sphinxstylestrong{Content:}
\begin{itemize}
\item {} 
Attitude

\item {} 
Personality

\item {} 
School (timetable, school-leaving qualification, Engagement)

\item {} 
Recreational activities

\item {} 
Social and family surroundings

\item {} 
Life circumstances

\end{itemize}


\subsubsection{Early Youth Questionnaire}
\label{\detokenize{Contents of SOEPcore/index:early-youth-questionnaire}}\label{\detokenize{Contents of SOEPcore/index:id26}}
The questionnaire for early youth is largely similar to the questionnaire for pupils in order to provide an appropriate data structure for questions relevant to developmental psychology. Fewer questions are asked about homework and the learning environment, but the question is asked whether the young person is involved in the school (e.g. as class spokesperson or in a working group) and social capital is acquired in this way. The current importance of various family members and friends is asked and, in addition to their own educational aspirations, also that of the three best friends. With regard to parents, the question is asked how long the young person is allowed to travel and stay up alone before school days and what things the 14-year-old has already done without parents (e.g. holidays, going to the doctor, exchanging something in the shop, drinking alcohol, smoking cigarettes). They ask again for the pocket money and also whether the young person has the opportunity to save money. Another new topic in this age group is the interest in politics and the inclination towards a certain party.

\sphinxstylestrong{Availability:} Since 2015

\sphinxstylestrong{Respondent:} 13-14-year-olds in the household

\sphinxstylestrong{Content:}
\begin{itemize}
\item {} 
self-perception

\item {} 
School (timetable, school-leaving qualification, Engagement)

\item {} 
Recreational activities

\item {} 
Friends

\item {} 
Siblings

\item {} 
Parents

\item {} 
Pocket money

\item {} 
Party preferences

\item {} 
Self-Perception

\item {} 
Willingness to take risks

\item {} 
Life satisfaction

\item {} 
Attitudes/Opinions

\item {} 
Future

\end{itemize}


\subsubsection{Youth Questionnaire}
\label{\detokenize{Contents of SOEPcore/index:youth-questionnaire}}\label{\detokenize{Contents of SOEPcore/index:id27}}
In the SOEP, people who turn 17 in the corresponding survey year are considered adult respondents. Like other first-time adult participants, you will thus receive a CV and a individual questionnaire. Since part of the adult biography (such as the employment biography or the relationship biography) does not yet apply to the young participants and other aspects such as the relationship with parents, leisure activities, the school situation or vocational training play a greater role, a youth questionnaire was developed in 2000 which replaces the CV questionnaire in this age group and has been used since then. The content of this questionnaire corresponds in many respects to the adult CV questionnaire, so that the data can be used to supplement the information  on parents (if they do not live in the household; data set: BIOPAREN). Health status, personality, willingness to take risks, locus of control, trust, time preference, political preferences, knowledge of German as well as information on the living situation, work situation, training, career plans and educational aspirations are also surveyed. For the period from 2000 to 2005, the youth questionnaire was surveyed in addition to the personal questionnaire. Since 2006, only the youth questionnaire has been recorded for 17-year-olds. Since then, it has been available in a version extended by a few indicators, and instead a test has been used to assess cognitive potential. Based on the I-S-T 2000R (Amthauer et al. 2001) the components analogies, number series and matrices with 20 subtasks each were selected for the SOEP (cf. Solga et al. 2005). With the help of these tasks, the fluid cognitive abilities are to be recorded. This is a strongly biologically determined dimension of cognitive abilities that is not influenced by education and is primarily based on reasoning, processing rate and working memory capacity (Cattell 1971; Horn 1982). Although the format of the test differs from the usual questionnaires in surveys, the willingness of young people to participate is high (Schupp and Hermann 2009).

\sphinxstylestrong{Availability:} Since 2000

\sphinxstylestrong{Respondent:} 16-17 year olds in the household

\sphinxstylestrong{Content:}
\begin{itemize}
\item {} 
Living

\item {} 
Relationships

\item {} 
Leisure and Sport

\item {} 
School (Graduation, Foreign languages, Engagement)

\item {} 
Pocket money

\item {} 
Education

\item {} 
Career Plans

\item {} 
Future

\item {} 
Origin

\item {} 
Childhood and Parental Home

\item {} 
Attitudes/Opinions

\item {} 
Self-Perception

\item {} 
Life satisfaction

\item {} 
Party preferences

\end{itemize}


\subsubsection{„Lust auf DJ“ (Denksport und Jugend) Questionnaire}
\label{\detokenize{Contents of SOEPcore/index:lust-auf-dj-denksport-und-jugend-questionnaire}}
In SOEP 2006, a separate questionnaire with cognitive tests for adolescents was used for the first time: “Lust auf DJ”. In this case, “DJ” stands for “Thinking Sports and Youth (Denksport und Jugend)”, but was also specifically selected to arouse the more common association of “Disc Jockey”. For all interviewees aged 16 - 17 years, the questionnaire “Lust auf DJ” was used and created.

\sphinxstylestrong{Availability:} Since 2007

\sphinxstylestrong{Respondent:} 16-17-year-olds in the household as a supplement to the youth questionnaire

\sphinxstylestrong{Content:}
\begin{itemize}
\item {} 
Assignment of word pairs

\item {} 
Complete incomplete equations

\item {} 
Assign figures

\end{itemize}


\subsection{Additional Instruments}
\label{\detokenize{Contents of SOEPcore/index:additional-instruments}}

\subsubsection{„Lücke“ Questionnaire - Re-questioning of the Individual Questionnaire (Summary)}
\label{\detokenize{Contents of SOEPcore/index:lucke-questionnaire-re-questioning-of-the-individual-questionnaire-summary}}
The “Lücke” (english:gap) questionnaire relates to temporary drop outs for which significant missing data from the previous year are collected.

\sphinxstylestrong{Availability:} Since 1987

\sphinxstylestrong{Respondent:} SOEP respondents who are temporarily unavailable.

\sphinxstylestrong{Content:}

All data refer to the previous survey year
\begin{itemize}
\item {} 
Status of the respondent

\item {} 
Occupational change

\item {} 
Receipt of social benefits within the last year

\item {} 
Completion of education

\item {} 
Type of educational attainment

\item {} 
Change of family status

\end{itemize}


\subsubsection{Deceased Persons Questionnaire}
\label{\detokenize{Contents of SOEPcore/index:deceased-persons-questionnaire}}\label{\detokenize{Contents of SOEPcore/index:id28}}
For the first time in the main wave of 2009, information should be collected on former SOEP participants who have died since the survey in 2008 or until the time of the survey in 2009. Through the questionnaire “The deceased person”, the SOEP curriculum vitae principle is thus consistently “completed”. The primary aim of the chosen concept is to obtain as much information as possible about the death circumstances of former SOEP participants. However, it also generates information about people who have never participated in the SOEP survey. The information collected in this way about otherwise “unknown” persons, however, can also be used for various analysis purposes on causes of death and the context of death can also be used in the socio-scientific analysis.

\sphinxstylestrong{Availability:} Since 2009

\sphinxstylestrong{Respondent:} SOEP respondents who lost a loved one.

\sphinxstylestrong{Content:}
\begin{itemize}
\item {} 
Relationship to the deceased

\item {} 
Deceased part of the survey?

\item {} 
Domestic environment of the deceased person

\item {} 
Cause and place of death

\item {} 
Legacies

\item {} 
Health condition of the deceased

\item {} 
Life satisfaction of the deceased

\item {} 
Influence of loss on one’s own life

\end{itemize}


\subsubsection{Gripping Strength Test}
\label{\detokenize{Contents of SOEPcore/index:gripping-strength-test}}
\sphinxstylestrong{Availability:} Since 2008

\sphinxstylestrong{Respondent:} Persons over 17 years in the household

\sphinxstylestrong{Content:}

This test measures the strength a person can exert when gripping. This can be important for assessing the physical condition.


\subsection{IAB-SOEP-Migrationsstichprobe}
\label{\detokenize{Contents of SOEPcore/index:iab-soep-migrationsstichprobe}}

\subsubsection{Personal Biography Questionnaire (New Respondents)}
\label{\detokenize{Contents of SOEPcore/index:personal-biography-questionnaire-new-respondents}}
\sphinxstylestrong{Availability:} Since 2014

\sphinxstylestrong{Respondent:} Persons with a migrant background aged 18 and over in the household

\sphinxstylestrong{Content:}
\begin{itemize}
\item {} 
Citizenship

\item {} 
Origin

\item {} 
Knowledge/Skills before entering

\item {} 
Migration background

\item {} 
Migration biography (the way to Germany)

\item {} 
Current living situation

\item {} 
Childhood and Parental Home

\item {} 
Life course since the age of 15

\item {} 
Education/Degrees

\item {} 
Family Situation

\item {} 
Partnership situation before immigration

\item {} 
Employment (current and past)

\item {} 
Occupational Change

\item {} 
Current income

\item {} 
Education/further training

\item {} 
Earnings

\item {} 
Well-being

\item {} 
Attitudes/Opinions

\end{itemize}


\subsubsection{Individual Questionnaire (Reinterviewed)}
\label{\detokenize{Contents of SOEPcore/index:individual-questionnaire-reinterviewed}}
\sphinxstylestrong{Availability:} Since 2014

\sphinxstylestrong{Respondent:} Persons with a migrant background aged 18 and over in the household

\sphinxstylestrong{Content:}

Like {\hyperref[\detokenize{Contents of SOEPcore/index:individual-questionnaire}]{\sphinxcrossref{\DUrole{std,std-ref}{Individual Questionnaire}}}} + the following migration-specific topics:
\begin{itemize}
\item {} 
Training/further training at home and abroad

\item {} 
Discrimination/Pursuit/War

\item {} 
Employment before moving to Germany

\item {} 
Amount of income in local currency

\item {} 
Religious community

\item {} 
Immigration parents and/or grandparents + place

\end{itemize}


\subsubsection{Youth Questionnaire}
\label{\detokenize{Contents of SOEPcore/index:id29}}
\sphinxstylestrong{Availability:} Since 2014

\sphinxstylestrong{Respondent:} 16-17 year olds in household with a migration background

\sphinxstylestrong{Content:}

Like {\hyperref[\detokenize{Contents of SOEPcore/index:biography-questionnaire}]{\sphinxcrossref{\DUrole{std,std-ref}{Biography Questionnaire}}}}  + the following migration-specific topics:
\begin{itemize}
\item {} 
Circle of friends

\item {} 
Degree at home or abroad

\item {} 
German lessons as a foreign language

\item {} 
Training at home or abroad

\item {} 
Year of the parents’ immigration

\item {} 
Acquired degree of parents at home or abroad

\item {} 
Religious community

\end{itemize}


\subsubsection{Household Questionnaire}
\label{\detokenize{Contents of SOEPcore/index:id30}}
\sphinxstylestrong{Availability:} Since 2014

\sphinxstylestrong{Respondent:} Head of household

\sphinxstylestrong{Content:}

Like {\hyperref[\detokenize{Contents of SOEPcore/index:household-questionnaire}]{\sphinxcrossref{\DUrole{std,std-ref}{Household Questionnaire}}}} + the following migration-specific topics:
\begin{itemize}
\item {} 
Valuables in Germany or abroad

\end{itemize}


\subsection{IAB-BAMF-SOEP-Befragung von Geflüchteten}
\label{\detokenize{Contents of SOEPcore/index:iab-bamf-soep-befragung-von-gefluchteten}}

\subsubsection{Personal Biography Questionnaire (New Respondents)}
\label{\detokenize{Contents of SOEPcore/index:id31}}
In 2017 (second wave of surveys), the survey instruments used and the respective survey content will come closer to the SOEP standard. In addition to the personal and household questionnaire, age-specific children’s instruments are used. As a rule, mothers of children of specific birth cohorts living in households (2016/2017, 2014, 2011, 2009, 2007) are asked about their educational participation in Germany, as well as information which includes the SOEP standard of the mother-child survey instruments and refugee specific additions, such as educational pathways before fleeing to Germany, language acquisition and mental illness.  In addition, with the consent of the parents, specific birth cohorts (2005, 2003 and 2000) of the growing up children/young people in the participating fugitive households themselves are interviewed. The age-specific SOEP standard instruments (students, early youth and youth) also serve as a model here. In addition to specific extensions for the fugitives, the adolescents undergo a test of basic cognitive skills developed by the Institute for Quality Development in Education (IQB). The selection of questions in the personal questionnaire, which is aimed at all adult refugees, should make it possible to trace the course of integration in many areas, also in comparison to other population groups. Thus, topics specific to refugees in the first wave of the survey are updated, such as the current status of the asylum procedure or language course participation. Classic SOEP topics, such as questions about current employment, will be given more scope. Flight-specific innovations that have become established in recent immigrant samples (IAB-SOEP Migration Surveys M1 and M2 from 2013 and 2015), such as the recording of educational qualifications acquired abroad with the help of the CAMCES tool and the question module on the recognition of qualifications acquired abroad, will also be used in the personal questionnaire in 2017.

\sphinxstylestrong{Availability:} Since 2017

\sphinxstylestrong{Respondent:} Refugees over 18 years of age

\sphinxstylestrong{Content:}
\begin{itemize}
\item {} 
Origin and route to Germany

\item {} 
Escape/Travel Reasons

\item {} 
Flight/Travel expenses

\item {} 
Escape/route

\item {} 
Accommodation in Germany

\item {} 
Reasons why Germany as a target country

\item {} 
Status of the asylum procedure

\item {} 
Residence permit

\item {} 
Satisfaction on various aspects

\item {} 
Intention to stay

\item {} 
Writing and language skills (mother tongue and foreign language)

\item {} 
Integration courses in Germany

\item {} 
Awareness, need and use of support and consulting services

\item {} 
Employment and income abroad and in Germany

\item {} 
Life situation before arrival

\item {} 
Schools, colleges and vocational training abroad and in Germany (+ recognition)

\item {} 
Curriculum vitae from age 15

\item {} 
Well-being

\item {} 
Perception of life

\item {} 
Attitude towards the future

\item {} 
Religious community

\item {} 
party preferences

\item {} 
Assessment of the current situation in the country of origin

\item {} 
Attitude and values

\item {} 
Personality

\item {} 
Social Networks

\item {} 
Family situation

\item {} 
Participation of children in education

\item {} 
Declaration of consent for register linking

\end{itemize}


\chapter{Target Population and Samples}
\label{\detokenize{Target Population and Samples/index:target-population-and-samples}}\label{\detokenize{Target Population and Samples/index::doc}}
The target population covered in the SOEP is defined as the residential population living in private households within the current boundaries of the Federal Republic of Germany (FRG). Because of changes in these boundaries (in 1990) and changes in the residential population due to migration, various adaptations have been applied to the initial sampling structure to keep the sample’s representativity. In addition, certain groups have been oversampled to increase the statistical power.
In 1984, the survey started with a sample covering the entire population in then West Germany (FRG), where the five biggest groups of foreigners (the so-called “guestworkers”) were oversampled.

The institutionalized population, in the true sense of the word (hospitals, nursing homes, military installations) is generally not representatively included in new samples. E.g. in 1984 only 57 institutionalized households are included. Later, however, persons from the initial households who have taken up residence temporarily or permanently in institutions of this kind are followed.

The SOEP was expanded to the territory of the German Democratic Republic in June 1990, only six months after the fall of the Berlin Wall. A further addition in 1994/95 was a sample of migrants who came to Germany after 1984, to take the influx of ethnic Germans from former Soviet countries into account. Two samples respresentative of the entire population in Germany were added in 1998 and 2000, to counter effects of panel attrition and to increase the overall sample size. In 2002, a high income sample was added, while in 2006 and 2009, additional refreshment samples were drawn.

To increase the overall sample size SOEP has started adding refreshment samples in 2011. While the first (in 2011) and second (2012) extensions are representative of the whole population, the third (2013) is supposed to explicitly cover migrants. For the fourth extension in 2014, the related study “Families in Germany”, covering mainly families, will be integrated into the SOEP.

The different samples in the SOEP are identified by letters: sample “A” refers to the German sample drawn in 1984, “C” to the East Germans from 1990, and so on. Even though these samples are kept separate, the respondents received identical questionnaires for the most part and distinctions by sample are usually not be necessary in an analysis.
However, one of the ideas of SOEP is, that the users have full information available about survey methodological issues and survey design. Which means in this case that you can of course identify the corresponding sample for each observation. In the following section, we present details on each of the samples, which - unless stated otherwise - are multi-stage random samples with regional clusters. The respondent’s households are selected by random-walk routines.

For an extensive discussion on sampling (and weighting): \sphinxhref{http://www.diw.de/de/diw\_02.c.299052.de/surveymethoden.html}{Survey methods}.


\section{The SOEP Samples in Detail}
\label{\detokenize{Target Population and Samples/index:the-soep-samples-in-detail}}
\sphinxstylestrong{Sample A} “Residents in the Federal Republic of Germany” covers persons in private households with a household head, who does not belong to one of the main foreigner groups of “guestworkers” (i.e. Turkish, Greek, Yugoslavian, Spanish or Italian households). Because only a few foreigners are in Sample A it is often called the “West German Sample” of the SOEP. In 1984 it covered 4,528 households with a sampling probability of about 0.0002.

\sphinxstylestrong{Sample B} “Foreigners in the Federal Republic of Germany” adds persons in private households with a Turkish, Greek, Yugoslavian, Spanish or Italian household head, which in 1984 constituted the main groups of foreigners in the FRG. Compared to Sample A the population of Sample B is oversampled with a sampling probability of about 0.002. The first wave included 1,393 households in Sample B.

\sphinxstylestrong{Sample C} “German Residents in the German Democratic Republic (GDR)” consists of persons in private households where the household head was a citizen of the German Democratic Republic (GDR). This meant that approximately 1.7\% of the residential population in the GDR in June 1990 was excluded from the sample as foreigners (who were mostly institutionalized). All in all, 2,179 households represent the starting size of this sample with a sampling probability of about 0.0005.

\sphinxstylestrong{Sample D} “Immigrants” started in 1994/95 with two different samples. In 1994, the first sample D1 had 236 households and in 1995, the second sample D2 had 295 households, leading to a total of 531 households (D1 and D2) in 1995. This sample consisted of households in which at least one household member had moved from abroad to West Germany after 1984. The sampling probability is about 0.0002.

\sphinxstylestrong{Sample E} “Refreshment” was added in 1998, selected from the entire population of private households in Germany. The households were chosen independently from the ongoing panel and its subsamples A through D, with the targets of increasing the number of observations of the general population and preserving its representativity. The selection scheme used for sample E essentially resembles the one used in subsample A. The number of households in the first wave of subsample E was \$1,060\$, with a sampling probability of about 0.00005. With the data distribution of 2012, parts of subsample E have been extracted into the SOEP Innovation Sample. It is also the first sample in which the Computer Assisted Personal Interview (CAPI) was implemented. Interviews in Samples A-D at this time were completely conducted using Paperand-Pencil-lnterviews (PAPI). To study mode effects, households of sample E were randomly allocated to CAPI and PAPI mode.

\sphinxstylestrong{Sample F} “Refreshment” was selected independently from all other subsamples from the population of private households in 2000. The selection scheme was slightly altered compared to the previous addition in Sample E: while the ’German’ households (all adults greater or equal 16 in the household have German nationality) were selected with a sampling probability of \$0.00028\$, the ’non-German’ households (at least one adult does not have German nationality) where oversampled with a probability of 0.0005. Overall, the number of added households in subsample F’s first wave amounts to 6,043.

\sphinxstylestrong{Sample G} “High Income” entered the SOEP in 2002 independently from all other subsamples. The original selection scheme required that the responding households had a monthly income of at least DM 7,500 (EUR 3,835), which - due to the lack of an adequate sampling frame - were identified using a screening procedure. This sample of overall 1,224 households increased the potential for analyses in the high income areas, which previously were difficult to conduct because of low case numbers. The derived sampling probability is about 0.0014. Starting with Wave 2 in 2003, the selection scheme for this subsample was changed such that only households with a net monthly income of at least EUR 4,500 were followed.

\sphinxstylestrong{Sample H} “Refreshment” started in 2006 as a random sample, again independently of all previous subsamples, covering all residential households in Germany. The addition of 1,506 households was drawn with a sampling probability of 0.0001.

\sphinxstylestrong{Sample I} “Incentive Sample” started in 2009, where in the first wave, a new incentive scheme was tested to increase participation rates (see also {[}sec:PanelCare{]}. The sampling was independent of all other SOEP-samples, adding a total number of 1,531 households to the SOEP. Their sampling probability was 0.00013. This sample remained in the main data distribution for its first two waves (i.e. 2010 and 2011, or waves Z and BA). With the data distribution of 2012, subsample I has been extracted into the SOEP Innovation Sample.

\sphinxstylestrong{Sample J} “Refreshment Sample” started in 2011 as a random sample that was drawn independently of all previous subsamples, covering the residential households in Germany. The addition of 3,136 households was drawn with a sampling probability of 0.0002.

\sphinxstylestrong{Sample K} “Refreshment Sample” started in 2012 as a random sample, drawn independently of all previous subsamples, covering the residential households in Germany. The addition of 1,526 households was drawn with a sampling probability of 0.0001.

\sphinxstylestrong{Sample L1} “Cohort Sample” covers private households in Germany, in which at least one household member is a child that was born between January 2007 and March 2010. Again migrants identified by an “onomastic procedure” are oversampled. Sample L1 (as well as L2 and L3) was part of the SOEP-related study “Familien in Deutschland” (FiD), which was later integrated into the SOEP in 2014. As part of an evaluation project of the Federal Ministry for Family Affairs, Senior Citizens, Women and Youth (BMFSFJ) and the Federal Ministry of Finance (BMF) the study focused on public benefits in Germany for married people and families. Therefore, the survey instruments of waves BA to BD differ in some parts from those of the other samples.

\sphinxstylestrong{Sample L2} “Family Types I” covers private households in Germany that meet at least one of the following criteria regarding their household composition: single parents, low income families and large families with three or more children. Similar to Sample G we face the problem that the eligible sub-population is relatively small and an adequate sampling frame is lacking. So again, a preceding telephone screening procedure identifies eligible households.

\sphinxstylestrong{Sample L3} “Family Types II” covers private households in Germany that meet at least one of the following criteria regarding their household composition: single parents or large families with three or more children. It is conducted analogical to Sample L2 in order to increase the number of cases in these sub-populations.

\sphinxstylestrong{Sample M1} “Migration Sample” In 2013 a new migration sample was added with around 2,700 households drawn by using register information of the German Federal Employment Agency.

\sphinxstylestrong{Sample M2} “Migration Sample” in 2015 another migration sample was added with around 1,100 households drawn by using register information of the German Federal Employment Agency.

\sphinxstylestrong{Sample M3} “Refugee Sample” in 2016 a new refugee sample was drawn for the IAB-BAMF-SOEP Refugee Survey in which roughly 1,769 households of displaced persons are repeatedly interviewed. Respondents aged 18 and older who entered Germany between January 2013 and December 2016 and who filed an asylum application (regardless of their current legal status) were interviewed as well as the members of their households.

\sphinxstylestrong{Sample M4} “Refugee Family Sample” The 2016 “IAB-BAMF-SOEP Refugee Survey” (Samples M3 and M4) is a joint project of the Institute for Employment Research (IAB), the Research Centre of the Federal Office for Migration and Refugees (BAMF-FZ) as well as the Socio-economic Panel (SOEP). The target population of the samples consists of 1,769 households with individuals who arrived in Germany between January 2013 and January 2016 and applied for asylum or were hosted as part of specific programs of the federal states (irrespective of their asylum procedure and their current legal status). The first part of the sample (M3) was financed with funds from the research budget of the Federal Employment Agency (BA) allocated to the IAB. Sample M4 was funded by the Federal Ministry of Education and Research (BMBF) and has a focus on refugee families.

\sphinxstylestrong{Sample M5} “Refugee Sample” M5 is the acronym for the third top-up sample of refugee households. The population of M5 covers adult refugees who have applied for asylum in Germany since January 1, 2013, and are currently living in Germany. The first wave of M5 was conducted in 2017. M5 added another 1,519 households of refugees who have migrated to Germany since 2013 to the SOEP framework.

\sphinxstylestrong{Sample N} “Refreshment Sample (PIAAC-L)” Sample N integrated 2,314 households of former participants of the Programme for the International Assessment of Adult Competencies (PIAAC and PIAAC-L) in 2017. This is the most recent addition to the SOEPCore samples. Fieldwork in sample N was conducted between Mid-March and Mid-August and thus slightly later than the majority of samples A\textendash{}L1.

More information about “Sample Sizes” and “Panel Attrition” can be found \sphinxhref{http://www.diw.de/documents/publikationen/73/diw\_01.c.579464.de/diw\_ssp0480.pdf}{here}

\begin{figure}[H]
\centering

\noindent\sphinxincludegraphics{{sample_overview}.PNG}
\end{figure}


\section{Eligibility and Follow-up}
\label{\detokenize{Target Population and Samples/index:eligibility-and-follow-up}}
As mentioned, the SOEP’s goal is to be representative of the residential population of Germany. All household members 16 and older are eligible for a personal interview, starting with the youth questionnaire at that age, followed by “regular” person questionnaires thereafter. As years go by, the children of the first wave reach age-eligibility and become panel members. If they move out and form their own families, they and their new families are still part of the survey. “New” persons become part of the SOEP population due to birth or residential mobility. In case a person enters a SOEP household after the initial wave, this person is asked to fill out the regular person questionnaire if age-eligible, or will be asked to participate once old enough. Thus in the absence of panel attrition the SOEP would be a self-sustaining survey.

The concept of how to follow the respondents and sample members over time is important for the representativeness of the study. The basic principle for follow-up in the SOEP is that all persons participating in a wave of any subsample are to be surveyed in the following years as long as they stay within the boundaries of Germany. This rule also extends to respondents who entered a SOEP-household after the first wave due to residential mobility or birth. If there is a “split-off”, i.e. people move out of the household they were last interviewed in, the members of the new household receive a new household identifier. The table conceptualizes how new sample members and households are realized in the SOEP. The figure shows that as a result of the follow-up concept, up to , several thousand “new” households became part of the SOEP population.

Persons or households who could not be interviewed in a given year are termed “temporary drop-outs”. These are followed until there are two consecutive waves of missing interviews for all household members or a final refusal of the complete household. In the case of a cooperation after a temporary drop-out, the respondent is asked to fill out an additional short questionnaire on central information on employment and demographics during the year of absence.


\begin{savenotes}\sphinxattablestart
\centering
\begin{tabulary}{\linewidth}[t]{|T|T|T|}
\hline
\sphinxstyletheadfamily &\sphinxstyletheadfamily 
Existing Households
&\sphinxstyletheadfamily 
New Households
\\
\hline
Existing Persons
&
classic case: without change of adress entire household moves
&
Move-out
\\
\hline
New Persons
&
Birth Move-In
&
Move-In or birth into move-out household
\\
\hline
\end{tabulary}
\par
\sphinxattableend\end{savenotes}

\sphinxstylestrong{Changes to the Sample:} Old and new household in the SOEP

\begin{figure}[H]
\centering

\noindent\sphinxincludegraphics{{old-new-hh}.PNG}
\end{figure}

\sphinxcode{\sphinxupquote{Download R Code to create figure}}


\section{Development of Sample Sizes}
\label{\detokenize{Target Population and Samples/index:development-of-sample-sizes}}
Individuals who refuse participation or are not available for an interview are kept in the so-called “gross” sample of the study as long as they continue to live in households with at least one participating person. Once the entire household declines to respond in two consecutive waves of data collection, all individuals from the household are removed from the SOEP. The table shows the starting sample sizes of samples A through M4, the years when the samples were first collected, as well as the percentage of those persons who were eligible for an interview but declined participation (“partial unit non-response”, PUNR) in the first wave. The figure illustrates the development of the number of successful person interviews since 1984. The reduction in the population size for all individual samples is mainly the result of person-level drop-outs, refusals, moving abroad, etc. However, due to new persons moving into already existing households, and children reaching the minimum respondent’s age of 16, and thereby increasing the sample size, this negative development is offset somewhat.


\subsection{Starting Sample Size of the SOEP Samples}
\label{\detokenize{Target Population and Samples/index:starting-sample-size-of-the-soep-samples}}

\begin{savenotes}\sphinxattablestart
\centering
\begin{tabulary}{\linewidth}[t]{|T|T|T|T|T|T|T|}
\hline
\sphinxstyletheadfamily 
Sample
&\sphinxstyletheadfamily 
Year
&\sphinxstyletheadfamily 
Households (net)
&\sphinxstyletheadfamily 
Persons(gross)
&\sphinxstyletheadfamily 
Respondents (net)
&\sphinxstyletheadfamily 
Partial Unit Non-Response (percent)
&\sphinxstyletheadfamily 
Children (gross)
\\
\hline
A
&
1984
&
4528
&
11422
&
9076
&
0.6
&
2290
\\
\hline
B
&
1984
&
1393
&
4830
&
3169
&
0.7
&
1636
\\
\hline
C
&
1990
&
2179
&
6131
&
4453
&
1.9
&
1591
\\
\hline
D1
&
1994
&
236
&
733
&
471
&
2.9
&
248
\\
\hline
D1/D2
&
1995
&
541
&
1668
&
1078
&
6.1
&
517
\\
\hline
E
&
1998
&
1057
&
2446
&
1910
&
3.5
&
466
\\
\hline
F
&
2000
&
6043
&
14510
&
10880
&
5.5
&
2991
\\
\hline
G
&
2002
&
1224
&
3538
&
2671
&
6.1
&
693
\\
\hline
H
&
2006
&
1506
&
3407
&
2616
&
6.0
&
623
\\
\hline
I
&
2009
&
1495
&
3428
&
2432
&
13.4
&
620
\\
\hline
J
&
2011
&
3136
&
6873
&
5161
&
9.9
&
1147
\\
\hline
K
&
2012
&
1526
&
3286
&
2473
&
9.2
&
563
\\
\hline
L1
&
2010
&
2074
&
7939
&
3770
&
6.7
&
3900
\\
\hline
L2
&
2010
&
2500
&
9063
&
4227
&
5.1
&
4611
\\
\hline
L3
&
2011
&
924
&
3645
&
1487
&
4.2
&
2092
\\
\hline
M1
&
2013
&
2723
&
8522
&
4964
&
17.8
&
2481
\\
\hline
M2
&
2015
&
1096
&
3048
&
1711
&
19.3
&
927
\\
\hline
M3
&
2016
&
1775
&
4823
&
2351
&
22.0
&
1808
\\
\hline
M4
&
2016
&
1779
&
7297
&
2465
&
27.1
&
3915
\\
\hline
\end{tabulary}
\par
\sphinxattableend\end{savenotes}


\subsection{Cross-Sectional Development of Sample Size (Respondents) Cross-Sectional Development of}
\label{\detokenize{Target Population and Samples/index:cross-sectional-development-of-sample-size-respondents-cross-sectional-development-of}}
\begin{figure}[H]
\centering

\noindent\sphinxincludegraphics{{crossdevel}.PNG}
\end{figure}

\sphinxcode{\sphinxupquote{Download Stata Code to create figure}}

This cross-sectional view is insufficient when examining the longitudinal development of the sample, which is influenced by different demographic and field-work related factors. As already shown, demographic reasons for entering the panel are birth and residential mobility. Analogously, the demographic reasons for a panel exit are death and moving abroad. Fieldwork related reasons are different, in that they relate to the interaction between the interviewer and the responding household. Respondents are either not reached for an interview (non-contact) or they decline to participate for the current year. The figure illustrates the longitudinal development of first-wave respondents in 1984, as well as their children, of samples A and B.


\subsection{Longitudinal Development of the 1984 Population}
\label{\detokenize{Target Population and Samples/index:longitudinal-development-of-the-1984-population}}
\begin{figure}[H]
\centering

\noindent\sphinxincludegraphics{{where2}.png}
\end{figure}

\sphinxcode{\sphinxupquote{Download Stata Code to create figure}}


\chapter{Survey Design}
\label{\detokenize{Survey Design/index:survey-design}}\label{\detokenize{Survey Design/index::doc}}

\section{Survey Instruments}
\label{\detokenize{Survey Design/index:survey-instruments}}
The interview methodology of the SOEP is based on a set of pre-tested questionnaires for households and individuals. Interviewers try to obtain face-to-face interviews with all members aged 16 years and over of a given survey household. Thus, there are no proxy interviews for adult household members. Additionally, one person (the so called “head of household”) is asked to answer a household related questionnaire covering information on housing, housing costs, and different sources of income (e.g. social transfers like social assistance or housing allowances). This questionnaire also covers some questions on children in the household up to the age of 16, mainly concerning their attendance in day care, kindergarten and school.

The questions in the SOEP are in principle identical for all participants of the survey to ensure comparability across the participants within any given year (of course, there are differences across years. There are a few exceptions to this rule, which are due to different requirements in the target population. Up to 1996 the questionnaires for the foreigner’s sample (B) and immigrant sample (D) covered additional measures of integration or information on re-migration behavior. Between 1990 and 1992, i.e. during the first years of the German unification process, the questionnaire for the East German sample (C) also contained some additional specific variables. Since 1996, all questionnaires are uniform and completely integrated for all main SOEP samples. The related studies use SOEP related content, but also have specific questions, so the contents may differ to various degrees in every year.

Another type of questionnaires is implemented because first time respondents are not treated identically to those with a repeated interview, since some information does not have to be asked every year unless a change occurred. Additionally, each respondent is asked to fill out a biography questionnaire covering information on the life course up to the first SOEP interview (e.g. marital history, social background, and employment biography).

Additional information - not provided directly by the respondents - can be obtained from the so-called “address logs”, which are stored for every year in the \$PBRUTTO and \$HBRUTTO files. Every address log is filled in by the interviewer even in the case of non-response, thus providing very valuable information, e.g. for attrition analyses. For researchers interested in methodological issues these data also contain information on the field work process, e.g. the number of contacts, reason for eventual drop-outs, or the interview mode. For successfully contacted households, the address logs cover the size of the household, some regional information, survey status etc., while the individual data for all household members include the relation to the household head, survey status of the individual and some demographic information.


\section{Survey Concepts}
\label{\detokenize{Survey Design/index:survey-concepts}}
Measuring stability and detecting changes means to repeat (almost) identical measures over time. Furthermore, the SOEP-questions capture stability and change by varying with regard to the time dimension, asking about events in the past, the present, and the future. Conceptually, different measurements of time are used:
\begin{itemize}
\item {} 
Questions about a point in time (present) e.g. current employment status or current levels of satisfaction

\item {} 
Single retrospective questions on certain events in the past e.g. how often did you change your job during the last ten years?

\item {} 
Retrospective life event history since the age of 15 (in the past) e.g. employment or marital history

\item {} 
Monthly calendar information on income and labor market participation (in the past) e.g. employment status January through December last year

\item {} 
Questions concerning a period of time (in the past) e.g. demographic changes since the last interview like marriage or death of spouse

\item {} 
Questions concerning future prospects (future) e.g. satisfaction with life five years from now, or job expectations

\end{itemize}


\section{Survey Modes}
\label{\detokenize{Survey Design/index:survey-modes}}
The SOEP uses several different modes to collect the data. Originally, the respondent’s answers were recorded by an interviewer who filled in a paper questionnaire, the so called pen-and-paper interview or PAPI. The personal contact between interviewer and respondent is important for the success of the survey; however, before losing a respondent due to a scheduling conflict between interviewer and respondent, the SOEP allows mailing in the questionnaire starting from the second wave of subsamples A-I. This concept does not resemble the concept of a regular mail survey, because the interviewer still keeps the personal contact with the household and schedules appointments with its respondents if possible. Starting with subsample J, only the computer assisted mode (CAPI) is allowed, and thus mailing in the questionnaires is no longer possible.

While the interviewer is in the household she/he directly conducts an interview with any household member, but can also hand out a questionnaire to other household members, who fill it in with or without her/his help (self-administered questionnaires, SAQ). This is much more time efficient for the interviewer, because household members can work in parallel on their questionnaires.

In 1998, interviews were conducted with computers for the first time, in computer-assisted personal interviews, or in CAPI mode. Compared to PAPI, CAPI is much more efficient in transferring the data into an electronic format, which was an important asset especially with the extensions of the panel starting in the year 2000. The CAPI mode was first conducted in parallel to the PAPI mode, meaning that interviewers and respondents were free to chose how they wanted to do the interview. This was important for the “older” sample members (respondents as well as interviewers), who were used to the PAPI concept. Only in the most recent samples (starting in subsample J), CAPI is the only mode. The figure depicts the development of modes up to 2011, showing that the CAPI mode has gained importance since its implementation.

Since the questionnaires have to be identical in both modes, the CAPI implementation is relatively simple compared to what would be technically feasible. For example, the SOEP basically does not use any form of dependent interviewing (i.e. referring to respondent data from previous waves), because this cannot be easily implemented in the PAPI-mode. Also, the filtering structure is very simple in the SOEP, because any respondent must be able to follow the interview path on her/his own on paper. Still, some technical features like the control of value ranges (e.g. month of birth, year of first marriage) or the randomization of scale items are implemented in the CAPI version of the questionnaire.

In the future, new modes will be introduced into the SOEP as they develop. The computer-assisted web interview (CAWI) is close to implementation, it will, however, not be used as a replacement of the current CAPI and PAPI modes, but rather as an extension the respondents may use similar to the mail-in or self-administered questionnaires. The core interview concept of the SOEP survey, the personal contact between respondent and interviewer, will not change.

\begin{figure}[H]
\centering

\noindent\sphinxincludegraphics{{mode}.png}
\end{figure}

\sphinxcode{\sphinxupquote{Download STATA Code to create figure}}.


\section{Panel Care}
\label{\detokenize{Survey Design/index:panel-care}}
To cope with panel attrition and to keep the longitudinal response rates at high levels, the SOEP has implemented so-called “panel care” efforts to maintain the personal contact between respondents and the survey. Panel care can be divided into incentives directly given to the respondent and other measures undertaken to keep the respondent in the study.

The study has honored the respondents with gifts and tokens of appreciation from the very beginning. For the most part, these gifts are small in-kind incentives like flowers, for which the interviewers have their own budget. In addition, the interviewers are asked to hand out a brochure with recent results from the study. Up to 2007, the respondents also received a lottery ticket as a thank you upon completion of the interview. The lottery collects money for social projects in Germany. Since 2008, the lottery ticket is included in the contact letter which is sent out about two weeks prior to the interview. It is thus given unconditionally, as long as the person has participated in the previous wave. After any successful interview, the respondent receives a thank you letter from the field work organization, which also includes a stamp for a regular letter.

In 2009, different incentive schemes were tested in the new subsample I to increase the first-wave response rates. The basic experiment included four randomized groups of households: (1) those with the default setup of the conditional lottery ticket; (2) those with a “low” cash incentive involving 5 Euros per household and 5 Euros per adult respondent; (3) those with a “high” cash incentive involving 5 Euros per household and 10 Euros per adult respondent; and (4) those with a choice between a “low” cash incentive and a lottery ticket. The results showed slightly higher response rates in the cash groups, although the extra money in group (3) did not pay of. Additional work is done by the field work agency: Addresses are kept up to date throughout the year in order to be informed about residential mobility. This is achieved for example by sending out a brochure containing some results based on previously collected data, or seasonal greeting cards.

In addition, the face-to-face interview ensures a personal relationship, which increase the likelihood to stay in the survey. Thus, keeping the same interviewer over time is one important goal - some of the respondents have indeed had the same interviewer since the beginning in 1984.


\chapter{Principles of Data Structure}
\label{\detokenize{Principles of Data Structure/index:principles-of-data-structure}}\label{\detokenize{Principles of Data Structure/index::doc}}

\section{Panel Data Analysis}
\label{\detokenize{Principles of Data Structure/index:panel-data-analysis}}\label{\detokenize{Principles of Data Structure/index:analysis}}
The data structure for panel data consists of three dimensions. At first, the respective examination units (n) and a matrix of dependent and independent variables (y,x) are completely analogous to a cross-sectional design. Another level is the dimension of time (t), whereby a distinction is made between two data formats for panel data structures - “wide” or “long” (with wide format the variable matrix is indexed with the dimension of time and with long format the respective examination units). Regardless of the selected data format, when using panel data with several survey waves, the data matrices are often not completely provided with information due to the panel mortality of individual survey units or because data from new panel members are only collected at a later point in time. In both cases, the term “unbalanced panel data” is used. In contrast, the classical panel data structure, on the other hand, is “balanced”, i.e. as many observations of dependent and independent variables are available for all study units as there are waves of data collection.
The data of social science panel data often show a data structure, which is characterized by many investigation units (large n) as well as, in relation to it, few waves and therefore measuring time (small t).
When data from a panel study are available, even descriptive forms of data analysis are often of particular interest, since the identification of changes in a variable over time and the corresponding separation of interindividual and intraindividual changes can represent important social facts, particularly in the case of generalizable samples. It is of social scientific interest whether a constant 15 \% proportion of people whose income is below the poverty risk level is repeatedly found in the same person over time, or whether there was a even balance of increases and decreases in poverty risks and only half of the population was permanently exposed to the risk.
The choice of complex analysis methods for panel data depends first and foremost on the respective measurement level of the dependent and independent variables, but also on whether they are time-constant variables (such as gender or migration background) or time-invariant variables (for an overview see Andreß et al. 2013). The statistical analysis models of panel data range from structural equation models (Finkel 1995), various regression models (Giesselmann/Windzio 2012), event analysis (Blossfeld 2010), sequence data analysis (Brüderl/Scherer 2005), latent growth models (Schiedeck/Wolff 2010) to causal analyses using matching methods (Gangl 2010). A particular advantage of panel data is that the chronological sequence of changes can be modelled and calculated and the problem of unobserved heterogeneity, which is often encountered in the social sciences, can be significantly reduced, at least in comparison with cross-sectional data (Brüderl 2010).


\section{Data Structure of SOEP-Core}
\label{\detokenize{Principles of Data Structure/index:data-structure-of-soep-core}}
SOEP-Core contains a multitude of different datasets. To get an overview of the data, a somewhat simplified categorization helps: There are {\hyperref[\detokenize{Principles of Data Structure/index:tracking-data}]{\sphinxcrossref{Tracking Data}}} and {\hyperref[\detokenize{Principles of Data Structure/index:survey-data}]{\sphinxcrossref{Survey Data}}} files which describe the development of the sample, such that the user knows which person or household was part of the interviewed sample in any given year. Then there are {\hyperref[\detokenize{Principles of Data Structure/index:original-data}]{\sphinxcrossref{Original Data}}} files, which contain the data from each year’s questionnaires without any changes except for very basic consistency checks. To help the user with the data, there also are {\hyperref[\detokenize{Principles of Data Structure/index:generated-data}]{\sphinxcrossref{Generated Data}}}. These contain consistently coded variables across all waves with common names, such that the users can easily use this information when combining datasets across waves. The SOEP also provides various data on the respondent’s background, called biographical data. Biography data in general can conceptually be separated into biographical data which are unchanging (such as information on parent’s education, or data from the mother-child questionnaires) and data which may be updated through changes in a respondent’s life (such as new children in the birth biography, or a job change in the job history). Some of the changing data is stored as {\hyperref[\detokenize{Principles of Data Structure/index:spell-data}]{\sphinxcrossref{Spell Data}}}. For each spell there is a definition of the spell type, begin, end point and the censoring status, indicating if a given employment or income spell is censored (left and/or right) or uncensored. One of the biggest assets of the SOEP data is their longitudinal nature, i.e. repeated observations of the same unit (person or household) over time. That{}`s why we provide longitudinal data sets, such as pl or hl. Finally, there are some files which cannot be easily categorized - some are one-time datasets, some provide information about the interviewers, some about respondents outside of Germany.

There are two datasets which should be the building block of any analysis, as they allow to define longitudinal populations very easily: PPFADL and HPFADL. HPFADL includes all households which have been interviewed successfully at least once. Similarly, PPFADL contains all persons who have ever lived in a household that has participated in the SOEP, i.e. that has been captured in HPFADL, including non-respondents and children. Both data files contain one record per household or person, respectively, with wave-specific variables for each year’s survey status. In addition to some time-invariant information (like gender, year of birth, migrant status), these files contain all necessary identifiers to combine other files with PPFADL and HPFADL.

Although they provide essential information, PPFADL and HPFADL alone are of little use for actual analyses. The most often used sources for additional information in SOEP-Core are the cross-sectional data files provided in each survey year (or “wave”) or the data sets in the long-format.


\subsection{Cross-sectional data files (CS)}
\label{\detokenize{Principles of Data Structure/index:cross-sectional-data-files-cs}}\label{\detokenize{Principles of Data Structure/index:cross}}
\begin{figure}[H]
\centering

\noindent\sphinxincludegraphics{{cross_sectional}.PNG}
\end{figure}

Each wave is identified by letters of the alphabet: the first wave in 1984 is wave “A”, 1985 is wave “B”, and so on. To simplify the notation, the “\$” sign is used, when all waves of one group of datasets are referred to. For example, \$H refers to all household level datasets AH to now. For each year of SOEP data there are single data files for households (e.g. \$H) as well as for individual respondents (e.g. \$P) and children (e.g. \$KIND) based on interview information. These observations make up the “net” population, with each of these files containing as many records as interviews could be conducted. Additional data files with a limited number of variables based on the “address log” constitute the “gross” number of households and persons, i.e. all households and their members which were eligible for an interview in any given year.

\sphinxstylestrong{Data structure}

Cross sectional data is a type of data, which observes many subjects at the same point of time. Each person is assigned a row in the data set and is only included once in such a data set. By merging cross-sectional SOEP data across waves (e.g. „bfp“ and „bgp“), you receive a dataset in wide-format.


\subsection{Data Structure in wide-format (wide)}
\label{\detokenize{Principles of Data Structure/index:data-structure-in-wide-format-wide}}
The SOEP data is offered in different data structures. In wide format, a respondent’s repeated responses are displayed in a single row and each response in a separate column. Each column represents a variable. We provide 4 datasets in wide-format: ppfad, phrf, hpfad, hhrf


\begin{savenotes}\sphinxattablestart
\centering
\begin{tabulary}{\linewidth}[t]{|T|T|T|T|T|}
\hline
\sphinxstyletheadfamily 
row
&\sphinxstyletheadfamily 
ID
&\sphinxstyletheadfamily 
syear
&\sphinxstyletheadfamily 
sex
&\sphinxstyletheadfamily 
income
\\
\hline
1
&
1
&
2016
&
m
&
1500
\\
\hline
2
&
2
&
2016
&
m
&
1000
\\
\hline
3
&
6
&
2016
&
f
&
2000
\\
\hline
4
&
8
&
2016
&
m
&
5500
\\
\hline
\end{tabulary}
\par
\sphinxattableend\end{savenotes}


\subsection{Data Structure in long Format (long)}
\label{\detokenize{Principles of Data Structure/index:data-structure-in-long-format-long}}\label{\detokenize{Principles of Data Structure/index:datasets-long}}
The long format is a compressed and user-friendly data set structure for longitudinal section analysis. Here, each person has one line per survey year. This means that you do not have several data sets for the different waves, but a data set in which all survey waves are represented. A person can occur more than once in such a data set. In long format, one line describes a person-year combination.


\begin{savenotes}\sphinxattablestart
\centering
\begin{tabulary}{\linewidth}[t]{|T|T|T|T|T|}
\hline
\sphinxstyletheadfamily 
Row
&\sphinxstyletheadfamily 
ID
&\sphinxstyletheadfamily 
syear
&\sphinxstyletheadfamily 
sex
&\sphinxstyletheadfamily 
income
\\
\hline
1
&
1
&
2010
&
f
&
1500
\\
\hline
2
&
1
&
2011
&
f
&
1500
\\
\hline
3
&
1
&
2012
&
f
&
2000
\\
\hline
4
&
2
&
1999
&
m
&
5500
\\
\hline
5
&
2
&
2000
&
m
&
5500
\\
\hline
\end{tabulary}
\par
\sphinxattableend\end{savenotes}


\subsection{Data Structure in spell format (spell)}
\label{\detokenize{Principles of Data Structure/index:data-structure-in-spell-format-spell}}
In the strict sense of the word, spell data are about time periods with a defined start and end. When handling spell data it is necessary to take potential censoring into account. Censoring denotes that the beginning (left censored) or ending (right censored) of a spell is imprecise because of missing information or the beginning or ending of a spell is outside of the period of observation.  It is quite conceivable that a person has only one spell over a given period, such as a male who is full-time employed. For a ten year period, there may be just the one spell “full-time employed”. In panel data, the same person would have 10 observations, one per year. A person may have many spells over a time period, and even have overlapping spells, like working part-time and receiving a disability pension. Spell data is useful for looking at stays in a certain state, and transitions in and out of that state.


\begin{savenotes}\sphinxattablestart
\centering
\begin{tabulary}{\linewidth}[t]{|T|T|T|T|T|T|T|}
\hline
\sphinxstyletheadfamily 
Row
&\sphinxstyletheadfamily 
ID
&\sphinxstyletheadfamily 
spellnr
&\sphinxstyletheadfamily 
spelltype
&\sphinxstyletheadfamily 
begin
&\sphinxstyletheadfamily 
end
&\sphinxstyletheadfamily 
censored
\\
\hline
1
&
1
&
1
&
Retired
&
1983
&
2007
&
left and right censored
\\
\hline
2
&
1
&
2
&
Housewife/husband
&
1983
&
1984
&
left censored
\\
\hline
3
&
1
&
3
&
Housewife/husband
&
1994
&
1994
&
uncensored
\\
\hline
4
&
1
&
4
&
Housewife/husband
&
1998
&
1998
&
uncensored
\\
\hline
5
&
2
&
1
&
Full-Time Employment
&
1984
&
1984
&
left censored
\\
\hline
6
&
2
&
2
&
Full-Time Employment
&
1985
&
1985
&
uncensored
\\
\hline
\end{tabulary}
\par
\sphinxattableend\end{savenotes}

Here are some recommended literature suggestions:

\sphinxstylestrong{Working with spell data:}

\sphinxhref{https://www.diw.de/documents/publikationen/73/diw\_01.c.581580.de/diw\_ssp0492.pdf}{Working with spell data (pdf)}:

\sphinxhref{https://www.diw.de/documents/dokumentenarchiv/17/diw\_01.c.581431.de/do-files\_spell-data.zip}{Working with spell data (do-files)}:

\sphinxstylestrong{How to generate spell data from data in wide format: Based on the Migration Biographies of the IAB-SOEP Migration Sample:}

\sphinxhref{https://www.econstor.eu/handle/10419/122163}{Generating spell data}:


\section{Data Sets SOEP-Core}
\label{\detokenize{Principles of Data Structure/index:data-sets-soep-core}}\label{\detokenize{Principles of Data Structure/index:datasets}}
In the SOEP, each survey year is allocated to a data wave, which is abbreviated with the letters of the alphabet. The current data wave can contain several versions, which are displayed in SOEP with a “v” for version and the respective version number. The version number represents the survey years since the beginning of the survey. The SOEP has recently published the 34th version since the survey began in 1984. Within a data wave, updates may occur over time, such as v34.1. If updates have been carried out, users are informed about them via various information channels and asked to order the data again. After ordering the data, the data will be sent to you as a zip-file.

\begin{figure}[H]
\centering

\noindent\sphinxincludegraphics{{SOEP_1}.PNG}
\end{figure}

Within this zip file you will find various data sets and a “RAW” subdirectory.

\begin{figure}[H]
\centering

\noindent\sphinxincludegraphics{{SOEP}.PNG}
\end{figure}

The data sets above the “RAW” subdirectory are highly compressed and an easy to analyze version of the SOEP data.

\begin{figure}[H]
\centering

\noindent\sphinxincludegraphics{{SOEP_2}.PNG}
\end{figure}

The data in SOEP-Core are no longer only provided as wave-specific individual files but rather pooled across all available years (in “long” format). In some cases, variables are harmonized to ensure that they are defined consistently over time. For example, the income information provided up to 2001 is given in euros, and categories are modified over time when versions of the questionnaire have been changed. The longitudinal nature is one of the biggest assets of the SOEP. That{}`s why we provide longitudinal data sets, such as pl or hl. The advantage of such a data set is that longitudinal analyses can be carried out without great effort.

If you need more information about the long data structure visit the chapter {\hyperref[\detokenize{Principles of Data Structure/index:datasets-long}]{\sphinxcrossref{\DUrole{std,std-ref}{Data Structure in long Format (long)}}}}.

In the “RAW” directory you will find all wave-specific data sets that were used to generate the long data sets on the previously presented level.

\begin{figure}[H]
\centering

\noindent\sphinxincludegraphics{{SOEP_4}.PNG}
\end{figure}

\begin{figure}[H]
\centering

\noindent\sphinxincludegraphics{{SOEP_3}.PNG}
\end{figure}

Within this “RAW” directory, the data sets are stored on a wave-specific basis and are the generation basis for the majority of the long data sets described above. In addition to these wave-specific data sets, the “RAW” directory also contains additional data sets in cross-sectional format that have not yet been distributed in long format (\$school, \$school2, ev, exit, \$pkalost and pbr\_hhchch).

To understand the data set and variable names, visit the {\hyperref[\detokenize{Principles of Data Structure/index:label}]{\sphinxcrossref{\DUrole{std,std-ref}{Labeling SOEP-Core}}}} chapter.


\subsection{Overview Data Sets}
\label{\detokenize{Principles of Data Structure/index:overview-data-sets}}\label{\detokenize{Principles of Data Structure/index:overview}}
\sphinxstylestrong{Your data distribution file contains five different types of data sets:}


\begin{savenotes}\sphinxatlongtablestart\begin{longtable}{|l|l|l|l|l|}
\hline
\sphinxstyletheadfamily 
Tracking Data
&\sphinxstyletheadfamily 
Original Data
&\sphinxstyletheadfamily 
Survey Data
&\sphinxstyletheadfamily 
Generated Data
&\sphinxstyletheadfamily 
Spell Data
\\
\hline
\endfirsthead

\multicolumn{5}{c}%
{\makebox[0pt]{\sphinxtablecontinued{\tablename\ \thetable{} -- continued from previous page}}}\\
\hline
\sphinxstyletheadfamily 
Tracking Data
&\sphinxstyletheadfamily 
Original Data
&\sphinxstyletheadfamily 
Survey Data
&\sphinxstyletheadfamily 
Generated Data
&\sphinxstyletheadfamily 
Spell Data
\\
\hline
\endhead

\hline
\multicolumn{5}{r}{\makebox[0pt][r]{\sphinxtablecontinued{Continued on next page}}}\\
\endfoot

\endlastfoot

hpfad
&
abroad
&
csamp
&
bioage17
&
artkalen
\\
\hline
hpfadl
&
biol
&
design
&
bioagel
&
biocouplm
\\
\hline
\$hbrutto
&
ev
&
exit
&
biobirth
&
biocouply
\\
\hline
hbrutto
&
\$h
&
hhrf
&
bioedu
&
biomarsm
\\
\hline
\$pbrutto
&
hl
&
pbr\_hhch
&
bioimmig
&
biomarsy
\\
\hline
pbrutto
&
\$host
&
phrf
&
biojob
&
einkalen
\\
\hline
pbr\_exit
&
\$jugend
&&
bioparen
&
lifespell
\\
\hline
ppfad
&
jugendl
&&
bioresid
&
migspell
\\
\hline
ppfadl
&
lueckel
&&
biosib
&
pbiospe
\\
\hline&
\$p
&&
biosoc
&
refugspell
\\
\hline&
pl
&&
biotwin
&
sozkalen
\\
\hline&
\$pausl
&&
camces
&\\
\hline&
\$pluecke
&&
cogdj
&\\
\hline&
\$post
&&
cognit
&\\
\hline&
\$school
&&
gripstr
&\\
\hline&
\$school2
&&
hconsum
&\\
\hline&
\$vp
&&
health
&\\
\hline&
vpl
&&
\$hgen
&\\
\hline&&&
hgen
&\\
\hline&&&
hwealth
&\\
\hline&&&
interviewer
&\\
\hline&&&
kidl
&\\
\hline&&&
\$kind
&\\
\hline&&&
mihinc
&\\
\hline&&&
\$pequiv
&\\
\hline&&&
pequiv
&\\
\hline&&&
pflege
&\\
\hline&&&
\$pgen
&\\
\hline&&&
pgen
&\\
\hline&&&
\$pkal
&\\
\hline&&&
pkal
&\\
\hline&&&
\$pkalost
&\\
\hline&&&
pwealth
&\\
\hline&&&
timepref
&\\
\hline&&&
trust
&\\
\hline
\end{longtable}\sphinxatlongtableend\end{savenotes}


\subsection{Tracking Data}
\label{\detokenize{Principles of Data Structure/index:tracking-data}}\label{\detokenize{Principles of Data Structure/index:tracking}}
Tracking data are the basis for linking your research-relevant variables. In addition to various demographic information, tracking data also provide information on how the interview is conducted. These data sets should be understood by you as initial data. You can use the tracking data to merge your research-relevant variables via the person and household numbers.


\begin{savenotes}\sphinxattablestart
\centering
\begin{tabulary}{\linewidth}[t]{|T|T|T|T|T|}
\hline
\sphinxstyletheadfamily 
Dataset
&\sphinxstyletheadfamily 
Label
&\sphinxstyletheadfamily 
Format
&\sphinxstyletheadfamily 
Identifier (ID)
&\sphinxstyletheadfamily 
Special Identifier
\\
\hline
hpfad
&
Household Tracking File
&
wide
&
hhnrakt, \$hhnr
&\\
\hline
hpfadl
&
Household Tracking File
&
long
&
hid, syear, cid
&\\
\hline
\$hbrutto
&
Gross Household Data
&
wide
&
hhnrakt, hhnr
&
intid1, intid
\\
\hline
hbrutto
&
Gross Household Data
&
long
&
hid, syear, cid
&
intid1, intid
\\
\hline
pbr\_exit
&
Cumulated Exit
&
long
&
pid, hid, syear, cid
&
hhnrold
\\
\hline
\$pbrutto
&
Gross Individual Data
&
wide
&
persnr, hhnrakt, hhnr
&
\$hhnrold
\\
\hline
pbrutto
&
Gross Individual Data
&
long
&
pid, hid,  syear,  cid
&
intid, hhnrold
\\
\hline
ppfad
&
Individual Tracking File
&
wide
&
persnr, hhnr, \$hhnr,
&\\
\hline
ppfadl
&
Individual Tracking File
&
long
&
pid, hid, syear, cid
&
parid
\\
\hline
\end{tabulary}
\par
\sphinxattableend\end{savenotes}

 \href{https://paneldata.org/soep-core/data/hpfad}{\textbf{hpfad „Household Tracking File" (wide)}}: For all years since 1984, the HPFAD data set contains information on all households that have ever participated in the SOEP survey at any point in time. HPFAD is important for the delimitation of the examination unit (household), especially for longitudinal analyses. HPFAD is particularly suitable for household analyses and can be used for preselection of specific households.

 \href{https://paneldata.org/soep-long/data/hpfadl}{\textbf{hpfadl „Household Tracking File“ (long):}} HPFADL consists of all waves of the data sets  \href{https://paneldata.org/soep-core/data/hpfad}{\textbf{hpfad „Household Tracking File" (wide)}} and  \href{https://paneldata.org/soep-core/data/hhrf}{\textbf{hhrf „Weighting and staying probabilities“ (wide)}} of SOEP-Core.

 \href{https://paneldata.org/soep-core/data/bghbrutto}{\textbf{\$hbrutto  „Gross Household Data“ (CS):}} \$HBRUTTO covers all households, who were successfully interviewed for the first time in wave \$ or were contacted for the purpose of being interviewed again in wave \$. The data sets provide gross cross-sectional information on all SOEP households’ interviews as well as their positions in the panel frame work.

 \href{https://paneldata.org/soep-long/data/hbrutto}{\textbf{hbrutto  „Gross Household Data“ (long):}} HBRUTTO consists of all waves of the data sets  \href{https://paneldata.org/soep-core/data/bghbrutto}{\textbf{\$hbrutto  „Gross Household Data“ (CS):}} of SOEP-Core.

 \href{https://paneldata.org/soep-long/data/pbr_exit}{\textbf{pbr\_exit„Cumulated Exit“ (long):}}:

 \href{https://paneldata.org/soep-core/data/bgpbrutto}{\textbf{\$pbrutto  „Gross Individual Data“ (CS)}} : \$PBRUTTO covers all respondents, who were successfully interviewed for the first time in wave \$ or were contacted for the purpose of being interviewed again in wave \$. The data set provides gross cross-sectional information on all SOEP respondents’ interviews as well as their positions in the panel frame work.

 \href{https://paneldata.org/soep-long/data/hbrutto}{\textbf{pbrutto  „Gross Individual Data“ (long):}} PBRUTTO consists of all waves of the data sets  \href{https://paneldata.org/soep-core/data/bgpbrutto}{\textbf{\$pbrutto  „Gross Individual Data“ (CS)}} of SOEP-Core.

 \href{https://paneldata.org/soep-core/data/ppfad}{\textbf{ppfad „Individual Tracking File“ (wide)}}: For all years since 1984, the PPFAD data set contains information on all persons who have ever lived in a SOEP household at a survey time (i.e. all respondents, but also children under 17 years of age and persons who have never given an interview). PPFAD is important for the delimitation of the examination units (persons), especially for longitudinal analyses.

 \href{https://paneldata.org/soep-long/data/ppfadl}{\textbf{ppfadl „Individual Tracking File“ (long):}} PPFADL consists of all waves of the data sets  \href{https://paneldata.org/soep-core/data/ppfad}{\textbf{ppfad „Individual Tracking File“ (wide)}} and  \href{https://paneldata.org/soep-core/data/phrf}{\textbf{phrf „Weighting and staying probabilities“ (wide)}} of SOEP-Core. It contains one record for each individual and year a person has been a member of a respondent household. It is keyed on PID, the Cross-Wave Person Identifier, and SYEAR, the survey year identifier. It contains the Household ID, and never changing individual characteristics, individual weights, as well as the response status, for that individual at each wave.


\subsection{Original Data}
\label{\detokenize{Principles of Data Structure/index:original-data}}
These data sets contain the direct information of the respondents. The contents of these variables are 1:1 the contents of the survey instruments. By searching in the questionnaires you can determine the exact wording of the question or also possible filter guidance.


\begin{savenotes}\sphinxattablestart
\centering
\begin{tabular}[t]{|\X{5}{30}|\X{10}{30}|\X{5}{30}|\X{5}{30}|\X{5}{30}|}
\hline
\sphinxstyletheadfamily 
Dataset
&\sphinxstyletheadfamily 
Label
&\sphinxstyletheadfamily 
Format
&\sphinxstyletheadfamily 
Identifier (ID)
&\sphinxstyletheadfamily 
Special Identifier
\\
\hline
abroad
&
Questionnaire for people moved abroad
&
long
&
persnr, hhnrakt, syear, hhnr
&\\
\hline
biol
&
Biographical Data
&
long
&
pid, hid, syear, cid
&
intid
\\
\hline
ev
&
First wealth module
&
wide
&
persnr, hhnrakt, hhnr
&\\
\hline
\$h
&
Household questionnaire
&
wide
&
hhnrakt, syear, hhnr
&
intid
\\
\hline
hl
&
Household questionnaire
&
long
&
hid, syear, cid
&
intid
\\
\hline
\$h\_refugees
&
Household questionnaire Refugee Sample
&
wide
&
hhnrakt, syear, hhnr
&
intid
\\
\hline
ghost
&
East specific questions from the Household questionnaire
&
wide
&
hhnrakt, hhnr
&
intid
\\
\hline
\$jugend
&
Youth questionnaire for first time respondents at age 17
&
wide
&
persnr, hhnrakt, syear, hhnr
&
intid
\\
\hline
jugendl
&
Youth questionnaire for first time respondents at age 18
&
long
&
pid, hid, syear, cid
&
intid
\\
\hline
\$p
&
Personal questionnaire
&
wide
&
persnr, hhnrakt, syear, hhnr
&
intid
\\
\hline
pl
&
Personal questionnaire
&
long
&
pid, hid, syear, cid
&
intid
\\
\hline
\$p\_mig
&
IAB-SOEP Migration Sample: Original Individual questionnaire
&
wide
&
pid, hid, syear, cid
&
intid
\\
\hline
\$p\_refugees
&
Personal questionnaire Refugee Sample, incl. Biography
&
wide
&
persnr, hhnrakt, syear, hhnr
&
intid
\\
\hline
\$pausl
&
Migrant specific questions in the Personal Questionnaire
&
wide
&
persnr, hhnrakt, hhnr
&\\
\hline
\$pluecke
&
Follow-Up Questioning
&
wide
&
persnr, hhnrakt, hhnr
&
intid
\\
\hline
\$school
&
Questionnaire: Early Youth, 12-13 years old
&
wide
&
persnr, hhnrakt, syear, hhnr
&
intid
\\
\hline
\$school2
&
Questionnaire: Early Youth, 14-15 years old
&
wide
&
persnr, hhnrakt, syear, hhnr
&
intid
\\
\hline
\$vp
&
Questionnaire: the deceased person
&
wide
&
persnr, hhnrakt, syear, hhnr
&
vpersnr, intid
\\
\hline
\end{tabular}
\par
\sphinxattableend\end{savenotes}

 \href{https://paneldata.org/soep-core/data/abroad}{\textbf{abroad „Questionnaire for people moved abroad“ (CS):}} With the pilot study ”Life outside Germany” in 2008, the longitudinal German Socio-Economic Panel Study (SOEP) ventured into completely uncharted methodological territory by attempting to locate the addresses of former participants in the German household panel study SOEP who have since immigrated abroad, and to survey these individuals with the help of a specially developed written questionnaire on the reasons for their international move. The project was discontinued due to insufficient case numbers in 2014.

 \href{https://paneldata.org/soep-long/data/biol}{\textbf{biol "Biographical Data" (long):}} BIOL contains cumulated individual-level data from the biographical questionnaire.

 \href{https://paneldata.org/soep-core/data/ev}{\textbf{ev „First wealth module“ (long):}}

 \href{https://paneldata.org/soep-core/data/bgh}{\textbf{\$h „Household questionnaire“ (CS):}} The \$H-files contain  all questions of the household questionnaire.

 \href{https://paneldata.org/soep-long/data/hl}{\textbf{hl „Household questionnaire“ (long):}} HL contains all waves of the data sets  from SOEP-Core.

 \href{https://paneldata.org/soep-core/data/bgh_refugees}{\textbf{h\_refugees „Household questionnaire Refugee Sample“ (CS):}} The \$H-files contain  all questions of the household refugees questionnaire.

\sphinxcode{\sphinxupquote{only 1990}}
 \href{https://paneldata.org/soep-core/data/ghost}{\textbf{ghost „East specific questions from the Household questionnaire“ (CS):}} The \$host file contains east specific questions from the household questionnaire. For the year 1990 the data provides information about east specific topics about the German reunification i.e. presents from the BRD.

 \href{https://paneldata.org/soep-core/data/bgjugend}{\textbf{\$jugend „Youth questionnaire for first time respondents at age 17“ (CS)}}: Since 2000 (wave Q), first-time respondents between the ages of 16 and 17 have received a separate biographical questionnaire with additional age-group-specific questions, for instance, about their relationship to their parents or about what they do in their free time. Up to now, only some of the data collected from this survey have been processed and provided to users in dataset BIOAGE17. The complete data will be provided in individual \$JUGEND datasets.

 \href{https://paneldata.org/soep-long/data/jugendl}{\textbf{jugendl „Youth questionnaire for first time respondents at age 17“ (long):}} JUGENDL contains the waves q (2000) up to the current wave of  \href{https://paneldata.org/soep-core/data/bgjugend}{\textbf{\$jugend „Youth questionnaire for first time respondents at age 17“ (CS)}} of SOEP-Core.

 \href{https://paneldata.org/soep-core/data/bgp}{\textbf{\$p „Individual questionnaire“ (CS):}} The \$P-files contain all variables of the individual questionnaire for the wave \$. In addition, the individual-specific data of the samples IAB-SOEP Migration and IAB-BAMF-SOEP Refugee Survey are integrated in the original \$P data set.

 \href{https://paneldata.org/soep-long/data/pl}{\textbf{pl „Individual questionnaire“ (long):}} The PL data set contains all waves of the  \href{https://paneldata.org/soep-core/data/bgp}{\textbf{\$p „Individual questionnaire“ (CS):}} data sets of SOEP-Core. In addition, the PL file contains all variables of all waves of the data sets  \href{https://paneldata.org/soep-core/data/hpost}{\textbf{\$post „East specific questions from the Individual questionnaire“ (CS)}} and  \href{https://paneldata.org/soep-core/data/lpausl}{\textbf{\$pausl „Migrant specific questions in the Individual Questionnaire“ (CS)}}.

\sphinxcode{\sphinxupquote{2013-2016}}
 \href{https://paneldata.org/soep-core/data/bgp_mig}{\textbf{\$p\_mig „IAB-SOEP Migration Sample: Original Individual questionnaire“ (CS):}} The original data from the Sample M specific survey instrument can be found in the dataset \$P\_MIG, combining the individual and the biographical questionnaire. \sphinxstylestrong{Since the current version “v34”, the data set is not part of the SOEP-Core distribution file anymore and has to be ordered separately}. The variables are included in original or generated datasets. Variables equivalent to variables in the individual questionnaire of other samples are included in the dataset \$P, Variables equivalent to variables in the biography questionnaire of other samples are included in the respective biography dataset (e.g. BIOMARSM), the comprehensively surveyed migration biography can be found in the new dataset MIGSPELL.

\sphinxcode{\sphinxupquote{only 2016}}
 \href{https://paneldata.org/soep-core/data/bgp_refugees}{\textbf{\$p\_refugees „IAB--BAMF-SOEP Survey of Refugees in Germany: Original Individual questionnaire“ (CS):}} The original data from the survey instruments used in Samples M3 and M4 can be found in original format in the dataset \$P\_REFUGEES, where the individual and the biographical questionnaires are combined. \sphinxstylestrong{Since the current version “v34”, the data set is not part of the SOEP-Core distribution file anymore and has to be ordered separately}. The variables are integrated in original or generated datasets. Variables equivalent to those in the individual questionnaire of other samples are included in the dataset \$P. Also included in \$P are all variables which will be asked more than once, but specific to the refugee questionnaire, Variables equivalent to those in the biographical questionnaires in other samples are included in the respective biographical datasets (e.g., BIOMARSM), the comprehensively surveyed migration biography can be found in the new dataset REFUGSPELL.

\sphinxcode{\sphinxupquote{1984-1995}}
 \href{https://paneldata.org/soep-core/data/lpausl}{\textbf{\$pausl „Migrant specific questions in the Individual Questionnaire“ (CS)}}:

 \href{https://paneldata.org/soep-core/data/bfpluecke}{\textbf{\$pluecke „Follow-Up Questioning“ (CS):}} Temporary drop-outs (“gaps”) can cause problems for longitudinal analyses. This is especially true for the employment and income data stored. That is why the SOEP tries to fill in at least some of the central missing information. \$PLUECKE is a small questionnaire covering information on the year previous to which the drop-out occurred. This covers questions on job-related changes, calendar of occupation, income, education and qualification.

 \href{https://paneldata.org/soep-core/data/hpost}{\textbf{\$post „East specific questions from the Individual questionnaire“ (CS)}}: The \$post files contain east specific questions from the individual questionnaire. For the years 1990 and 1991 the data provides information about east specific topics.

 \href{https://paneldata.org/soep-core/data/bgschool}{\textbf{\$school „Questionnaire: Early Youth, 12-13 years old“ (CS):}} Since 2014 the \$SCHOOL-files contain all variables of the „Pre-teen (Schülerinnen und Schüler)“ questionnaire. Therefore the data sets provide variables about school, home, leisure time, health, self-perception and relationships with friends, siblings and parents.

 \href{https://paneldata.org/soep-core/data/bgschool2}{\textbf{\$school2 „Questionnaire: Early Youth, 14-15 years old“ (CS):}} Since 2016 the \$SCHOOL2-files contain all variables of the „Early Youth (Frühe Jugend)“ questionnaire. Therefore the data sets provide variables about self-perception, independence, school, leisure time or relationships with friends, siblings and parents.

 \href{https://paneldata.org/soep-core/data/bgvp}{\textbf{\$vp „Questionnaire: the deceased person“ (CS):}} The \$VP-files contain information about respondents who lost a person in the previous year. It provides information about the deceased person and the respondent who reported the case of death.


\subsection{Survey Data}
\label{\detokenize{Principles of Data Structure/index:survey-data}}\label{\detokenize{Principles of Data Structure/index:survey}}
These data sets contain surveymethodical information for SOEP core. The various data sets provide detailed exit information from respondents or household weighting factors that you need for representative analyses.


\begin{savenotes}\sphinxattablestart
\centering
\begin{tabular}[t]{|\X{5}{30}|\X{10}{30}|\X{5}{30}|\X{5}{30}|\X{5}{30}|}
\hline
\sphinxstyletheadfamily 
Dataset
&\sphinxstyletheadfamily 
Label
&\sphinxstyletheadfamily 
Format
&\sphinxstyletheadfamily 
Identifier (ID)
&\sphinxstyletheadfamily 
Special Identifier
\\
\hline
csamp
&
Sample Definition
&
long
&
cid
&\\
\hline
design
&
Survey Design
&
wide
&
hhnr
&
intid
\\
\hline
exit
&
Cumulative drop-outs
&
wide
&
persnr, hhnr, syear
&\\
\hline
hhrf
&
Weighting and staying probabilities
&
wide
&
hhnrakt, hhnr
&\\
\hline
pbr\_hhch
&
PBR\_HHCH
&
wide
&
persnr, hhnrakt, syear, hhnr
&
pnralt, pnrneu, hhnrold
\\
\hline
phrf
&
Weighting and staying probabilities
&
wide
&
persnr, hhnr
&\\
\hline
\end{tabular}
\par
\sphinxattableend\end{savenotes}

 \href{https://paneldata.org/soep-long/data/csamp}{\textbf{csamp „Sample Definition“ (long):}}

 \href{https://paneldata.org/soep-core/data/design}{\textbf{design „Survey design“ (CS)}}: The dataset DESIGN provides information on the stratified sampling of the SOEP in form of two variables. The variable STRAT identifies each of the discrete sampling groups described above. Altogether, the SOEP consists of 40 strata: one stratum in sample A, twenty-seven in sample B, one in sample C, three in sample D, one in sample E, two in sample F, four in sample G, and one in sample H. Unique inclusion probabilities pertain to each of these strata. The variable DESIGN contains the inverse of this probability, i.e., the design weight.

 \href{https://paneldata.org/soep-core/data/exit}{\textbf{exit „Cumulative drop-outs“ (CS):}}

 \href{https://paneldata.org/soep-core/data/hhrf}{\textbf{hhrf „Weighting and staying probabilities“ (wide)}}: In the SOEP database, different weighting variables for cross-sectional as well as for different kinds of longitudinal weighting are set aside for each household in the HHRF-file.

 \href{https://paneldata.org/soep-core/data/pbr_hhch}{\textbf{pbr\_hhch „PBR\_HHCH“ (CS):}}

 \href{https://paneldata.org/soep-core/data/phrf}{\textbf{phrf „Weighting and staying probabilities“ (wide)}}: In the SOEP database, different weighting variables for cross-sectional as well as for different kinds of longitudinal weighting are set aside for each person in the PHRF-file.


\subsection{Generated Data}
\label{\detokenize{Principles of Data Structure/index:generated-data}}
The SOEP team has prepared these data sets for you in a special way. The data sets are prepared in a research-friendly manner and are subjected to additional plausibility checks and quality controls. They usually consist of several variables, of different survey instruments and are described by the documentation provided. Therefore, these data sets cannot be assigned 1:1 to a survey instrument.


\begin{savenotes}\sphinxatlongtablestart\begin{longtable}{|\X{5}{31}|\X{12}{31}|\X{3}{31}|\X{6}{31}|\X{5}{31}|}
\hline
\sphinxstyletheadfamily 
Dataset
&\sphinxstyletheadfamily 
Label
&\sphinxstyletheadfamily 
Format
&\sphinxstyletheadfamily 
Identifier (ID)
&\sphinxstyletheadfamily 
Special Identifier
\\
\hline
\endfirsthead

\multicolumn{5}{c}%
{\makebox[0pt]{\sphinxtablecontinued{\tablename\ \thetable{} -- continued from previous page}}}\\
\hline
\sphinxstyletheadfamily 
Dataset
&\sphinxstyletheadfamily 
Label
&\sphinxstyletheadfamily 
Format
&\sphinxstyletheadfamily 
Identifier (ID)
&\sphinxstyletheadfamily 
Special Identifier
\\
\hline
\endhead

\hline
\multicolumn{5}{r}{\makebox[0pt][r]{\sphinxtablecontinued{Continued on next page}}}\\
\endfoot

\endlastfoot

bioage17
&
Generated biographical youth information
&
wide
&
persnr, hhnrakt, syear, hhnr
&
bymnr, byvnr, intid
\\
\hline
bioagel
&
Generated biographical information
&
long
&
persnr, hhnrakt, syear, hhnr
&
persnre
\\
\hline
biobirth
&
Generated biographical information
&
wide
&
persnr, hhnr
&
kidpnr01-kidpnr15
\\
\hline
bioedu
&
Generated biographical information
&
wide
&
persnr, hhnr
&\\
\hline
bioimmig
&
Generated biographical information
&
long
&
persnr, hhnrakt, syear, hhnr
&\\
\hline
biojob
&
Generated biographical information
&
wide
&
persnr, hhnr
&\\
\hline
bioresid
&
Generated biographical information
&
wide
&
persnr, hhnrakt, syear, hhnr
&
intid
\\
\hline
biosib
&
Generated biographical information
&
wide
&
persnr, hhnr
&
sibpnr1-sibpnr11
\\
\hline
biosoc
&
Generated biographical information
&
wide
&
persnr, hhnrakt, syear, hhnr
&
intid
\\
\hline
biotwin
&
Generated biographical information
&
wide
&
persnr, hhnr
&
pnrtwin, pnrtrip, pnrquad
\\
\hline
camces
&
Highest Educational Qualification, Migrants Sample M1 and M2
&
wide
&
persnr, hhnrakt, syear, hhnr
&\\
\hline
cogdj
&
Data on cognitive tests (Youth)
&
wide
&
persnr, syear, hhnr
&\\
\hline
cognit
&
Data on cognitive potential
&
wide
&
persnr, syear, hhnr
&
intid
\\
\hline
gripstr
&
Measures grip strength
&
wide
&
persnr, syear, hhnr
&
intid
\\
\hline
hconsum
&
Hosehold Consume Module
&
wide
&
hhnrakt, syear, hhnr
&\\
\hline
health
&
Data on health indicators
&
wide
&
persnr, syear, hhnr
&\\
\hline
\$hgen
&
Generated Household Data
&
wide
&
hhnrakt, hhnr
&\\
\hline
hgen
&
Generated Household Data
&
long
&
hid syear cid
&\\
\hline
hwealth
&
Wealth Module
&
long
&
hhnrakt, syear, hhnr
&\\
\hline
interviewer
&
Data on the SOEP Interviewer
&
long
&
hhnr, syear
&
intid
\\
\hline
kidlong
&
Data on children
&
long
&
persnr, hhnrakt, syear, hhnr
&\\
\hline
\$kind
&
Data on children (from HH-Questionnaire)
&
wide
&
persnr, hhnrakt, hhnr
&\\
\hline
mihinc
&
Multiple imputed data on monthly household income
&
long
&
hhnrakt, syear, hhnr
&\\
\hline
\$pequiv
&
Cross-national Equivalent File
&
wide
&
persnr, hhnrakt, syear, hhnr
&\\
\hline
pflege
&
Persons needing care within the household
&
long
&
persnr, syear, hhnr
&\\
\hline
\$pgen
&
Generated Individual Data
&
wide
&
persnr, hhnrakt, hhnr
&\\
\hline
pgen
&
Generated Individual Data
&
long
&
pid, hid, syear, cid
&\\
\hline
\$pkal
&
Individual Calendar
&
wide
&
persnr, hhnrakt, hhnr
&\\
\hline
pkal
&
Individual Calendar
&
long
&
pid, hid, syear, cid
&\\
\hline
\$pkalost
&
Individual Calender
&
wide
&
persnr, hhnrakt, hhnr
&\\
\hline
pwealth
&
Wealth Module
&
long
&
persnr, hhnrakt, syear
&\\
\hline
timepref
&
Experiment on time preferences
&
wide
&
persnr, hhnrakt, syear, hhnr
&\\
\hline
trust
&
Experiment on trust
&
long
&
persnr, hhnrakt, syear, hhnr
&\\
\hline
\end{longtable}\sphinxatlongtableend\end{savenotes}

 \href{https://paneldata.org/soep-core/data/bioage17}{\textbf{bioage17 „Generated biographical information“ (CS):}} The design of the dataset BIOAGE17 is patterned after the 2001 Youth Questionnaire, which is the standard version for subsequent years.  A special group of first time respondents are young persons living in a panel household, who reach the surveying age of 17 years. From this specific group of panel entrants, we are able to obtain some more detailed information on youth and socialisation than from other new sample members.

 \href{https://paneldata.org/soep-core/data/bioagel}{\textbf{bioagel „Generated biographical information“ (long):}} The BIOAGEL data files are generated using information collected in the “Mother \& Child” and “Parent” questionnaires. BIOAGEL is now provided in one dataset.

 \href{https://paneldata.org/soep-core/data/biobirth}{\textbf{biobirth „Generated biographical information“ (CS):}} The file BIOBIRTH provides information on fertility histories of adult respondents in the SOEP. Until 2014 (version 30, wave BD) the data was stored in two separate files: BIOBIRTH containing female fertility histories, and BIOBRTHM providing male fertility histories. Fertility histories in BIOBIRTH provide information on every woman (as well as every man with a panel entry since 2001) who has ever provided at least one successful SOEP interview.

 \href{https://paneldata.org/soep-core/data/bioedu}{\textbf{bioedu „Generated biographical information“ (CS):}} The Socio-Economic Panel Study (SOEP) contains a broad range of variables which cover early child education and care, educational participation, educational degrees and other related topics. It is the aim of the BIOEDU dataset to provide ready-made variables on educational transitions and related topics in order to support analyses in a longitudinal perspective.

 \href{https://paneldata.org/soep-core/data/bioimmig}{\textbf{bioimmig „Generated biographical information“ (long):}} The variables contained in BIOIMMIG deal with questions related to foreigners in (and migrants to) Germany. Specifically, questions concerning desire to return to the home country, the presence of relatives in the home country, reasons for coming to Germany, and conditions upon initial arrival in Germany.

 \href{https://paneldata.org/soep-core/data/biojob}{\textbf{biojob „Generated biographical information“ (CS):}} The purpose of BIOJOB is to provide a file, that offers the user convenient access to biographical information on past job activities. BIOJOB consists of generated variables as well as plain questionnaire information. Up to now all but two variables of BIOJOB are time-invariant. Information on occupational changes and on the age at the most recent change of occupation refer to the date of the respondent’s biography interview.

 \href{https://paneldata.org/soep-core/data/bioresid}{\textbf{bioresid „Generated biographical information“ (CS):}} In 1994 questions with a focus on occupancy were introduced to the Biographical Questionnaire asking for the duration of residence in the current dwelling and any second residence. The information surveyed in the Biographical Questionnaire is stored in the file BIORESID.

 \href{https://paneldata.org/soep-core/data/biosibd}{\textbf{biosib „Generated biographical information“ (CS):}} BIOSIB provides information on siblings living within the SOEP households. The data set contains the person numbers of all siblings in an observed family. It includes information on their sex, their year of birth, the number of siblings, the individual’s position within the birth order, and on the relationship between the observed siblings.

 \href{https://paneldata.org/soep-core/data/biosoc}{\textbf{biosoc „Generated biographical information“ (CS):}} BIOSOC contains retrospective data on youth and socialization. Respondents of all ages describe aspects of their life at the age of 15, including their relationship with parents, grades in school, the federal state where they last attained educational qualifications, detailed information on vocational qualifications, as well as intentions to complete further education or vocational training. Questions concerning military and alternative services are also included in this data set.

 \href{https://paneldata.org/soep-core/data/biotwin}{\textbf{biotwin „Generated biographical information“ (CS):}} The file BIOTWIN contains all twins that were ever identified within the SOEP. To be classified as a twin, a person is required to  have exactly the same age as his or her sibling (year \& month of birth), have a relationship to the head of the household that indicates that he or her and a second persons are siblings, and have the same mother (as far as a pointer to the mother is available). Furthermore, it is not only twins that are recorded in the BIOTWIN data set, but also triplets or quadruple siblings.

 \href{https://paneldata.org/soep-core/data/camces}{\textbf{camces „Highest Educational Qualification, Migrants Sample M1 and M2“ (CS):}} The CAMCES-File provides information about Computer-Assisted Measurement and Coding of Educational Qualifications in Surveys.

 \href{https://paneldata.org/soep-core/data/cogdj}{\textbf{cogdj „Data on cognitive tests (Youth)“ (CS):}} In SOEP 2006, a separate questionnaire with cognitive tests for adolescents was used for the first time: “Lust auf DJ”. In this case, “DJ” stands for “Thinking Sports and Youth (Denksport und Jugend)”, but was also specifically selected to arouse the more common association of “Disc Jockey”. For all interviewees aged 16 - 17 years, the questionnaire “Lust auf DJ” was used and created.

 \href{https://paneldata.org/soep-core/data/cognit}{\textbf{cognit „Data on cognitive potential“ (long):}} In the 2006 survey year, for the first time, short cognitive tests were carried out with a subsample of the SOEP. The goal was to employ a robust set of instruments that could be administered easily by trained interviewers within just a few minutes. Im COGNIT06 werden den Nutzern die aggregierten Summen-Scores (jeweils Gesamtwerte für drei Zeitpakete, sog. „parcels“ von 30, 60 und 90 Sekunden) zur Verfügung gestellt.

 \href{https://paneldata.org/soep-core/data/gripstr}{\textbf{gripstr „Measures grip strength (left and right hand)“ (long):}} The data on grip strength from the survey year 2012 is now included in the GRIPSTR dataset.

 \href{https://paneldata.org/soep-core/data/hconsum}{\textbf{hconsum „HH consume module“ (CS)“:}} We were faced with three methodological challenges in generating the final consumption data. Firstly, due to the design of the consumption module, inconsistent answers arose between the monthly and annual amounts spent for consumption. Secondly, we encountered the well-known phenomenon of missing data, here in particular item nonresponse. And thirdly, consumption data are usually blurred by heaping. For researchers who do not want their consumption variables to include changes from all steps of data preparation, the new data set “HCONSUM” contains not only the prepared consumption variables but also flag variables providing researchers the opportunity to select individual solutions.

 \href{https://paneldata.org/soep-core/data/health}{\textbf{health „Data on health indicators“ (long):}} Starting in 2002 the SOEP health module in the individual questionnaire has been revised and put into a two year replication period. In the HEALTH-File users find i.e. the generated variables on height and weight with imputation flags and a user-friendly longitudinal checked generated variable of the Body Mass Index (BMI).

 \href{https://paneldata.org/soep-core/data/bghgen}{\textbf{\$hgen „Generated Household Data“ (CS)}}: In order to minimize computing efforts for the user, the SOEP provides yearly status variables on household level. The \$HGEN data provides a set of time-consistent variables generated from the SOEP household questionnaire. It only includes households who participated in the respective year.

 \href{https://paneldata.org/soep-long/data/hgenlong}{\textbf{hgen „Generated Household Data“ (long):}} HGEN contains all waves of the  \href{https://paneldata.org/soep-core/data/bghgen}{\textbf{\$hgen „Generated Household Data“ (CS)}} data sets of SOEP-Core.

 \href{https://paneldata.org/soep-core/data/hwealth}{\textbf{hwealth „Wealth module“ (long):}} The generated SOEP wealth data is stored in two separate data files called PWEALTH for information at the individual level and HWEALTH for correspondingly aggregated data at the household level. HWEALTH contains all information on the household level; it is purely the result of aggregating the person-level information in PWEALTH. However for all persons with valid household level information that did refuse to respond to the Individual questionnaire (partial unit non-response) imputations have been carried out and the results are included in HWEALTH.

 \href{https://paneldata.org/soep-core/data/interviewer}{\textbf{interviewer „Data on the SOEP Interviewer“ (long):}} The SOEP does not only aim at collecting high-quality data on the living conditions and well-being of households, but \textendash{}as a by-product of internal quality assurance processes\textendash{} it lends itself increasingly as a empirical source for survey research. The purpose of the INTERVIEWER file is to provide user convenient access to all available, longitudinal information on the SOEP interviewers.

 \href{https://paneldata.org/soep-core/data/kidlong}{\textbf{kidlong „Data on children“ (long)}}: The variables stored in the KIDLONG file are based on the information annually collected and stored in the wave-specific \$KIND files. The relevant information is not provided by children themselves but by answers to the questions in the household questionnaire given by the respondent within the household (mostly the head of the household). This data is reaggregated at the person level and stored as child-specific entries in the file  \href{https://paneldata.org/soep-core/data/bgkind}{\textbf{\$kind „Data on children (from HH-Questionnaire)“ (CS):}}.

 \href{https://paneldata.org/soep-core/data/bgkind}{\textbf{\$kind „Data on children (from HH-Questionnaire)“ (CS):}} The variables from the annual \$kind files  are not based on answers provided by the children themselves, but by answers   provided by the head of household. This data is re-aggregated on the person level and saved as child-specific entries in the file \$kind. The annual \$kind datasets also contain additional information on institutional care and school attendance for children and young people.

 \href{https://paneldata.org/soep-core/data/mihinc}{\textbf{mihinc „Multiple imputed data on monthly household income (long)“:}} The dataset MIHINC contains the complete imputation results and is separately available. To be compatible with methods for analysing multiply imputed data, MIHINC is constructed in the so called stacked or MIM Dataset Format. It contains the following variables: HHNRAKT, SVYYEAR, MJ, MI, IHINC and IMPFLAG. Since 1995 for every survey household in all survey years there are ten imputed values for the current household income.

 \href{https://paneldata.org/soep-core/data/bgpequiv}{\textbf{\$pequiv „Cross-national Equivalent File“ (CS)}}: The \$PEQUV-File is based on the Cross-National Equivalent File (CNEF) with extended income information for the SOEP. This file comprises not only the aggregated income figures provided in the CNEF but also further single income components.

 \href{https://paneldata.org/soep-long/data/pequiv}{\textbf{pequiv „Cross-national Equivalent File“ (long)}} PEQUIV contains all waves of the  \href{https://paneldata.org/soep-core/data/bgpequiv}{\textbf{\$pequiv „Cross-national Equivalent File“ (CS)}} data sets of SOEP-Core.

 \href{https://paneldata.org/soep-core/data/pflege}{\textbf{pflege „Persons needing care within the household“ (long):}} Since wave B (1985) the SOEP household questionnaire includes questions on household members in need of care. In order to support analyses on an individual level, this information has been restructured and stored in the cumulative file PFLEGE.

 \href{https://paneldata.org/soep-core/data/bgpgen}{\textbf{\$pgen „Generated Individual Data“ (CS):}} The \$PGEN-files contain user friendly data on the individual level which are consolidated from different sources. The plausibility is in many respects longitudinally validated, therefore the data here are in most situations superior compared to the data in \$P. The file contains one row for each person (persnr is unique) with a completed individual or youth questionnaire.

 \href{https://paneldata.org/soep-long/data/pgen}{\textbf{pgen „Generated Individual Data“ (long):}} PGEN contains all waves of the  \href{https://paneldata.org/soep-core/data/bgpgen}{\textbf{\$pgen „Generated Individual Data“ (CS):}} data sets of SOEP-Core.

 \href{https://paneldata.org/soep-core/data/bgpkal}{\textbf{\$pkal „Individual Calendar“ (CS)}}: The \$pkal datasets contain calender variables from the Individual questionnaire.  The dataset includes the activity status on a monthly basis as well as the income status of a person.

 \href{https://paneldata.org/soep-long/data/pkal}{\textbf{pkal „Individual Calendar“ (long)}} PKAL contains all waves of the  \href{https://paneldata.org/soep-core/data/bgpkal}{\textbf{\$pkal „Individual Calendar“ (CS)}} data sets of SOEP-Core.

\sphinxcode{\sphinxupquote{1990-1991}}
 \href{https://paneldata.org/soep-core/data/hpkalost}{\textbf{\$pkalost „Individual Calendar“ (CS):}}

 \href{https://paneldata.org/soep-core/data/pwealth}{\textbf{pwealth „Wealth module“ (long):}} In the year 2002, the individual questionnaire included for the first time a special module focusing on wealth. This section included questions on seven different wealth components: Owner-occupied property (including debt), other property (including debt),  financial assets, private pensions (including life insurance and building savings contracts),  business assets,  tangible assets and  consumer credit. The generated SOEP wealth data is stored in two separate data files called PWEALTH for information at the individual level and HWEALTH for correspondingly aggregated data at the household level. Wealth-related variable names in the file PWEALTH consist of six digits. The first digit tells the user which wealth component is referred to, and the second to sixth digits provide more detailed information about possible filter information, the personal share, the gross amount, and the amount of any outstanding debt. In principle a digit is coded “1” if a given variable does indeed contain this specific piece of information and “0” otherwise.  The wealth information in the SOEP questionnaire is surveyed at the individual level and thus also imputed or edited at the individual level (although checked against household information for consistency).

 \href{https://paneldata.org/soep-core/data/timepref}{\textbf{timepref „Experiment on time preferences“ (CS):}} Following on the behavioral experiment on trust and trustworthiness carried out in the 2003, 2004, and 2005 SOEP surveys, the experiment “time preferences” was run in 2006. In this experiment on economic behavior, respondents were asked to decide how they would want to receive \texteuro{}200 in prize money: if they would want to receive it immediately by check, or if they would want to wait and receive a larger amount later—that is, with interest.

 \href{https://paneldata.org/soep-core/data/trust}{\textbf{trust „Experiment on trust“ (long):}} Data set of the economic behavior experiment on trust and trustworthiness from the survey years 2003, 2004 \& 2005, which serves to measure trust, based on an investment game. This is a one-off game for two actors who relate to each other anonymously. The first player receives a credit of ten points and can overwrite any number of points of the second player. Each overwritten point is doubled. The second player also receives a credit of ten points. After receiving the (doubled) points from the first player, it decides how much of its own credit it will transfer to the first player (zero to ten points). As with the first transfer, your points at the recipient are doubled. After the decision of the second player, the game ends and the other players are paid their income (one point corresponds to one euro, the sum is sent out as a cheque a few days later). The TRUST data set thus contains the information from all three waves in which the behavioral experiment was conducted.


\subsection{Spell Data}
\label{\detokenize{Principles of Data Structure/index:spell-data}}
General information about spell data in the SOEP can be found in the chapter {\hyperref[\detokenize{Principles of Data Structure/index:data-structure-in-spell-format-spell}]{\sphinxcrossref{Data Structure in spell format (spell)}}}


\begin{savenotes}\sphinxattablestart
\centering
\begin{tabular}[t]{|\X{5}{35}|\X{15}{35}|\X{5}{35}|\X{5}{35}|\X{5}{35}|}
\hline
\sphinxstyletheadfamily 
Dataset
&\sphinxstyletheadfamily 
Label
&\sphinxstyletheadfamily 
Format
&\sphinxstyletheadfamily 
spellnr
&\sphinxstyletheadfamily 
Special Identifier
\\
\hline
artkalen
&
Spell data from the activity calendar
&
wide
&
persnr, hhnr
&\\
\hline
biocouplm
&
Generated biographical information
&
long
&
persnr, hhnr
&
coupid
\\
\hline
biocouply
&
Generated biographical information
&
long
&
persnr, hhnr
&\\
\hline
biomarsm
&
Generated biographical information
&
long
&
persnr, hhnr
&\\
\hline
biomarsy
&
Generated biographical information
&
long
&
persnr, hhnr
&\\
\hline
einkalen
&
{[}deprecated{]} Spell data on income
&
long
&
persnr, hhnr
&\\
\hline
lifespell
&
Spell Information on the Pre- and Post-Survey History of SOEP-Respondents
&
long
&
persnr, hhnr
&\\
\hline
migspell
&
Migration history
&
long
&
persnr, hhnr
&\\
\hline
pbiospe
&
Generated biographical information
&
long
&
persnr, hhnr
&\\
\hline
refugspell
&
Migration history
&
long
&
persnr, hhnr
&\\
\hline
sozkalen
&
{[}deprecated{]} Spell data on social benefits
&
long
&
hhnrakt, hhnr
&\\
\hline
\end{tabular}
\par
\sphinxattableend\end{savenotes}

 \href{https://paneldata.org/soep-core/data/artkalen}{\textbf{artkalen „Spell data from the activity calendar“ (long)}}: The ARTKALEN contains spells (monthly) for events starting in January 1983. This is in contrast to PBIOSPE, where spells were in yearly durations, and events previous to 1983 were included. The information on activity status are collected on a monthly basis in the yearly Individual questionnaire and stored in the file ARTKALEN.

 \href{https://paneldata.org/soep-core/data/biocouplm}{\textbf{biocouplm „Generated biographical information“ (long):}} With the BIOCOUPLM the SOEP provides consistent and continuous partnership histories for nearly all adult respondents. BIOCOUPLM is build on the prospective information at the time of each interview. The relationsship histories are collected on a monthly basis from all adult SOEP-participants since their entry into the SOEP.

 \href{https://paneldata.org/soep-core/data/biocouply}{\textbf{biocouply „Generated biographical information“ (long):}} With the BIOCOUPLY the SOEP provides consistent and continuous partnership histories for nearly all adult respondents. BIOCOUPLY is build on retrospective and prospective information at the time of each interview. The relationsship histories are provided on an annual basis.

 \href{https://paneldata.org/soep-core/data/biomarsm}{\textbf{biomarsm „Generated biographical information“ (long)}}: With BIOMARSM the SOEP provides consistent and continuous marital histories for nearly all adult respondents. BIOMARSM is build on the prospective information at the time of each interview. The martial histories are collected on a monthly basis from all adult SOEP-participants since their entry into the SOEP.

 \href{https://paneldata.org/soep-core/data/biomarsy}{\textbf{biomarsy „Generated biographical information“ (long)}}: With BIOMARSY the SOEP provides consistent and continuous marital histories for nearly all adult respondents. BIOMARSY is build on retrospective and prospective information at the time of each interview. The marital histories are provided on an annual basis.

 \href{https://paneldata.org/soep-core/data/einkalen}{\textbf{einkalen „[deprecated] Spell data on income“ (long)}}: The income calendar is used to gain information about sources of income throughout the year. The respondent checks off for each month all appropriate sources of income.

 \href{https://paneldata.org/soep-core/data/lifespell}{\textbf{lifespell „Spell Information on the Pre- and Post-Survey History of SOEP-Respondents"}}: The SOEP team regularly conducts drop-out studies to identify the whereabouts of attritors. These studies draw on official register data and allow us to determine whether a person is still living in Germany, is deceased, or has moved abroad since the last SOEP interview. The information is combined in a spell file LIFESPELL. This dataset reports all available information on the pre- and the post-survey history of all persons who have ever been a member of a SOEP household.

 \href{https://paneldata.org/soep-core/data/migspell}{\textbf{migspell „Migration history“(long)}}: MIGSPELL is derived from the migration biographies, which are collected from each new respondent of the IAB-SOEP migration samples M1 and M2. It contains data on the moves of foreign-born migrants as well as on the stays abroad of German-born respondents.

 \href{https://paneldata.org/soep-core/data/pbiospe}{\textbf{pbiospe „Generated biographical information“ (long)}}: The spell file PBIOSPE is based on the information on activity status over the life course, which is collected as a matrix from every respondent answering the Biography Questionnaire. The observations start at the age of 15 and end at the current age (up to age 65). To update the ongoing occupational career in PBIOSPE, information from the yearly Individual Questionnaire is also used.

 \href{https://paneldata.org/soep-core/data/refugspell}{\textbf{refugspell „Migration history“ (long)}}: For migration biographies in the refugee samples, we created the spell data set REFUGSPELL. The variables in MIGSPELL and REFUGSPELL are derived from different instruments and only partially overlap. The data structure allows the data set to be linked with MIGSPELL if desired.

\sphinxcode{\sphinxupquote{1992-2000}}
 \href{https://paneldata.org/soep-core/data/sozkalen}{\textbf{sozkalen „[deprecated] Spell data on social benefits“}}: The file SOZKALEN provides spell data on receiving social assistance of households, defining begin, end, and censoring status of any period of receiving 3 different types of assistance. This file is set up, using information from the calendar, asked for the previous year (asked for the years 1992-2000). Thus, it contains information on a monthly basis.


\section{Labeling SOEP-Core}
\label{\detokenize{Principles of Data Structure/index:labeling-soep-core}}\label{\detokenize{Principles of Data Structure/index:label}}
The following explanations refer to the data sets of the subdirectory “raw” in your distribution file. There is no systematic variable naming for the long files above the subdirectory “raw”.


\subsection{Labeling Scheme of Data Sets and Variables in SOEP-Core}
\label{\detokenize{Principles of Data Structure/index:labeling-scheme-of-data-sets-and-variables-in-soep-core}}
To distinguish the multitude of data sets and variables, the SOEP uses systematic dataset and variable names for data in cross-sectional format. These names provide a lot of information for data users.
Example of a data set name:

xp

\begin{figure}[H]
\centering

\noindent\sphinxincludegraphics{{dataset_example}.PNG}
\end{figure}

\sphinxstylestrong{The first identifier of each data set name is the wave identifier (“x”). It can contain one or two letters. .}

Each wave or survey year can be assigned using a letter in the alphabet:


\begin{savenotes}\sphinxattablestart
\centering
\begin{tabulary}{\linewidth}[t]{|T|T|T|T|T|T|T|T|T|T|T|T|T|T|T|T|T|}
\hline

1984
&
1985
&
1986
&
1987
&
1988
&
1989
&
1990
&
1991
&
1992
&
1993
&
1994
&
1995
&
1996
&
1997
&
1998
&
1999
&
2000
\\
\hline
a
&
b
&
c
&
d
&
e
&
f
&
g
&
h
&
i
&
j
&
k
&
l
&
m
&
n
&
o
&
p
&
q
\\
\hline
2001
&
2002
&
2003
&
2004
&
2005
&
2006
&
2007
&
2008
&
2009
&
2010
&
2011
&
2012
&
2013
&
2014
&
2015
&
2016
&
2017
\\
\hline
r
&
s
&
t
&
u
&
v
&
w
&
x
&
y
&
z
&
ba
&
bb
&
bc
&
bd
&
be
&
bf
&
bg
&
bh
\\
\hline
\end{tabulary}
\par
\sphinxattableend\end{savenotes}

As can be seen from the table, the sample data set “xp” contains survey information from the survey year 2007.

\sphinxstylestrong{The second identifier of each data set name is the abbreviation for the respective survey instrument or, for generated data sets, the name of the content (“p”).}
\begin{itemize}
\item {} 
h= Household

\item {} 
hbrutto= Household Gross

\item {} 
hgen= Generated Household Data

\item {} 
p= Individuals

\item {} 
pbrutto= Person Gross

\item {} 
p\_mig= Migrants

\item {} 
pgen= Generated individual data

\item {} 
jugend = Youth (Ages 16-17)

\item {} 
school= Pupils (Ages 11-12)

\item {} 
vp= Deceased persons

\item {} 
luecke= Gap Questionnaire

\item {} 
hkind= Information for children from household questionnaire

\item {} 
pequiv= Cross National Equivalent File

\item {} 
pkal= Calendar

\end{itemize}

Further examples:
\begin{itemize}
\item {} 
bah = Wave „ba“ (Survey year 2010), Household data sets

\item {} 
bfschool= Wave „bf“ (Survey year 2015), Pupils data sets

\item {} 
zhgen = Wave „z“(Survey year 2009), Generated Household data sets

\end{itemize}

Variable names in the SOEPcore data files follow basic conventions:
First, there are datasets with “speaking” variable names, where the variable name itself conveys something about the information stored in this variable. This is usally the case when the dataset is generated.

For the original datasets such as \$H, \$P and \$KIND, the variable names are set up “around” the unit of analysis (individual - “p”, household - “h”, and child - “k”) and show before this indicator the wave in which the data was collected and after it the reference where the question can be found in the original survey instrument (see Figure 9 for an overview).

\begin{figure}[H]
\centering

\noindent\sphinxincludegraphics{{wuqi}.PNG}
\end{figure}

Example for a variable name:
bfp0103

\begin{figure}[H]
\centering

\noindent\sphinxincludegraphics{{variable_example}.PNG}
\end{figure}

The first identifier of a variable name is the wave (i.e. „bf“)
Every wave or rather every year can be assigned to a specific letter in the alphabet:


\begin{savenotes}\sphinxattablestart
\centering
\begin{tabulary}{\linewidth}[t]{|T|T|T|T|T|T|T|T|T|T|T|T|T|T|T|T|T|}
\hline

1984
&
1985
&
1986
&
1987
&
1988
&
1989
&
1990
&
1991
&
1992
&
1993
&
1994
&
1995
&
1996
&
1997
&
1998
&
1999
&
2000
\\
\hline
a
&
b
&
c
&
d
&
e
&
f
&
g
&
h
&
i
&
j
&
k
&
l
&
m
&
n
&
o
&
p
&
q
\\
\hline
2001
&
2002
&
2003
&
2004
&
2005
&
2006
&
2007
&
2008
&
2009
&
2010
&
2011
&
2012
&
2013
&
2014
&
2015
&
2016
&
2017
\\
\hline
r
&
s
&
t
&
u
&
v
&
w
&
x
&
y
&
z
&
ba
&
bb
&
bc
&
bd
&
be
&
bf
&
bg
&
bh
\\
\hline
\end{tabulary}
\par
\sphinxattableend\end{savenotes}

As can be seen from the table, the variable „bfp0103” contains information from the survey year 2015.

The second identifier of a variable is the abbreviation for the respective survey instrument or the type of information („p“)
\begin{itemize}
\item {} 
h= Household

\item {} 
hbrutto= Household gross

\item {} 
hgen= Generated household data

\item {} 
p=Individual data

\item {} 
pbrutto= Person gross

\item {} 
p\_mig= Person migrants (M1 und M2)

\item {} 
pgen= Generated individual data

\item {} 
jugend = Youth (Ages 16-17)

\item {} 
school= Pupils (Ages 11-12)

\item {} 
vp= Deceased people

\item {} 
luecke= Gap Questionnaire

\item {} 
hkind= Children information from the household questionnaire

\item {} 
pequiv= Cross National Equivalent File

\item {} 
pkal= Calender

\end{itemize}

The third identifier of a variable name describes the question number („01“) and a possible fourth identifier describes the position of the answer category („03“).

\begin{figure}[H]
\centering

\noindent\sphinxincludegraphics{{question_example_2}.PNG}
\end{figure}

The example variable „bfp0103“ describes the „satisfaction of work“. The variable was raised in 2015 („bf“) and it can be found in the individual questionnaire („p“).  In the associated individual questionnaire, the variable can be found in the first question („01“) under the third position of all answers categories („03“).

More examples:
-       ap06 = Wave „a“ (survey year 1984), Individual Dataset, Question 6
-       th1603 = Wave „t“ (survey year 2003), Household Dataset, Question 16, Item 3
-       lp10312= Wave „l“ (survey year 1995), Individual Dataset, Question 3, Item 12
-       bap15604 = Wave „ba“ (survey year 2010), Individual Dataset, Question 156, Item 4

Since the data structure is getting richer every year, we extended the common variable naming convention WUQI, starting with the wave „bh“(2017).
Additionally, we provide our users with an „instrument“ variable that contains all our survey instruments for each analyzing unit.


\subsection{Extended Variable Naming Conventions}
\label{\detokenize{Principles of Data Structure/index:extended-variable-naming-conventions}}
\begin{figure}[H]
\centering

\noindent\sphinxincludegraphics{{wu_q_i_q}.PNG}
\end{figure}

We added an underscore between question identifier and item identifier to separate question and item visually. In addition, a questionnaire identifier was introduced, which is also separated by an underscore from the item. This new version of naming variables only comes to use, if the survey instrument differs from the „original“ instrument.

Due to our different samples in the SOEP, there are some samples groups that are getting sample specific questions, like the migrant sample that started in 2013. For that specific group, we created an extended individual questionnaire, with migrant specific question and standard SOEP questions that are asked every year. For the specific questions, you can use the instrument variable to see the variables{}` source.

Let{}`s take a look at the variable bhp109\_01\_q57
\begin{itemize}
\item {} 
bh= Year 2017

\item {} 
P= Person questionnaire

\item {} 
109= Question 109

\item {} 
\_01= First Item

\item {} 
\_q57= ?

\end{itemize}

To know which questionnaire is the right one, you have to take a look at the instrument variable.


\begin{savenotes}\sphinxattablestart
\centering
\begin{tabulary}{\linewidth}[t]{|T|T|}
\hline
\sphinxstyletheadfamily 
Value
&\sphinxstyletheadfamily 
Questionnaire
\\
\hline
50
&
2017 Individual Questionnaire  (A-L1  ; PAPI) {[}soep-core-2017-pe{]}
\\
\hline
51
&
2017  Individual Questionnaire (A-L3 ; CAPI) {[}soep-core-2017-pe2{]}
\\
\hline
52
&
2017  Individual Questionnaire (L2-L3 ; CAWI) {[}soep-core-2017-pe3{]}
\\
\hline
53
&
2017  Individual Questionnaire  (N; CAPI) {[}soep-core-2017-pe4{]}
\\
\hline
54
&
2017  Individual Questionnaire  (M1-M2 Re-Surveyed; CAPI) {[}soep-core-2017-p-m12{]}
\\
\hline
55
&
2017 Questionnaire Individual-Biography (M1-M2 First-Surveyed; CAPI) {[}soep-core-2017-pb-m12-erst{]}
\\
\hline
56
&
2017 Questionnaire Individual-Biography  (M3-M5 First-Surveyed; CAPI) {[}soep-core-2017-pb-m345-erst{]}
\\
\hline
57
&
2017 Questionnaire Individual-Biography  (M3-M4 Re-Surveyed; CAPI) {[}soep-core-2017-pb-m34-wieder{]}
\\
\hline
58
&
2017 Biography Questionnaire  (A-L1  First-Surveyed; PAPI) {[}soep-core-2017-ll{]}
\\
\hline
59
&
2017 Biography Questionnaire  (A-L3; N First-Surveyed; CAPI) {[}soep-core-2017-ll2{]}
\\
\hline
\end{tabulary}
\par
\sphinxattableend\end{savenotes}

The instrument variable for identifying the exact questionnaire can be found in the respective data set. The value Q57 of the example identifies the individual biography questionnaire for re-surveyed respondents of the samples M3/M4 as the variable source.
If you are now interested in the direct question in the questionnaire, open the individual biography questionnaire for refugees (Re-Surveyed), look for question number 109 and look at the first item. The variable bhp109\_01\_q57 was raised with the following question:


\begin{savenotes}\sphinxattablestart
\centering
\begin{tabular}[t]{|*{1}{\X{1}{1}|}}
\hline

Q109: \sphinxstylestrong{When was the beginning of the integration course?}
\begin{itemize}
\item {} 
1  Year

\item {} 
2  Month

\item {} 
99 No Details

\end{itemize}
\\
\hline
\end{tabular}
\par
\sphinxattableend\end{savenotes}

Using the variable name and the instrument variable, you can easily identify the corresponding question in the corresponding questionnaire:
\begin{itemize}
\item {} 
bhp109\_01\_q57

\item {} 
bh= Year 2017

\item {} 
P= Individual questionnaire

\item {} 
109= Question 109

\item {} 
\_01= First Item

\item {} 
\_q57= 2017 Questionnaire Individual-Biography  (M3-M4 Re-Surveyed; CAPI) {[}soep-core-2017-pb-m34-wieder{]}

\end{itemize}


\subsection{Missing Conventions}
\label{\detokenize{Principles of Data Structure/index:missing-conventions}}\label{\detokenize{Principles of Data Structure/index:missings}}
Survey variables might be missing, i.e. without a valid code or value for different reasons. In the SOEP, negative values are not valid for any variable, but are used instead to code different reasons for missing information. There are two distinctions for missing values: they may originate in the respondent’s answer or in the survey design. The respondent may refuse or not know an answer or she may report invalid values on the one hand, and the interview design may exclude respondents with certain characteristics from some questions on the other (e.g. men will never be asked if they are pregnant). The following codes apply both for SOEPCore and SOEPlong, also shown here:


\begin{savenotes}\sphinxattablestart
\centering
\begin{tabulary}{\linewidth}[t]{|T|T|}
\hline
\sphinxstyletheadfamily 
Code
&\sphinxstyletheadfamily 
Label
\\
\hline
-1
&
no answer / don’t know
\\
\hline
-2
&
does not apply
\\
\hline
-3
&
implausible value
\\
\hline
-4
&
Inadmissable multiple response
\\
\hline
-5
&
Not included in this version of the questionnaire
\\
\hline
-6
&
Version of questionnaire with modified filtering
\\
\hline
-8
&
Question not part of the survey program this year\(\sp{\text{1}}\)
\\
\hline
\end{tabulary}
\par
\sphinxattableend\end{savenotes}

\sphinxstyleemphasis{\(\sp{\text{1}}\)Only applicable for datasets in long format.}

A person might refuse to answer a question, which happens more often in sensitive questions (e.g. income related questions), or may just not know the answer to a question. In such a case, the missing code is “-1” for “no answer / don’t know”. Note that the SOEP does not distinguish between the refusal to answer and a true “don’t know”. Information may be missing when a question is not asked because it is not relevant for a specific person, e.g. owner-occupiers will not be asked about the amount of rent they pay. In such cases, the question “Does not apply” to this person, and the variable receives a code of “-2”. Sometimes invalid answers are encountered, when respondents fill out a PAPI interview themselves or the interviewer mistypes an answer, e.g. persons cannot work more than 168 hours a week. In such a case, multiple checks are carried out, and if the inconsistency remains, the variable is recoded “-3 Implausible value”. Some questions contain multiple answer possibilities, where the respondents are asked to pick one and only one answer. In the SOEP PAPI instruments, sometimes respondents ignore this request and provide more than one answer, e.g. they mark “very good” and “good” when asked about their current health status. In such cases, if the correct answer cannot be determined from the questionnaire itself, the code “-4 Invalid Multiple Answers” is given to this variable. With the extension of the SOEP in recent years, entirely new samples have been added to the core. In these samples, sometimes questions are left out completely, e.g. to shorten the questionnaire or because the focus of the sample is different as in some of the related studies. In such a case, the variable will be set to “-5 Not included in this version of the questionnaire” for an entire subsample.
With the use of CAPI, recent developments include an “integrated” person questionnaire, i.e. the biography part and the “regular” part of the questionnaire are asked as one. Some of the questions in the biography part are repeated in the regular part. While in the PAPI mode, the respondent will answer the same question twice, the CAPI allows to filter the respondent around the question if it has already been asked. These cases are very rare - if they occur, they receive a code “-6 Version of questionnaire with modified filtering”.


\chapter{Working with SOEP Data}
\label{\detokenize{Working with SOEP Data/index:working-with-soep-data}}\label{\detokenize{Working with SOEP Data/index::doc}}

\section{Working with Tracking Data (PPFAD)}
\label{\detokenize{Working with SOEP Data/index:working-with-tracking-data-ppfad}}\label{\detokenize{Working with SOEP Data/index:working-ppfad}}
For all years since 1984, the PPFAD data set contains information on all persons who have ever lived in a SOEP household at a survey time (i.e. all respondents, but also children under 17 years of age and persons who have never given an interview). PPFAD is important for the distinction of the research units (persons), especially for longitudinal analyses. In addition, paneldata.org uses PPFAD to differentiate the study population.

\sphinxstylestrong{Time constant information of persons:}
\begin{itemize}
\item {} 
Never changing Person ID (adults, adolescents, children)

\item {} 
Original Household Number

\item {} 
Gender, year of birth, month of birth, year of death if applicable

\item {} 
Migrant Background

\item {} 
Sample Membership (psample)

\end{itemize}

\sphinxstylestrong{Time-varying information from people:}
\begin{itemize}
\item {} 
Current Household Number: If you move to another household, the household number changes (hhnrakt or \$hhnr)

\item {} 
Survey Status (\$netto, \$netold)

\item {} 
Population Membership (private household, institutional households)

\item {} 
Survey Region (East or West Germany)

\end{itemize}

The data set is explained in more detail in a documentation:

\sphinxhref{http://www.diw.de/documents/publikationen/73/diw\_01.c.581313.de/diw\_ssp0487.pdf}{Dokumentation PPFAD}:

\sphinxstylestrong{Create an exercise path with four subfolders:}

\begin{figure}[H]
\centering

\noindent\sphinxincludegraphics{{uebungspfade}.PNG}
\end{figure}

\sphinxstylestrong{Example:}
\begin{itemize}
\item {} 
H:/material/exercises/do

\item {} 
H:/material/exercises/output

\item {} 
H:/material/exercises/temp

\item {} 
H:/material/exercises/log

\end{itemize}

These are used to store your script, log files, datasets and temporary datasets. Open an empty do file and define your created paths with globals:

\fvset{hllines={, ,}}%
\begin{sphinxVerbatim}[commandchars=\\\{\},numbers=left,firstnumber=1,stepnumber=1]
\PYG{o}{*}\PYG{o}{*}\PYG{o}{*}\PYG{o}{*}\PYG{o}{*}\PYG{o}{*}\PYG{o}{*}\PYG{o}{*}\PYG{o}{*}\PYG{o}{*}\PYG{o}{*}\PYG{o}{*}\PYG{o}{*}\PYG{o}{*}\PYG{o}{*}\PYG{o}{*}\PYG{o}{*}\PYG{o}{*}\PYG{o}{*}\PYG{o}{*}\PYG{o}{*}\PYG{o}{*}\PYG{o}{*}\PYG{o}{*}\PYG{o}{*}\PYG{o}{*}\PYG{o}{*}\PYG{o}{*}\PYG{o}{*}\PYG{o}{*}\PYG{o}{*}\PYG{o}{*}\PYG{o}{*}\PYG{o}{*}\PYG{o}{*}\PYG{o}{*}\PYG{o}{*}\PYG{o}{*}\PYG{o}{*}\PYG{o}{*}\PYG{o}{*}\PYG{o}{*}\PYG{o}{*}\PYG{o}{*}\PYG{o}{*}\PYG{o}{*}\PYG{o}{*}
\PYG{o}{*} \PYG{n}{Set} \PYG{n}{relative} \PYG{n}{paths} \PYG{n}{to} \PYG{n}{the} \PYG{n}{working} \PYG{n}{directory}
\PYG{o}{*}\PYG{o}{*}\PYG{o}{*}\PYG{o}{*}\PYG{o}{*}\PYG{o}{*}\PYG{o}{*}\PYG{o}{*}\PYG{o}{*}\PYG{o}{*}\PYG{o}{*}\PYG{o}{*}\PYG{o}{*}\PYG{o}{*}\PYG{o}{*}\PYG{o}{*}\PYG{o}{*}\PYG{o}{*}\PYG{o}{*}\PYG{o}{*}\PYG{o}{*}\PYG{o}{*}\PYG{o}{*}\PYG{o}{*}\PYG{o}{*}\PYG{o}{*}\PYG{o}{*}\PYG{o}{*}\PYG{o}{*}\PYG{o}{*}\PYG{o}{*}\PYG{o}{*}\PYG{o}{*}\PYG{o}{*}\PYG{o}{*}\PYG{o}{*}\PYG{o}{*}\PYG{o}{*}\PYG{o}{*}\PYG{o}{*}\PYG{o}{*}\PYG{o}{*}\PYG{o}{*}\PYG{o}{*}\PYG{o}{*}\PYG{o}{*}\PYG{o}{*}
\PYG{k}{global} \PYG{n}{AVZ} 	\PYG{l+s+s2}{\PYGZdq{}}\PYG{l+s+s2}{H:}\PYG{l+s+s2}{\PYGZbs{}}\PYG{l+s+s2}{material}\PYG{l+s+s2}{\PYGZbs{}}\PYG{l+s+s2}{exercises}\PYG{l+s+s2}{\PYGZdq{}}
\PYG{k}{global} \PYG{n}{MY\PYGZus{}IN\PYGZus{}PATH} \PYG{l+s+s2}{\PYGZdq{}}\PYG{l+s+se}{\PYGZbs{}\PYGZbs{}}\PYG{l+s+s2}{hume}\PYG{l+s+se}{\PYGZbs{}r}\PYG{l+s+s2}{dc\PYGZhy{}prod}\PYG{l+s+s2}{\PYGZbs{}}\PYG{l+s+s2}{complete}\PYG{l+s+s2}{\PYGZbs{}}\PYG{l+s+s2}{soep\PYGZhy{}core}\PYG{l+s+s2}{\PYGZbs{}}\PYG{l+s+s2}{soep.v33.2}\PYG{l+s+s2}{\PYGZbs{}}\PYG{l+s+s2}{stata\PYGZus{}en}\PYG{l+s+se}{\PYGZbs{}\PYGZdq{}}
\PYG{k}{global} \PYG{n}{MY\PYGZus{}DO\PYGZus{}FILES} \PYG{l+s+s2}{\PYGZdq{}}\PYG{l+s+s2}{\PYGZdl{}AVZ}\PYG{l+s+s2}{\PYGZbs{}}\PYG{l+s+s2}{do}\PYG{l+s+se}{\PYGZbs{}\PYGZdq{}}
\PYG{k}{global} \PYG{n}{MY\PYGZus{}LOG\PYGZus{}OUT} \PYG{l+s+s2}{\PYGZdq{}}\PYG{l+s+s2}{\PYGZdl{}AVZ}\PYG{l+s+s2}{\PYGZbs{}}\PYG{l+s+s2}{log}\PYG{l+s+se}{\PYGZbs{}\PYGZdq{}}
\PYG{k}{global} \PYG{n}{MY\PYGZus{}OUT\PYGZus{}DATA} \PYG{l+s+s2}{\PYGZdq{}}\PYG{l+s+s2}{\PYGZdl{}AVZ}\PYG{l+s+s2}{\PYGZbs{}}\PYG{l+s+s2}{output}\PYG{l+s+se}{\PYGZbs{}\PYGZdq{}}
\PYG{k}{global} \PYG{n}{MY\PYGZus{}OUT\PYGZus{}TEMP} \PYG{l+s+s2}{\PYGZdq{}}\PYG{l+s+s2}{\PYGZdl{}AVZ}\PYG{l+s+se}{\PYGZbs{}t}\PYG{l+s+s2}{emp}\PYG{l+s+se}{\PYGZbs{}\PYGZdq{}}
\end{sphinxVerbatim}

The global „AVZ“ defines the main path. The main paths are subdivided using the globals “MY\_IN\_PATH”, “MY\_DO\_FILES”, “MY\_LOG\_OUT”, “MY\_OUT\_DATA”, “MY\_OUT\_TEMP”. The global “MY\_IN\_PATH” contains the path to your ordered data.

\sphinxstylestrong{Based on the data in PPFAD, answer the following questions:}

\sphinxstylestrong{1. Look at the two people with the person ID (variable persnr) 2102 and 19202}

\sphinxstylestrong{a) What gender are they? When were they born and possibly died?}

Open the PPFAD dataset. Search the data set for variables that describe gender, year of birth and year of death. Display the information of the variables for persons 2102 and 19202.

\fvset{hllines={, ,}}%
\begin{sphinxVerbatim}[commandchars=\\\{\},numbers=left,firstnumber=1,stepnumber=1]
use \PYGZdq{}\PYGZdl{}\PYGZob{}MY\PYGZus{}IN\PYGZus{}PATH\PYGZcb{}ppfad.dta\PYGZdq{}, clear

* a) What gender are they? When were they born and eventually died?
list persnr sex gebjahr todjahr if persnr == 2102 \textbar{} persnr == 19202
\end{sphinxVerbatim}

\begin{figure}[H]
\centering

\noindent\sphinxincludegraphics{{aufgabe_1.a}.PNG}
\end{figure}

\sphinxstylestrong{b) Were these people and their parents born in Germany?}

In the data set, search for a variable that describes the migration background.
Display the information of the variable for persons 2102 and 19202.

\fvset{hllines={, ,}}%
\begin{sphinxVerbatim}[commandchars=\\\{\},numbers=left,firstnumber=1,stepnumber=1]
* b) Were these people and their parents born in Germany?
list persnr migback if persnr == 2102 \textbar{} persnr == 19202
\end{sphinxVerbatim}

\begin{figure}[H]
\centering

\noindent\sphinxincludegraphics{{aufgabe_1.b}.PNG}
\end{figure}

\sphinxstylestrong{c) If they have immigrated: In which year and from which country?}

Search the data set for a variable that describes the country of birth and the year of moving to Germany. Display the information of the variables for persons 2102 and 19202.

\fvset{hllines={, ,}}%
\begin{sphinxVerbatim}[commandchars=\\\{\},numbers=left,firstnumber=1,stepnumber=1]
*c) If they have immigrated: In which year and from which country?
list persnr immiyear corigin if persnr  == 2102 \textbar{} persnr == 19202
\end{sphinxVerbatim}

\begin{figure}[H]
\centering

\noindent\sphinxincludegraphics{{aufgabe_1.c}.PNG}
\end{figure}

\sphinxstylestrong{d) Are these people from East or West Germany?}

Search the data set for a variable that describes east-west affiliation.
Display the information of the variables for persons 2102 and 19202.

\fvset{hllines={, ,}}%
\begin{sphinxVerbatim}[commandchars=\\\{\},numbers=left,firstnumber=1,stepnumber=1]
*d) Are these people from East or West Germany?
list persnr loc1989 psample if persnr  == 2102 \textbar{} persnr == 19202
\end{sphinxVerbatim}

\begin{figure}[H]
\centering

\noindent\sphinxincludegraphics{{aufgabe_1.d}.PNG}
\end{figure}

\sphinxstylestrong{e) From which sources does the information on the migration background and the year of death come?}

Search the data set for info variables that show you sources of information for the year of death and the migration background. Display the information of the variables for persons 2102 and 19202.

\fvset{hllines={, ,}}%
\begin{sphinxVerbatim}[commandchars=\\\{\},numbers=left,firstnumber=1,stepnumber=1]
*e) From which sources does the information on the migration background and the year of death come?
list miginfo todinfo if persnr  == 2102 \textbar{} persnr == 19202
\end{sphinxVerbatim}

\begin{figure}[H]
\centering

\noindent\sphinxincludegraphics{{aufgabe_1.e}.PNG}
\end{figure}

\sphinxstylestrong{2. How many people lived in a realised private household in 2016 and answered the individual questionnaire?}

Remember that the wave-specific survey year in SOEP is abbreviated with letters. SOEP started in 1984 (wave a) and was in a survey wave “bg” in 2016. For more information on this topic, please refer to the DTC subchapter {\hyperref[\detokenize{Principles of Data Structure/index:label}]{\sphinxcrossref{\DUrole{std,std-ref}{Labeling SOEP-Core}}}}.

If you are interested in the 2016 survey year, the wave name indicates that you should be interested in variables with the abbreviation “bg”.
Search the data set for variables with the abbreviation “bg” that describe the population. Display the characteristics of the population variables:

\fvset{hllines={, ,}}%
\begin{sphinxVerbatim}[commandchars=\\\{\},numbers=left,firstnumber=1,stepnumber=1]
********************************************************************************
*** Exercise 2) ***
* How many people lived in a realised private household in 2016 and answered the 
* personal questionnaire?

********************************************************************************

* informationen from:
* 2016 \PYGZhy{}\PYGZgt{} Wave bg
* private household \PYGZhy{}\PYGZgt{} bgpop
* Individual questionnaire \PYGZhy{}\PYGZgt{} bgnetto

tab bgpop
\end{sphinxVerbatim}

\begin{figure}[H]
\centering

\noindent\sphinxincludegraphics{{aufgabe_2_1}.PNG}
\end{figure}

Values 1 and 2 are relevant to answer the question because they describe realized households. Search the data set for variables with the abbreviation “bg” that describe the survey status. Display the characteristics of the survey status:

\fvset{hllines={, ,}}%
\begin{sphinxVerbatim}[commandchars=\\\{\},numbers=left,firstnumber=1,stepnumber=1]
\PYG{n}{tab} \PYG{n}{bgnetto}
\end{sphinxVerbatim}

\begin{figure}[H]
\centering

\noindent\sphinxincludegraphics{{aufgabe_2_2}.PNG}
\end{figure}

Respondents with survey status between 10 and 15 or survey status 19 completed the individual questionnaire. Cross-tab the variables bgpop and bgnetto with an appropriate restricting condition to answer the question.

\fvset{hllines={, ,}}%
\begin{sphinxVerbatim}[commandchars=\\\{\},numbers=left,firstnumber=1,stepnumber=1]
\PYG{n}{tab} \PYG{n}{bgnetto} \PYG{n}{bgpop} \PYG{k}{if} \PYG{p}{(}\PYG{p}{(}\PYG{n}{bgnetto} \PYG{o}{\PYGZgt{}}\PYG{o}{=} \PYG{l+m+mi}{10} \PYG{o}{\PYGZam{}} \PYG{n}{bgnetto} \PYG{o}{\PYGZlt{}}\PYG{o}{=} \PYG{l+m+mi}{15}\PYG{p}{)} \PYG{o}{\textbar{}} \PYG{n}{bgnetto}\PYG{o}{==}\PYG{l+m+mi}{19}\PYG{p}{)} \PYG{o}{\PYGZam{}} \PYG{p}{(}\PYG{n}{bgpop}\PYG{o}{==}\PYG{l+m+mi}{1} \PYG{o}{\textbar{}} \PYG{n}{bgpop}\PYG{o}{==}\PYG{l+m+mi}{2}\PYG{p}{)}
\end{sphinxVerbatim}

\begin{figure}[H]
\centering

\noindent\sphinxincludegraphics{{aufgabe_2_3}.PNG}
\end{figure}

\sphinxstylestrong{3. PPFAD allows you to see which populations can be viewed from a longitudinal perspective:}

\sphinxstylestrong{a) How many people who answered the individual questionnaire in 2000 also took part in the survey in 2014?}

Remember that the wave-specific survey year in SOEP is abbreviated with letters. SOEP started in 1984 (wave a) and was in a survey wave “bg” in 2016. For more information on the subject, see the subchapter {\hyperref[\detokenize{Principles of Data Structure/index:label}]{\sphinxcrossref{\DUrole{std,std-ref}{Labeling SOEP-Core}}}}.
The wave name shows that you are interested in the survey years 2000 and 2014. The survey years include the wave names “q”(2000) and “be”(2014). Search the data set for variables with the abbreviations “q” and “be” that describe the survey status. Display the characteristics of the survey status under the condition that the individual questionnaire has been answered:

\fvset{hllines={, ,}}%
\begin{sphinxVerbatim}[commandchars=\\\{\},numbers=left,firstnumber=1,stepnumber=1]
* a)How many people who answered the personal questionnaire in 2000 also took 
*   part in the survey in 2014?

* informationen from:
*	2000 \PYGZhy{}\PYGZgt{} wave q
*  	2014 \PYGZhy{}\PYGZgt{} wave be     
* 	Individual questionnaire \PYGZhy{}\PYGZgt{} \PYGZdl{}netto

tab qnetto benetto  if qnetto\PYGZgt{}=10 \PYGZam{} qnetto\PYGZlt{}=19 \PYGZam{} benetto\PYGZgt{}=10 \PYGZam{} benetto\PYGZlt{}=19
*or:
//fre qnetto benetto  if qnetto\PYGZgt{}=10 \PYGZam{} qnetto\PYGZlt{}=19 \PYGZam{} benetto\PYGZgt{}=10 \PYGZam{} benetto\PYGZlt{}=19
\end{sphinxVerbatim}

\begin{figure}[H]
\centering

\noindent\sphinxincludegraphics{{aufgabe_3_a}.PNG}
\end{figure}

A total of 7639 respondents completed the individual questionnaire in 2000 and 2014.

\sphinxstylestrong{b) How many people answered the individual questionnaire every year from 2000 to 2014?}

The survey years include the wave designations from “q”(2000) to “be”(2014).
View the relevant survey status codes to answer the question. Please consider all persons who have answered the individual questionnaire:

\fvset{hllines={, ,}}%
\begin{sphinxVerbatim}[commandchars=\\\{\},numbers=left,firstnumber=1,stepnumber=1]
* b) How many people answered the individual questionnaire every year from 2000 
*    to 2014?

/* to see all the codes */
lab list bgnetto
\end{sphinxVerbatim}

\begin{figure}[H]
\centering

\noindent\sphinxincludegraphics{{aufgabe_3_b}.PNG}
\end{figure}

Define a variable list that shows all survey statuses (\$netto) of the 15 survey waves considered in total.

\fvset{hllines={, ,}}%
\begin{sphinxVerbatim}[commandchars=\\\{\},numbers=left,firstnumber=1,stepnumber=1]
\PYG{n}{local} \PYG{n}{v} \PYG{l+s+s2}{\PYGZdq{}}\PYG{l+s+s2}{netto}\PYG{l+s+s2}{\PYGZdq{}}
\PYG{n}{local} \PYG{n}{vlist} \PYG{l+s+s2}{\PYGZdq{}}\PYG{l+s+s2}{q{}`v}\PYG{l+s+s2}{\PYGZsq{}}\PYG{l+s+s2}{ r{}`v}\PYG{l+s+s2}{\PYGZsq{}}\PYG{l+s+s2}{ s{}`v}\PYG{l+s+s2}{\PYGZsq{}}\PYG{l+s+s2}{ t{}`v}\PYG{l+s+s2}{\PYGZsq{}}\PYG{l+s+s2}{ u{}`v}\PYG{l+s+s2}{\PYGZsq{}}\PYG{l+s+s2}{ v{}`v}\PYG{l+s+s2}{\PYGZsq{}}\PYG{l+s+s2}{ w{}`v}\PYG{l+s+s2}{\PYGZsq{}}\PYG{l+s+s2}{ x{}`v}\PYG{l+s+s2}{\PYGZsq{}}\PYG{l+s+s2}{ y{}`v}\PYG{l+s+s2}{\PYGZsq{}}\PYG{l+s+s2}{ z{}`v}\PYG{l+s+s2}{\PYGZsq{}}\PYG{l+s+s2}{ ba{}`v}\PYG{l+s+s2}{\PYGZsq{}}\PYG{l+s+s2}{ bb{}`v}\PYG{l+s+s2}{\PYGZsq{}}\PYG{l+s+s2}{ bc{}`v}\PYG{l+s+s2}{\PYGZsq{}}\PYG{l+s+s2}{ bd{}`v}\PYG{l+s+s2}{\PYGZsq{}}\PYG{l+s+s2}{ be{}`v}\PYG{l+s+s2}{\PYGZsq{}}\PYG{l+s+s2}{\PYGZdq{}}  
\PYG{o}{/}\PYG{o}{*} \PYG{o}{\PYGZhy{}}\PYG{o}{\PYGZhy{}}\PYG{o}{\PYGZgt{}} \PYG{l+m+mi}{15} \PYG{n}{waves} \PYG{o}{*}\PYG{o}{/}
\end{sphinxVerbatim}

Generate a variable that shows the number of waves of completed person interviews. Note that the values 10,12,13,14,15,16,18,19 of the \$netto variable mean realized interviews.

\fvset{hllines={, ,}}%
\begin{sphinxVerbatim}[commandchars=\\\{\},numbers=left,firstnumber=1,stepnumber=1]
capture drop h1
egen h1 = anycount({}`vlist\PYGZsq{}), values(10 12 13 14 15 16 18 19)
\end{sphinxVerbatim}

Display a table with its newly generated variable.

\fvset{hllines={, ,}}%
\begin{sphinxVerbatim}[commandchars=\\\{\},numbers=left,firstnumber=1,stepnumber=1]
\PYG{n}{tab} \PYG{n}{h1} \PYG{k}{if} \PYG{n}{h1} \PYG{o}{==} \PYG{l+m+mi}{15}
\end{sphinxVerbatim}

\begin{figure}[H]
\centering

\noindent\sphinxincludegraphics{{aufgabe_3_b4}.PNG}
\end{figure}

A total of 6665 people completed the individual questionnaire every year over the period 2000-2014.

\sphinxstylestrong{c) How many people who turned 15 in 2011 and lived as children in a survey household took part in the survey in 2016?}

The survey year 2011 is represented by the wave “bb” and the survey year 2016 is represented by the wave “bg”. To answer the question, a variable must be generated that identifies people who were 15 years old in 2011. The age of the respondent can be determined with the year of birth and you can limit children using the net code. Generate a variable with people who turned 15 in 2011 and lived in a survey household as a child.

\fvset{hllines={, ,}}%
\begin{sphinxVerbatim}[commandchars=\\\{\},numbers=left,firstnumber=1,stepnumber=1]
* c) How many people who turned 15 in 2011 and lived as children in a survey 
*    household took part in the survey in 2016?

*   informationen from:
*  	2011 \PYGZhy{}\PYGZgt{} wave bb
*	Age  \PYGZhy{}\PYGZgt{} 15  
*	Child \PYGZhy{}\PYGZgt{} bbnetto   
*	2016 \PYGZhy{}\PYGZgt{} wave bg
* 	Individual Questionnaire \PYGZhy{}\PYGZgt{} bgnetto

/* People who turned 15 in 2011 and lived in a survey household as a child...*/
capture drop a15kind
gen a15kind = 1 if 2011\PYGZhy{}gebjahr == 15 \PYGZam{} bbnetto \PYGZgt{}= 20 \PYGZam{} bbnetto \PYGZlt{} 30

\end{sphinxVerbatim}

In order to identify all persons who were 15 years old in 2011, lived in a survey household as a child and completed the individual questionnaire in 2016, you must use the net codes again. Create a table from the net code of 2016 to narrow down the cases appropriately.

\fvset{hllines={, ,}}%
\begin{sphinxVerbatim}[commandchars=\\\{\},numbers=left,firstnumber=1,stepnumber=1]
\PYG{o}{/}\PYG{o}{/} \PYG{n}{fre} \PYG{n}{bgnetto} \PYG{k}{if} \PYG{n}{a15kind} \PYG{o}{==} \PYG{l+m+mi}{1} \PYG{o}{\PYGZam{}} \PYG{n}{bgnetto} \PYG{o}{\PYGZgt{}}\PYG{o}{=} \PYG{l+m+mi}{10} \PYG{o}{\PYGZam{}} \PYG{n}{bgnetto} \PYG{o}{\PYGZlt{}} \PYG{l+m+mi}{20}
\PYG{o}{*} \PYG{n}{oder}\PYG{p}{:}
\PYG{n}{tab} \PYG{n}{bgnetto} \PYG{k}{if} \PYG{n}{a15kind} \PYG{o}{==} \PYG{l+m+mi}{1} \PYG{o}{\PYGZam{}} \PYG{n}{bgnetto} \PYG{o}{\PYGZgt{}}\PYG{o}{=} \PYG{l+m+mi}{10} \PYG{o}{\PYGZam{}} \PYG{n}{bgnetto} \PYG{o}{\PYGZlt{}} \PYG{l+m+mi}{20}

\end{sphinxVerbatim}

\begin{figure}[H]
\centering

\noindent\sphinxincludegraphics{{aufgabe_3_c2}.PNG}
\end{figure}

In 2016, a total of 309 people who were 15 years old and were part of a survey household as a child in 2011, completed a individual interview.

\sphinxstylestrong{d) The person with persnr=588010 was born in 1984 in a panel household and was still part of the sample in 2009. The person has changed households twice during this time. In which years?}

To identify how often and when a person has changed the household, you must display all available household numbers in ppfad for person 588010.

\fvset{hllines={, ,}}%
\begin{sphinxVerbatim}[commandchars=\\\{\},numbers=left,firstnumber=1,stepnumber=1]
* still part of the sample in 2009. The person has changed households twice during
* this time. In which years?

* Information from:
* \PYGZhy{}\PYGZgt{} household numbers

list *hhnr if persnr == 588010
/* \PYGZhy{}\PYGZgt{} changed household 
 in year d (1987)
 in year y (2008)
 no participation since bb (2011) 
*/
\end{sphinxVerbatim}

\begin{figure}[H]
\centering

\noindent\sphinxincludegraphics{{aufgabe_3_d}.PNG}
\end{figure}

The person 588010 has participated in the survey since the wave “b” (1985) in household 58807. From wave “d” (1987) to wave “x” (2007) the person was in household 73407, from wave “y” (2008) the person was in household 132608.


\section{Generating a cross-section Data Set}
\label{\detokenize{Working with SOEP Data/index:generating-a-cross-section-data-set}}\label{\detokenize{Working with SOEP Data/index:cross-data}}
This example involves generating a data set to analyze health satisfaction determinants in 2008, and you can either use the Paneldata.org syntax generator or write your own syntax file to perform this task. You can search for the variable names in Paneldata.org (or use the variables below directly).

\sphinxstylestrong{1. Generate a cross-section dataset for the year 2008, which should contain all persons with the following characteristics:}
\begin{itemize}
\item {} 
Respondents in 2008  \href{https://paneldata.org/soep-core/data/ppfad/ynetto}{\textbf{"ynetto"}}

\item {} 
Lives 2008 in private household  \href{https://paneldata.org/soep-core/data/ppfad/ypop}{\textbf{"ypop"}}

\end{itemize}

The data set should contain the following variables of interest.
\begin{itemize}
\item {} 
Satisfaction with health  \href{https://paneldata.org/soep-core/data/yp/yp0101}{\textbf{"yp0101"}}

\item {} 
Smoking currently yes/no  \href{https://paneldata.org/soep-core/data/yp/yp10601}{\textbf{"yp10601"}}

\item {} 
current employment status  \href{https://paneldata.org/soep-core/data/ypgen/emplst08}{\textbf{"emplst08"}}

\item {} 
monthly household net income  \href{https://paneldata.org/soep-core/data/yhgen/hinc08}{\textbf{"hinc08"}}

\end{itemize}

In addition, the data set should contain the following additional information for a 2008 cross-sectional analysis (these variables are automatically generated by paneldata.org):
\begin{itemize}
\item {} 
Current cross-section weighting factor  \href{https://paneldata.org/soep-core/data/phrf/yphrf}{\textbf{"yphrf"}}

\item {} 
Personal number  \href{https://paneldata.org/soep-core/data/ppfad/persnr}{\textbf{"persnr"}}

\item {} 
Original household number  \href{https://paneldata.org/soep-core/data/ppfad/hhnr}{\textbf{"hhnr"}}

\item {} 
Current household number  \href{https://paneldata.org/soep-core/data/ppfad/yhhnr}{\textbf{"yhhnr"}}

\item {} 
Sample affiliation  \href{https://paneldata.org/soep-core/data/ppfad/psample}{\textbf{"psample"}}

\item {} 
Gender  \href{https://paneldata.org/soep-core/data/ppfad/sex}{\textbf{"sex"}}

\item {} 
Year of birth  \href{https://paneldata.org/soep-core/data/ppfad/gebjahr}{\textbf{"gebjahr"}}

\end{itemize}

\sphinxstylestrong{Create an exercise path with four subfolders:}

\begin{figure}[H]
\centering

\noindent\sphinxincludegraphics{{uebungspfade}.PNG}
\end{figure}

\sphinxstylestrong{Example:}
\begin{itemize}
\item {} 
H:/material/exercises/do

\item {} 
H:/material/exercises/output

\item {} 
H:/material/exercises/temp

\item {} 
H:/material/exercises/log

\end{itemize}

These are used to store commands, log files, data sets and temporary data sets.
Open an empty do file and define your created paths with globals:

\fvset{hllines={, ,}}%
\begin{sphinxVerbatim}[commandchars=\\\{\},numbers=left,firstnumber=1,stepnumber=1]
\PYG{o}{*}\PYG{o}{*}\PYG{o}{*}\PYG{o}{*}\PYG{o}{*}\PYG{o}{*}\PYG{o}{*}\PYG{o}{*}\PYG{o}{*}\PYG{o}{*}\PYG{o}{*}\PYG{o}{*}\PYG{o}{*}\PYG{o}{*}\PYG{o}{*}\PYG{o}{*}\PYG{o}{*}\PYG{o}{*}\PYG{o}{*}\PYG{o}{*}\PYG{o}{*}\PYG{o}{*}\PYG{o}{*}\PYG{o}{*}\PYG{o}{*}\PYG{o}{*}\PYG{o}{*}\PYG{o}{*}\PYG{o}{*}\PYG{o}{*}\PYG{o}{*}\PYG{o}{*}\PYG{o}{*}\PYG{o}{*}\PYG{o}{*}\PYG{o}{*}\PYG{o}{*}\PYG{o}{*}\PYG{o}{*}\PYG{o}{*}\PYG{o}{*}\PYG{o}{*}\PYG{o}{*}\PYG{o}{*}\PYG{o}{*}\PYG{o}{*}\PYG{o}{*}
\PYG{o}{*} \PYG{n}{Set} \PYG{n}{relative} \PYG{n}{paths} \PYG{n}{to} \PYG{n}{the} \PYG{n}{working} \PYG{n}{directory}
\PYG{o}{*}\PYG{o}{*}\PYG{o}{*}\PYG{o}{*}\PYG{o}{*}\PYG{o}{*}\PYG{o}{*}\PYG{o}{*}\PYG{o}{*}\PYG{o}{*}\PYG{o}{*}\PYG{o}{*}\PYG{o}{*}\PYG{o}{*}\PYG{o}{*}\PYG{o}{*}\PYG{o}{*}\PYG{o}{*}\PYG{o}{*}\PYG{o}{*}\PYG{o}{*}\PYG{o}{*}\PYG{o}{*}\PYG{o}{*}\PYG{o}{*}\PYG{o}{*}\PYG{o}{*}\PYG{o}{*}\PYG{o}{*}\PYG{o}{*}\PYG{o}{*}\PYG{o}{*}\PYG{o}{*}\PYG{o}{*}\PYG{o}{*}\PYG{o}{*}\PYG{o}{*}\PYG{o}{*}\PYG{o}{*}\PYG{o}{*}\PYG{o}{*}\PYG{o}{*}\PYG{o}{*}\PYG{o}{*}\PYG{o}{*}\PYG{o}{*}\PYG{o}{*}
\PYG{k}{global} \PYG{n}{AVZ} 	\PYG{l+s+s2}{\PYGZdq{}}\PYG{l+s+s2}{H:}\PYG{l+s+s2}{\PYGZbs{}}\PYG{l+s+s2}{material}\PYG{l+s+s2}{\PYGZbs{}}\PYG{l+s+s2}{exercises}\PYG{l+s+s2}{\PYGZdq{}}
\PYG{k}{global} \PYG{n}{MY\PYGZus{}IN\PYGZus{}PATH} \PYG{l+s+s2}{\PYGZdq{}}\PYG{l+s+se}{\PYGZbs{}\PYGZbs{}}\PYG{l+s+s2}{hume}\PYG{l+s+se}{\PYGZbs{}r}\PYG{l+s+s2}{dc\PYGZhy{}prod}\PYG{l+s+s2}{\PYGZbs{}}\PYG{l+s+s2}{complete}\PYG{l+s+s2}{\PYGZbs{}}\PYG{l+s+s2}{soep\PYGZhy{}core}\PYG{l+s+s2}{\PYGZbs{}}\PYG{l+s+s2}{soep.v33.2}\PYG{l+s+s2}{\PYGZbs{}}\PYG{l+s+s2}{stata\PYGZus{}en}\PYG{l+s+se}{\PYGZbs{}\PYGZdq{}}
\PYG{k}{global} \PYG{n}{MY\PYGZus{}DO\PYGZus{}FILES} \PYG{l+s+s2}{\PYGZdq{}}\PYG{l+s+s2}{\PYGZdl{}AVZ}\PYG{l+s+s2}{\PYGZbs{}}\PYG{l+s+s2}{do}\PYG{l+s+se}{\PYGZbs{}\PYGZdq{}}
\PYG{k}{global} \PYG{n}{MY\PYGZus{}LOG\PYGZus{}OUT} \PYG{l+s+s2}{\PYGZdq{}}\PYG{l+s+s2}{\PYGZdl{}AVZ}\PYG{l+s+s2}{\PYGZbs{}}\PYG{l+s+s2}{log}\PYG{l+s+se}{\PYGZbs{}\PYGZdq{}}
\PYG{k}{global} \PYG{n}{MY\PYGZus{}OUT\PYGZus{}DATA} \PYG{l+s+s2}{\PYGZdq{}}\PYG{l+s+s2}{\PYGZdl{}AVZ}\PYG{l+s+s2}{\PYGZbs{}}\PYG{l+s+s2}{output}\PYG{l+s+se}{\PYGZbs{}\PYGZdq{}}
\PYG{k}{global} \PYG{n}{MY\PYGZus{}OUT\PYGZus{}TEMP} \PYG{l+s+s2}{\PYGZdq{}}\PYG{l+s+s2}{\PYGZdl{}AVZ}\PYG{l+s+se}{\PYGZbs{}t}\PYG{l+s+s2}{emp}\PYG{l+s+se}{\PYGZbs{}\PYGZdq{}}
\end{sphinxVerbatim}

The global „AVZ“ defines the main path. The main paths are subdivided using the globals “MY\_IN\_PATH”, “MY\_DO\_FILES”, “MY\_LOG\_OUT”, “MY\_OUT\_DATA”, “MY\_OUT\_TEMP”. The global “MY\_IN\_PATH” contains the path to your ordered data.

Use ppfad as the source file together with the required variables. Keep all cases with completed interviews. In addition, your data set should only contain respondents who can make a statement on the content of the question. For example, you can use the net code to identify and remove children from your data set.

\fvset{hllines={, ,}}%
\begin{sphinxVerbatim}[commandchars=\\\{\},numbers=left,firstnumber=1,stepnumber=1]
\PYG{o}{*} \PYG{o}{*} \PYG{o}{*} \PYG{n}{PFAD} \PYG{o}{*} \PYG{o}{*} \PYG{o}{*}

\PYG{n}{use} \PYG{n}{hhnr} \PYG{n}{persnr} \PYG{n}{sex} \PYG{n}{gebjahr} \PYG{n}{psample} \PYG{n}{yhhnr} \PYG{n}{ynetto} \PYG{n}{ypop} \PYG{n}{using} \PYG{l+s+s2}{\PYGZdq{}}\PYG{l+s+s2}{\PYGZdl{}}\PYG{l+s+si}{\PYGZob{}MY\PYGZus{}IN\PYGZus{}PATH\PYGZcb{}}\PYG{l+s+s2}{ppfad.dta}\PYG{l+s+s2}{\PYGZdq{}}


\PYG{o}{*} \PYG{o}{*} \PYG{o}{*} \PYG{n}{BALANCED} \PYG{n}{VS} \PYG{n}{UNBALANCED} \PYG{o}{*} \PYG{o}{*} \PYG{o}{*}

\PYG{n}{keep} \PYG{k}{if} \PYG{p}{(} \PYG{p}{(}\PYG{n}{ynetto} \PYG{o}{\PYGZgt{}}\PYG{o}{=} \PYG{l+m+mi}{10} \PYG{o}{\PYGZam{}} \PYG{n}{ynetto} \PYG{o}{\PYGZlt{}} \PYG{l+m+mi}{20}\PYG{p}{)} \PYG{p}{)}


\PYG{o}{*} \PYG{o}{*} \PYG{o}{*} \PYG{n}{PRIATVE} \PYG{n}{VS} \PYG{n}{ALL} \PYG{n}{HOUSEHOLDS} \PYG{o}{*} \PYG{o}{*} \PYG{o}{*}

\PYG{n}{keep} \PYG{k}{if} \PYG{p}{(} \PYG{p}{(}\PYG{n}{ypop} \PYG{o}{==} \PYG{l+m+mi}{1} \PYG{o}{\textbar{}} \PYG{n}{ypop} \PYG{o}{==} \PYG{l+m+mi}{2}\PYG{p}{)} \PYG{p}{)}


\PYG{o}{*} \PYG{o}{*} \PYG{o}{*} \PYG{n}{SORT} \PYG{n}{PFAD} \PYG{o}{*} \PYG{o}{*} \PYG{o}{*}

\PYG{n}{sort} \PYG{n}{persnr}
\PYG{n}{save} \PYG{l+s+s2}{\PYGZdq{}}\PYG{l+s+s2}{\PYGZdl{}}\PYG{l+s+si}{\PYGZob{}MY\PYGZus{}OUT\PYGZus{}TEMP\PYGZcb{}}\PYG{l+s+s2}{ppfad.dta}\PYG{l+s+s2}{\PYGZdq{}}\PYG{p}{,} \PYG{n}{replace}
\PYG{n}{clear}
\end{sphinxVerbatim}

Save the modified data record temporarily.
Now link your data set with the weights of the SOEP and save your data set as a master file.

\fvset{hllines={, ,}}%
\begin{sphinxVerbatim}[commandchars=\\\{\},numbers=left,firstnumber=1,stepnumber=1]
\PYG{o}{*} \PYG{o}{*} \PYG{o}{*} \PYG{n}{HRF} \PYG{o}{*} \PYG{o}{*} \PYG{o}{*}

\PYG{n}{use} \PYG{l+s+s2}{\PYGZdq{}}\PYG{l+s+s2}{\PYGZdl{}}\PYG{l+s+si}{\PYGZob{}MY\PYGZus{}IN\PYGZus{}PATH\PYGZcb{}}\PYG{l+s+s2}{phrf.dta}\PYG{l+s+s2}{\PYGZdq{}}
\PYG{n}{sort} \PYG{n}{persnr}
\PYG{n}{save} \PYG{l+s+s2}{\PYGZdq{}}\PYG{l+s+s2}{\PYGZdl{}}\PYG{l+s+si}{\PYGZob{}MY\PYGZus{}OUT\PYGZus{}TEMP\PYGZcb{}}\PYG{l+s+s2}{hrf.dta}\PYG{l+s+s2}{\PYGZdq{}}\PYG{p}{,} \PYG{n}{replace}
\PYG{n}{clear}


\PYG{o}{*} \PYG{o}{*} \PYG{o}{*} \PYG{n}{CREATE} \PYG{n}{MASTER} \PYG{o}{*} \PYG{o}{*} \PYG{o}{*}

\PYG{n}{use} \PYG{l+s+s2}{\PYGZdq{}}\PYG{l+s+s2}{\PYGZdl{}}\PYG{l+s+si}{\PYGZob{}MY\PYGZus{}OUT\PYGZus{}TEMP\PYGZcb{}}\PYG{l+s+s2}{ppfad.dta}\PYG{l+s+s2}{\PYGZdq{}}
\PYG{n}{merge} \PYG{l+m+mi}{1}\PYG{p}{:}\PYG{l+m+mi}{1} \PYG{n}{persnr} \PYG{n}{using} \PYG{l+s+s2}{\PYGZdq{}}\PYG{l+s+s2}{\PYGZdl{}}\PYG{l+s+si}{\PYGZob{}MY\PYGZus{}OUT\PYGZus{}TEMP\PYGZcb{}}\PYG{l+s+s2}{hrf.dta}\PYG{l+s+s2}{\PYGZdq{}}
\PYG{n}{drop} \PYG{k}{if} \PYG{n}{\PYGZus{}merge} \PYG{o}{==} \PYG{l+m+mi}{2}
\PYG{n}{drop} \PYG{n}{\PYGZus{}merge}
\PYG{n}{sort} \PYG{n}{persnr}
\PYG{n}{save} \PYG{l+s+s2}{\PYGZdq{}}\PYG{l+s+s2}{\PYGZdl{}}\PYG{l+s+si}{\PYGZob{}MY\PYGZus{}OUT\PYGZus{}TEMP\PYGZcb{}}\PYG{l+s+s2}{master.dta}\PYG{l+s+s2}{\PYGZdq{}}\PYG{p}{,} \PYG{n}{replace}
\PYG{n}{clear}
\end{sphinxVerbatim}

Now prepare the content variables. Search for the content variables you are looking for from the various data records and temporarily save the created data records.

\fvset{hllines={, ,}}%
\begin{sphinxVerbatim}[commandchars=\\\{\},numbers=left,firstnumber=1,stepnumber=1]
\PYG{o}{*} \PYG{o}{*} \PYG{o}{*} \PYG{n}{READ} \PYG{n}{DATA} \PYG{o}{*} \PYG{o}{*} \PYG{o}{*}

\PYG{n}{use} \PYG{n}{hinc08} \PYG{n}{yhhnr} \PYG{n}{using} \PYG{l+s+s2}{\PYGZdq{}}\PYG{l+s+s2}{\PYGZdl{}}\PYG{l+s+si}{\PYGZob{}MY\PYGZus{}IN\PYGZus{}PATH\PYGZcb{}}\PYG{l+s+s2}{yhgen.dta}\PYG{l+s+s2}{\PYGZdq{}}
\PYG{n}{sort} \PYG{n}{yhhnr}
\PYG{n}{save} \PYG{l+s+s2}{\PYGZdq{}}\PYG{l+s+s2}{\PYGZdl{}}\PYG{l+s+si}{\PYGZob{}MY\PYGZus{}OUT\PYGZus{}TEMP\PYGZcb{}}\PYG{l+s+s2}{yhgen.dta}\PYG{l+s+s2}{\PYGZdq{}}\PYG{p}{,} \PYG{n}{replace}
\PYG{n}{clear}


\PYG{n}{use} \PYG{n}{yp10601} \PYG{n}{yhhnr} \PYG{n}{yp0101} \PYG{n}{persnr} \PYG{n}{using} \PYG{l+s+s2}{\PYGZdq{}}\PYG{l+s+s2}{\PYGZdl{}}\PYG{l+s+si}{\PYGZob{}MY\PYGZus{}IN\PYGZus{}PATH\PYGZcb{}}\PYG{l+s+s2}{yp.dta}\PYG{l+s+s2}{\PYGZdq{}}
\PYG{n}{sort} \PYG{n}{persnr}
\PYG{n}{save} \PYG{l+s+s2}{\PYGZdq{}}\PYG{l+s+s2}{\PYGZdl{}}\PYG{l+s+si}{\PYGZob{}MY\PYGZus{}OUT\PYGZus{}TEMP\PYGZcb{}}\PYG{l+s+s2}{yp.dta}\PYG{l+s+s2}{\PYGZdq{}}\PYG{p}{,} \PYG{n}{replace}
\PYG{n}{clear}


\PYG{n}{use} \PYG{n}{emplst08} \PYG{n}{yhhnr} \PYG{n}{persnr} \PYG{n}{using} \PYG{l+s+s2}{\PYGZdq{}}\PYG{l+s+s2}{\PYGZdl{}}\PYG{l+s+si}{\PYGZob{}MY\PYGZus{}IN\PYGZus{}PATH\PYGZcb{}}\PYG{l+s+s2}{ypgen.dta}\PYG{l+s+s2}{\PYGZdq{}}
\PYG{n}{sort} \PYG{n}{persnr}
\PYG{n}{save} \PYG{l+s+s2}{\PYGZdq{}}\PYG{l+s+s2}{\PYGZdl{}}\PYG{l+s+si}{\PYGZob{}MY\PYGZus{}OUT\PYGZus{}TEMP\PYGZcb{}}\PYG{l+s+s2}{ypgen.dta}\PYG{l+s+s2}{\PYGZdq{}}\PYG{p}{,} \PYG{n}{replace}
\PYG{n}{clear}
\end{sphinxVerbatim}

Link your created data sets to your masterfile and save your analysis data set.

\fvset{hllines={, ,}}%
\begin{sphinxVerbatim}[commandchars=\\\{\},numbers=left,firstnumber=1,stepnumber=1]
\PYG{o}{*} \PYG{o}{*} \PYG{o}{*} \PYG{n}{MERGE} \PYG{n}{DATA} \PYG{o}{*} \PYG{o}{*} \PYG{o}{*}

\PYG{n}{use}   \PYG{l+s+s2}{\PYGZdq{}}\PYG{l+s+s2}{\PYGZdl{}}\PYG{l+s+si}{\PYGZob{}MY\PYGZus{}OUT\PYGZus{}TEMP\PYGZcb{}}\PYG{l+s+s2}{master.dta}\PYG{l+s+s2}{\PYGZdq{}}

\PYG{n}{sort} \PYG{n}{yhhnr}
\PYG{n}{merge} \PYG{n}{yhhnr} \PYG{n}{using} \PYG{l+s+s2}{\PYGZdq{}}\PYG{l+s+s2}{\PYGZdl{}}\PYG{l+s+si}{\PYGZob{}MY\PYGZus{}OUT\PYGZus{}TEMP\PYGZcb{}}\PYG{l+s+s2}{yhgen.dta}\PYG{l+s+s2}{\PYGZdq{}}
\PYG{n}{drop} \PYG{k}{if} \PYG{n}{\PYGZus{}merge} \PYG{o}{==} \PYG{l+m+mi}{2}
\PYG{n}{drop} \PYG{n}{\PYGZus{}merge}

\PYG{n}{sort} \PYG{n}{persnr}
\PYG{n}{merge} \PYG{n}{persnr} \PYG{n}{using} \PYG{l+s+s2}{\PYGZdq{}}\PYG{l+s+s2}{\PYGZdl{}}\PYG{l+s+si}{\PYGZob{}MY\PYGZus{}OUT\PYGZus{}TEMP\PYGZcb{}}\PYG{l+s+s2}{yp.dta}\PYG{l+s+s2}{\PYGZdq{}}
\PYG{n}{drop} \PYG{k}{if} \PYG{n}{\PYGZus{}merge} \PYG{o}{==} \PYG{l+m+mi}{2}
\PYG{n}{drop} \PYG{n}{\PYGZus{}merge}

\PYG{n}{sort} \PYG{n}{persnr}
\PYG{n}{merge} \PYG{n}{persnr} \PYG{n}{using} \PYG{l+s+s2}{\PYGZdq{}}\PYG{l+s+s2}{\PYGZdl{}}\PYG{l+s+si}{\PYGZob{}MY\PYGZus{}OUT\PYGZus{}TEMP\PYGZcb{}}\PYG{l+s+s2}{ypgen.dta}\PYG{l+s+s2}{\PYGZdq{}}
\PYG{n}{drop} \PYG{k}{if} \PYG{n}{\PYGZus{}merge} \PYG{o}{==} \PYG{l+m+mi}{2}
\PYG{n}{drop} \PYG{n}{\PYGZus{}merge}


\PYG{o}{*} \PYG{o}{*} \PYG{o}{*} \PYG{n}{DONE} \PYG{o}{*} \PYG{o}{*} \PYG{o}{*}

\PYG{n}{save} \PYG{l+s+s2}{\PYGZdq{}}\PYG{l+s+s2}{\PYGZdl{}}\PYG{l+s+si}{\PYGZob{}MY\PYGZus{}OUT\PYGZus{}DATA\PYGZcb{}}\PYG{l+s+s2}{my\PYGZus{}dataset.dta}\PYG{l+s+s2}{\PYGZdq{}}\PYG{p}{,} \PYG{n}{replace}
\PYG{n}{desc}
\end{sphinxVerbatim}

You have successfully created a cross-sectional data set for the year 2008.

\sphinxstylestrong{2. Encode missing values into missing values in system failings (STATA)!}

In SOEP the missing codes of variables are described in detail with the values -1 to -8. To learn more about missing codes, see the chapter {\hyperref[\detokenize{Principles of Data Structure/index:missings}]{\sphinxcrossref{\DUrole{std,std-ref}{Missing Conventions}}}}. For content analyses it is not always necessary to differentiate missing codes. Therefore you should be able to convert missing codes:

\fvset{hllines={, ,}}%
\begin{sphinxVerbatim}[commandchars=\\\{\},numbers=left,firstnumber=1,stepnumber=1]
use \PYGZdq{}\PYGZdl{}MY\PYGZus{}OUT\PYGZus{}DATA\PYGZbs{}my\PYGZus{}dataset.dta\PYGZdq{}, clear


********************************************************************************
*** Exercise 2) ***
* Encode missing values into missing values in system missings (STATA)!
********************************************************************************

* mvdecode = Change missing values to numeric values and vice versa
	mvdecode \PYGZus{}all, mv(\PYGZhy{}1=. \PYGZbs{} \PYGZhy{}2=.t \PYGZbs{} \PYGZhy{}3=.x \PYGZbs{} \PYGZhy{}5=.y \PYGZbs{} \PYGZhy{}8=.z)
\end{sphinxVerbatim}

Open your analysis data set and summarize all missing codes.

\sphinxstylestrong{3. How does average health satisfaction differ}
\sphinxstylestrong{a) by sex}

Satisfaction was measured on a scale of 10. To compare the average satisfaction with health between women and men, you should display the mean value for both sexes.

\fvset{hllines={, ,}}%
\begin{sphinxVerbatim}[commandchars=\\\{\},numbers=left,firstnumber=1,stepnumber=1]
	\PYG{o}{*}\PYG{n}{unweighted}\PYG{o}{*}
	\PYG{n}{tabstat} \PYG{n}{yp0101}\PYG{p}{,} \PYG{n}{by}\PYG{p}{(}\PYG{n}{sex}\PYG{p}{)}
\end{sphinxVerbatim}

\begin{figure}[H]
\centering

\noindent\sphinxincludegraphics{{quer_06}.PNG}
\end{figure}

Since you have previously added the SOEP weighting factors to your analysis data set, you should use the weighting for a representative analysis.

\fvset{hllines={, ,}}%
\begin{sphinxVerbatim}[commandchars=\\\{\},numbers=left,firstnumber=1,stepnumber=1]
	\PYG{o}{*}\PYG{n}{weighted}\PYG{o}{*} 
	\PYG{n}{tabstat} \PYG{n}{yp0101} \PYG{p}{[}\PYG{n}{aw}\PYG{o}{=}\PYG{n}{yphrf}\PYG{p}{]}\PYG{p}{,} \PYG{n}{by}\PYG{p}{(}\PYG{n}{sex}\PYG{p}{)}		
\end{sphinxVerbatim}

\begin{figure}[H]
\centering

\noindent\sphinxincludegraphics{{quer_07}.PNG}
\end{figure}

\sphinxstylestrong{b) Employment status}

Now proceed in a similar way when comparing satisfaction with health and employment status. Compare the mean values again:

\fvset{hllines={, ,}}%
\begin{sphinxVerbatim}[commandchars=\\\{\},numbers=left,firstnumber=1,stepnumber=1]
\PYG{o}{*}\PYG{n}{b}\PYG{p}{)} \PYG{n}{by} \PYG{n}{job} \PYG{n}{status}\PYG{p}{:}
	\PYG{o}{*}\PYG{n}{unweighted}\PYG{o}{*}
	\PYG{n}{tabstat} \PYG{n}{yp0101}\PYG{p}{,} \PYG{n}{by}\PYG{p}{(}\PYG{n}{emplst08}\PYG{p}{)}
\end{sphinxVerbatim}

\begin{figure}[H]
\centering

\noindent\sphinxincludegraphics{{quer_08}.PNG}
\end{figure}

Since you have previously added the SOEP weighting factors to your analysis data set, you should use the weighting for a representative analysis.

\fvset{hllines={, ,}}%
\begin{sphinxVerbatim}[commandchars=\\\{\},numbers=left,firstnumber=1,stepnumber=1]
	\PYG{o}{*}\PYG{n}{weighted}\PYG{o}{*}
	\PYG{n}{tabstat} \PYG{n}{yp0101} \PYG{p}{[}\PYG{n}{aw}\PYG{o}{=}\PYG{n}{yphrf}\PYG{p}{]}\PYG{p}{,} \PYG{n}{by}\PYG{p}{(}\PYG{n}{emplst08}\PYG{p}{)}
\end{sphinxVerbatim}

\begin{figure}[H]
\centering

\noindent\sphinxincludegraphics{{quer_09}.PNG}
\end{figure}

\sphinxstylestrong{c) Age}

Since you do not have a variable that represents the age, you must generate a suitable age variable using the Birth year variable. The year of birth is metric and should be categorized for analysis. Define categories for your age variable and assign suitable labels.

\fvset{hllines={, ,}}%
\begin{sphinxVerbatim}[commandchars=\\\{\},numbers=left,firstnumber=1,stepnumber=1]
\PYG{o}{*}\PYG{n}{c}\PYG{p}{)} \PYG{n}{by} \PYG{n}{age} \PYG{o+ow}{in} \PYG{l+m+mi}{2008} \PYG{p}{(}\PYG{o}{\PYGZlt{}}\PYG{l+m+mi}{30}\PYG{p}{,} \PYG{l+m+mi}{30}\PYG{o}{\PYGZhy{}}\PYG{l+m+mi}{64}\PYG{p}{,} \PYG{l+m+mi}{65}\PYG{o}{+}\PYG{p}{)}
	
	\PYG{n}{gen} \PYG{n}{age}\PYG{o}{=}\PYG{l+m+mi}{2008}\PYG{o}{\PYGZhy{}}\PYG{n}{gebjahr}
	\PYG{n}{gen} \PYG{n}{age\PYGZus{}3}\PYG{o}{=}\PYG{n}{age}
	\PYG{n}{recode} \PYG{n}{age\PYGZus{}3} \PYG{p}{(}\PYG{l+m+mi}{17}\PYG{o}{/}\PYG{l+m+mi}{29}\PYG{o}{=}\PYG{l+m+mi}{1}\PYG{p}{)} \PYG{p}{(}\PYG{l+m+mi}{30}\PYG{o}{/}\PYG{l+m+mi}{64}\PYG{o}{=}\PYG{l+m+mi}{2}\PYG{p}{)} \PYG{p}{(}\PYG{l+m+mi}{65}\PYG{o}{/}\PYG{l+m+mi}{120}\PYG{o}{=}\PYG{l+m+mi}{3}\PYG{p}{)}
	\PYG{n}{label} \PYG{n}{define} \PYG{n}{age\PYGZus{}3} \PYG{l+m+mi}{1} \PYG{l+s+s2}{\PYGZdq{}}\PYG{l+s+s2}{17\PYGZhy{}29}\PYG{l+s+s2}{\PYGZdq{}} \PYG{l+m+mi}{2} \PYG{l+s+s2}{\PYGZdq{}}\PYG{l+s+s2}{30\PYGZhy{}64}\PYG{l+s+s2}{\PYGZdq{}} \PYG{l+m+mi}{3} \PYG{l+s+s2}{\PYGZdq{}}\PYG{l+s+s2}{65+}\PYG{l+s+s2}{\PYGZdq{}}
	\PYG{n}{label} \PYG{n}{values} \PYG{n}{age\PYGZus{}3} \PYG{n}{age\PYGZus{}3}
\end{sphinxVerbatim}

Create a mean value comparison with your age variable and health satisfaction in weighted and unweighted form.

\fvset{hllines={, ,}}%
\begin{sphinxVerbatim}[commandchars=\\\{\},numbers=left,firstnumber=1,stepnumber=1]
	\PYG{o}{*}\PYG{n}{unweighted}\PYG{o}{*}
	\PYG{n}{tabstat} \PYG{n}{yp0101}\PYG{p}{,} \PYG{n}{by}\PYG{p}{(}\PYG{n}{age\PYGZus{}3}\PYG{p}{)}
\end{sphinxVerbatim}

\begin{figure}[H]
\centering

\noindent\sphinxincludegraphics{{quer_11}.PNG}
\end{figure}

\fvset{hllines={, ,}}%
\begin{sphinxVerbatim}[commandchars=\\\{\},numbers=left,firstnumber=1,stepnumber=1]
	\PYG{o}{*}\PYG{n}{weighted}\PYG{o}{*}
	\PYG{n}{tabstat} \PYG{n}{yp0101} \PYG{p}{[}\PYG{n}{aw}\PYG{o}{=}\PYG{n}{yphrf}\PYG{p}{]}\PYG{p}{,} \PYG{n}{by}\PYG{p}{(}\PYG{n}{age\PYGZus{}3}\PYG{p}{)} 
\end{sphinxVerbatim}

\begin{figure}[H]
\centering

\noindent\sphinxincludegraphics{{quer_12}.PNG}
\end{figure}

\sphinxstylestrong{d) Income}

As with age, generate a categorized version of the income for the household net income:

\fvset{hllines={, ,}}%
\begin{sphinxVerbatim}[commandchars=\\\{\},numbers=left,firstnumber=1,stepnumber=1]
\PYG{o}{*}\PYG{n}{d}\PYG{p}{)} \PYG{n}{by} \PYG{n}{monthly} \PYG{n}{houshold} \PYG{n}{net} \PYG{n}{income} \PYG{p}{(}\PYG{o}{\PYGZhy{}}\PYG{l+m+mf}{1.999}\PYG{p}{,} \PYG{l+m+mf}{2.000}\PYG{o}{\PYGZhy{}}\PYG{l+m+mf}{3.999}\PYG{p}{,} \PYG{l+m+mi}{4000}\PYG{o}{+} \PYG{n}{Euro}\PYG{p}{)}
	\PYG{n}{gen} \PYG{n}{hinc08\PYGZus{}3} \PYG{o}{=} \PYG{n}{hinc08}
	\PYG{n}{recode} \PYG{n}{hinc08\PYGZus{}3} \PYG{p}{(}\PYG{l+m+mi}{0}\PYG{o}{/}\PYG{l+m+mi}{1999}\PYG{o}{=}\PYG{l+m+mi}{1}\PYG{p}{)} \PYG{p}{(}\PYG{l+m+mi}{2000}\PYG{o}{/}\PYG{l+m+mi}{3999}\PYG{o}{=}\PYG{l+m+mi}{2}\PYG{p}{)} \PYG{p}{(}\PYG{l+m+mi}{4000}\PYG{o}{/}\PYG{l+m+mi}{99999}\PYG{o}{=}\PYG{l+m+mi}{3}\PYG{p}{)}
	\PYG{n}{label} \PYG{n}{define} \PYG{n}{hinc08\PYGZus{}3} \PYG{l+m+mi}{1} \PYG{l+s+s2}{\PYGZdq{}}\PYG{l+s+s2}{\PYGZlt{}2000 Euro}\PYG{l+s+s2}{\PYGZdq{}} \PYG{l+m+mi}{2} \PYG{l+s+s2}{\PYGZdq{}}\PYG{l+s+s2}{2000\PYGZhy{}\PYGZlt{}4000 Euro}\PYG{l+s+s2}{\PYGZdq{}} \PYG{l+m+mi}{3} \PYG{l+s+s2}{\PYGZdq{}}\PYG{l+s+s2}{4000+ Euro}\PYG{l+s+s2}{\PYGZdq{}}
	\PYG{n}{label} \PYG{n}{values} \PYG{n}{hinc08\PYGZus{}3} \PYG{n}{hinc08\PYGZus{}3}
\end{sphinxVerbatim}

Display the mean values in weighted and unweighted form:

\fvset{hllines={, ,}}%
\begin{sphinxVerbatim}[commandchars=\\\{\},numbers=left,firstnumber=1,stepnumber=1]
	\PYG{o}{*}\PYG{n}{unweighted}\PYG{o}{*}
	\PYG{n}{tabstat} \PYG{n}{yp0101}\PYG{p}{,} \PYG{n}{by}\PYG{p}{(}\PYG{n}{hinc08\PYGZus{}3}\PYG{p}{)}
\end{sphinxVerbatim}

\begin{figure}[H]
\centering

\noindent\sphinxincludegraphics{{quer_14}.PNG}
\end{figure}

\fvset{hllines={, ,}}%
\begin{sphinxVerbatim}[commandchars=\\\{\},numbers=left,firstnumber=1,stepnumber=1]
	\PYG{o}{*}\PYG{n}{weighted}\PYG{o}{*}
	\PYG{n}{tabstat} \PYG{n}{yp0101} \PYG{p}{[}\PYG{n}{aw}\PYG{o}{=}\PYG{n}{yphrf}\PYG{p}{]}\PYG{p}{,} \PYG{n}{by}\PYG{p}{(}\PYG{n}{hinc08\PYGZus{}3}\PYG{p}{)}
\end{sphinxVerbatim}

\begin{figure}[H]
\centering

\noindent\sphinxincludegraphics{{quer_15}.PNG}
\end{figure}

\sphinxstylestrong{e) Smoking}

Since this variable is nominal, adjustments to this variable are not necessary. Display the average satisfaction with health for smokers and non-smokers in weighted and unweighted form:

\fvset{hllines={, ,}}%
\begin{sphinxVerbatim}[commandchars=\\\{\},numbers=left,firstnumber=1,stepnumber=1]
\PYG{o}{*}\PYG{n}{e}\PYG{p}{)} \PYG{n}{by} \PYG{n}{smoking} \PYG{n}{yes}\PYG{o}{/}\PYG{n}{no}

	\PYG{o}{*}\PYG{n}{unweighted}\PYG{o}{*}
	\PYG{n}{tabstat} \PYG{n}{yp0101}\PYG{p}{,} \PYG{n}{by}\PYG{p}{(}\PYG{n}{yp10601}\PYG{p}{)}
\end{sphinxVerbatim}

\begin{figure}[H]
\centering

\noindent\sphinxincludegraphics{{quer_16}.PNG}
\end{figure}

\fvset{hllines={, ,}}%
\begin{sphinxVerbatim}[commandchars=\\\{\},numbers=left,firstnumber=1,stepnumber=1]
	\PYG{o}{*}\PYG{n}{weighted}\PYG{o}{*}
	\PYG{n}{tabstat} \PYG{n}{yp0101} \PYG{p}{[}\PYG{n}{aw}\PYG{o}{=}\PYG{n}{yphrf}\PYG{p}{]}\PYG{p}{,} \PYG{n}{by}\PYG{p}{(}\PYG{n}{yp10601}\PYG{p}{)}  
\end{sphinxVerbatim}

\begin{figure}[H]
\centering

\noindent\sphinxincludegraphics{{quer_17}.PNG}
\end{figure}


\section{Working with Migration Data (BIOIMMIG)}
\label{\detokenize{Working with SOEP Data/index:working-with-migration-data-bioimmig}}
With its migration and refugee samples, SOEP provides a broad spectrum of information on persons with a refugee and migration background.

In the BIOIMMIG data set you will find relevant information on the history of flight and migration, such as motives for fleeing and migration, the circumstances after arrival in Germany, but also information on relatives in the country of origin and the desire to return to the country of origin in edited form. For more information about this data set and a list of the variables it contains, see the
\sphinxhref{http://panel.gsoep.de/soep-docs/surveypapers/diw\_ssp0418.pdf\#page=214}{BIOIMMIG Documentation}.

In the following, we will use this record and other information from the SOEP to create a status variable that you can use to distinguish whether or not people with a migration background also have an escape background.

\sphinxstylestrong{Create an exercise path with four subfolders:}

\begin{figure}[H]
\centering

\noindent\sphinxincludegraphics{{uebungspfade}.PNG}
\end{figure}

\sphinxstylestrong{Example:}
\begin{itemize}
\item {} 
H:/material/exercises/do

\item {} 
H:/material/exercises/output

\item {} 
H:/material/exercises/temp

\item {} 
H:/material/exercises/log

\end{itemize}

These are used to store commands, log files, data sets and temporary data sets.
Open an empty do file and define your created paths with globals:

\fvset{hllines={, ,}}%
\begin{sphinxVerbatim}[commandchars=\\\{\},numbers=left,firstnumber=1,stepnumber=1]
\PYG{o}{*}\PYG{o}{*}\PYG{o}{*}\PYG{o}{*}\PYG{o}{*}\PYG{o}{*}\PYG{o}{*}\PYG{o}{*}\PYG{o}{*}\PYG{o}{*}\PYG{o}{*}\PYG{o}{*}\PYG{o}{*}\PYG{o}{*}\PYG{o}{*}\PYG{o}{*}\PYG{o}{*}\PYG{o}{*}\PYG{o}{*}\PYG{o}{*}\PYG{o}{*}\PYG{o}{*}\PYG{o}{*}\PYG{o}{*}\PYG{o}{*}\PYG{o}{*}\PYG{o}{*}\PYG{o}{*}\PYG{o}{*}\PYG{o}{*}\PYG{o}{*}\PYG{o}{*}\PYG{o}{*}\PYG{o}{*}\PYG{o}{*}\PYG{o}{*}\PYG{o}{*}\PYG{o}{*}\PYG{o}{*}\PYG{o}{*}\PYG{o}{*}\PYG{o}{*}\PYG{o}{*}\PYG{o}{*}\PYG{o}{*}\PYG{o}{*}\PYG{o}{*}
\PYG{o}{*} \PYG{n}{Set} \PYG{n}{relative} \PYG{n}{paths} \PYG{n}{to} \PYG{n}{the} \PYG{n}{working} \PYG{n}{directory}
\PYG{o}{*}\PYG{o}{*}\PYG{o}{*}\PYG{o}{*}\PYG{o}{*}\PYG{o}{*}\PYG{o}{*}\PYG{o}{*}\PYG{o}{*}\PYG{o}{*}\PYG{o}{*}\PYG{o}{*}\PYG{o}{*}\PYG{o}{*}\PYG{o}{*}\PYG{o}{*}\PYG{o}{*}\PYG{o}{*}\PYG{o}{*}\PYG{o}{*}\PYG{o}{*}\PYG{o}{*}\PYG{o}{*}\PYG{o}{*}\PYG{o}{*}\PYG{o}{*}\PYG{o}{*}\PYG{o}{*}\PYG{o}{*}\PYG{o}{*}\PYG{o}{*}\PYG{o}{*}\PYG{o}{*}\PYG{o}{*}\PYG{o}{*}\PYG{o}{*}\PYG{o}{*}\PYG{o}{*}\PYG{o}{*}\PYG{o}{*}\PYG{o}{*}\PYG{o}{*}\PYG{o}{*}\PYG{o}{*}\PYG{o}{*}\PYG{o}{*}\PYG{o}{*}
\PYG{k}{global} \PYG{n}{AVZ} 	\PYG{l+s+s2}{\PYGZdq{}}\PYG{l+s+s2}{H:}\PYG{l+s+s2}{\PYGZbs{}}\PYG{l+s+s2}{material}\PYG{l+s+s2}{\PYGZbs{}}\PYG{l+s+s2}{exercises}\PYG{l+s+s2}{\PYGZdq{}}
\PYG{k}{global} \PYG{n}{MY\PYGZus{}IN\PYGZus{}PATH} \PYG{l+s+s2}{\PYGZdq{}}\PYG{l+s+se}{\PYGZbs{}\PYGZbs{}}\PYG{l+s+s2}{hume}\PYG{l+s+se}{\PYGZbs{}r}\PYG{l+s+s2}{dc\PYGZhy{}prod}\PYG{l+s+s2}{\PYGZbs{}}\PYG{l+s+s2}{complete}\PYG{l+s+s2}{\PYGZbs{}}\PYG{l+s+s2}{soep\PYGZhy{}core}\PYG{l+s+s2}{\PYGZbs{}}\PYG{l+s+s2}{soep.v33.2}\PYG{l+s+s2}{\PYGZbs{}}\PYG{l+s+s2}{stata\PYGZus{}en}\PYG{l+s+se}{\PYGZbs{}\PYGZdq{}}
\PYG{k}{global} \PYG{n}{MY\PYGZus{}DO\PYGZus{}FILES} \PYG{l+s+s2}{\PYGZdq{}}\PYG{l+s+s2}{\PYGZdl{}AVZ}\PYG{l+s+s2}{\PYGZbs{}}\PYG{l+s+s2}{do}\PYG{l+s+se}{\PYGZbs{}\PYGZdq{}}
\PYG{k}{global} \PYG{n}{MY\PYGZus{}LOG\PYGZus{}OUT} \PYG{l+s+s2}{\PYGZdq{}}\PYG{l+s+s2}{\PYGZdl{}AVZ}\PYG{l+s+s2}{\PYGZbs{}}\PYG{l+s+s2}{log}\PYG{l+s+se}{\PYGZbs{}\PYGZdq{}}
\PYG{k}{global} \PYG{n}{MY\PYGZus{}OUT\PYGZus{}DATA} \PYG{l+s+s2}{\PYGZdq{}}\PYG{l+s+s2}{\PYGZdl{}AVZ}\PYG{l+s+s2}{\PYGZbs{}}\PYG{l+s+s2}{output}\PYG{l+s+se}{\PYGZbs{}\PYGZdq{}}
\PYG{k}{global} \PYG{n}{MY\PYGZus{}OUT\PYGZus{}TEMP} \PYG{l+s+s2}{\PYGZdq{}}\PYG{l+s+s2}{\PYGZdl{}AVZ}\PYG{l+s+se}{\PYGZbs{}t}\PYG{l+s+s2}{emp}\PYG{l+s+se}{\PYGZbs{}\PYGZdq{}}
\end{sphinxVerbatim}

The global „AVZ“ defines the main path. The main paths are subdivided using the globals “MY\_IN\_PATH”, “MY\_DO\_FILES”, “MY\_LOG\_OUT”, “MY\_OUT\_DATA”, “MY\_OUT\_TEMP”. The global “MY\_IN\_PATH” contains the path to your ordered data.

\sphinxstylestrong{Task 1: Preparation of BIOIMMIG}

\sphinxstylestrong{a) In which variable can you find information about the status of each person when they immigrated to Germany?}

Open the record or browse the \sphinxhref{http://panel.gsoep.de/soep-docs/surveypapers/diw\_ssp0418.pdf\#page=214}{BIOIMMIG documentation} and search for a variable describing the immigration status. The biimgrp variable from the BIOIMMIG data set is the appropriate variable.

\fvset{hllines={, ,}}%
\begin{sphinxVerbatim}[commandchars=\\\{\},numbers=left,firstnumber=1,stepnumber=1]
*** Exercise 1 ******************************************************************

/*
a)	In which variable can you find information about the status of each person when they immigrated to Germany?
*/

* Immigration status is stored in the variable biimgrp.

use \PYGZdl{}MY\PYGZus{}IN\PYGZus{}PATH\PYGZbs{}bioimmig.dta, clear
\end{sphinxVerbatim}

\sphinxstylestrong{b) Identify this variable in the BIOIMMIG data set and load it from the data set, together with the person number and the survey year.}

Open your data set only with the required variables to maintain clarity in your analysis data set.

\fvset{hllines={, ,}}%
\begin{sphinxVerbatim}[commandchars=\\\{\},numbers=left,firstnumber=1,stepnumber=1]
/*
b)	Identify this variable in the BIOIMMIG data set and load it from the data set, together with the person number and the survey year.
*/

use persnr syear biimgrp using \PYGZdl{}MY\PYGZus{}IN\PYGZus{}PATH\PYGZbs{}bioimmig.dta, clear
\end{sphinxVerbatim}

\sphinxstylestrong{c) What are the values of this variable?}

Familiarize yourself with your research-relevant analysis variable and check coding and case numbers.

\fvset{hllines={, ,}}%
\begin{sphinxVerbatim}[commandchars=\\\{\},numbers=left,firstnumber=1,stepnumber=1]
/*
c)	What are the values of this variable? 
*/

tab biimgrp, m //Characteristics of the variable are examined.
\end{sphinxVerbatim}

\begin{figure}[H]
\centering

\noindent\sphinxincludegraphics{{mig_1}.PNG}
\end{figure}

\sphinxstylestrong{d) On the basis of this variable, generate the variable “Escape”, which only distinguishes between three groups:}
\begin{itemize}
\item {} 
0 = Cases where no information is available

\item {} 
1 = All persons without escape background

\item {} 
2 = Asylum seekers / fugitives

\end{itemize}

After you have familiarized yourself with the research-relevant analysis variable, recode the variable to suit your project. Then check the case numbers of your generated variable with the source variable.

\fvset{hllines={, ,}}%
\begin{sphinxVerbatim}[commandchars=\\\{\},numbers=left,firstnumber=1,stepnumber=1]
\PYG{o}{/}\PYG{o}{*}
\PYG{n}{d}\PYG{p}{)}	\PYG{n}{On} \PYG{n}{the} \PYG{n}{basis} \PYG{n}{of} \PYG{n}{this} \PYG{n}{variable}\PYG{p}{,} \PYG{n}{generate} \PYG{n}{the} \PYG{n}{variable} \PYG{l+s+s2}{\PYGZdq{}}\PYG{l+s+s2}{Escape}\PYG{l+s+s2}{\PYGZdq{}}\PYG{p}{,} \PYG{n}{which} \PYG{n}{only} \PYG{n}{distinguishes} \PYG{n}{between} \PYG{n}{three} \PYG{n}{groups}\PYG{p}{:}
    \PYG{l+m+mi}{0} \PYG{o}{=} \PYG{n}{Cases} \PYG{n}{where} \PYG{n}{no} \PYG{n}{information} \PYG{o+ow}{is} \PYG{n}{available}
    \PYG{l+m+mi}{1} \PYG{o}{=} \PYG{n}{All} \PYG{n}{persons} \PYG{n}{without} \PYG{n}{escape} \PYG{n}{background} 
    \PYG{l+m+mi}{2} \PYG{o}{=} \PYG{n}{Asylum} \PYG{n}{seekers} \PYG{o}{/} \PYG{n}{refugees}
\PYG{o}{*}\PYG{o}{/}

\PYG{n}{recode} \PYG{n}{biimgrp} \PYG{p}{(}\PYG{o}{\PYGZhy{}}\PYG{l+m+mi}{5} \PYG{o}{\PYGZhy{}}\PYG{l+m+mi}{2} \PYG{o}{\PYGZhy{}}\PYG{l+m+mi}{1} \PYG{o}{=} \PYG{l+m+mi}{0} \PYG{l+s+s2}{\PYGZdq{}}\PYG{l+s+s2}{No Answer}\PYG{l+s+s2}{\PYGZdq{}}\PYG{p}{)} \PYG{p}{(}\PYG{l+m+mi}{1} \PYG{l+m+mi}{2} \PYG{l+m+mi}{3} \PYG{l+m+mi}{4} \PYG{l+m+mi}{6} \PYG{o}{=} \PYG{l+m+mi}{1} \PYG{l+s+s2}{\PYGZdq{}}\PYG{l+s+s2}{no Escape}\PYG{l+s+s2}{\PYGZdq{}}\PYG{p}{)} \PYG{p}{(}\PYG{l+m+mi}{5} \PYG{o}{=} \PYG{l+m+mi}{2} \PYG{l+s+s2}{\PYGZdq{}}\PYG{l+s+s2}{Escape}\PYG{l+s+s2}{\PYGZdq{}}\PYG{p}{)}\PYG{p}{,} \PYG{n}{gen}\PYG{p}{(}\PYG{n}{Escape}\PYG{p}{)}
\PYG{n}{tab} \PYG{n}{biimgrp} \PYG{n}{Escape}\PYG{p}{,} \PYG{n}{m} \PYG{o}{/}\PYG{o}{/} \PYG{n}{biimgrp} \PYG{o+ow}{and} \PYG{n}{escape} \PYG{n}{are} \PYG{n}{compared}\PYG{o}{.}
\end{sphinxVerbatim}

\begin{figure}[H]
\centering

\noindent\sphinxincludegraphics{{mig_2}.PNG}
\end{figure}

\sphinxstylestrong{e) It may happen that initially there is no information on the status of immigration, but this will change in a later year. Limit the data record to the last observation that is available for the respective person, since this way the specification with the most information content is used.}

\fvset{hllines={, ,}}%
\begin{sphinxVerbatim}[commandchars=\\\{\},numbers=left,firstnumber=1,stepnumber=1]
\PYG{n}{e}\PYG{p}{)}	\PYG{n}{It} \PYG{n}{may} \PYG{n}{happen} \PYG{n}{that} \PYG{n}{tinitially} \PYG{n}{there} \PYG{o+ow}{is} \PYG{n}{no} \PYG{n}{information} \PYG{n}{on} \PYG{n}{the} \PYG{n}{status} \PYG{n}{of} 
\PYG{o}{*}   \PYG{n}{immigration}\PYG{p}{,} \PYG{n}{but} \PYG{n}{this} \PYG{n}{will} \PYG{n}{change} \PYG{o+ow}{in} \PYG{n}{a} \PYG{n}{later} \PYG{n}{year}\PYG{o}{.} \PYG{n}{Limit} \PYG{n}{the} \PYG{n}{data} \PYG{n}{record} \PYG{n}{to} 
\PYG{o}{*}   \PYG{n}{the} \PYG{n}{last} \PYG{n}{observation} \PYG{n}{that} \PYG{o+ow}{is} \PYG{n}{available} \PYG{k}{for} \PYG{n}{the} \PYG{n}{respective} \PYG{n}{person}\PYG{p}{,} \PYG{n}{since} \PYG{n}{this} 
\PYG{o}{*}   \PYG{n}{way} \PYG{n}{the} \PYG{n}{specification} \PYG{k}{with} \PYG{n}{the} \PYG{n}{most} \PYG{n}{information} \PYG{n}{content} \PYG{o+ow}{is} \PYG{n}{used}\PYG{o}{.} 
\PYG{o}{*}\PYG{o}{/}

\PYG{n}{bysort} \PYG{n}{persnr}\PYG{p}{:} \PYG{n}{egen} \PYG{n}{syear\PYGZus{}max} \PYG{o}{=} \PYG{n+nb}{max}\PYG{p}{(}\PYG{n}{syear}\PYG{p}{)} \PYG{o}{/}\PYG{o}{/}\PYG{n}{A} \PYG{n}{variable} \PYG{o+ow}{is} \PYG{n}{created}\PYG{p}{,} \PYG{n}{which} \PYG{n}{shows} \PYG{n}{the} \PYG{n}{last} \PYG{n}{existing} \PYG{n}{yearly} \PYG{n}{observation}
\PYG{n}{keep} \PYG{k}{if} \PYG{n}{syear\PYGZus{}max} \PYG{o}{==} \PYG{n}{syear} \PYG{o}{/}\PYG{o}{/}\PYG{n}{Annual} \PYG{n}{observations} \PYG{n}{which} \PYG{n}{are} \PYG{o+ow}{not} \PYG{n}{the} \PYG{n}{last} \PYG{n}{observation} \PYG{n}{are} \PYG{n}{deleted}\PYG{o}{.}
\end{sphinxVerbatim}

\sphinxstylestrong{f) Save the generated data record on your personal drive temporarily .}

\fvset{hllines={, ,}}%
\begin{sphinxVerbatim}[commandchars=\\\{\},numbers=left,firstnumber=1,stepnumber=1]
f)	Save the generated data record on your personal drive temporarily 
*/

save \PYGZdl{}MY\PYGZus{}OUT\PYGZus{}TEMP\PYGZbs{}biimgrp.dta, replace
\end{sphinxVerbatim}

\sphinxstylestrong{Aufgabe 2: Add basic variables from PPFAD and weights}

\sphinxstylestrong{a) Load the following information from PPFAD:}
\begin{itemize}
\item {} 
Never changing Person ID  \href{https://paneldata.org/soep-core/data/ppfad/persnr}{\textbf{"persnr"}}

\item {} 
Household number  \href{https://paneldata.org/soep-core/data/ppfad/hhnr}{\textbf{"hhnr"}} and the current household number  \href{https://paneldata.org/soep-core/data/ppfad/bghhnr}{\textbf{"bghhnr"}}

\item {} 
The net variable with information about the interview type  \href{https://paneldata.org/soep-core/data/ppfad/bgnetto}{\textbf{"bgnetto"}}

\item {} 
The sex of the person  \href{https://paneldata.org/soep-core/data/ppfad/sex}{\textbf{"sex"}}

\item {} 
The year of birth  \href{https://paneldata.org/soep-core/data/ppfad/gebjahr}{\textbf{"gebjahr"}}

\item {} 
Variables on the migration background  \href{https://paneldata.org/soep-core/data/ppfad/migback}{\textbf{"migback"}},  \href{https://paneldata.org/soep-core/data/ppfad/germborn}{\textbf{"germborn"}},  \href{https://paneldata.org/soep-core/data/ppfad/corigin}{\textbf{"corigin"}},  \href{https://paneldata.org/soep-core/data/ppfad/immiyear}{\textbf{"immiyear"}}

\item {} 
Information about the survey status:  \href{https://paneldata.org/soep-core/data/ppfad/psample}{\textbf{"psample"}}

\end{itemize}

If you want to familiarize yourself with the PPFAD data set, visit the chapter {\hyperref[\detokenize{Working with SOEP Data/index:working-ppfad}]{\sphinxcrossref{\DUrole{std,std-ref}{Working with Tracking Data (PPFAD)}}}}.

\fvset{hllines={, ,}}%
\begin{sphinxVerbatim}[commandchars=\\\{\},numbers=left,firstnumber=1,stepnumber=1]
/*
a)	Use the following information from PPFAD: 
  \PYGZhy{} Never changing Person ID „persnr“
  \PYGZhy{} Household number \PYGZdq{}hhnr\PYGZdq{} and the current household number \PYGZdq{}bghhnr\PYGZdq{}. 
  \PYGZhy{} the net variable with information about the interview type \PYGZdq{}bgnetto\PYGZdq{}.
  \PYGZhy{} the sex of the person \PYGZdq{}sex\PYGZdq{}
  \PYGZhy{} the year of birth \PYGZdq{}semester\PYGZdq{}
  \PYGZhy{} Variables on the migration background \PYGZdq{}migback\PYGZdq{}, \PYGZdq{}germborn\PYGZdq{} \PYGZdq{}corigin\PYGZdq{} \PYGZdq{}immiyear\PYGZdq{}
  \PYGZhy{} Information about the survey status: \PYGZdq{}bgnetto\PYGZdq{} and \PYGZdq{}psample\PYGZdq{}.
*/

use persnr hhnr bghhnr bgnetto psample sex gebjahr germborn corigin immiyear migback  using \PYGZdl{}MY\PYGZus{}IN\PYGZus{}PATH\PYGZbs{}ppfad.dta, clear
\end{sphinxVerbatim}

\sphinxstylestrong{b)  Merge the previously generated data record using the person number.}

If you don’t understand how to create your own cross-section dataset, visit the chapter {\hyperref[\detokenize{Working with SOEP Data/index:cross-data}]{\sphinxcrossref{\DUrole{std,std-ref}{Generating a cross-section Data Set}}}}.

\fvset{hllines={, ,}}%
\begin{sphinxVerbatim}[commandchars=\\\{\},numbers=left,firstnumber=1,stepnumber=1]
/*
b)	Merge the previously generated data record using the person number.
*/

merge 1:1 persnr using \PYGZdl{}MY\PYGZus{}OUT\PYGZus{}TEMP\PYGZbs{}biimgrp.dta, nogen
\end{sphinxVerbatim}

\sphinxstylestrong{c) Add the corresponding person extrapolation factors to the data record.}

\fvset{hllines={, ,}}%
\begin{sphinxVerbatim}[commandchars=\\\{\},numbers=left,firstnumber=1,stepnumber=1]
c)	Add the corresponding person extrapolation factors to the data record.
*/

merge 1:1 persnr using \PYGZdl{}MY\PYGZus{}IN\PYGZus{}PATH\PYGZbs{}phrf.dta, keepus(bgphrf) nogen
\end{sphinxVerbatim}

\sphinxstylestrong{d) Only keep respondents for whom a youth or individual questionnaire was realized in 2016.}

For example, to exclude children who have not provided immigration status information, use the net code from PPFAD. Only keep persons who have conducted a completed individual or youth interview.

\fvset{hllines={, ,}}%
\begin{sphinxVerbatim}[commandchars=\\\{\},numbers=left,firstnumber=1,stepnumber=1]
\PYG{o}{/}\PYG{o}{*}
\PYG{n}{d}\PYG{p}{)}	\PYG{n}{Only} \PYG{n}{keep} \PYG{n}{individuals} \PYG{k}{for} \PYG{n}{whom} \PYG{n}{a} \PYG{n}{youth} \PYG{o+ow}{or} \PYG{n}{personal} \PYG{n}{questionnaire} \PYG{n}{was} \PYG{n}{realized} \PYG{o+ow}{in} \PYG{l+m+mf}{2016.}
\PYG{o}{*}\PYG{o}{/}

\PYG{n}{tab} \PYG{n}{bgnetto}\PYG{p}{,} \PYG{n}{m} \PYG{o}{/}\PYG{o}{/}\PYG{n}{Variable} \PYG{n}{values} \PYG{n}{are} \PYG{n}{displayed}

\PYG{n}{keep} \PYG{k}{if} \PYG{n}{inrange}\PYG{p}{(}\PYG{n}{bgnetto}\PYG{p}{,} \PYG{l+m+mi}{10}\PYG{p}{,} \PYG{l+m+mi}{19}\PYG{p}{)} \PYG{o}{/}\PYG{o}{/} \PYG{n}{People} \PYG{n}{who} \PYG{n}{have} \PYG{n}{a} \PYG{n}{code} \PYG{n}{between} \PYG{l+m+mi}{10} \PYG{o+ow}{and} \PYG{l+m+mi}{19} \PYG{n}{will} \PYG{n}{be} \PYG{n}{kept}\PYG{o}{.}
\end{sphinxVerbatim}

\begin{figure}[H]
\centering

\noindent\sphinxincludegraphics{{mig_3}.PNG}
\end{figure}

\sphinxstylestrong{Task 3: Generate a status variable with the following categories:}.
\begin{itemize}
\item {} 
No immigrant background

\item {} 
Migration 2nd generation

\item {} 
Immigration without information

\item {} 
Immigration, not flight

\item {} 
Immigration, Flight

\end{itemize}

To generate this status variable, check the contents of the existing migration variables from PPFAD (migback germborn).

\fvset{hllines={, ,}}%
\begin{sphinxVerbatim}[commandchars=\\\{\},numbers=left,firstnumber=1,stepnumber=1]
\PYG{o}{/}\PYG{o}{*}
\PYG{n}{Generate} \PYG{n}{a} \PYG{n}{status} \PYG{n}{variable} \PYG{k}{with} \PYG{n}{the} \PYG{n}{following} \PYG{n}{categories}\PYG{p}{:}
\PYG{o}{*}\PYG{o}{/}

\PYG{n}{tab} \PYG{n}{migback}
\end{sphinxVerbatim}

\begin{figure}[H]
\centering

\noindent\sphinxincludegraphics{{mig_4}.PNG}
\end{figure}

\fvset{hllines={, ,}}%
\begin{sphinxVerbatim}[commandchars=\\\{\},numbers=left,firstnumber=1,stepnumber=1]
\PYG{n}{tab} \PYG{n}{germborn}
\end{sphinxVerbatim}

\begin{figure}[H]
\centering

\noindent\sphinxincludegraphics{{mig_5}.PNG}
\end{figure}

Use the migration variables from PPFAD (migback, germborn) and link this information with your previously generated escape variable to build the described status variable from Task 3.

\fvset{hllines={, ,}}%
\begin{sphinxVerbatim}[commandchars=\\\{\},numbers=left,firstnumber=1,stepnumber=1]
\PYG{n}{gen} \PYG{n}{Status} \PYG{o}{=} \PYG{l+m+mi}{0} \PYG{o}{/}\PYG{o}{/} \PYG{n}{All} \PYG{n}{persons} \PYG{n}{will} \PYG{n}{first} \PYG{n}{receive} \PYG{n}{the} \PYG{n}{missing} \PYG{n}{code} \PYG{k}{for} \PYG{l+s+s2}{\PYGZdq{}}\PYG{l+s+s2}{no info}\PYG{l+s+s2}{\PYGZdq{}}\PYG{o}{.}
\PYG{n}{replace} \PYG{n}{Status} \PYG{o}{=} \PYG{l+m+mi}{1} \PYG{k}{if} \PYG{n}{migback} \PYG{o}{==} \PYG{l+m+mi}{1} \PYG{o}{\PYGZam{}} \PYG{n}{germborn} \PYG{o}{==} \PYG{l+m+mi}{1} \PYG{o}{/}\PYG{o}{/} \PYG{l+s+s2}{\PYGZdq{}}\PYG{l+s+s2}{no migback}\PYG{l+s+s2}{\PYGZdq{}}
\PYG{n}{replace} \PYG{n}{Status} \PYG{o}{=} \PYG{l+m+mi}{2} \PYG{k}{if} \PYG{n}{migback} \PYG{o}{==} \PYG{l+m+mi}{3}                 \PYG{o}{/}\PYG{o}{/} \PYG{l+s+s2}{\PYGZdq{}}\PYG{l+s+s2}{2nd generation}\PYG{l+s+s2}{\PYGZdq{}} \PYG{p}{(}\PYG{l+m+mi}{2}\PYG{n}{nd} \PYG{n}{generation} \PYG{n}{migrants} \PYG{n}{born} \PYG{n}{by} \PYG{n}{definition} \PYG{o+ow}{in} \PYG{n}{Germany}\PYG{p}{,} \PYG{n}{therefore} \PYG{l+s+s2}{\PYGZdq{}}\PYG{l+s+s2}{\PYGZam{} germborn == 1}\PYG{l+s+s2}{\PYGZdq{}} \PYG{n}{here} \PYG{n}{unnecessary}
\PYG{n}{replace} \PYG{n}{Status} \PYG{o}{=} \PYG{l+m+mi}{3} \PYG{k}{if} \PYG{n}{germborn} \PYG{o}{==} \PYG{l+m+mi}{2} \PYG{o}{\PYGZam{}} \PYG{n}{Escape} \PYG{o}{==} \PYG{l+m+mi}{0}  \PYG{o}{/}\PYG{o}{/} \PYG{l+s+s2}{\PYGZdq{}}\PYG{l+s+s2}{Immigrants without information}\PYG{l+s+s2}{\PYGZdq{}} 
\PYG{n}{replace} \PYG{n}{Status} \PYG{o}{=} \PYG{l+m+mi}{4} \PYG{k}{if} \PYG{n}{germborn} \PYG{o}{==} \PYG{l+m+mi}{2} \PYG{o}{\PYGZam{}} \PYG{n}{Escape} \PYG{o}{==} \PYG{l+m+mi}{1}  \PYG{o}{/}\PYG{o}{/} \PYG{l+s+s2}{\PYGZdq{}}\PYG{l+s+s2}{Immigrants, no escape}\PYG{l+s+s2}{\PYGZdq{}}
\PYG{n}{replace} \PYG{n}{Status} \PYG{o}{=} \PYG{l+m+mi}{5} \PYG{k}{if} \PYG{n}{germborn} \PYG{o}{==} \PYG{l+m+mi}{2} \PYG{o}{\PYGZam{}} \PYG{n}{Escape} \PYG{o}{==} \PYG{l+m+mi}{2}  \PYG{o}{/}\PYG{o}{/} \PYG{l+s+s2}{\PYGZdq{}}\PYG{l+s+s2}{Immigrant, escape}\PYG{l+s+s2}{\PYGZdq{}}

\PYG{n}{label} \PYG{k}{def} \PYG{n+nf}{Statuslbl} \PYG{l+m+mi}{0}\PYG{l+s+s2}{\PYGZdq{}}\PYG{l+s+s2}{no info}\PYG{l+s+s2}{\PYGZdq{}} \PYG{l+m+mi}{1}\PYG{l+s+s2}{\PYGZdq{}}\PYG{l+s+s2}{no migback}\PYG{l+s+s2}{\PYGZdq{}} \PYG{l+m+mi}{2}\PYG{l+s+s2}{\PYGZdq{}}\PYG{l+s+s2}{2. Generation}\PYG{l+s+s2}{\PYGZdq{}} \PYG{l+m+mi}{3}\PYG{l+s+s2}{\PYGZdq{}}\PYG{l+s+s2}{Immigrants without information}\PYG{l+s+s2}{\PYGZdq{}}  \PYG{l+m+mi}{4}\PYG{l+s+s2}{\PYGZdq{}}\PYG{l+s+s2}{Immigrants, no escape}\PYG{l+s+s2}{\PYGZdq{}} \PYG{l+m+mi}{5}\PYG{l+s+s2}{\PYGZdq{}}\PYG{l+s+s2}{Immigrant, escape}\PYG{l+s+s2}{\PYGZdq{}}
\PYG{n}{label} \PYG{n}{val} \PYG{n}{Status} \PYG{n}{Statuslbl} \PYG{o}{/}\PYG{o}{/} \PYG{n}{Values} \PYG{n}{of} \PYG{n}{the} \PYG{n}{status} \PYG{n}{veriable} \PYG{n}{receive} \PYG{n}{label}
\end{sphinxVerbatim}

\sphinxstylestrong{Task 4: Content analysis:}

\sphinxstylestrong{a) How many refugees (foreign-born with refugee/asylum titles) are now in your record?}

Look at your status variable previously generated in task 3 to answer the question

\fvset{hllines={, ,}}%
\begin{sphinxVerbatim}[commandchars=\\\{\},numbers=left,firstnumber=1,stepnumber=1]
*** Exercise 4 ******************************************************************

/*
a)	How many refugees (foreign\PYGZhy{}born with refugee/asylum titles) are now in your record?
*/

tab Status, m //Display Generated Status Variable
\end{sphinxVerbatim}

\begin{figure}[H]
\centering

\noindent\sphinxincludegraphics{{mig_6}.PNG}
\end{figure}

All 4,514 respondents who received the value 5 for the generated status variable have a direct migration background (migback==2), were not born in Germany (germborn==2) and fled their home country (flight==2 and biimgrp==5).

\sphinxstylestrong{b) How many are there if you take the person extrapolation factors into account? Interpret the results.}

Look at your status variable previously generated in task 3 to answer the question

\fvset{hllines={, ,}}%
\begin{sphinxVerbatim}[commandchars=\\\{\},numbers=left,firstnumber=1,stepnumber=1]
/*
b)	How many are there if you take the person extrapolation factors into account? Interpret the results.
*/

tab Status [aw=bgphrf], m  //Display generated status variable weighted with analytic weights
\end{sphinxVerbatim}

\begin{figure}[H]
\centering

\noindent\sphinxincludegraphics{{mig_7}.PNG}
\end{figure}

After weighting, there are only about 675 fugitives in the data set.
The weighting thus corrected the number of fugitives downwards.

\sphinxstylestrong{c) How many persons are represented by the sample taking the extrapolation factors into account?}

To use frequency weights in STATA, integer weights are required. Create an integer frequency weight from the weighting factor provided so that you can make representative statements. Then take a look at the new results.

\fvset{hllines={, ,}}%
\begin{sphinxVerbatim}[commandchars=\\\{\},numbers=left,firstnumber=1,stepnumber=1]
/*
c)	How many persons are represented by the sample taking the extrapolation factors into account?
*/

gen fweight = round(bgphrf) //Frequency weights for stata require integer weight
tab Status [fw=fweight], m  //Display generated status variable weighted with frequency weights
\end{sphinxVerbatim}

\begin{figure}[H]
\centering

\noindent\sphinxincludegraphics{{mig_8}.PNG}
\end{figure}

Around 1,600,000 people are represented.

\sphinxstylestrong{d) What is the proportion of people over 40 years of age among the fugitives?}

Since the data in this exercise come from the wave “bg”, we are currently in the survey year 2016; if you need a description of the wave designations, please refer to the chapter {\hyperref[\detokenize{Principles of Data Structure/index:label}]{\sphinxcrossref{\DUrole{std,std-ref}{Labeling SOEP-Core}}}}. To generate a suitable age variable, you can use the year of birth (year of birth). If we look at the survey year 2016, all persons born in 1976 or earlier were over 40 years old. Generate a suitable age variable and look at the proportion of fugitives over 40 years of age in weighted form:

\fvset{hllines={, ,}}%
\begin{sphinxVerbatim}[commandchars=\\\{\},numbers=left,firstnumber=1,stepnumber=1]
/*
d)	What is the proportion of people over 40 years of age among the fugitives?
*/

gen ue\PYGZus{}40 = 0
replace ue\PYGZus{}40 = 1 if gebjahr \PYGZlt{}= 1976 // Persons receive proficiency 1 if they were born before 1975.

tab Status ue\PYGZus{}40 [aw=bgphrf], m row nofreq
\end{sphinxVerbatim}

\begin{figure}[H]
\centering

\noindent\sphinxincludegraphics{{mig_9}.PNG}
\end{figure}

The proportion of refugees over 40 years of age is about 47\%.


\section{Generating a longitudinal Data Set}
\label{\detokenize{Working with SOEP Data/index:generating-a-longitudinal-data-set}}
This example is about generating a data set to analyze determinants of health satisfaction. You can either use the syntax generator of paneldata.org or write a syntax file yourself. You can search for variable names in Paneldata.org.

In the previous examples you have already created an exercise path with four subfolders, as well as corresponding globals in the STATA do-file. You can use the same folders and globals for this exercise.

\sphinxstylestrong{1.Generate an unbalanced panel dataset for the years 2006 to 2008 using paneldata.org if you wish. The data set should contain all respondents in private households:}

The data set should contain the following variables of interest:
\begin{itemize}
\item {} 
Health satisfaction  \href{https://paneldata.org/soep-core/data/wp/wp0101}{\textbf{"wp0101"}}  \href{https://paneldata.org/soep-core/data/xp/xp0101}{\textbf{"xp0101"}}  \href{https://paneldata.org/soep-core/data/yp/yp0101}{\textbf{"yp0101"}}

\item {} 
Smoking at present yes/no  \href{https://paneldata.org/soep-core/data/wp/wp9301}{\textbf{"wp9301"}}  \href{https://paneldata.org/soep-core/data/yp/yp10601}{\textbf{"yp10601"}}

\item {} 
Current employment status  \href{https://paneldata.org/soep-core/data/wpgen/emplst06}{\textbf{"emplst06"}}  \href{https://paneldata.org/soep-core/data/xpgen/emplst07}{\textbf{"emplst07"}}  \href{https://paneldata.org/soep-core/data/ypgen/emplst08}{\textbf{"emplst08"}}

\item {} 
Monthly household net income  \href{https://paneldata.org/soep-core/data/whgen/hinc06}{\textbf{"hinc06"}}  \href{https://paneldata.org/soep-core/data/xhgen/hinc07}{\textbf{"hinc07"}}  \href{https://paneldata.org/soep-core/data/yhgen/hinc08}{\textbf{"hinc08"}}

\end{itemize}

In addition, the data set should include the following additional information for analysis from 2006 to 2008:
\begin{itemize}
\item {} 
Cross-sectional weighting factors for all relevant years  \href{https://paneldata.org/soep-core/data/phrf/wphrf}{\textbf{"wphrf"}}  \href{https://paneldata.org/soep-core/data/phrf/xphrf}{\textbf{"xphrf"}}  \href{https://paneldata.org/soep-core/data/phrf/yphrf}{\textbf{"yphrf"}}

\item {} 
Person ID  \href{https://paneldata.org/soep-core/data/ppfad/persnr}{\textbf{"persnr"}}

\item {} 
Original household number  \href{https://paneldata.org/soep-core/data/ppfad/hhnr}{\textbf{"hhnr"}}

\item {} 
Household number for all relevant years  \href{https://paneldata.org/soep-core/data/ppfad/whhnr}{\textbf{"whhnr"}}  \href{https://paneldata.org/soep-core/data/ppfad/xhhnr}{\textbf{"xhhnr"}}  \href{https://paneldata.org/soep-core/data/ppfad/yhhnr}{\textbf{"yhhnr"}}

\item {} 
Sample membership  \href{https://paneldata.org/soep-core/data/ppfad/psample}{\textbf{"psample"}}

\item {} 
Sex  \href{https://paneldata.org/soep-core/data/ppfad/sex}{\textbf{"sex"}}

\item {} 
Year of birth  \href{https://paneldata.org/soep-core/data/ppfad/gebjahr}{\textbf{"gebjahr"}}

\item {} 
population membership  \href{https://paneldata.org/soep-core/data/ppfad/wpop}{\textbf{"wpop"}}  \href{https://paneldata.org/soep-core/data/ppfad/xpop}{\textbf{"xpop"}}  \href{https://paneldata.org/soep-core/data/ppfad/ypop}{\textbf{"ypop"}}

\end{itemize}

If you need detailed instructions on how the script generator works in paneldata.org, you can find them in the chapter
{\hyperref[\detokenize{Working with SOEP Documentation/index:syntax}]{\sphinxcrossref{\DUrole{std,std-ref}{Syntax Generator on paneldata.org}}}}.

If you would like to assemble your data set yourself, you can do this with the data sets you have supplied. From the previous exercise with tracking data, you may already have an idea where to get most of the variables.

Since we want to have an unbalanced panel record, the \$netto variable for the years 2006 to 2008 must also be used. In addition, our analysis must limit population membership, as we are only interested in household respondents.

\begin{sphinxadmonition}{tip}{Tip:}
If a data set is created from several variables of different data sets, it is worth sorting the person number before saving the individual data sets in order to be able to merge the data sets more easily afterwards.
\end{sphinxadmonition}

\sphinxstylestrong{1.1.  Create a Master-Files}

Use ppfad as the source file together with the required variables that you may have already researched in Paneldata or identified from the variable label of the data set. Note that only variables of the years to be analyzed should be used.

\fvset{hllines={, ,}}%
\begin{sphinxVerbatim}[commandchars=\\\{\},numbers=left,firstnumber=1,stepnumber=1]

\PYG{n}{use} \PYG{n}{hhnr} \PYG{n}{persnr} \PYG{n}{sex} \PYG{n}{gebjahr} \PYG{n}{psample} \PYG{n}{xhhnr} \PYG{n}{xnetto} \PYG{n}{xpop} \PYG{n}{yhhnr} \PYG{n}{ynetto} \PYG{n}{ypop} \PYG{n}{whhnr} \PYG{n}{wnetto} \PYG{n}{wpop} \PYG{n}{using} \PYG{l+s+s2}{\PYGZdq{}}\PYG{l+s+s2}{\PYGZdl{}}\PYG{l+s+si}{\PYGZob{}MY\PYGZus{}PATH\PYGZus{}IN\PYGZcb{}}\PYG{l+s+s2}{ppfad.dta}\PYG{l+s+s2}{\PYGZdq{}}

\end{sphinxVerbatim}

Since we want to receive an unbalanced data set, i.e. persons who have completed a personal questionnaire at least once within the 3 years, you must restrict the variable \$netto (survey status). Also, we only want to analyze private households, so we need a further restriction of the \$pop (sample membership) variable.

\fvset{hllines={, ,}}%
\begin{sphinxVerbatim}[commandchars=\\\{\},numbers=left,firstnumber=1,stepnumber=1]

\PYG{n}{keep} \PYG{k}{if} \PYG{p}{(} \PYG{p}{(}\PYG{n}{xnetto} \PYG{o}{\PYGZgt{}}\PYG{o}{=} \PYG{l+m+mi}{10} \PYG{o}{\PYGZam{}} \PYG{n}{xnetto} \PYG{o}{\PYGZlt{}} \PYG{l+m+mi}{20}\PYG{p}{)} \PYG{o}{\textbar{}} \PYG{p}{(}\PYG{n}{ynetto} \PYG{o}{\PYGZgt{}}\PYG{o}{=} \PYG{l+m+mi}{10} \PYG{o}{\PYGZam{}} \PYG{n}{ynetto} \PYG{o}{\PYGZlt{}} \PYG{l+m+mi}{20}\PYG{p}{)} \PYG{o}{\textbar{}} \PYG{p}{(}\PYG{n}{wnetto} \PYG{o}{\PYGZgt{}}\PYG{o}{=} \PYG{l+m+mi}{10} \PYG{o}{\PYGZam{}} \PYG{n}{wnetto} \PYG{o}{\PYGZlt{}} \PYG{l+m+mi}{20}\PYG{p}{)} \PYG{p}{)}


\PYG{o}{*} \PYG{o}{*} \PYG{o}{*} \PYG{n}{PRIVATE} \PYG{n}{VS} \PYG{n}{ALL} \PYG{n}{HOUSEHOLDS} \PYG{o}{*} \PYG{o}{*} \PYG{o}{*}

\PYG{n}{keep} \PYG{k}{if} \PYG{p}{(} \PYG{p}{(}\PYG{n}{xpop} \PYG{o}{==} \PYG{l+m+mi}{1} \PYG{o}{\textbar{}} \PYG{n}{xpop} \PYG{o}{==} \PYG{l+m+mi}{2}\PYG{p}{)} \PYG{o}{\textbar{}} \PYG{p}{(}\PYG{n}{ypop} \PYG{o}{==} \PYG{l+m+mi}{1} \PYG{o}{\textbar{}} \PYG{n}{ypop} \PYG{o}{==} \PYG{l+m+mi}{2}\PYG{p}{)} \PYG{o}{\textbar{}} \PYG{p}{(}\PYG{n}{wpop} \PYG{o}{==} \PYG{l+m+mi}{1} \PYG{o}{\textbar{}} \PYG{n}{wpop} \PYG{o}{==} \PYG{l+m+mi}{2}\PYG{p}{)} \PYG{p}{)}

\end{sphinxVerbatim}

Then we sort the persnr (personal number) of the data record and save it.

\fvset{hllines={, ,}}%
\begin{sphinxVerbatim}[commandchars=\\\{\},numbers=left,firstnumber=1,stepnumber=1]

\PYG{n}{sort} \PYG{n}{persnr}
\PYG{n}{save} \PYG{l+s+s2}{\PYGZdq{}}\PYG{l+s+s2}{\PYGZdl{}}\PYG{l+s+si}{\PYGZob{}MY\PYGZus{}PATH\PYGZus{}OUT\PYGZcb{}}\PYG{l+s+s2}{ppfad.dta}\PYG{l+s+s2}{\PYGZdq{}}\PYG{p}{,} \PYG{n}{replace}
\PYG{n}{clear}

\end{sphinxVerbatim}

What is still missing is the cross-section weighting factor and the variables of interest in terms of content. To apply the weighting factors to the data set, open the weighting data set for the person level phrf, sort it and save it again.

\fvset{hllines={, ,}}%
\begin{sphinxVerbatim}[commandchars=\\\{\},numbers=left,firstnumber=1,stepnumber=1]

\PYG{n}{use} \PYG{n}{persnr} \PYG{n}{wphrf} \PYG{n}{xphrf} \PYG{n}{yphrf} \PYG{n}{using} \PYG{l+s+s2}{\PYGZdq{}}\PYG{l+s+s2}{\PYGZdl{}}\PYG{l+s+si}{\PYGZob{}MY\PYGZus{}PATH\PYGZus{}IN\PYGZcb{}}\PYG{l+s+s2}{phrf.dta}\PYG{l+s+s2}{\PYGZdq{}}
\PYG{n}{sort} \PYG{n}{persnr}
\PYG{n}{save} \PYG{l+s+s2}{\PYGZdq{}}\PYG{l+s+s2}{\PYGZdl{}}\PYG{l+s+si}{\PYGZob{}MY\PYGZus{}PATH\PYGZus{}OUT\PYGZcb{}}\PYG{l+s+s2}{phrf.dta}\PYG{l+s+s2}{\PYGZdq{}}\PYG{p}{,} \PYG{n}{replace}
\PYG{n}{clear}

\end{sphinxVerbatim}

Now we come to the variables of content. In order not to have to click through all delivered data sets, it is recommended to enter the label of the variable of interest on paneldata.org.

Use the filter to narrow your search. Select our main study SOEP Core, the search type “variable”, the analysis unit “p” or “h” and the corresponding year. Once you have clicked on the year of interest, a variable history is displayed. You can use this to see in which years the variable was collected and what the variable is called.

Example: Variable Label „Satisfaction Health“

\begin{figure}[H]
\centering

\noindent\sphinxincludegraphics{{satisfaction_health}.PNG}
\end{figure}

Example: Variable  Label „currently smoking yes/no“

\begin{figure}[H]
\centering

\noindent\sphinxincludegraphics{{currently_smoke}.PNG}
\end{figure}

Example: Variable  Label „current employment status“

\begin{figure}[H]
\centering

\noindent\sphinxincludegraphics{{employment_status}.PNG}
\end{figure}

Example: Variable  Label „monthly net household income“

\begin{figure}[H]
\centering

\noindent\sphinxincludegraphics{{household_income}.PNG}
\end{figure}

To merge the data you can either use the script generator on paneldata.org or write the syntax manually into a do-file.

We now have all the information we need to create a master file. As already mentioned with \sphinxstylestrong{TIP!}, it is recommended to save the data records sorted by the persnr (person number) before merging.

\fvset{hllines={, ,}}%
\begin{sphinxVerbatim}[commandchars=\\\{\},numbers=left,firstnumber=1,stepnumber=1]
\PYG{n}{use} \PYG{n}{persnr} \PYG{n}{wp0101} \PYG{n}{wp9301} \PYG{n}{using} \PYG{l+s+s2}{\PYGZdq{}}\PYG{l+s+s2}{\PYGZdl{}}\PYG{l+s+si}{\PYGZob{}MY\PYGZus{}PATH\PYGZus{}IN\PYGZcb{}}\PYG{l+s+s2}{wp.dta}\PYG{l+s+s2}{\PYGZdq{}}
\PYG{n}{sort} \PYG{n}{persnr}
\PYG{n}{save} \PYG{l+s+s2}{\PYGZdq{}}\PYG{l+s+s2}{\PYGZdl{}}\PYG{l+s+si}{\PYGZob{}MY\PYGZus{}PATH\PYGZus{}OUT\PYGZcb{}}\PYG{l+s+s2}{wp.dta}\PYG{l+s+s2}{\PYGZdq{}}\PYG{p}{,} \PYG{n}{replace}
\PYG{n}{clear}

\PYG{o}{*} \PYG{o}{*} \PYG{o}{*} \PYG{n}{Persons} \PYG{l+m+mi}{2007} \PYG{o}{*} \PYG{o}{*} \PYG{o}{*}
\PYG{n}{use} \PYG{n}{persnr} \PYG{n}{xp0101} \PYG{n}{using} \PYG{l+s+s2}{\PYGZdq{}}\PYG{l+s+s2}{\PYGZdl{}}\PYG{l+s+si}{\PYGZob{}MY\PYGZus{}PATH\PYGZus{}IN\PYGZcb{}}\PYG{l+s+s2}{xp.dta}\PYG{l+s+s2}{\PYGZdq{}}
\PYG{n}{sort} \PYG{n}{persnr}
\PYG{n}{save} \PYG{l+s+s2}{\PYGZdq{}}\PYG{l+s+s2}{\PYGZdl{}}\PYG{l+s+si}{\PYGZob{}MY\PYGZus{}PATH\PYGZus{}OUT\PYGZcb{}}\PYG{l+s+s2}{xp.dta}\PYG{l+s+s2}{\PYGZdq{}}\PYG{p}{,} \PYG{n}{replace}
\PYG{n}{clear}

\PYG{o}{*} \PYG{o}{*} \PYG{o}{*} \PYG{n}{Persons} \PYG{l+m+mi}{2008} \PYG{o}{*} \PYG{o}{*} \PYG{o}{*}
\PYG{n}{use} \PYG{n}{persnr} \PYG{n}{yp0101} \PYG{n}{yp10601} \PYG{n}{using} \PYG{l+s+s2}{\PYGZdq{}}\PYG{l+s+s2}{\PYGZdl{}}\PYG{l+s+si}{\PYGZob{}MY\PYGZus{}PATH\PYGZus{}IN\PYGZcb{}}\PYG{l+s+s2}{yp.dta}\PYG{l+s+s2}{\PYGZdq{}}
\PYG{n}{sort} \PYG{n}{persnr}
\PYG{n}{save} \PYG{l+s+s2}{\PYGZdq{}}\PYG{l+s+s2}{\PYGZdl{}}\PYG{l+s+si}{\PYGZob{}MY\PYGZus{}PATH\PYGZus{}OUT\PYGZcb{}}\PYG{l+s+s2}{yp.dta}\PYG{l+s+s2}{\PYGZdq{}}\PYG{p}{,} \PYG{n}{replace}
\PYG{n}{clear}

\end{sphinxVerbatim}

With the help of a unique indicator, which is either the household number (\$hhnr) or the person number (persnr), you can now merge all data records or individual variables to ppfad. Which indicator to use and when depends on the unit of analysis. Since we are on the person level, our indicator is persnr (person ID).

We load the dataset ppfad and merge our datasets or variables to ppfad.

\fvset{hllines={, ,}}%
\begin{sphinxVerbatim}[commandchars=\\\{\},numbers=left,firstnumber=1,stepnumber=1]

merge 1:1 persnr using \PYGZdq{}\PYGZdl{}\PYGZob{}MY\PYGZus{}PATH\PYGZus{}OUT\PYGZcb{}phrf.dta\PYGZdq{}, keep(master match) nogen


* merge data from \PYGZdl{}p.dta 
merge 1:1 persnr using \PYGZdq{}\PYGZdl{}\PYGZob{}MY\PYGZus{}PATH\PYGZus{}IN\PYGZcb{}/wp.dta\PYGZdq{}, keepus(wp0101 wp9301)  keep(master match) nogen // health \PYGZam{} smoking
merge 1:1 persnr using \PYGZdq{}\PYGZdl{}\PYGZob{}MY\PYGZus{}PATH\PYGZus{}IN\PYGZcb{}/xp.dta\PYGZdq{}, keepus(xp0101) 		  keep(master match) nogen // health
merge 1:1 persnr using \PYGZdq{}\PYGZdl{}\PYGZob{}MY\PYGZus{}PATH\PYGZus{}IN\PYGZcb{}/yp.dta\PYGZdq{}, keepus(yp0101 yp10601) keep(master match) nogen // health \PYGZam{} smoking

* merge data from \PYGZdl{}pgen.dta 
local y = 6
foreach wave in w x y \PYGZob{}
	merge 1:1 persnr using \PYGZdq{}\PYGZdl{}\PYGZob{}MY\PYGZus{}PATH\PYGZus{}IN\PYGZcb{}/{}`wave\PYGZsq{}pgen.dta\PYGZdq{}, keepus(emplst0{}`y\PYGZsq{})nogen keep(master match) 
	local y = {}`y\PYGZsq{} + 1
\PYGZcb{}

* merge data from \PYGZdl{}hgen.dta 
local y = 6
foreach wave in w x y \PYGZob{}
	merge m:1 {}`wave\PYGZsq{}hhnr using \PYGZdq{}\PYGZdl{}\PYGZob{}MY\PYGZus{}PATH\PYGZus{}IN\PYGZcb{}/{}`wave\PYGZsq{}hgen.dta\PYGZdq{}, keepus(hinc0{}`y\PYGZsq{}) nogen keep(master match) 
	local y = {}`y\PYGZsq{} + 1
\PYGZcb{}

\end{sphinxVerbatim}

\sphinxstylestrong{2. Encode missing values in system failings (STATA)!}

After the master file has been created with all required information, the missing values, which can take between -1 to -8 in SOEP, must be recoded into missings. This step is important for converting a wide-format data set to a long format.

\fvset{hllines={, ,}}%
\begin{sphinxVerbatim}[commandchars=\\\{\},numbers=left,firstnumber=1,stepnumber=1]
********************************************************************************
*** Task 2) ***
* Encode missing values in system failings (STATA)!
********************************************************************************

	mvdecode \PYGZus{}all, mv(\PYGZhy{}1=. \PYGZbs{} \PYGZhy{}2=.t \PYGZbs{} \PYGZhy{}3=.x \PYGZbs{} \PYGZhy{}5=.y \PYGZbs{} \PYGZhy{}8=.z)
\end{sphinxVerbatim}

\sphinxstylestrong{3. The data set is in wide-format, i.e. additional years are displayed as additional variables (columns). For many analyses it makes sense to convert data sets into the long format. In long format, additional years are displayed as additional lines. If the data record covers three years, as in this example, there are three lines for each person. Convert the data set to long format using the STATA command reshape.!}

Since these are cross-section variables, it can be assumed that each variable has at least one wave abbreviation, which makes the variable unique. Conversely, this means that the variables must be renamed before the reshape command.

Before renaming all original variables (e.g. from \$P data records) it must be checked whether the question and the answer categories were the same in all years (you can also look up the exact wording of the question in the corresponding questionnaire). If changes are made, the variables may have to be recoded.

\fvset{hllines={, ,}}%
\begin{sphinxVerbatim}[commandchars=\\\{\},numbers=left,firstnumber=1,stepnumber=1]
\PYG{o}{*}\PYG{n}{Check} \PYG{k}{if} \PYG{n}{original} \PYG{n}{variable} \PYG{n}{have} \PYG{n}{changed} \PYG{n}{over} \PYG{n}{time}
	\PYG{n}{tab1} \PYG{n}{wp0101} \PYG{n}{xp0101} \PYG{n}{yp0101}
	\PYG{n}{tab1} \PYG{n}{wp9301} \PYG{n}{yp10601}
	\PYG{o}{/}\PYG{o}{*}\PYG{n}{additionally} \PYG{n}{check} \PYG{n}{questionaires} \PYG{k}{for} \PYG{n}{exact} \PYG{n}{wording}\PYG{o}{*}\PYG{o}{/}
\end{sphinxVerbatim}

How you rename the variables is largely up to you. However, you should ensure that the name remains consistent over time and that the variable only differs according to the year (variable name + four-digit year suffix, e.g. zufr2006, zufr2007, zufr2008). You can rename the variables either manually, line by line, or for advanced users using a loop.

Example of manual renaming:

\fvset{hllines={, ,}}%
\begin{sphinxVerbatim}[commandchars=\\\{\},numbers=left,firstnumber=1,stepnumber=1]
\PYG{o}{*}\PYG{n}{rename} \PYG{n}{time}\PYG{o}{\PYGZhy{}}\PYG{n}{variant} \PYG{n}{variables}
\PYG{o}{*}\PYG{k}{with} \PYG{n}{examples} \PYG{n}{how} \PYG{n}{to} \PYG{n}{use} \PYG{n}{loops} \PYG{p}{(}\PYG{n}{but} \PYG{n}{can} \PYG{n}{also} \PYG{n}{be} \PYG{n}{done} \PYG{l+s+s2}{\PYGZdq{}}\PYG{l+s+s2}{manually}\PYG{l+s+s2}{\PYGZdq{}}\PYG{p}{)}
	\PYG{n}{rename} \PYG{n}{wp9301} \PYG{n}{smoke2006}
	\PYG{n}{rename} \PYG{n}{yp10601} \PYG{n}{smoke2008}
	\PYG{n}{rename} \PYG{n}{wp0101} \PYG{n}{health2006}
	\PYG{n}{rename} \PYG{n}{xp0101} \PYG{n}{health2007}
	\PYG{n}{rename} \PYG{n}{yp0101} \PYG{n}{health2008}
	\PYG{o}{.}\PYG{o}{.}\PYG{o}{.}
\end{sphinxVerbatim}

Example of a loop:

\fvset{hllines={, ,}}%
\begin{sphinxVerbatim}[commandchars=\\\{\},numbers=left,firstnumber=1,stepnumber=1]
	foreach  x in 6 7 8 \PYGZob{}
		rename hinc0{}`x\PYGZsq{} hinc200{}`x\PYGZsq{}
		rename emplst0{}`x\PYGZsq{} emplst200{}`x\PYGZsq{}
		\PYGZcb{}

		
	local y=2006
	foreach w in w x y \PYGZob{}
		rename {}`w\PYGZsq{}hhnr hhnrakt{}`y\PYGZsq{}
		rename {}`w\PYGZsq{}netto netto{}`y\PYGZsq{}
		rename {}`w\PYGZsq{}pop pop{}`y\PYGZsq{}
		rename {}`w\PYGZsq{}phrf phrf{}`y\PYGZsq{}
		local y={}`y\PYGZsq{}+1
		\PYGZcb{}
\end{sphinxVerbatim}

\sphinxstylestrong{3.1. The reshape-command}

Now that we have made all relevant preparations, you can start to convert the data set.
If you want to convert a data set, you can do this in both directions:

\begin{figure}[H]
\centering

\noindent\sphinxincludegraphics{{aufgabe_3_reshape}.PNG}
\end{figure}

In our case we reshap from wide to long. This means that a new variable name must be assigned for the year of the survey (j). The variable is then generated automatically.  Currently, each person is assigned a line in Stata.


\begin{savenotes}\sphinxattablestart
\centering
\begin{tabulary}{\linewidth}[t]{|T|T|T|T|T|T|}
\hline
\sphinxstyletheadfamily 
persnr
&\sphinxstyletheadfamily 
hhnr
&\sphinxstyletheadfamily 
wave
&\sphinxstyletheadfamily 
sex
&\sphinxstyletheadfamily 
smoke2006
&\sphinxstyletheadfamily 
smoke2008
\\
\hline
12345
&
123
&
x
&
m
&
yes
&
yes
\\
\hline
54321
&
211
&
x
&
m
&
no
&
no
\\
\hline
\end{tabulary}
\par
\sphinxattableend\end{savenotes}

\fvset{hllines={, ,}}%
\begin{sphinxVerbatim}[commandchars=\\\{\},numbers=left,firstnumber=1,stepnumber=1]
\PYG{o}{*}\PYG{n}{reshape} \PYG{n}{dataset} \PYG{n}{to} \PYG{n}{long}\PYG{o}{\PYGZhy{}}\PYG{n+nb}{format}
	\PYG{n}{reshape} \PYG{n}{long} \PYG{n}{health} \PYG{n}{smoke} \PYG{n}{emplst} \PYG{n}{hinc} \PYG{n}{netto} \PYG{n}{pop} \PYG{n}{hhnrakt} \PYG{n}{phrf}\PYG{p}{,} \PYG{n}{i}\PYG{p}{(}\PYG{n}{persnr}\PYG{p}{)} \PYG{n}{j}\PYG{p}{(}\PYG{n}{year}\PYG{p}{)}
	\PYG{n}{bys} \PYG{n}{persnr}\PYG{p}{:} \PYG{n}{gen} \PYG{n}{waves}\PYG{o}{=}\PYG{n}{\PYGZus{}N}		\PYG{o}{/}\PYG{o}{*}\PYG{n}{additional} \PYG{n}{information}\PYG{p}{:} \PYG{n}{count} \PYG{n}{number} \PYG{n}{of} \PYG{n}{waves} \PYG{n}{per} \PYG{n}{person}\PYG{o}{*}\PYG{o}{/}
	\PYG{n}{tab} \PYG{n}{waves}
\end{sphinxVerbatim}

After the reshape command you have one line per year for each person:


\begin{savenotes}\sphinxattablestart
\centering
\begin{tabulary}{\linewidth}[t]{|T|T|T|T|T|T|}
\hline
\sphinxstyletheadfamily 
persnr
&\sphinxstyletheadfamily 
hhnr
&\sphinxstyletheadfamily 
wave
&\sphinxstyletheadfamily 
year
&\sphinxstyletheadfamily 
sex
&\sphinxstyletheadfamily 
smoke
\\
\hline
12345
&
123
&
x
&
2006
&
m
&
yes
\\
\hline
12345
&
123
&
y
&
2007
&
m
&
.
\\
\hline
12345
&
123
&
z
&
2008
&
m
&
yes
\\
\hline
\end{tabulary}
\par
\sphinxattableend\end{savenotes}

\sphinxstylestrong{4. Perform analyses based on the data. Try to answer the following questions:}

\sphinxstylestrong{a. Has average satisfaction with men’s and women’s health changed over the three years?}

Satisfaction with health was measured on a scale of 10, with a value of 10 representing an extraordinarily high level of satisfaction. To compare the average satisfaction with health between women and men, you should display the mean value for both sexes. The mean value is displayed weighted here.

\fvset{hllines={, ,}}%
\begin{sphinxVerbatim}[commandchars=\\\{\},numbers=left,firstnumber=1,stepnumber=1]
*a) Has the average satisfaction with men\PYGZsq{}s health and women changed 
*   over the three years?

	  mean health [pw=phrf], over(sex year)
\end{sphinxVerbatim}

\begin{figure}[H]
\centering

\noindent\sphinxincludegraphics{{mean_health}.PNG}
\end{figure}

The output shows the average values for men and women for all three years. The first three values show average satisfaction with men’s health between 2006 and 2008, while the last three values show average satisfaction with women’s health.

\sphinxstylestrong{b. What is the proportion of people for whom health satisfaction has increased from 2006 to 2007?}

To answer this question, the difference between 2006 and 2007 should be displayed. You should make sure that only within one persnr (person ID) and the satisfaction of the following year should be analyzed.

\fvset{hllines={, ,}}%
\begin{sphinxVerbatim}[commandchars=\\\{\},numbers=left,firstnumber=1,stepnumber=1]
*b) What is the proportion of people for whom health satisfaction has increased 
*   from 2006 to 2007?? 
	sort persnr year
	gen diff=health\PYGZhy{}health[\PYGZus{}n\PYGZhy{}1] if persnr==persnr[\PYGZus{}n\PYGZhy{}1] \PYGZam{} year==year[\PYGZus{}n\PYGZhy{}1]+1
	tab diff if year==2007				/*unweighted*/
\end{sphinxVerbatim}

\begin{figure}[H]
\centering

\noindent\sphinxincludegraphics{{compare_health_unweighted}.PNG}
\end{figure}

Since you have previously added the SOEP weighting factors to your analysis data set, you should use the weighting for a representative analysis.

\fvset{hllines={, ,}}%
\begin{sphinxVerbatim}[commandchars=\\\{\},numbers=left,firstnumber=1,stepnumber=1]
	\PYG{n}{tab} \PYG{n}{diff} \PYG{k}{if} \PYG{n}{year}\PYG{o}{==}\PYG{l+m+mi}{2007} \PYG{p}{[}\PYG{n}{aw}\PYG{o}{=}\PYG{n}{phrf}\PYG{p}{]}	\PYG{o}{/}\PYG{o}{*}\PYG{n}{weighted}\PYG{o}{*}\PYG{o}{/}
\end{sphinxVerbatim}

\begin{figure}[H]
\centering

\noindent\sphinxincludegraphics{{compare_health_weighted}.PNG}
\end{figure}

The values less than 0 show a deterioration in health satisfaction. The value 0 means a constant health satisfaction and all values above 0 show a positive change in satisfaction with their health. With a value of 10, it can be assumed that these people were interviewed for the first time in 2007 or 2008.

\sphinxstylestrong{c. In what direction and how much has satisfaction with the health of people who quit smoking after 2006 changed from 2006 to 2008?}

The procedure is similar to the previous question, except that the element “smoke yes/no” is added.

\fvset{hllines={, ,}}%
\begin{sphinxVerbatim}[commandchars=\\\{\},numbers=left,firstnumber=1,stepnumber=1]
*c) In what direction and how much has satisfaction with the health of 
*   people who quit smoking after 2006 changed from 2006 to 2008?

	gen diff2=health\PYGZhy{}health[\PYGZus{}n\PYGZhy{}2] if persnr==persnr[\PYGZus{}n\PYGZhy{}2] \PYGZam{} year==year[\PYGZus{}n\PYGZhy{}2]+2 \PYGZam{} year==2008
	gen quit=.
	replace quit=0 if smoke==1 \PYGZam{} smoke[\PYGZus{}n\PYGZhy{}2]==1 \PYGZam{} persnr==persnr[\PYGZus{}n\PYGZhy{}2] \PYGZam{} year==year[\PYGZus{}n\PYGZhy{}2]+2 \PYGZam{} year==2008
	replace quit=1 if smoke==2 \PYGZam{} smoke[\PYGZus{}n\PYGZhy{}2]==1 \PYGZam{} persnr==persnr[\PYGZus{}n\PYGZhy{}2] \PYGZam{} year==year[\PYGZus{}n\PYGZhy{}2]+2 \PYGZam{} year==2008
	replace quit=2 if smoke==2 \PYGZam{} smoke[\PYGZus{}n\PYGZhy{}2]==2 \PYGZam{} persnr==persnr[\PYGZus{}n\PYGZhy{}2] \PYGZam{} year==year[\PYGZus{}n\PYGZhy{}2]+2 \PYGZam{} year==2008
	replace quit=3 if smoke==1 \PYGZam{} smoke[\PYGZus{}n\PYGZhy{}2]==2 \PYGZam{} persnr==persnr[\PYGZus{}n\PYGZhy{}2] \PYGZam{} year==year[\PYGZus{}n\PYGZhy{}2]+2 \PYGZam{} year==2008
	label define quit 0 \PYGZdq{}smoker\PYGZdq{} 1 \PYGZdq{}quit\PYGZdq{} 2 \PYGZdq{}non\PYGZhy{}smoker\PYGZdq{} 3 \PYGZdq{}begin\PYGZdq{}
	label values quit quit
	tabstat diff2, by(quit)
\end{sphinxVerbatim}

\begin{figure}[H]
\centering

\noindent\sphinxincludegraphics{{smoke_vs_health}.PNG}
\end{figure}

To obtain a weighted mean value, address the analysis weight after the generated variable.

\fvset{hllines={, ,}}%
\begin{sphinxVerbatim}[commandchars=\\\{\},numbers=left,firstnumber=1,stepnumber=1]
	\PYG{n}{tabstat} \PYG{n}{diff2} \PYG{p}{[}\PYG{n}{aw}\PYG{o}{=}\PYG{n}{phrf}\PYG{p}{]}\PYG{p}{,} \PYG{n}{by}\PYG{p}{(}\PYG{n}{quit}\PYG{p}{)}	\PYG{o}{/}\PYG{o}{*}\PYG{n}{weighted}\PYG{o}{*}\PYG{o}{/}
\end{sphinxVerbatim}

\begin{figure}[H]
\centering

\noindent\sphinxincludegraphics{{smoke_vs_health_weight}.PNG}
\end{figure}

This illustration shows the mean of the health variable under the condition of the variable quit we generated beforehand. With a mean of -0.24 (weighted -0.35) the biggest change in health satisfaction is seen in people who quit smoking after 2006. For example, if a person smoked in 2006 and indicated a satisfaction value of 8, the person after he/she stopped smoking in 2008 indicates a satisfaction value of 7.76. So you can assume that when a person stops smoking, the state of health that a person perceives deteriorates. Now we have to test if the assumption is correct.

\sphinxstylestrong{d. Does quit smoking make your health worse? To what extent can the result of the analysis “Stop smoking” be distorted?}

In order to establish a connection between health satisfaction and stopping smoking, one should use the ttest or to be more specific, the one-sample t test. It checks whether the mean value of a sample deviates significantly from a known expected value (specified in the null hypothesis).

\fvset{hllines={, ,}}%
\begin{sphinxVerbatim}[commandchars=\\\{\},numbers=left,firstnumber=1,stepnumber=1]
*d) Does quitting smoking make your health worse? To what extent can the 
*   result of the analysis \PYGZdq{}Stop smoking\PYGZdq{} be distorted?
	
	* Notes: So far we have not tested whether the difference is statistically significant
		ttest diff2==0 if quit==1 		
\end{sphinxVerbatim}

\begin{figure}[H]
\centering

\noindent\sphinxincludegraphics{{ttest}.PNG}
\end{figure}

\sphinxstyleemphasis{H0 Hypothesis: If one stops smoking it has no effect on health.}

For this test we assume a 95\% probability. What we want to check now is whether the H0 hypothesis can be rejected or not. If you look at the output of the test, you first see the mean value of value 1 (quit smoking) of the variable quit. The last line of the output shows the significance level. If it falls below the value 0.05, one can speak of a statistically significant result. In our example, the null hypothesis can be discarded because its value is less than 0.05 percent. So quitting smoking has a significant impact on a person’s perceived health.


\section{Longitudinal Data Analysis}
\label{\detokenize{Working with SOEP Data/index:longitudinal-data-analysis}}
Simple cross section analyses show that married people have a higher life satisfaction than singles. You want to check this on the basis of longitudinal analyses with the SOEP.

\sphinxstylestrong{Create an exercise path with four subfolders:}

\begin{figure}[H]
\centering

\noindent\sphinxincludegraphics{{uebungspfade}.PNG}
\end{figure}

\sphinxstylestrong{Example:}
\begin{itemize}
\item {} 
H:/material/exercises/do

\item {} 
H:/material/exercises/output

\item {} 
H:/material/exercises/temp

\item {} 
H:/material/exercises/log

\end{itemize}

These are used to store your script, log files, datasets and temporary datasets. Open an empty do file and define your created paths with globals:

\fvset{hllines={, ,}}%
\begin{sphinxVerbatim}[commandchars=\\\{\},numbers=left,firstnumber=1,stepnumber=1]
***********************************************
* Set some useful commands
***********************************************
version 13
clear all
set more off
**increase buffer size
set scrollbufsize 2000000
**now restart stata!

***********************************************
* Set relative paths to the working directory
***********************************************
global AVZ 	\PYGZdq{}H:\PYGZbs{}material\PYGZbs{}exercises\PYGZdq{}
global MY\PYGZus{}IN\PYGZus{}PATH \PYGZdq{}\PYGZbs{}\PYGZbs{}hume\PYGZbs{}rdc\PYGZhy{}prod\PYGZbs{}distribution\PYGZbs{}soep\PYGZhy{}long\PYGZbs{}soep.v33.1\PYGZbs{}stata\PYGZus{}en\PYGZbs{}\PYGZdq{}
global MY\PYGZus{}DO\PYGZus{}FILES \PYGZdq{}\PYGZdl{}AVZ\PYGZbs{}do\PYGZbs{}\PYGZdq{}
global MY\PYGZus{}LOG\PYGZus{}OUT \PYGZdq{}\PYGZdl{}AVZ\PYGZbs{}log\PYGZbs{}\PYGZdq{}
global MY\PYGZus{}OUT\PYGZus{}DATA \PYGZdq{}\PYGZdl{}AVZ\PYGZbs{}output\PYGZbs{}\PYGZdq{}
global MY\PYGZus{}OUT\PYGZus{}TEMP \PYGZdq{}\PYGZdl{}AVZ\PYGZbs{}temp\PYGZbs{}\PYGZdq{}
\end{sphinxVerbatim}

The global „AVZ“ defines the main path. The main paths are subdivided using the globals “MY\_IN\_PATH”, “MY\_DO\_FILES”, “MY\_LOG\_OUT”, “MY\_OUT\_DATA”, “MY\_OUT\_TEMP”. The global “MY\_IN\_PATH” contains the path to your ordered data.

\sphinxstylestrong{Create a master file that uses the important variables from ppfadl.}

You should always add some variables from PPFADL to your data set by default.
Download the following information from PPFADL:
\begin{itemize}
\item {} 
Person ID  \href{https://paneldata.org/soep-long/data/ppfadl/pid}{\textbf{"pid"}}

\item {} 
Household number  \href{https://paneldata.org/soep-long/data/ppfadl/pid}{\textbf{"pid"}}

\item {} 
Survey year  \href{https://paneldata.org/soep-long/data/ppfadl/syear}{\textbf{"syear"}}

\item {} 
The net variable with information on the interview type  \href{https://paneldata.org/soep-long/data/ppfadl/netto}{\textbf{"netto"}}

\item {} 
The weighting variable  \href{https://paneldata.org/soep-long/data/ppfadl/phrf}{\textbf{"phrf"}}

\item {} 
The sex of the person  \href{https://paneldata.org/soep-long/data/ppfadl/sex}{\textbf{"sex"}}

\item {} 
The migration background  \href{https://paneldata.org/soep-long/data/ppfadl/migback}{\textbf{"migback"}}

\end{itemize}

\fvset{hllines={, ,}}%
\begin{sphinxVerbatim}[commandchars=\\\{\},numbers=left,firstnumber=1,stepnumber=1]
*\PYGZhy{}\PYGZhy{}\PYGZhy{}\PYGZhy{}\PYGZhy{}\PYGZhy{}\PYGZhy{}\PYGZhy{}\PYGZhy{}\PYGZhy{}\PYGZhy{}\PYGZhy{}\PYGZhy{}\PYGZhy{}\PYGZhy{}\PYGZhy{}\PYGZhy{}\PYGZhy{}\PYGZhy{}\PYGZhy{}\PYGZhy{}\PYGZhy{}\PYGZhy{}\PYGZhy{}\PYGZhy{}\PYGZhy{}\PYGZhy{}\PYGZhy{}\PYGZhy{}\PYGZhy{}\PYGZhy{}\PYGZhy{}\PYGZhy{}\PYGZhy{}\PYGZhy{}\PYGZhy{}\PYGZhy{}\PYGZhy{}\PYGZhy{}\PYGZhy{}\PYGZhy{}\PYGZhy{}\PYGZhy{}\PYGZhy{}\PYGZhy{}\PYGZhy{}\PYGZhy{}\PYGZhy{}\PYGZhy{}\PYGZhy{}\PYGZhy{}\PYGZhy{}\PYGZhy{}\PYGZhy{}\PYGZhy{}\PYGZhy{}\PYGZhy{}\PYGZhy{}\PYGZhy{}\PYGZhy{}\PYGZhy{}\PYGZhy{}\PYGZhy{}\PYGZhy{}\PYGZhy{}\PYGZhy{}\PYGZhy{}\PYGZhy{}\PYGZhy{}\PYGZhy{}\PYGZhy{}\PYGZhy{}\PYGZhy{}\PYGZhy{}\PYGZhy{}\PYGZhy{}\PYGZhy{}\PYGZhy{}\PYGZhy{}
*** Step 1) Start with basic information from PPFADL ***

use pid hid syear netto phrf migback sex using \PYGZdl{}\PYGZob{}MY\PYGZus{}IN\PYGZus{}PATH\PYGZcb{}\PYGZbs{}ppfadl.dta 
\end{sphinxVerbatim}

\sphinxstylestrong{Search for matching variables and add them to your data set}

To perform your analysis, you need different SOEP variables. The SOEP offers various options for a variable search:
\begin{itemize}
\item {} 
Search the questionnaires for useful variables. (for more information visit the chapter {\hyperref[\detokenize{Working with SOEP Documentation/index:quest-search}]{\sphinxcrossref{\DUrole{std,std-ref}{Variable Search with Questionnaires}}}})

\item {} 
Find a suitable variable via the topic list of paneldata.org (for more information visit the chapter {\hyperref[\detokenize{Working with SOEP Documentation/index:topic}]{\sphinxcrossref{\DUrole{std,std-ref}{Topic Search with paneldata.org}}}})

\item {} 
Search for a suitable variable using a search term in paneldata.org (for more information visit the chapter {\hyperref[\detokenize{Working with SOEP Documentation/index:var-search}]{\sphinxcrossref{\DUrole{std,std-ref}{Variable Search with paneldata.org}}}})

\item {} 
Use the documentation provided by the generated variables (for more information visit the chapter {\hyperref[\detokenize{Working with SOEP Documentation/index:documentation}]{\sphinxcrossref{\DUrole{std,std-ref}{Documentation of Generated Data}}}})

\end{itemize}

In this case you need the variables  \href{https://paneldata.org/soep-long/data/pgen/pgfamstd}{\textbf{"pgfamstd"}} (martial status) and  \href{https://paneldata.org/soep-long/data/pl/plh0182}{\textbf{"plh0182"}} (life satisfaction).

\fvset{hllines={, ,}}%
\begin{sphinxVerbatim}[commandchars=\\\{\},numbers=left,firstnumber=1,stepnumber=1]
*\PYGZhy{}\PYGZhy{}\PYGZhy{}\PYGZhy{}\PYGZhy{}\PYGZhy{}\PYGZhy{}\PYGZhy{}\PYGZhy{}\PYGZhy{}\PYGZhy{}\PYGZhy{}\PYGZhy{}\PYGZhy{}\PYGZhy{}\PYGZhy{}\PYGZhy{}\PYGZhy{}\PYGZhy{}\PYGZhy{}\PYGZhy{}\PYGZhy{}\PYGZhy{}\PYGZhy{}\PYGZhy{}\PYGZhy{}\PYGZhy{}\PYGZhy{}\PYGZhy{}\PYGZhy{}\PYGZhy{}\PYGZhy{}\PYGZhy{}\PYGZhy{}\PYGZhy{}\PYGZhy{}\PYGZhy{}\PYGZhy{}\PYGZhy{}\PYGZhy{}\PYGZhy{}\PYGZhy{}\PYGZhy{}\PYGZhy{}\PYGZhy{}\PYGZhy{}\PYGZhy{}\PYGZhy{}\PYGZhy{}\PYGZhy{}\PYGZhy{}\PYGZhy{}\PYGZhy{}\PYGZhy{}\PYGZhy{}\PYGZhy{}\PYGZhy{}\PYGZhy{}\PYGZhy{}\PYGZhy{}\PYGZhy{}\PYGZhy{}\PYGZhy{}\PYGZhy{}\PYGZhy{}\PYGZhy{}\PYGZhy{}\PYGZhy{}\PYGZhy{}\PYGZhy{}\PYGZhy{}\PYGZhy{}\PYGZhy{}\PYGZhy{}\PYGZhy{}\PYGZhy{}\PYGZhy{}\PYGZhy{}\PYGZhy{}
*** Step 2) Add the relavant variables: here: family status and life satisfaction ***
merge 1:1 pid syear using \PYGZdl{}\PYGZob{}MY\PYGZus{}IN\PYGZus{}PATH\PYGZcb{}\PYGZbs{}pgen, keepusing(pgfamstd) keep(1 3) nogen	

		// merges family status from pgen
		// Documentation for PGEN can be found here
		// http://panel.gsoep.de/soep\PYGZhy{}docs/surveypapers/diw\PYGZus{}ssp0307.pdf)

		
*describe using pl (directory)
		// for checking out variable names without opening the dataset
		
merge 1:1 pid syear using \PYGZdl{}\PYGZob{}MY\PYGZus{}IN\PYGZus{}PATH\PYGZcb{}\PYGZbs{}pl, keepusing(plh0182) keep(1 3) nogen
		// merges life satisfaction from pl 

save \PYGZdl{}MY\PYGZus{}OUT\PYGZus{}DATA\PYGZbs{}ppfad.dta, replace
\end{sphinxVerbatim}

\sphinxstylestrong{Clean and inspect the data}

Recode all missings into the format of a point.

\fvset{hllines={, ,}}%
\begin{sphinxVerbatim}[commandchars=\\\{\},numbers=left,firstnumber=1,stepnumber=1]
\PYG{o}{*}\PYG{o}{\PYGZhy{}}\PYG{o}{\PYGZhy{}}\PYG{o}{\PYGZhy{}}\PYG{o}{\PYGZhy{}}\PYG{o}{\PYGZhy{}}\PYG{o}{\PYGZhy{}}\PYG{o}{\PYGZhy{}}\PYG{o}{\PYGZhy{}}\PYG{o}{\PYGZhy{}}\PYG{o}{\PYGZhy{}}\PYG{o}{\PYGZhy{}}\PYG{o}{\PYGZhy{}}\PYG{o}{\PYGZhy{}}\PYG{o}{\PYGZhy{}}\PYG{o}{\PYGZhy{}}\PYG{o}{\PYGZhy{}}\PYG{o}{\PYGZhy{}}\PYG{o}{\PYGZhy{}}\PYG{o}{\PYGZhy{}}\PYG{o}{\PYGZhy{}}\PYG{o}{\PYGZhy{}}\PYG{o}{\PYGZhy{}}\PYG{o}{\PYGZhy{}}\PYG{o}{\PYGZhy{}}\PYG{o}{\PYGZhy{}}\PYG{o}{\PYGZhy{}}\PYG{o}{\PYGZhy{}}\PYG{o}{\PYGZhy{}}\PYG{o}{\PYGZhy{}}\PYG{o}{\PYGZhy{}}\PYG{o}{\PYGZhy{}}\PYG{o}{\PYGZhy{}}\PYG{o}{\PYGZhy{}}\PYG{o}{\PYGZhy{}}\PYG{o}{\PYGZhy{}}\PYG{o}{\PYGZhy{}}\PYG{o}{\PYGZhy{}}\PYG{o}{\PYGZhy{}}\PYG{o}{\PYGZhy{}}\PYG{o}{\PYGZhy{}}\PYG{o}{\PYGZhy{}}\PYG{o}{\PYGZhy{}}\PYG{o}{\PYGZhy{}}\PYG{o}{\PYGZhy{}}\PYG{o}{\PYGZhy{}}\PYG{o}{\PYGZhy{}}\PYG{o}{\PYGZhy{}}\PYG{o}{\PYGZhy{}}\PYG{o}{\PYGZhy{}}\PYG{o}{\PYGZhy{}}\PYG{o}{\PYGZhy{}}\PYG{o}{\PYGZhy{}}\PYG{o}{\PYGZhy{}}\PYG{o}{\PYGZhy{}}\PYG{o}{\PYGZhy{}}\PYG{o}{\PYGZhy{}}\PYG{o}{\PYGZhy{}}\PYG{o}{\PYGZhy{}}\PYG{o}{\PYGZhy{}}\PYG{o}{\PYGZhy{}}\PYG{o}{\PYGZhy{}}\PYG{o}{\PYGZhy{}}\PYG{o}{\PYGZhy{}}\PYG{o}{\PYGZhy{}}\PYG{o}{\PYGZhy{}}\PYG{o}{\PYGZhy{}}\PYG{o}{\PYGZhy{}}\PYG{o}{\PYGZhy{}}\PYG{o}{\PYGZhy{}}\PYG{o}{\PYGZhy{}}\PYG{o}{\PYGZhy{}}\PYG{o}{\PYGZhy{}}\PYG{o}{\PYGZhy{}}\PYG{o}{\PYGZhy{}}\PYG{o}{\PYGZhy{}}\PYG{o}{\PYGZhy{}}\PYG{o}{\PYGZhy{}}\PYG{o}{\PYGZhy{}}\PYG{o}{\PYGZhy{}}
\PYG{o}{*}\PYG{o}{*}\PYG{o}{*} \PYG{n}{Step} \PYG{l+m+mi}{3}\PYG{p}{)} \PYG{n}{Clean} \PYG{o+ow}{and} \PYG{n}{inspect} \PYG{n}{the} \PYG{n}{data}
\PYG{n}{mvdecode} \PYG{n}{\PYGZus{}all}\PYG{p}{,} \PYG{n}{mv}\PYG{p}{(}\PYG{o}{\PYGZhy{}}\PYG{l+m+mi}{8}\PYG{o}{/}\PYG{o}{\PYGZhy{}}\PYG{l+m+mi}{1}\PYG{p}{)}
\end{sphinxVerbatim}

Since you are interested in individual characteristics in your analysis: Delete all measurements that are not based on successful personal interviews.

\fvset{hllines={, ,}}%
\begin{sphinxVerbatim}[commandchars=\\\{\},numbers=left,firstnumber=1,stepnumber=1]
\PYG{n}{tab} \PYG{n}{netto}
\PYG{n}{drop} \PYG{k}{if} \PYG{n}{netto}\PYG{o}{\PYGZgt{}}\PYG{l+m+mi}{19}
\end{sphinxVerbatim}

\begin{figure}[H]
\centering

\noindent\sphinxincludegraphics{{SOEPlong_01}.PNG}
\end{figure}

\sphinxstylestrong{How many people contribute measurements and what is the proportion of people contributing at least 10 measurements?}

Define the data set as a panel data set.

\fvset{hllines={, ,}}%
\begin{sphinxVerbatim}[commandchars=\\\{\},numbers=left,firstnumber=1,stepnumber=1]
\PYG{o}{*}\PYG{o}{*}\PYG{n}{define} \PYG{n}{the} \PYG{n}{data} \PYG{n+nb}{set} \PYG{k}{as} \PYG{n}{panel} \PYG{n}{data}
\PYG{n}{xtset} \PYG{n}{pid} \PYG{n}{syear}
\PYG{n}{xtdes}
\end{sphinxVerbatim}

\begin{figure}[H]
\centering

\noindent\sphinxincludegraphics{{SOEPlong_02}.PNG}
\end{figure}

86079 respondents have contributed information within waves a (1984) - bg (2016) and 75\% of the 86079 respondents have provided information for at least 10 waves

\sphinxstylestrong{How many people took part in the survey in 2010 and contributed to continuous measurements until 2014?}

\fvset{hllines={, ,}}%
\begin{sphinxVerbatim}[commandchars=\\\{\},numbers=left,firstnumber=1,stepnumber=1]
\PYG{n}{xtdes} \PYG{k}{if} \PYG{n}{syear}\PYG{o}{\PYGZgt{}}\PYG{o}{=}\PYG{l+m+mi}{2010} \PYG{o}{\PYGZam{}} \PYG{n}{syear}\PYG{o}{\PYGZlt{}}\PYG{o}{=}\PYG{l+m+mi}{2014}
\end{sphinxVerbatim}

\begin{figure}[H]
\centering

\noindent\sphinxincludegraphics{{SOEPlong_03}.PNG}
\end{figure}

14673 respondents provided continuous information from 2010 to 2014.

\sphinxstylestrong{Univariate inspection \& analysis}

\sphinxstylestrong{How does the mean of life satisfaction change over time?}

\fvset{hllines={, ,}}%
\begin{sphinxVerbatim}[commandchars=\\\{\},numbers=left,firstnumber=1,stepnumber=1]
\PYG{o}{*}\PYG{o}{\PYGZhy{}}\PYG{o}{\PYGZhy{}}\PYG{o}{\PYGZhy{}}\PYG{o}{\PYGZhy{}}\PYG{o}{\PYGZhy{}}\PYG{o}{\PYGZhy{}}\PYG{o}{\PYGZhy{}}\PYG{o}{\PYGZhy{}}\PYG{o}{\PYGZhy{}}\PYG{o}{\PYGZhy{}}\PYG{o}{\PYGZhy{}}\PYG{o}{\PYGZhy{}}\PYG{o}{\PYGZhy{}}\PYG{o}{\PYGZhy{}}\PYG{o}{\PYGZhy{}}\PYG{o}{\PYGZhy{}}\PYG{o}{\PYGZhy{}}\PYG{o}{\PYGZhy{}}\PYG{o}{\PYGZhy{}}\PYG{o}{\PYGZhy{}}\PYG{o}{\PYGZhy{}}\PYG{o}{\PYGZhy{}}\PYG{o}{\PYGZhy{}}\PYG{o}{\PYGZhy{}}\PYG{o}{\PYGZhy{}}\PYG{o}{\PYGZhy{}}\PYG{o}{\PYGZhy{}}\PYG{o}{\PYGZhy{}}\PYG{o}{\PYGZhy{}}\PYG{o}{\PYGZhy{}}\PYG{o}{\PYGZhy{}}\PYG{o}{\PYGZhy{}}\PYG{o}{\PYGZhy{}}\PYG{o}{\PYGZhy{}}\PYG{o}{\PYGZhy{}}\PYG{o}{\PYGZhy{}}\PYG{o}{\PYGZhy{}}\PYG{o}{\PYGZhy{}}\PYG{o}{\PYGZhy{}}\PYG{o}{\PYGZhy{}}\PYG{o}{\PYGZhy{}}\PYG{o}{\PYGZhy{}}\PYG{o}{\PYGZhy{}}\PYG{o}{\PYGZhy{}}\PYG{o}{\PYGZhy{}}\PYG{o}{\PYGZhy{}}\PYG{o}{\PYGZhy{}}\PYG{o}{\PYGZhy{}}\PYG{o}{\PYGZhy{}}\PYG{o}{\PYGZhy{}}\PYG{o}{\PYGZhy{}}\PYG{o}{\PYGZhy{}}\PYG{o}{\PYGZhy{}}\PYG{o}{\PYGZhy{}}\PYG{o}{\PYGZhy{}}\PYG{o}{\PYGZhy{}}\PYG{o}{\PYGZhy{}}\PYG{o}{\PYGZhy{}}\PYG{o}{\PYGZhy{}}\PYG{o}{\PYGZhy{}}\PYG{o}{\PYGZhy{}}\PYG{o}{\PYGZhy{}}\PYG{o}{\PYGZhy{}}\PYG{o}{\PYGZhy{}}\PYG{o}{\PYGZhy{}}\PYG{o}{\PYGZhy{}}\PYG{o}{\PYGZhy{}}\PYG{o}{\PYGZhy{}}\PYG{o}{\PYGZhy{}}\PYG{o}{\PYGZhy{}}\PYG{o}{\PYGZhy{}}\PYG{o}{\PYGZhy{}}\PYG{o}{\PYGZhy{}}\PYG{o}{\PYGZhy{}}\PYG{o}{\PYGZhy{}}\PYG{o}{\PYGZhy{}}\PYG{o}{\PYGZhy{}}\PYG{o}{\PYGZhy{}}\PYG{o}{\PYGZhy{}}
\PYG{o}{*}\PYG{o}{*}\PYG{o}{*} \PYG{n}{Step} \PYG{l+m+mi}{4}\PYG{p}{)} \PYG{n}{univariate} \PYG{n}{inspection} \PYG{o}{\PYGZam{}} \PYG{n}{analysis}
\PYG{n}{table} \PYG{n}{syear}\PYG{p}{,} \PYG{n}{content} \PYG{p}{(}\PYG{n}{mean} \PYG{n}{plh0182}\PYG{p}{)}
\end{sphinxVerbatim}

\begin{figure}[H]
\centering

\noindent\sphinxincludegraphics{{SOEPlong_04}.PNG}
\end{figure}

\sphinxstylestrong{How high is the proportion of people who will be a) married in 2014 or b) have a migration background. Compare weighted with unweighted frequency tables: Which people are overrepresented in SOEP?}

\fvset{hllines={, ,}}%
\begin{sphinxVerbatim}[commandchars=\\\{\},numbers=left,firstnumber=1,stepnumber=1]
\PYG{n}{tab1} \PYG{n}{pgfamstd} \PYG{n}{migback} \PYG{k}{if} \PYG{n}{syear}\PYG{o}{==}\PYG{l+m+mi}{2014}
\PYG{n}{tab} \PYG{n}{pgfamstd} \PYG{p}{[}\PYG{n}{aw}\PYG{o}{=}\PYG{n}{phrf}\PYG{p}{]} \PYG{k}{if} \PYG{n}{syear}\PYG{o}{==}\PYG{l+m+mi}{2014}
\PYG{n}{tab} \PYG{n}{migback} \PYG{p}{[}\PYG{n}{aw}\PYG{o}{=}\PYG{n}{phrf}\PYG{p}{]} \PYG{k}{if} \PYG{n}{syear}\PYG{o}{==}\PYG{l+m+mi}{2014}
\end{sphinxVerbatim}

\begin{figure}[H]
\centering

\noindent\sphinxincludegraphics{{SOEPlong_05}.PNG}
\end{figure}

\begin{figure}[H]
\centering

\noindent\sphinxincludegraphics{{SOEPlong_06}.PNG}
\end{figure}

The data show that married people are overrepresented in the SOEP and single people are underrepresented. The weighting makes it representative for Germany again.

\begin{figure}[H]
\centering

\noindent\sphinxincludegraphics{{SOEPlong_05b}.PNG}
\end{figure}

\begin{figure}[H]
\centering

\noindent\sphinxincludegraphics{{SOEPlong_07}.PNG}
\end{figure}

In the SOEP sample, respondents with a direct or indirect migration background are overrepresented.

\sphinxstylestrong{How many of those persons who report an life satisfaction (scale value 7) in a survey year also indicate the scale value 7 in the following survey year?}

\fvset{hllines={, ,}}%
\begin{sphinxVerbatim}[commandchars=\\\{\},numbers=left,firstnumber=1,stepnumber=1]
\PYG{n}{xttrans} \PYG{n}{plh0182}
\end{sphinxVerbatim}

\begin{figure}[H]
\centering

\noindent\sphinxincludegraphics{{SOEPlong_08}.PNG}
\end{figure}

34.57\% of the respondents who reported a life satisfaction of 7 again reported a value of 7 in the following year.

\sphinxstylestrong{Is it more likely that a highly dissatisfied person (value: 0) will be less dissatisfied the following year, or that a very satisfied (value: 10) person will be less satisfied the following year?}

\fvset{hllines={, ,}}%
\begin{sphinxVerbatim}[commandchars=\\\{\},numbers=left,firstnumber=1,stepnumber=1]
\PYG{n}{xttrans} \PYG{n}{plh0182}
\end{sphinxVerbatim}

\begin{figure}[H]
\centering

\noindent\sphinxincludegraphics{{SOEPlong_08}.PNG}
\end{figure}

The rows reflect the initial values, and the columns reflect the final values.
People who were completely dissatisfied (value: 0) in the base year remain completely dissatisfied with around 20 \% in the following year. About 80\% of these dissatisfied people from the base year improve their life satisfaction in the following year. Of the completely satisfied persons (value: 10), about 37\% remain just as satisfied in the following year. For 63\%, however, life satisfaction worsens. It is more likely that a completely dissatisfied person (value: 0) will become more satisfied in the following year.

\sphinxstylestrong{Which transitions in marital status can be observed particularly frequently in the data?}

\fvset{hllines={, ,}}%
\begin{sphinxVerbatim}[commandchars=\\\{\},numbers=left,firstnumber=1,stepnumber=1]
\PYG{n}{xttrans} \PYG{n}{pgfamstd}
\end{sphinxVerbatim}

\begin{figure}[H]
\centering

\noindent\sphinxincludegraphics{{SOEPlong_09}.PNG}
\end{figure}

Survey respondents who were married but separated in the base year and declared a divorce as family status in the following year can be observed particularly frequently. (About 19\%).

\sphinxstylestrong{Simple cross sectional analyses}

You now want to discover the correlation between marital status and life satisfaction. Is there an effect of marriage on life satisfaction? And if so, is this a sustainable effect?

\sphinxstylestrong{First, calculate the correlation between family status and life satisfaction in cross section for 2010: Are married people happier than singles?}

\fvset{hllines={, ,}}%
\begin{sphinxVerbatim}[commandchars=\\\{\},numbers=left,firstnumber=1,stepnumber=1]
\PYG{o}{*}\PYG{o}{\PYGZhy{}}\PYG{o}{\PYGZhy{}}\PYG{o}{\PYGZhy{}}\PYG{o}{\PYGZhy{}}\PYG{o}{\PYGZhy{}}\PYG{o}{\PYGZhy{}}\PYG{o}{\PYGZhy{}}\PYG{o}{\PYGZhy{}}\PYG{o}{\PYGZhy{}}\PYG{o}{\PYGZhy{}}\PYG{o}{\PYGZhy{}}\PYG{o}{\PYGZhy{}}\PYG{o}{\PYGZhy{}}\PYG{o}{\PYGZhy{}}\PYG{o}{\PYGZhy{}}\PYG{o}{\PYGZhy{}}\PYG{o}{\PYGZhy{}}\PYG{o}{\PYGZhy{}}\PYG{o}{\PYGZhy{}}\PYG{o}{\PYGZhy{}}\PYG{o}{\PYGZhy{}}\PYG{o}{\PYGZhy{}}\PYG{o}{\PYGZhy{}}\PYG{o}{\PYGZhy{}}\PYG{o}{\PYGZhy{}}\PYG{o}{\PYGZhy{}}\PYG{o}{\PYGZhy{}}\PYG{o}{\PYGZhy{}}\PYG{o}{\PYGZhy{}}\PYG{o}{\PYGZhy{}}\PYG{o}{\PYGZhy{}}\PYG{o}{\PYGZhy{}}\PYG{o}{\PYGZhy{}}\PYG{o}{\PYGZhy{}}\PYG{o}{\PYGZhy{}}\PYG{o}{\PYGZhy{}}\PYG{o}{\PYGZhy{}}\PYG{o}{\PYGZhy{}}\PYG{o}{\PYGZhy{}}\PYG{o}{\PYGZhy{}}\PYG{o}{\PYGZhy{}}\PYG{o}{\PYGZhy{}}\PYG{o}{\PYGZhy{}}\PYG{o}{\PYGZhy{}}\PYG{o}{\PYGZhy{}}\PYG{o}{\PYGZhy{}}\PYG{o}{\PYGZhy{}}\PYG{o}{\PYGZhy{}}\PYG{o}{\PYGZhy{}}\PYG{o}{\PYGZhy{}}\PYG{o}{\PYGZhy{}}\PYG{o}{\PYGZhy{}}\PYG{o}{\PYGZhy{}}\PYG{o}{\PYGZhy{}}\PYG{o}{\PYGZhy{}}\PYG{o}{\PYGZhy{}}\PYG{o}{\PYGZhy{}}\PYG{o}{\PYGZhy{}}\PYG{o}{\PYGZhy{}}\PYG{o}{\PYGZhy{}}\PYG{o}{\PYGZhy{}}\PYG{o}{\PYGZhy{}}\PYG{o}{\PYGZhy{}}\PYG{o}{\PYGZhy{}}\PYG{o}{\PYGZhy{}}\PYG{o}{\PYGZhy{}}\PYG{o}{\PYGZhy{}}\PYG{o}{\PYGZhy{}}\PYG{o}{\PYGZhy{}}\PYG{o}{\PYGZhy{}}\PYG{o}{\PYGZhy{}}\PYG{o}{\PYGZhy{}}\PYG{o}{\PYGZhy{}}\PYG{o}{\PYGZhy{}}\PYG{o}{\PYGZhy{}}\PYG{o}{\PYGZhy{}}\PYG{o}{\PYGZhy{}}\PYG{o}{\PYGZhy{}}\PYG{o}{\PYGZhy{}}
\PYG{o}{*}\PYG{o}{*}\PYG{o}{*} \PYG{n}{Step} \PYG{l+m+mi}{5}\PYG{p}{)}\PYG{n}{simple} \PYG{n}{cross} \PYG{n}{sectional} \PYG{n}{analyses}
\PYG{n}{table} \PYG{n}{pgfamstd} \PYG{k}{if} \PYG{n}{syear}\PYG{o}{==}\PYG{l+m+mi}{2010}\PYG{p}{,} \PYG{n}{content} \PYG{p}{(}\PYG{n}{mean} \PYG{n}{plh0182}\PYG{p}{)}
\end{sphinxVerbatim}

\begin{figure}[H]
\centering

\noindent\sphinxincludegraphics{{SOEPlong_10}.PNG}
\end{figure}

At first glance, married couples seem happier than singles.

Now generate a variable that indicates a transition from “single” to “married”.

\sphinxstylestrong{How many such transitions can you find in the data?}

\fvset{hllines={, ,}}%
\begin{sphinxVerbatim}[commandchars=\\\{\},numbers=left,firstnumber=1,stepnumber=1]
\PYG{o}{*}\PYG{o}{*}\PYG{o}{*}\PYG{n}{perform} \PYG{n}{longitudinal} \PYG{n}{analysis}
\PYG{o}{*}\PYG{o}{*}\PYG{n}{define} \PYG{n}{event}\PYG{p}{:} \PYG{n}{transition} \PYG{n}{to} \PYG{n}{marriage}	
\PYG{n}{generate} \PYG{n}{to\PYGZus{}mar}\PYG{o}{=}\PYG{l+m+mi}{1} \PYG{k}{if} \PYG{n}{pgfamstd}\PYG{o}{==}\PYG{l+m+mi}{1} \PYG{o}{\PYGZam{}} \PYG{n}{l}\PYG{o}{.}\PYG{n}{pgfamstd}\PYG{o}{==}\PYG{l+m+mi}{3}
\PYG{n}{tab} \PYG{n}{to\PYGZus{}mar}
\end{sphinxVerbatim}

\begin{figure}[H]
\centering

\noindent\sphinxincludegraphics{{SOEPlong_11}.PNG}
\end{figure}

A total of 4834 people can be observed changing from single to married.

\sphinxstylestrong{What is the average level of life satisfaction immediately after the transition to marriage (i.e. in the first survey in which the transition can be observed) and how high is life satisfaction immediately before the transition to marriage?}

\fvset{hllines={, ,}}%
\begin{sphinxVerbatim}[commandchars=\\\{\},numbers=left,firstnumber=1,stepnumber=1]
\PYG{o}{*}\PYG{o}{*}\PYG{n}{standard} \PYG{n}{way} \PYG{n}{of} \PYG{n}{life}\PYG{o}{\PYGZhy{}}\PYG{n}{event} \PYG{n}{analysis}
\PYG{n+nb}{sum} \PYG{n}{plh0182} \PYG{k}{if} \PYG{n}{to\PYGZus{}mar}\PYG{o}{==}\PYG{l+m+mi}{1}
\PYG{n+nb}{sum} \PYG{n}{l}\PYG{o}{.}\PYG{n}{plh0182} \PYG{k}{if} \PYG{n}{to\PYGZus{}mar}\PYG{o}{==}\PYG{l+m+mi}{1}

\PYG{o}{*}\PYG{o}{*}\PYG{n}{alternative} \PYG{n}{way}
\PYG{n}{generate} \PYG{n}{dif\PYGZus{}sat}\PYG{o}{=} \PYG{n}{plh0182}\PYG{o}{\PYGZhy{}} \PYG{n}{l}\PYG{o}{.}\PYG{n}{plh0182}
\PYG{n}{mean} \PYG{n}{dif\PYGZus{}sat} \PYG{k}{if} \PYG{n}{to\PYGZus{}mar}\PYG{o}{==}\PYG{l+m+mi}{1}
\end{sphinxVerbatim}

\begin{figure}[H]
\centering

\noindent\sphinxincludegraphics{{SOEPlong_12}.PNG}
\end{figure}

\begin{figure}[H]
\centering

\noindent\sphinxincludegraphics{{SOEPlong_13}.PNG}
\end{figure}

Before the transition to marriage, the average life satisfaction of the respondents is 7.54. in the following year, i.e. after the transition to marriage, the average life satisfaction of the respondents is 7.65. It can be seen that with the transition to marriage, the average life satisfaction rises slightly by 0.11.

\sphinxstylestrong{Map the complete satisfaction history around the “marriage entry” event {[}3 years before; 3 years after{]}.}

\fvset{hllines={, ,}}%
\begin{sphinxVerbatim}[commandchars=\\\{\},numbers=left,firstnumber=1,stepnumber=1]
\PYG{o}{*}\PYG{o}{*}\PYG{n}{preparing} \PYG{n}{illustration} \PYG{n}{of} \PYG{n}{trajectory}
\PYG{n}{generate} \PYG{n}{t}\PYG{o}{=}\PYG{l+m+mi}{0} \PYG{k}{if} \PYG{n}{to\PYGZus{}mar}\PYG{o}{==}\PYG{l+m+mi}{1} \PYG{o}{\PYGZam{}} \PYG{n}{l}\PYG{o}{.}\PYG{n}{to\PYGZus{}mar}\PYG{o}{\PYGZti{}}\PYG{o}{=}\PYG{l+m+mi}{1} \PYG{o}{\PYGZam{}}\PYG{n}{l2}\PYG{o}{.}\PYG{n}{to\PYGZus{}mar}\PYG{o}{\PYGZti{}}\PYG{o}{=}\PYG{l+m+mi}{1} \PYG{o}{\PYGZam{}} \PYG{n}{l3}\PYG{o}{.}\PYG{n}{to\PYGZus{}mar}\PYG{o}{\PYGZti{}}\PYG{o}{=}\PYG{l+m+mi}{1} \PYG{o}{\PYGZam{}} \PYG{n}{l4}\PYG{o}{.}\PYG{n}{to\PYGZus{}mar}\PYG{o}{\PYGZti{}}\PYG{o}{=}\PYG{l+m+mi}{1} \PYG{o}{\PYGZam{}} \PYG{n}{l5}\PYG{o}{.}\PYG{n}{to\PYGZus{}mar}\PYG{o}{\PYGZti{}}\PYG{o}{=}\PYG{l+m+mi}{1} \PYG{o}{\PYGZam{}} \PYG{n}{l6}\PYG{o}{.}\PYG{n}{to\PYGZus{}mar}\PYG{o}{\PYGZti{}}\PYG{o}{=}\PYG{l+m+mi}{1} \PYG{o}{\PYGZam{}} \PYG{n}{l7}\PYG{o}{.}\PYG{n}{to\PYGZus{}mar}\PYG{o}{\PYGZti{}}\PYG{o}{=}\PYG{l+m+mi}{1} \PYG{o}{\PYGZam{}} \PYG{n}{l8}\PYG{o}{.}\PYG{n}{to\PYGZus{}mar}\PYG{o}{\PYGZti{}}\PYG{o}{=}\PYG{l+m+mi}{1} \PYG{o}{\PYGZam{}} \PYG{n}{l9}\PYG{o}{.}\PYG{n}{to\PYGZus{}mar}\PYG{o}{\PYGZti{}}\PYG{o}{=}\PYG{l+m+mi}{1} \PYG{o}{\PYGZam{}} \PYG{n}{l10}\PYG{o}{.}\PYG{n}{to\PYGZus{}mar}\PYG{o}{\PYGZti{}}\PYG{o}{=}\PYG{l+m+mi}{1} \PYG{o}{\PYGZam{}} \PYG{n}{l11}\PYG{o}{.}\PYG{n}{to\PYGZus{}mar}\PYG{o}{\PYGZti{}}\PYG{o}{=}\PYG{l+m+mi}{1} \PYG{o}{\PYGZam{}} \PYG{n}{l12}\PYG{o}{.}\PYG{n}{to\PYGZus{}mar}\PYG{o}{\PYGZti{}}\PYG{o}{=}\PYG{l+m+mi}{1} \PYG{o}{\PYGZam{}} \PYG{n}{l13}\PYG{o}{.}\PYG{n}{to\PYGZus{}mar}\PYG{o}{\PYGZti{}}\PYG{o}{=}\PYG{l+m+mi}{1} \PYG{o}{\PYGZam{}} \PYG{n}{l14}\PYG{o}{.}\PYG{n}{to\PYGZus{}mar}\PYG{o}{\PYGZti{}}\PYG{o}{=}\PYG{l+m+mi}{1}
\PYG{n}{replace} \PYG{n}{t}\PYG{o}{=}\PYG{l+m+mi}{1} \PYG{k}{if} \PYG{n}{l}\PYG{o}{.}\PYG{n}{t}\PYG{o}{==}\PYG{l+m+mi}{0}
\PYG{n}{replace} \PYG{n}{t}\PYG{o}{=}\PYG{l+m+mi}{2} \PYG{k}{if} \PYG{n}{l2}\PYG{o}{.}\PYG{n}{t}\PYG{o}{==}\PYG{l+m+mi}{0}
\PYG{n}{replace} \PYG{n}{t}\PYG{o}{=}\PYG{l+m+mi}{3} \PYG{k}{if} \PYG{n}{l3}\PYG{o}{.}\PYG{n}{t}\PYG{o}{==}\PYG{l+m+mi}{0}
\PYG{n}{replace} \PYG{n}{t}\PYG{o}{=}\PYG{o}{\PYGZhy{}}\PYG{l+m+mi}{1} \PYG{k}{if} \PYG{n}{f}\PYG{o}{.}\PYG{n}{t}\PYG{o}{==}\PYG{l+m+mi}{0}
\PYG{n}{replace} \PYG{n}{t}\PYG{o}{=}\PYG{o}{\PYGZhy{}}\PYG{l+m+mi}{2} \PYG{k}{if} \PYG{n}{f2}\PYG{o}{.}\PYG{n}{t}\PYG{o}{==}\PYG{l+m+mi}{0}
\PYG{n}{replace} \PYG{n}{t}\PYG{o}{=}\PYG{o}{\PYGZhy{}}\PYG{l+m+mi}{3} \PYG{k}{if} \PYG{n}{f3}\PYG{o}{.}\PYG{n}{t}\PYG{o}{==}\PYG{l+m+mi}{0}

\PYG{n}{table} \PYG{n}{t}\PYG{p}{,} \PYG{n}{content} \PYG{p}{(}\PYG{n}{mean} \PYG{n}{plh0182} \PYG{n}{n} \PYG{n}{plh0182}\PYG{p}{)}
\end{sphinxVerbatim}

\begin{figure}[H]
\centering

\noindent\sphinxincludegraphics{{SOEPlong_14}.PNG}
\end{figure}

Choose a suitable presentation for your results and let Stata create a graphic.

\fvset{hllines={, ,}}%
\begin{sphinxVerbatim}[commandchars=\\\{\},numbers=left,firstnumber=1,stepnumber=1]
** Preparing graph of event analysis												
sort t
cap drop meanplh0182
by t: egen meanplh0182 = mean(plh0182)

cap drop upper
gen upper = .
forval i = \PYGZhy{}3/3\PYGZob{} 
	su plh0182 if t == {}`i\PYGZsq{}
	replace upper = r(mean) + 1.96 * r(sd)/sqrt(r(N)) if t == {}`i\PYGZsq{}
\PYGZcb{}

cap drop lower
gen lower = .
forval i = \PYGZhy{}3/3\PYGZob{} 
	su plh0182 if t == {}`i\PYGZsq{}
	replace lower = r(mean) \PYGZhy{} 1.96 * r(sd)/sqrt(r(N)) if t == {}`i\PYGZsq{}
\PYGZcb{}

twoway (line meanplh0182 t) (rcap upper lower t, lcolor(\PYGZdq{}red\PYGZdq{})) , title(\PYGZdq{}Satisfaction with life relative to year of marriage\PYGZdq{}) legend(label(1 \PYGZdq{}Avg. life satisfaction\PYGZdq{}) label(2 \PYGZdq{}95\PYGZpc{} Conf. interval\PYGZdq{})) scheme(s1mono) xtitle(\PYGZdq{}Years relative to marriage\PYGZdq{}) ytitle(\PYGZdq{}Avg. life satisfaction\PYGZdq{})
\end{sphinxVerbatim}

\begin{figure}[H]
\centering

\noindent\sphinxincludegraphics{{SOEPlong_15}.PNG}
\end{figure}

The graph shows that a positive effect on life satisfaction can be observed when the family status changes from single to married. In the following years of the existing marriage, life satisfaction decreases again and approaches the initial satisfaction before the marriage.


\section{Fixed Effects Estimation}
\label{\detokenize{Working with SOEP Data/index:fixed-effects-estimation}}
You want to find out whether certain variables relevant to the labour market, such as work experience or education time, influence a person’s hourly wage. Other variables such as gender or marriage status should also be taken into account. You decide to use the SOEP data to set up a fixed effects estimation model.

\sphinxstylestrong{Create an exercise path with four subfolders:}

\begin{figure}[H]
\centering

\noindent\sphinxincludegraphics{{uebungspfade}.PNG}
\end{figure}

\sphinxstylestrong{Example:}
\begin{itemize}
\item {} 
H:/material/exercises/do

\item {} 
H:/material/exercises/output

\item {} 
H:/material/exercises/temp

\item {} 
H:/material/exercises/log

\end{itemize}

These are used to store your script, log files, datasets and temporary datasets. Open an empty do file and define your created paths with globals:

\fvset{hllines={, ,}}%
\begin{sphinxVerbatim}[commandchars=\\\{\},numbers=left,firstnumber=1,stepnumber=1]
\PYG{o}{*}\PYG{o}{*}\PYG{o}{*}\PYG{o}{*}\PYG{o}{*}\PYG{o}{*}\PYG{o}{*}\PYG{o}{*}\PYG{o}{*}\PYG{o}{*}\PYG{o}{*}\PYG{o}{*}\PYG{o}{*}\PYG{o}{*}\PYG{o}{*}\PYG{o}{*}\PYG{o}{*}\PYG{o}{*}\PYG{o}{*}\PYG{o}{*}\PYG{o}{*}\PYG{o}{*}\PYG{o}{*}\PYG{o}{*}\PYG{o}{*}\PYG{o}{*}\PYG{o}{*}\PYG{o}{*}\PYG{o}{*}\PYG{o}{*}\PYG{o}{*}\PYG{o}{*}\PYG{o}{*}\PYG{o}{*}\PYG{o}{*}\PYG{o}{*}\PYG{o}{*}\PYG{o}{*}\PYG{o}{*}\PYG{o}{*}\PYG{o}{*}\PYG{o}{*}\PYG{o}{*}\PYG{o}{*}\PYG{o}{*}\PYG{o}{*}\PYG{o}{*}
\PYG{o}{*} \PYG{n}{Set} \PYG{n}{relative} \PYG{n}{paths} \PYG{n}{to} \PYG{n}{the} \PYG{n}{working} \PYG{n}{directory}
\PYG{o}{*}\PYG{o}{*}\PYG{o}{*}\PYG{o}{*}\PYG{o}{*}\PYG{o}{*}\PYG{o}{*}\PYG{o}{*}\PYG{o}{*}\PYG{o}{*}\PYG{o}{*}\PYG{o}{*}\PYG{o}{*}\PYG{o}{*}\PYG{o}{*}\PYG{o}{*}\PYG{o}{*}\PYG{o}{*}\PYG{o}{*}\PYG{o}{*}\PYG{o}{*}\PYG{o}{*}\PYG{o}{*}\PYG{o}{*}\PYG{o}{*}\PYG{o}{*}\PYG{o}{*}\PYG{o}{*}\PYG{o}{*}\PYG{o}{*}\PYG{o}{*}\PYG{o}{*}\PYG{o}{*}\PYG{o}{*}\PYG{o}{*}\PYG{o}{*}\PYG{o}{*}\PYG{o}{*}\PYG{o}{*}\PYG{o}{*}\PYG{o}{*}\PYG{o}{*}\PYG{o}{*}\PYG{o}{*}\PYG{o}{*}\PYG{o}{*}\PYG{o}{*}
\PYG{k}{global} \PYG{n}{AVZ} 	\PYG{l+s+s2}{\PYGZdq{}}\PYG{l+s+s2}{H:}\PYG{l+s+s2}{\PYGZbs{}}\PYG{l+s+s2}{material}\PYG{l+s+s2}{\PYGZbs{}}\PYG{l+s+s2}{exercises}\PYG{l+s+s2}{\PYGZdq{}}
\PYG{k}{global} \PYG{n}{MY\PYGZus{}IN\PYGZus{}PATH} \PYG{l+s+s2}{\PYGZdq{}}\PYG{l+s+se}{\PYGZbs{}\PYGZbs{}}\PYG{l+s+s2}{hume}\PYG{l+s+se}{\PYGZbs{}r}\PYG{l+s+s2}{dc\PYGZhy{}prod}\PYG{l+s+s2}{\PYGZbs{}}\PYG{l+s+s2}{distribution}\PYG{l+s+s2}{\PYGZbs{}}\PYG{l+s+s2}{soep\PYGZhy{}long}\PYG{l+s+s2}{\PYGZbs{}}\PYG{l+s+s2}{soep.v33.1}\PYG{l+s+s2}{\PYGZbs{}}\PYG{l+s+s2}{stata\PYGZus{}en}\PYG{l+s+se}{\PYGZbs{}\PYGZdq{}}
\PYG{k}{global} \PYG{n}{MY\PYGZus{}DO\PYGZus{}FILES} \PYG{l+s+s2}{\PYGZdq{}}\PYG{l+s+s2}{\PYGZdl{}AVZ}\PYG{l+s+s2}{\PYGZbs{}}\PYG{l+s+s2}{do}\PYG{l+s+se}{\PYGZbs{}\PYGZdq{}}
\PYG{k}{global} \PYG{n}{MY\PYGZus{}LOG\PYGZus{}OUT} \PYG{l+s+s2}{\PYGZdq{}}\PYG{l+s+s2}{\PYGZdl{}AVZ}\PYG{l+s+s2}{\PYGZbs{}}\PYG{l+s+s2}{log}\PYG{l+s+se}{\PYGZbs{}\PYGZdq{}}
\PYG{k}{global} \PYG{n}{MY\PYGZus{}OUT\PYGZus{}DATA} \PYG{l+s+s2}{\PYGZdq{}}\PYG{l+s+s2}{\PYGZdl{}AVZ}\PYG{l+s+s2}{\PYGZbs{}}\PYG{l+s+s2}{output}\PYG{l+s+se}{\PYGZbs{}\PYGZdq{}}
\PYG{k}{global} \PYG{n}{MY\PYGZus{}OUT\PYGZus{}TEMP} \PYG{l+s+s2}{\PYGZdq{}}\PYG{l+s+s2}{\PYGZdl{}AVZ}\PYG{l+s+se}{\PYGZbs{}t}\PYG{l+s+s2}{emp}\PYG{l+s+se}{\PYGZbs{}\PYGZdq{}}
\end{sphinxVerbatim}

The global „AVZ“ defines the main path. The main paths are subdivided using the globals “MY\_IN\_PATH”, “MY\_DO\_FILES”, “MY\_LOG\_OUT”, “MY\_OUT\_DATA”, “MY\_OUT\_TEMP”. The global “MY\_IN\_PATH” contains the path to your ordered data.

\sphinxstylestrong{a) Generate your own SOEPWage.dta data set. The data set should contain information on gross monthly wage, marital status and other personal characteristics.}

To perform your analysis, you need different SOEP variables. The SOEP offers various options for a variable search:
\begin{itemize}
\item {} 
Search the questionnaires for useful variables. (for more information visit the chapter {\hyperref[\detokenize{Working with SOEP Documentation/index:quest-search}]{\sphinxcrossref{\DUrole{std,std-ref}{Variable Search with Questionnaires}}}})

\item {} 
Find a suitable variable via the topic list of paneldata.org (for more information visit the chapter {\hyperref[\detokenize{Working with SOEP Documentation/index:topic}]{\sphinxcrossref{\DUrole{std,std-ref}{Topic Search with paneldata.org}}}})

\item {} 
Search for a suitable variable using a search term in paneldata.org (for more information visit the chapter {\hyperref[\detokenize{Working with SOEP Documentation/index:var-search}]{\sphinxcrossref{\DUrole{std,std-ref}{Variable Search with paneldata.org}}}})

\item {} 
Use the documentation provided by the generated variables (for more information visit the chapter {\hyperref[\detokenize{Working with SOEP Documentation/index:documentation}]{\sphinxcrossref{\DUrole{std,std-ref}{Documentation of Generated Data}}}})

\end{itemize}

Use the various important variables of the ppfadl.dta data set as your start file. Your source file should contain the following variables:
\begin{itemize}
\item {} 
Person ID  \href{https://paneldata.org/soep-long/data/ppfadl/pid}{\textbf{"pid"}}

\item {} 
Survey year  \href{https://paneldata.org/soep-long/data/ppfadl/syear}{\textbf{"syear"}}

\item {} 
Birth Year  \href{https://paneldata.org/soep-long/data/ppfadl/gebjahr}{\textbf{"gebjahr"}}

\item {} 
The net variable with information on the interview type  \href{https://paneldata.org/soep-long/data/ppfadl/netto}{\textbf{"netto"}}

\item {} 
The weighting variable  \href{https://paneldata.org/soep-long/data/ppfadl/phrf}{\textbf{"phrf"}}

\item {} 
The sex of the person  \href{https://paneldata.org/soep-long/data/ppfadl/sex}{\textbf{"sex"}}

\item {} 
Sample Membership  \href{https://paneldata.org/soep-long/data/ppfadl/pop}{\textbf{"pop"}}

\end{itemize}

\fvset{hllines={, ,}}%
\begin{sphinxVerbatim}[commandchars=\\\{\},numbers=left,firstnumber=1,stepnumber=1]
\PYG{n}{use} \PYG{n}{pid} \PYG{n}{syear} \PYG{n}{sex} \PYG{n}{gebjahr} \PYG{n}{netto} \PYG{n}{pop} \PYG{n}{phrf} \PYG{n}{using} \PYG{l+s+s2}{\PYGZdq{}}\PYG{l+s+s2}{\PYGZdl{}}\PYG{l+s+si}{\PYGZob{}MY\PYGZus{}IN\PYGZus{}PATH\PYGZcb{}}\PYG{l+s+s2}{/ppfadl.dta}\PYG{l+s+s2}{\PYGZdq{}}\PYG{p}{,} \PYG{n}{clear}
\end{sphinxVerbatim}

Apply the necessary content variables to your starting data set. You need the following variables for your analysis:
\begin{itemize}
\item {} 
Employment Status  \href{https://paneldata.org/soep-long/data/pl/plb0022}{\textbf{"plb0022"}}

\item {} 
Current Gross Labor Income in Euro  \href{https://paneldata.org/soep-long/data/pgen/pglabgro}{\textbf{"pglabgro"}}

\item {} 
Actual Work Time Per Week  \href{https://paneldata.org/soep-long/data/pgen/pglabgro}{\textbf{"pgtatzeit"}}

\item {} 
Working Experience Full-Time Employment  \href{https://paneldata.org/soep-long/data/pgen/pgexpft}{\textbf{"pgexpft"}}

\item {} 
Amount Of Education Or Training In Years  \href{https://paneldata.org/soep-long/data/pgen/pgbilzeit}{\textbf{"pgbilzeit"}}

\item {} 
Marital Status In Survey Year  \href{https://paneldata.org/soep-long/data/pgen/pgfamstd}{\textbf{"pgfamstd"}}

\end{itemize}

\fvset{hllines={, ,}}%
\begin{sphinxVerbatim}[commandchars=\\\{\},numbers=left,firstnumber=1,stepnumber=1]
\PYG{n}{merge} \PYG{l+m+mi}{1}\PYG{p}{:}\PYG{l+m+mi}{1} \PYG{n}{pid} \PYG{n}{syear} \PYG{n}{using} \PYG{l+s+s2}{\PYGZdq{}}\PYG{l+s+s2}{\PYGZdl{}}\PYG{l+s+si}{\PYGZob{}MY\PYGZus{}IN\PYGZus{}PATH\PYGZcb{}}\PYG{l+s+s2}{/pl.dta}\PYG{l+s+s2}{\PYGZdq{}}\PYG{p}{,} \PYG{n}{keepus}\PYG{p}{(}\PYG{n}{plb0022}\PYG{p}{)} \PYG{n}{keep}\PYG{p}{(}\PYG{n}{master} \PYG{n}{match}\PYG{p}{)} \PYG{n}{nogen}
\PYG{n}{merge} \PYG{l+m+mi}{1}\PYG{p}{:}\PYG{l+m+mi}{1} \PYG{n}{pid} \PYG{n}{syear} \PYG{n}{using} \PYG{l+s+s2}{\PYGZdq{}}\PYG{l+s+s2}{\PYGZdl{}}\PYG{l+s+si}{\PYGZob{}MY\PYGZus{}IN\PYGZus{}PATH\PYGZcb{}}\PYG{l+s+s2}{/pgen.dta}\PYG{l+s+s2}{\PYGZdq{}}\PYG{p}{,} \PYG{n}{keepus}\PYG{p}{(}\PYG{n}{pglabgro} \PYG{n}{pgtatzeit} \PYG{n}{pgexpft} \PYG{n}{pgbilzeit} \PYG{n}{pgfamstd}\PYG{p}{)} \PYG{n}{keep}\PYG{p}{(}\PYG{n}{master} \PYG{n}{match}\PYG{p}{)} \PYG{n}{nogen}
\end{sphinxVerbatim}

Only keep people who have completed an interview and who live in a private household.

\fvset{hllines={, ,}}%
\begin{sphinxVerbatim}[commandchars=\\\{\},numbers=left,firstnumber=1,stepnumber=1]
\PYG{o}{*} \PYG{n}{Only} \PYG{n}{select} \PYG{n}{people} \PYG{k}{with} \PYG{n}{completed} \PYG{n}{interviews}
\PYG{n}{keep} \PYG{k}{if} \PYG{n}{inrange}\PYG{p}{(}\PYG{n}{netto}\PYG{p}{,} \PYG{l+m+mi}{10}\PYG{p}{,} \PYG{l+m+mi}{19}\PYG{p}{)}

\PYG{o}{*} \PYG{n}{Only} \PYG{n}{private} \PYG{n}{households}
\PYG{n}{keep} \PYG{k}{if} \PYG{n}{pop}\PYG{o}{==}\PYG{l+m+mi}{1} \PYG{o}{\textbar{}} \PYG{n}{pop}\PYG{o}{==}\PYG{l+m+mi}{2}
\end{sphinxVerbatim}

Since you are only interested in the period from 2012 to 2016 in your analysis, remove all survey information that does not fall within this period. To finish, save your data set.

\fvset{hllines={, ,}}%
\begin{sphinxVerbatim}[commandchars=\\\{\},numbers=left,firstnumber=1,stepnumber=1]
\PYG{o}{*} \PYG{n}{Period} \PYG{k+kn}{from} \PYG{l+m+mi}{2012} \PYG{n}{to} \PYG{l+m+mi}{2016}
\PYG{n}{keep} \PYG{k}{if} \PYG{n}{syear}\PYG{o}{\PYGZgt{}}\PYG{o}{=}\PYG{l+m+mi}{2012} \PYG{o}{\PYGZam{}} \PYG{n}{syear}\PYG{o}{\PYGZlt{}}\PYG{o}{=}\PYG{l+m+mi}{2016}
\end{sphinxVerbatim}

\sphinxstylestrong{Exercise 1: Prepare your data set}

\sphinxstylestrong{a) Load your created SOEPWage.dta data set. The data set contains information on gross monthly wage, marital status and other personal characteristics.}

\fvset{hllines={, ,}}%
\begin{sphinxVerbatim}[commandchars=\\\{\},numbers=left,firstnumber=1,stepnumber=1]
\PYG{o}{*}\PYG{o}{*}\PYG{o}{*} \PYG{n}{Exercise} \PYG{l+m+mi}{1}\PYG{p}{:} \PYG{n}{Prepare} \PYG{n}{your} \PYG{n}{data} \PYG{n+nb}{set}
\PYG{o}{*} \PYG{n}{a}\PYG{p}{)} \PYG{n}{Load} \PYG{n}{data} \PYG{n+nb}{set}
\PYG{n}{use} \PYG{l+s+s2}{\PYGZdq{}}\PYG{l+s+s2}{\PYGZdl{}}\PYG{l+s+si}{\PYGZob{}MY\PYGZus{}OUT\PYGZus{}DATA\PYGZcb{}}\PYG{l+s+s2}{/SOEPWage.dta}\PYG{l+s+s2}{\PYGZdq{}}\PYG{p}{,} \PYG{n}{clear}
\end{sphinxVerbatim}

\sphinxstylestrong{b) Recode all missing values in Stata Missings (.)}

\fvset{hllines={, ,}}%
\begin{sphinxVerbatim}[commandchars=\\\{\},numbers=left,firstnumber=1,stepnumber=1]
\PYG{o}{*} \PYG{n}{b}\PYG{p}{)} \PYG{n}{Recode} \PYG{n}{Missings}
\PYG{n}{mvdecode} \PYG{n}{\PYGZus{}all}\PYG{p}{,} \PYG{n}{mv}\PYG{p}{(}\PYG{o}{\PYGZhy{}}\PYG{l+m+mi}{8}\PYG{o}{/}\PYG{o}{\PYGZhy{}}\PYG{l+m+mi}{1} \PYG{o}{=} \PYG{o}{.}\PYG{p}{)}
\end{sphinxVerbatim}

For more information about the missing codes of SOEP data visit the chapter {\hyperref[\detokenize{Principles of Data Structure/index:missings}]{\sphinxcrossref{\DUrole{std,std-ref}{Missing Conventions}}}}

\sphinxstylestrong{c) Generate the variables “hourly wage” (gross monthly wage/4.33*working time) for persons who have earned at least 1 Euro and have worked at least one hour,  “Married vs. Unmarried” and age.}

\fvset{hllines={, ,}}%
\begin{sphinxVerbatim}[commandchars=\\\{\},numbers=left,firstnumber=1,stepnumber=1]
\PYG{o}{*} \PYG{n}{c}\PYG{p}{)} \PYG{n}{Generate} \PYG{n}{Variables}
\PYG{n}{gen} \PYG{n}{wage} \PYG{o}{=} \PYG{n}{pglabgro}\PYG{o}{/}\PYG{p}{(}\PYG{l+m+mf}{4.33}\PYG{o}{*}\PYG{n}{pgtatzeit}\PYG{p}{)} \PYG{k}{if} \PYG{n}{pglabgro}\PYG{o}{\PYGZgt{}}\PYG{o}{=}\PYG{l+m+mi}{1} \PYG{o}{\PYGZam{}} \PYG{n}{pgtatzeit}\PYG{o}{\PYGZgt{}}\PYG{o}{=}\PYG{l+m+mi}{1}

\PYG{n}{gen} \PYG{n}{married} \PYG{o}{=} \PYG{l+m+mi}{1} \PYG{k}{if} \PYG{n}{pgfamstd}\PYG{o}{==}\PYG{l+m+mi}{1} \PYG{o}{\textbar{}} \PYG{n}{pgfamstd}\PYG{o}{==}\PYG{l+m+mi}{6} \PYG{o}{\textbar{}} \PYG{n}{pgfamstd}\PYG{o}{==}\PYG{l+m+mi}{7} \PYG{o}{\textbar{}} \PYG{n}{pgfamstd}\PYG{o}{==}\PYG{l+m+mi}{8}
\PYG{n}{replace} \PYG{n}{married} \PYG{o}{=} \PYG{l+m+mi}{0} \PYG{k}{if} \PYG{n}{inrange}\PYG{p}{(}\PYG{n}{pgfamstd}\PYG{p}{,} \PYG{l+m+mi}{2}\PYG{p}{,} \PYG{l+m+mi}{5}\PYG{p}{)}

\PYG{n}{gen} \PYG{n}{age} \PYG{o}{=} \PYG{n}{syear} \PYG{o}{\PYGZhy{}} \PYG{n}{gebjahr}
\end{sphinxVerbatim}

\sphinxstylestrong{d) Adjust the variable “hourly wage” from outlier values by setting values smaller than the 1st percentile to the same value. Set values greater than 3 times the 99th percentile to 3*99th percentile. Then generate the variable lwage = log(wage).}

\fvset{hllines={, ,}}%
\begin{sphinxVerbatim}[commandchars=\\\{\},numbers=left,firstnumber=1,stepnumber=1]
\PYG{o}{*} \PYG{n}{d}\PYG{p}{)} \PYG{n}{Adjust} \PYG{n}{wage} \PYG{n}{variable}
\PYG{n+nb}{sum} \PYG{n}{wage}\PYG{p}{,} \PYG{n}{detail}
\PYG{n}{replace} \PYG{n}{wage} \PYG{o}{=} \PYG{l+m+mi}{1}\PYG{o}{/}\PYG{l+m+mi}{3}\PYG{o}{*}\PYG{n}{r}\PYG{p}{(}\PYG{n}{p1}\PYG{p}{)} \PYG{k}{if} \PYG{n}{wage}\PYG{o}{\PYGZlt{}}\PYG{l+m+mi}{1}\PYG{o}{/}\PYG{l+m+mi}{3}\PYG{o}{*}\PYG{n}{r}\PYG{p}{(}\PYG{n}{p1}\PYG{p}{)}
\PYG{n}{replace} \PYG{n}{wage} \PYG{o}{=} \PYG{l+m+mi}{3}\PYG{o}{*}\PYG{n}{r}\PYG{p}{(}\PYG{n}{p99}\PYG{p}{)} \PYG{k}{if} \PYG{n}{wage}\PYG{o}{\PYGZgt{}}\PYG{l+m+mi}{3}\PYG{o}{*}\PYG{n}{r}\PYG{p}{(}\PYG{n}{p99}\PYG{p}{)} \PYG{o}{\PYGZam{}} \PYG{n}{wage}\PYG{o}{\PYGZlt{}}\PYG{o}{.}

\PYG{n}{gen} \PYG{n}{lwage} \PYG{o}{=} \PYG{n}{log}\PYG{p}{(}\PYG{n}{wage}\PYG{p}{)}
\PYG{n}{label} \PYG{n}{variable} \PYG{n}{lwage} \PYG{l+s+s2}{\PYGZdq{}}\PYG{l+s+s2}{Log hourly wage}\PYG{l+s+s2}{\PYGZdq{}}

\PYG{n}{save} \PYG{l+s+s2}{\PYGZdq{}}\PYG{l+s+s2}{\PYGZdl{}}\PYG{l+s+si}{\PYGZob{}MY\PYGZus{}OUT\PYGZus{}DATA\PYGZcb{}}\PYG{l+s+s2}{/SOEPWage\PYGZus{}temp.dta}\PYG{l+s+s2}{\PYGZdq{}}\PYG{p}{,} \PYG{n}{replace}
\end{sphinxVerbatim}

\sphinxstylestrong{Exercise 2: Descriptive statistics}

\sphinxstylestrong{a) Define the data set as a panel data set.}

\fvset{hllines={, ,}}%
\begin{sphinxVerbatim}[commandchars=\\\{\},numbers=left,firstnumber=1,stepnumber=1]
\PYG{o}{*}\PYG{o}{*}\PYG{o}{*} \PYG{n}{Exercise} \PYG{l+m+mi}{2}\PYG{p}{:} \PYG{n}{Descriptive} \PYG{n}{statistics}
\PYG{o}{*} \PYG{n}{a}\PYG{p}{)}
\PYG{n}{xtset} \PYG{n}{pid} \PYG{n}{syear} \PYG{o}{/}\PYG{o}{/} \PYG{n}{Declaring} \PYG{n}{data} \PYG{k}{as} \PYG{n}{panel} \PYG{n}{data}
\end{sphinxVerbatim}

\sphinxstylestrong{b) What percentage of people participate in all five waves (xtdescribe)}

\fvset{hllines={, ,}}%
\begin{sphinxVerbatim}[commandchars=\\\{\},numbers=left,firstnumber=1,stepnumber=1]
\PYG{o}{*} \PYG{n}{b}\PYG{p}{)}
\PYG{n}{xtdescribe}\PYG{p}{,} \PYG{n}{patterns}\PYG{p}{(}\PYG{l+m+mi}{16}\PYG{p}{)} \PYG{o}{/}\PYG{o}{/} \PYG{o}{\PYGZhy{}}\PYG{o}{\PYGZgt{}} \PYG{n}{unbalanced} \PYG{n}{panel}
\end{sphinxVerbatim}

\begin{figure}[H]
\centering

\noindent\sphinxincludegraphics{{fixed_01}.PNG}
\end{figure}

42808 respondents have contributed information within waves bc (2012) - bg (2016) and about 40\% (17069) of the 42808 respondents have provided information for all waves.

\sphinxstylestrong{c) Describe the variable “Married” with xttab and xttrans. Take a look at some individual wage (pid=30320901, pid=30932501, pid==3101602, pid==3101801) developments with xtline.}

\fvset{hllines={, ,}}%
\begin{sphinxVerbatim}[commandchars=\\\{\},numbers=left,firstnumber=1,stepnumber=1]
\PYG{o}{*} \PYG{n}{c}\PYG{p}{)} 
\PYG{o}{*} \PYG{n}{Stability} \PYG{n}{of} \PYG{n}{the} \PYG{n}{relationship} \PYG{n}{status}
\PYG{n}{xttab} \PYG{n}{married} 
\end{sphinxVerbatim}

\begin{figure}[H]
\centering

\noindent\sphinxincludegraphics{{fixed_02}.PNG}
\end{figure}

You can observe 41.37 percent of person-year observations with Married==No. At least once 19717 people within the period from 2012 to 2016 have stated not to have been married. 25014 persons reported to have been married at least once during this period. Those who were not married for at least one year responded with “married==no” in 94.69\% of the observations. Whereas those who have been married at least once responded in 95.88 percent of the observations with”Married==Yes”. A very stable response behaviour can therefore be observed.

\fvset{hllines={, ,}}%
\begin{sphinxVerbatim}[commandchars=\\\{\},numbers=left,firstnumber=1,stepnumber=1]
\PYG{o}{*} \PYG{n}{Transition} \PYG{n}{probabilities}
\PYG{n}{xttrans} \PYG{n}{married}\PYG{p}{,} \PYG{n}{freq}
\end{sphinxVerbatim}

\begin{figure}[H]
\centering

\noindent\sphinxincludegraphics{{fixed_03}.PNG}
\end{figure}

96.87 percent of the person-year observations with “married==no” are also not yet married in the next period. 98.51 percent of the persons who are married indicate that they will also be married in the following period. A stable behaviour of the respondents can be seen.

\fvset{hllines={, ,}}%
\begin{sphinxVerbatim}[commandchars=\\\{\},numbers=left,firstnumber=1,stepnumber=1]
\PYG{o}{*} \PYG{n}{Individual} \PYG{n}{sequences} \PYG{n}{of} \PYG{l+s+s2}{\PYGZdq{}}\PYG{l+s+s2}{wage}\PYG{l+s+s2}{\PYGZdq{}}
\PYG{n}{xtline} \PYG{n}{wage} \PYG{k}{if} \PYG{n}{pid}\PYG{o}{==}\PYG{l+m+mi}{30320901} \PYG{o}{\textbar{}} \PYG{n}{pid}\PYG{o}{==}\PYG{l+m+mi}{30932501} \PYG{o}{\textbar{}} \PYG{n}{pid}\PYG{o}{==}\PYG{l+m+mi}{3101602} \PYG{o}{\textbar{}} \PYG{n}{pid}\PYG{o}{==}\PYG{l+m+mi}{3101801}\PYG{p}{,} \PYG{n}{overlay} 
\end{sphinxVerbatim}

\begin{figure}[H]
\centering

\noindent\sphinxincludegraphics{{fixed_04}.PNG}
\end{figure}

The graphic shows a comparison of the hourly wage for four different respondents.

\sphinxstylestrong{Exercise 3: Pooled OLS Regression}

\sphinxstylestrong{a) Execute a pooled OLS regression with “Log hourly wage” as dependent variable and “Married”, “Gender”, “Work experience” and “Training time” as independent variables. Interpret the coefficients for “married”, “gender” and “length of training”. Why are these not causal effects?}

\fvset{hllines={, ,}}%
\begin{sphinxVerbatim}[commandchars=\\\{\},numbers=left,firstnumber=1,stepnumber=1]
\PYG{o}{*}\PYG{o}{*}\PYG{o}{*} \PYG{n}{Exercise} \PYG{l+m+mi}{3}\PYG{p}{:} \PYG{n}{Pooled} \PYG{n}{OLS} \PYG{n}{Regression}
\PYG{o}{*} \PYG{n}{a}\PYG{p}{)} \PYG{n}{Pooled} \PYG{n}{OLS}
\PYG{n}{reg} \PYG{n}{lwage} \PYG{n}{married} \PYG{n}{sex} \PYG{n}{pgexpft} \PYG{n}{pgbilzeit}
\end{sphinxVerbatim}

\begin{figure}[H]
\centering

\noindent\sphinxincludegraphics{{fixed_05}.PNG}
\end{figure}

The variables married, sex and pgbilzeit most likely correlate with
other disregarded/unobserved variables that have an effect on the wage. For example, women work more frequently in occupations with lower wages.

\sphinxstylestrong{b) Run the regression again with the option “vce(cluster persnr)” to get clustered standard errors. How do the standard errors of the coefficients change?}

\fvset{hllines={, ,}}%
\begin{sphinxVerbatim}[commandchars=\\\{\},numbers=left,firstnumber=1,stepnumber=1]
\PYG{o}{*} \PYG{n}{b}\PYG{p}{)} \PYG{n}{Pooled} \PYG{n}{OLS} \PYG{k}{with} \PYG{n}{cluster} \PYG{n}{standard} \PYG{n}{errors}
\PYG{n}{reg} \PYG{n}{lwage} \PYG{n}{married} \PYG{n}{sex} \PYG{n}{pgexpft} \PYG{n}{pgbilzeit}\PYG{p}{,} \PYG{n}{vce}\PYG{p}{(}\PYG{n}{cluster} \PYG{n}{pid}\PYG{p}{)}
\end{sphinxVerbatim}

\begin{figure}[H]
\centering

\noindent\sphinxincludegraphics{{fixed_06}.PNG}
\end{figure}

The standard errors are getting bigger.

\sphinxstylestrong{Exercise 4: Fixed Effects}

\sphinxstylestrong{a) Subtract the person-specific mean value from each variable of the model. Use the “egen” function. Ideally you should also use a loop.}

\fvset{hllines={, ,}}%
\begin{sphinxVerbatim}[commandchars=\\\{\},numbers=left,firstnumber=1,stepnumber=1]
*** Exercise 4: Fixed Effects
* a) Subtract person\PYGZhy{}specific averages

gen sample = 1
foreach var in lwage married sex pgexpft pgbilzeit \PYGZob{}

	bysort pid: egen {}`var\PYGZsq{}Mean = mean({}`var\PYGZsq{})
	replace {}`var\PYGZsq{}Mean = . if {}`var\PYGZsq{}==.
	gen {}`var\PYGZsq{}Demeaned =  {}`var\PYGZsq{} \PYGZhy{} {}`var\PYGZsq{}Mean
	replace sample = 0 if {}`var\PYGZsq{}==.
\PYGZcb{}
bysort pid (sample): replace sample = sample[1]
\end{sphinxVerbatim}

\sphinxstylestrong{b) Estimate the Fixed Effects model with the previously generated variables. Why is no coefficient estimated for “gender”? How do the coefficients change compared to the pooled OLS estimate? Is the effect of “married” now causally interpretable?}

\fvset{hllines={, ,}}%
\begin{sphinxVerbatim}[commandchars=\\\{\},numbers=left,firstnumber=1,stepnumber=1]
\PYG{n}{reg} \PYG{n}{lwageDemeaned} \PYG{n}{marriedDemeaned} \PYG{n}{sexDemeaned} \PYG{n}{pgexpftDemeaned} \PYG{n}{pgbilzeitDemeaned}\PYG{p}{,} \PYG{n}{vce}\PYG{p}{(}\PYG{n}{cluster} \PYG{n}{pid}\PYG{p}{)} \PYG{n}{nocons}
\end{sphinxVerbatim}

\begin{figure}[H]
\centering

\noindent\sphinxincludegraphics{{fixed_07}.PNG}
\end{figure}

No coefficient was estimated for sex because sex was stable over time for all observations. The coefficient of married is now significant at the 5\% level!

\sphinxstylestrong{c) Now estimate the Fixed Effects model using the command
“xtreg lwage married sex pgexpft pgbilzeit, fe “. What do you notice about the coefficients compared to task 4 b)? And with the standard errors?}

\fvset{hllines={, ,}}%
\begin{sphinxVerbatim}[commandchars=\\\{\},numbers=left,firstnumber=1,stepnumber=1]
\PYG{o}{*} \PYG{n}{c}\PYG{p}{)} \PYG{n}{xtreg}\PYG{p}{,} \PYG{n}{fe}
\PYG{n}{xtreg} \PYG{n}{lwage} \PYG{n}{married} \PYG{n}{pgexpft} \PYG{n}{pgbilzeit}\PYG{p}{,} \PYG{n}{fe} \PYG{n}{vce}\PYG{p}{(}\PYG{n}{cluster} \PYG{n}{pid}\PYG{p}{)}
\end{sphinxVerbatim}

\begin{figure}[H]
\centering

\noindent\sphinxincludegraphics{{fixed_08}.PNG}
\end{figure}

The coefficients are not identical with 4 b) and the standard errors become larger, because model b) does not take into account the estimation of mean values in the standard errors.

\sphinxstylestrong{d) Now add dummy variables for the years (i.syear). What happens with the effect of “labour market experience”?}

\fvset{hllines={, ,}}%
\begin{sphinxVerbatim}[commandchars=\\\{\},numbers=left,firstnumber=1,stepnumber=1]
\PYG{o}{*} \PYG{n}{d}\PYG{p}{)} \PYG{n}{xtreg} \PYG{k}{with} \PYG{n}{dummy}
\PYG{n}{xtreg} \PYG{n}{lwage} \PYG{n}{married} \PYG{n}{pgexpft} \PYG{n}{pgbilzeit} \PYG{n}{i}\PYG{o}{.}\PYG{n}{syear}\PYG{p}{,} \PYG{n}{fe} \PYG{n}{vce}\PYG{p}{(}\PYG{n}{cluster} \PYG{n}{pid}\PYG{p}{)}
\end{sphinxVerbatim}

\begin{figure}[H]
\centering

\noindent\sphinxincludegraphics{{fixed_09}.PNG}
\end{figure}

Effects on the variables remain significant. The model could possibly be specified on a case by case basis. The Mincer equation is based on (potential) labour market experience squared.

\sphinxstylestrong{e) Now you can also square labour market experience into the model. To what extent does the effect of labour market experience change compared to task 5d)?}

\fvset{hllines={, ,}}%
\begin{sphinxVerbatim}[commandchars=\\\{\},numbers=left,firstnumber=1,stepnumber=1]
\PYG{o}{*} \PYG{n}{e}\PYG{p}{)} \PYG{n}{expft} \PYG{n}{squared}
\PYG{n}{xtreg} \PYG{n}{lwage} \PYG{n}{married} \PYG{n}{c}\PYG{o}{.}\PYG{n}{pgexpft}\PYG{c+c1}{\PYGZsh{}\PYGZsh{}c.pgexpft pgbilzeit i.syear, fe vce(cluster pid)}
\end{sphinxVerbatim}

\begin{figure}[H]
\centering

\noindent\sphinxincludegraphics{{fixed_10}.PNG}
\end{figure}

The coefficients of pgexpft and pgexpft\textasciicircum{}2 remain significant whereas the coefficient for married is no longer significant.

\fvset{hllines={, ,}}%
\begin{sphinxVerbatim}[commandchars=\\\{\},numbers=left,firstnumber=1,stepnumber=1]
\PYG{n}{graph} \PYG{n}{twoway} \PYG{p}{(}\PYG{n}{func} \PYG{n}{y} \PYG{o}{=} \PYG{n}{\PYGZus{}b}\PYG{p}{[}\PYG{n}{pgexpft}\PYG{p}{]}\PYG{o}{*}\PYG{n}{x} \PYG{o}{+} \PYG{n}{\PYGZus{}b}\PYG{p}{[}\PYG{n}{c}\PYG{o}{.}\PYG{n}{pgexpft}\PYG{c+c1}{\PYGZsh{}c.pgexpft]*x*x, range(0 40))}
\end{sphinxVerbatim}

\begin{figure}[H]
\centering

\noindent\sphinxincludegraphics{{fixed_11}.PNG}
\end{figure}

The graph shows that the effects of the labour market experience decrease after approximately 15 years of professional experience.

\sphinxstylestrong{f) Now estimate the model from task 5e) with longitudinal section weights. Why is the number of cases now significantly smaller? Why could the coefficient of “pgbilzeit” have changed?}

\begin{sphinxadmonition}{tip}{Tip:}
Create your own longitudinal person weights  e.g. longitudinal person weight from wave A to wave D. Take the starting wave cross-sectional weight (aphrf) and multiply through by each following wave staying factor, as in the following example:
gen adphrf=aphrf*bpbleib*cpbleib*dpbleib
\end{sphinxadmonition}

Since you are looking at the period 2012-2016, you must create a suitable longitudinal weight. To do this, use the phrf data set from the RAW subdirectory. Apply the required variables on your analysis data set and generate your period-related longitudinal section weight. To understand the structure of the data distribution file and the location of the different data sets, visit the chapter {\hyperref[\detokenize{Principles of Data Structure/index:datasets}]{\sphinxcrossref{\DUrole{std,std-ref}{Data Sets SOEP-Core}}}}. For more information about the weighting data sets and other survey data sets, visit the chapter {\hyperref[\detokenize{Principles of Data Structure/index:survey}]{\sphinxcrossref{\DUrole{std,std-ref}{Survey Data}}}}.

\fvset{hllines={, ,}}%
\begin{sphinxVerbatim}[commandchars=\\\{\},numbers=left,firstnumber=1,stepnumber=1]
\PYG{o}{*} \PYG{n}{f}\PYG{p}{)} \PYG{n}{Fixed} \PYG{n}{Effects} \PYG{n}{weighted}
\PYG{k}{global} \PYG{n}{MY\PYGZus{}IN\PYGZus{}PATH2} \PYG{l+s+s2}{\PYGZdq{}}\PYG{l+s+se}{\PYGZbs{}\PYGZbs{}}\PYG{l+s+s2}{hume}\PYG{l+s+se}{\PYGZbs{}r}\PYG{l+s+s2}{dc\PYGZhy{}prod}\PYG{l+s+s2}{\PYGZbs{}}\PYG{l+s+s2}{complete}\PYG{l+s+s2}{\PYGZbs{}}\PYG{l+s+s2}{soep\PYGZhy{}core}\PYG{l+s+s2}{\PYGZbs{}}\PYG{l+s+s2}{soep.v33.2}\PYG{l+s+s2}{\PYGZbs{}}\PYG{l+s+s2}{stata\PYGZus{}en}\PYG{l+s+se}{\PYGZbs{}\PYGZdq{}}
\PYG{n}{rename} \PYG{n}{pid} \PYG{n}{persnr}
\PYG{n}{merge} \PYG{n}{m}\PYG{p}{:}\PYG{l+m+mi}{1} \PYG{n}{persnr} \PYG{n}{using} \PYG{l+s+s2}{\PYGZdq{}}\PYG{l+s+s2}{\PYGZdl{}}\PYG{l+s+si}{\PYGZob{}MY\PYGZus{}IN\PYGZus{}PATH2\PYGZcb{}}\PYG{l+s+s2}{/phrf.dta}\PYG{l+s+s2}{\PYGZdq{}}\PYG{p}{,} \PYG{n}{nogen} \PYG{n}{keep}\PYG{p}{(}\PYG{n}{master} \PYG{n}{match}\PYG{p}{)} \PYG{n}{keepus}\PYG{p}{(}\PYG{n}{bcphrf} \PYG{n}{bdpbleib} \PYG{n}{bepbleib} \PYG{n}{bfpbleib} \PYG{n}{bgpbleib}\PYG{p}{)}
\PYG{n}{gen} \PYG{n}{wlong} \PYG{o}{=} \PYG{n}{bcphrf}\PYG{o}{*}\PYG{n}{bdpbleib}\PYG{o}{*}\PYG{n}{bepbleib}\PYG{o}{*}\PYG{n}{bfpbleib}\PYG{o}{*}\PYG{n}{bgpbleib}
\PYG{n}{label} \PYG{n}{variable} \PYG{n}{wlong} \PYG{l+s+s2}{\PYGZdq{}}\PYG{l+s+s2}{Weighting BC\PYGZhy{}BG}\PYG{l+s+s2}{\PYGZdq{}}
\PYG{n}{rename} \PYG{n}{persnr} \PYG{n}{pid}
\end{sphinxVerbatim}

Now estimate the model from 5e) and use the created weight.

\fvset{hllines={, ,}}%
\begin{sphinxVerbatim}[commandchars=\\\{\},numbers=left,firstnumber=1,stepnumber=1]
\PYG{n}{xtreg} \PYG{n}{lwage} \PYG{n}{married} \PYG{n}{c}\PYG{o}{.}\PYG{n}{pgexpft}\PYG{c+c1}{\PYGZsh{}\PYGZsh{}c.pgexpft pgbilzeit i.syear [pw=wlong], fe vce(cluster pid)}
\end{sphinxVerbatim}

\begin{figure}[H]
\centering

\noindent\sphinxincludegraphics{{fixed_12}.PNG}
\end{figure}

The number of observations is now much smaller. The effect of pgbilzeit is stronger than before. Pgbilzeit has a lower effect in the wlong==0 group, where the return is different for each additional educational year.  People in the wlong===0 group may not get the return for the additional education they expected on the local labour market and may therefore move -\textgreater{} higher probability for dropout.


\section{Working with SOEP Regional Data}
\label{\detokenize{Working with SOEP Data/index:working-with-soep-regional-data}}
SOEP offers diverse possibilities for regional and spatial analysis. With the anonymized regional information on the residences of SOEP respondents (households and individuals), it is possible to link numerous regional indicators on the levels of the states (Bundesländer), spatial planning regions, districts, and postal codes with the SOEP data on these households. However, specific security provisions must be observed due to the sensitivity of the data under data protection law. Accordingly, you are not allowed to make statements on, e.g., place of residence or administrative district in your analyses, but the data does provide valuable background information.

\begin{figure}[H]
\centering

\noindent\sphinxincludegraphics{{regionaldata_en}.jpg}
\end{figure}

For more Information and to get access visit  \href{https://www.diw.de/en/diw_02.c.222520.en/regional_data.html}{\textbf{Regional Data}}

For your research project you want to measure current (year 2016) urban-rural differences in the population. You are particularly interested in the differences in political interest and the different satisfaction variables provided by the SOEP.  You also want to take into account demographic differences in gender and age. In order to be able to evaluate the research potential, you should get an overview. For regional analyses, for example, the community size classes from the regional data are suitable.

\sphinxstylestrong{Create an exercise path with four subfolders:}

\begin{figure}[H]
\centering

\noindent\sphinxincludegraphics{{uebungspfade}.PNG}
\end{figure}

\sphinxstylestrong{Example:}
\begin{itemize}
\item {} 
H:/material/exercises/do

\item {} 
H:/material/exercises/output

\item {} 
H:/material/exercises/temp

\item {} 
H:/material/exercises/log

\end{itemize}

These are used to store your script, log files, datasets and temporary datasets. Open an empty do file and define your created paths with globals:

\fvset{hllines={, ,}}%
\begin{sphinxVerbatim}[commandchars=\\\{\},numbers=left,firstnumber=1,stepnumber=1]
\PYG{o}{*}\PYG{o}{*}\PYG{o}{*}\PYG{o}{*}\PYG{o}{*}\PYG{o}{*}\PYG{o}{*}\PYG{o}{*}\PYG{o}{*}\PYG{o}{*}\PYG{o}{*}\PYG{o}{*}\PYG{o}{*}\PYG{o}{*}\PYG{o}{*}\PYG{o}{*}\PYG{o}{*}\PYG{o}{*}\PYG{o}{*}\PYG{o}{*}\PYG{o}{*}\PYG{o}{*}\PYG{o}{*}\PYG{o}{*}\PYG{o}{*}\PYG{o}{*}\PYG{o}{*}\PYG{o}{*}\PYG{o}{*}\PYG{o}{*}\PYG{o}{*}\PYG{o}{*}\PYG{o}{*}\PYG{o}{*}\PYG{o}{*}\PYG{o}{*}\PYG{o}{*}\PYG{o}{*}\PYG{o}{*}\PYG{o}{*}\PYG{o}{*}\PYG{o}{*}\PYG{o}{*}\PYG{o}{*}\PYG{o}{*}\PYG{o}{*}\PYG{o}{*}
\PYG{o}{*} \PYG{n}{Set} \PYG{n}{relative} \PYG{n}{paths} \PYG{n}{to} \PYG{n}{the} \PYG{n}{working} \PYG{n}{directory}
\PYG{o}{*}\PYG{o}{*}\PYG{o}{*}\PYG{o}{*}\PYG{o}{*}\PYG{o}{*}\PYG{o}{*}\PYG{o}{*}\PYG{o}{*}\PYG{o}{*}\PYG{o}{*}\PYG{o}{*}\PYG{o}{*}\PYG{o}{*}\PYG{o}{*}\PYG{o}{*}\PYG{o}{*}\PYG{o}{*}\PYG{o}{*}\PYG{o}{*}\PYG{o}{*}\PYG{o}{*}\PYG{o}{*}\PYG{o}{*}\PYG{o}{*}\PYG{o}{*}\PYG{o}{*}\PYG{o}{*}\PYG{o}{*}\PYG{o}{*}\PYG{o}{*}\PYG{o}{*}\PYG{o}{*}\PYG{o}{*}\PYG{o}{*}\PYG{o}{*}\PYG{o}{*}\PYG{o}{*}\PYG{o}{*}\PYG{o}{*}\PYG{o}{*}\PYG{o}{*}\PYG{o}{*}\PYG{o}{*}\PYG{o}{*}\PYG{o}{*}\PYG{o}{*}
\PYG{k}{global} \PYG{n}{AVZ} 	\PYG{l+s+s2}{\PYGZdq{}}\PYG{l+s+s2}{H:}\PYG{l+s+s2}{\PYGZbs{}}\PYG{l+s+s2}{material}\PYG{l+s+s2}{\PYGZbs{}}\PYG{l+s+s2}{exercises}\PYG{l+s+s2}{\PYGZdq{}}
\PYG{k}{global} \PYG{n}{MY\PYGZus{}IN\PYGZus{}PATH} \PYG{l+s+s2}{\PYGZdq{}}\PYG{l+s+se}{\PYGZbs{}\PYGZbs{}}\PYG{l+s+s2}{hume}\PYG{l+s+se}{\PYGZbs{}r}\PYG{l+s+s2}{dc\PYGZhy{}prod}\PYG{l+s+s2}{\PYGZbs{}}\PYG{l+s+s2}{complete}\PYG{l+s+s2}{\PYGZbs{}}\PYG{l+s+s2}{soep\PYGZhy{}core}\PYG{l+s+s2}{\PYGZbs{}}\PYG{l+s+s2}{soep.v33.2}\PYG{l+s+s2}{\PYGZbs{}}\PYG{l+s+s2}{stata\PYGZus{}en}\PYG{l+s+se}{\PYGZbs{}\PYGZdq{}}
\PYG{k}{global} \PYG{n}{region} \PYG{l+s+s2}{\PYGZdq{}}\PYG{l+s+se}{\PYGZbs{}\PYGZbs{}}\PYG{l+s+s2}{hume}\PYG{l+s+s2}{\PYGZbs{}}\PYG{l+s+s2}{soep\PYGZhy{}region}\PYG{l+s+s2}{\PYGZbs{}}\PYG{l+s+s2}{DATA}\PYG{l+s+s2}{\PYGZbs{}}\PYG{l+s+s2}{soep33\PYGZus{}de}\PYG{l+s+se}{\PYGZbs{}\PYGZdq{}}
\PYG{k}{global} \PYG{n}{MY\PYGZus{}DO\PYGZus{}FILES} \PYG{l+s+s2}{\PYGZdq{}}\PYG{l+s+s2}{\PYGZdl{}AVZ}\PYG{l+s+s2}{\PYGZbs{}}\PYG{l+s+s2}{do}\PYG{l+s+se}{\PYGZbs{}\PYGZdq{}}
\PYG{k}{global} \PYG{n}{MY\PYGZus{}LOG\PYGZus{}OUT} \PYG{l+s+s2}{\PYGZdq{}}\PYG{l+s+s2}{\PYGZdl{}AVZ}\PYG{l+s+s2}{\PYGZbs{}}\PYG{l+s+s2}{log}\PYG{l+s+se}{\PYGZbs{}\PYGZdq{}}
\PYG{k}{global} \PYG{n}{MY\PYGZus{}OUT\PYGZus{}DATA} \PYG{l+s+s2}{\PYGZdq{}}\PYG{l+s+s2}{\PYGZdl{}AVZ}\PYG{l+s+s2}{\PYGZbs{}}\PYG{l+s+s2}{output}\PYG{l+s+se}{\PYGZbs{}\PYGZdq{}}
\PYG{k}{global} \PYG{n}{MY\PYGZus{}OUT\PYGZus{}TEMP} \PYG{l+s+s2}{\PYGZdq{}}\PYG{l+s+s2}{\PYGZdl{}AVZ}\PYG{l+s+se}{\PYGZbs{}t}\PYG{l+s+s2}{emp}\PYG{l+s+se}{\PYGZbs{}\PYGZdq{}}
\end{sphinxVerbatim}

The global „AVZ“ defines the main path. The main paths are subdivided using the globals “MY\_IN\_PATH”, “MY\_DO\_FILES”, “MY\_LOG\_OUT”, “MY\_OUT\_DATA”, “MY\_OUT\_TEMP”. The global “MY\_IN\_PATH” contains the path to your ordered data.

\sphinxstylestrong{a) Prepare a cross-sectional analysis data set covering the survey year 2016 (wave bg).}

To perform your analysis, you need different SOEP variables. The SOEP offers various options for a variable search:
\begin{itemize}
\item {} 
Search the questionnaires for useful variables. (for more information visit the chapter {\hyperref[\detokenize{Working with SOEP Documentation/index:quest-search}]{\sphinxcrossref{\DUrole{std,std-ref}{Variable Search with Questionnaires}}}})

\item {} 
Find a suitable variable via the topic list of paneldata.org (for more information visit the chapter {\hyperref[\detokenize{Working with SOEP Documentation/index:topic}]{\sphinxcrossref{\DUrole{std,std-ref}{Topic Search with paneldata.org}}}})

\item {} 
Search for a suitable variable using a search term in paneldata.org (for more information visit the chapter {\hyperref[\detokenize{Working with SOEP Documentation/index:var-search}]{\sphinxcrossref{\DUrole{std,std-ref}{Variable Search with paneldata.org}}}})

\item {} 
Use the documentation provided by the generated variables (for more information visit the chapter {\hyperref[\detokenize{Working with SOEP Documentation/index:documentation}]{\sphinxcrossref{\DUrole{std,std-ref}{Documentation of Generated Data}}}})

\end{itemize}

Your source file should contain the following variables:
\begin{itemize}
\item {} 
Never Changing Person ID  \href{https://paneldata.org/soep-core/data/ppfad/persnr}{\textbf{"persnr"}}

\item {} 
Original Household Number  \href{https://paneldata.org/soep-core/data/ppfad/hhnr}{\textbf{"hhnr"}}

\item {} 
Current Wave Household Number  \href{https://paneldata.org/soep-core/data/ppfad/bghhnr}{\textbf{"bghhnr"}}

\item {} 
The sex of the person  \href{https://paneldata.org/soep-core/data/ppfad/sex}{\textbf{"sex"}}

\item {} 
Year of birth  \href{https://paneldata.org/soep-core/data/ppfad/gebjahr}{\textbf{"gebjahr"}}

\item {} 
Survey Status 2016  \href{https://paneldata.org/soep-core/data/ppfad/bgnetto}{\textbf{"bgnetto"}}

\item {} 
Sample Membership 2016  \href{https://paneldata.org/soep-core/data/ppfad/bgpop}{\textbf{"bgpop"}}

\item {} 
Weighting Factor 2016  \href{https://paneldata.org/soep-core/data/phrf/bgphrf}{\textbf{"bgphrf"}}

\item {} 
Satisfaction With Health  \href{https://paneldata.org/soep-core/data/bgp/bgp0101}{\textbf{"bgp0101"}}

\item {} 
Satisfaction With Sleep  \href{https://paneldata.org/soep-core/data/bgp/bgp0102}{\textbf{"bgp0102"}}

\item {} 
Satisfaction With Work  \href{https://paneldata.org/soep-core/data/bgp/bgp0103}{\textbf{"bgp0103"}}

\item {} 
Satisfaction With Housework  \href{https://paneldata.org/soep-core/data/bgp/bgp0104}{\textbf{"bgp0104"}}

\item {} 
Satisfaction With Household Income  \href{https://paneldata.org/soep-core/data/bgp/bgp0105}{\textbf{"bgp0105"}}

\item {} 
Satisfaction With Personal Income  \href{https://paneldata.org/soep-core/data/bgp/bgp0106}{\textbf{"bgp0106"}}

\item {} 
Satisfaction With Dwelling  \href{https://paneldata.org/soep-core/data/bgp/bgp0107}{\textbf{"bgp0107"}}

\item {} 
Satisfaction With Amount Of Leisure Time  \href{https://paneldata.org/soep-core/data/bgp/bgp0108}{\textbf{"bgp0108"}}

\item {} 
Satisfaction With Child Care  \href{https://paneldata.org/soep-core/data/bgp/bgp0109}{\textbf{"bgp0109"}}

\item {} 
Satisfaction With Family Life  \href{https://paneldata.org/soep-core/data/bgp/bgp0110}{\textbf{"bgp0110"}}

\item {} 
Satisfaction With Social Life  \href{https://paneldata.org/soep-core/data/bgp/bgp0111}{\textbf{"bgp0111"}}

\item {} 
Zufriedenheit mit Demokratie  \href{https://paneldata.org/soep-core/data/bgp/bgp0112}{\textbf{"bgp0112"}}

\item {} 
Political Interests  \href{https://paneldata.org/soep-core/data/bgp/bgp143}{\textbf{"bgp143"}}

\item {} 
Current Sample Region  \href{https://paneldata.org/soep-core/data/bghbrutto/bgsampreg}{\textbf{"bgsampreg"}}

\item {} 
Federal State  \href{https://paneldata.org/soep-core/data/bghbrutto/bgbula}{\textbf{"bgbula"}}

\item {} 
Spatial category by BBSR  \href{https://paneldata.org/soep-core/data/bghbrutto/bgregtyp}{\textbf{"bgregtyp"}}

\item {} 
Community Class Sizes “ggk”

\end{itemize}

Use the various important variables of the ppfad.dta data set as your start file.

\fvset{hllines={, ,}}%
\begin{sphinxVerbatim}[commandchars=\\\{\},numbers=left,firstnumber=1,stepnumber=1]
use hhnr persnr bghhnr sex gebjahr bgnetto bgpop using \PYGZdl{}\PYGZob{}MY\PYGZus{}IN\PYGZus{}PATH\PYGZcb{}\PYGZbs{}ppfad.dta, clear
\end{sphinxVerbatim}

Keep people who completed a questionnaire in 2016 and live in a private household.

\fvset{hllines={, ,}}%
\begin{sphinxVerbatim}[commandchars=\\\{\},numbers=left,firstnumber=1,stepnumber=1]
* Keep people who completed a questionnaire in 2016 and live in a private household
keep if bghhnr\PYGZgt{}0 \PYGZam{} inrange(bgnetto, 10, 29) \PYGZam{} inlist(bgpop, 1, 2)
keep hhnr persnr bghhnr sex gebjahr bgnetto bgpop
merge 1:1  persnr using \PYGZdl{}\PYGZob{}MY\PYGZus{}IN\PYGZus{}PATH\PYGZcb{}\PYGZbs{}phrf.dta, keep(match master) keepusing (bgphrf) nogenerate
tempfile ppfad
save {}`ppfad\PYGZsq{}
\end{sphinxVerbatim}

Prepare the different data sets bgp, bghbrutto, regiobl

\fvset{hllines={, ,}}%
\begin{sphinxVerbatim}[commandchars=\\\{\},numbers=left,firstnumber=1,stepnumber=1]
* Prepare data set bgp
use \PYGZdl{}\PYGZob{}MY\PYGZus{}IN\PYGZus{}PATH\PYGZcb{}\PYGZbs{}bgp.dta, replace
keep persnr hhnr bghhnr bgp01* bgp143
tempfile bgp
save {}`bgp\PYGZsq{}

* Prepare data set bghbrutto
use \PYGZdl{}\PYGZob{}MY\PYGZus{}IN\PYGZus{}PATH\PYGZcb{}\PYGZbs{}bghbrutto.dta, replace
keep hhnr bghhnr bgsampreg bgbula bgregtyp
tempfile bghbrutto
save {}`bghbrutto\PYGZsq{}

* Prepare data set regionl
use \PYGZdl{}\PYGZob{}region\PYGZcb{}\PYGZbs{}regionl\PYGZus{}v33.dta, replace
keep if syear==2016
keep syear hhnr hhnrakt ggk
rename hhnrakt bghhnr
tempfile regionl
save {}`regionl\PYGZsq{}
\end{sphinxVerbatim}

Merge all data sets.

\fvset{hllines={, ,}}%
\begin{sphinxVerbatim}[commandchars=\\\{\},numbers=left,firstnumber=1,stepnumber=1]
* Merge all data sets
use {}`ppfad\PYGZsq{}
merge 1:1 persnr using {}`bgp\PYGZsq{}, keep(match master) nogenerate
merge m:1 bghhnr hhnr using {}`regionl\PYGZsq{}, keep(match master) nogenerate
merge m:1 bghhnr hhnr using {}`bghbrutto\PYGZsq{}, keep(match master) nogenerate
\end{sphinxVerbatim}

Recode negative values into missings.

\fvset{hllines={, ,}}%
\begin{sphinxVerbatim}[commandchars=\\\{\},numbers=left,firstnumber=1,stepnumber=1]
\PYG{o}{*} \PYG{n}{Recode} \PYG{n}{negative} \PYG{n}{values} \PYG{n}{into} \PYG{n}{missings}
\PYG{n}{mvdecode} \PYG{n}{sex} \PYG{n}{gebjahr} \PYG{n}{bgp01}\PYG{o}{*} \PYG{n}{bgp143}\PYG{p}{,}\PYG{n}{mv}\PYG{p}{(}\PYG{o}{\PYGZhy{}}\PYG{l+m+mi}{5}\PYG{o}{/}\PYG{o}{\PYGZhy{}}\PYG{l+m+mi}{1}\PYG{p}{)}
\end{sphinxVerbatim}

Categorize the community class sizes of the SOEP regional data set.

\fvset{hllines={, ,}}%
\begin{sphinxVerbatim}[commandchars=\\\{\},numbers=left,firstnumber=1,stepnumber=1]
\PYG{o}{*} \PYG{n}{Categorize} \PYG{n}{community} \PYG{k}{class} \PYG{n+nc}{size}
\PYG{n}{gen} \PYG{n}{ggk\PYGZus{}cat}\PYG{o}{=}\PYG{o}{.}
\PYG{n}{replace} \PYG{n}{ggk\PYGZus{}cat}\PYG{o}{=}\PYG{o}{\PYGZhy{}}\PYG{l+m+mi}{1} \PYG{k}{if} \PYG{n}{ggk}\PYG{o}{==}\PYG{o}{\PYGZhy{}}\PYG{l+m+mi}{1}
\PYG{n}{replace} \PYG{n}{ggk\PYGZus{}cat}\PYG{o}{=}\PYG{l+m+mi}{1} \PYG{k}{if} \PYG{n}{ggk}\PYG{o}{==}\PYG{l+m+mi}{1} \PYG{o}{\textbar{}} \PYG{n}{ggk}\PYG{o}{==}\PYG{l+m+mi}{2}
\PYG{n}{replace} \PYG{n}{ggk\PYGZus{}cat}\PYG{o}{=}\PYG{l+m+mi}{2} \PYG{k}{if} \PYG{n}{ggk}\PYG{o}{==}\PYG{l+m+mi}{3}
\PYG{n}{replace} \PYG{n}{ggk\PYGZus{}cat}\PYG{o}{=}\PYG{l+m+mi}{3} \PYG{k}{if} \PYG{n}{ggk}\PYG{o}{==}\PYG{l+m+mi}{4} \PYG{o}{\textbar{}} \PYG{n}{ggk}\PYG{o}{==}\PYG{l+m+mi}{5}
\PYG{n}{replace} \PYG{n}{ggk\PYGZus{}cat}\PYG{o}{=}\PYG{l+m+mi}{4} \PYG{k}{if} \PYG{n}{ggk}\PYG{o}{\PYGZgt{}}\PYG{l+m+mi}{5} \PYG{o}{\PYGZam{}} \PYG{n}{ggk}\PYG{o}{\PYGZlt{}}\PYG{o}{=}\PYG{l+m+mi}{7}

\PYG{n}{lab} \PYG{n}{var} \PYG{n}{ggk\PYGZus{}cat} \PYG{l+s+s2}{\PYGZdq{}}\PYG{l+s+s2}{Community Size categorised}\PYG{l+s+s2}{\PYGZdq{}}
\PYG{n}{lab} \PYG{k}{def} \PYG{n+nf}{ggk\PYGZus{}cat} \PYG{o}{\PYGZhy{}}\PYG{l+m+mi}{1} \PYG{l+s+s2}{\PYGZdq{}}\PYG{l+s+s2}{No information}\PYG{l+s+s2}{\PYGZdq{}} \PYG{l+m+mi}{1} \PYG{l+s+s2}{\PYGZdq{}}\PYG{l+s+s2}{\PYGZlt{}=5000}\PYG{l+s+s2}{\PYGZdq{}} \PYG{l+m+mi}{2} \PYG{l+s+s2}{\PYGZdq{}}\PYG{l+s+s2}{5001 \PYGZhy{} 20000}\PYG{l+s+s2}{\PYGZdq{}} \PYG{l+m+mi}{3} \PYG{l+s+s2}{\PYGZdq{}}\PYG{l+s+s2}{20001 \PYGZhy{} 100000}\PYG{l+s+s2}{\PYGZdq{}} \PYG{o}{/}\PYG{o}{/}\PYG{o}{/} 
\PYG{l+m+mi}{4} \PYG{l+s+s2}{\PYGZdq{}}\PYG{l+s+s2}{\PYGZgt{}100000}\PYG{l+s+s2}{\PYGZdq{}}
\PYG{n}{lab} \PYG{n}{val} \PYG{n}{ggk\PYGZus{}cat} \PYG{n}{ggk\PYGZus{}cat}
\end{sphinxVerbatim}

Generate an age variable.

\fvset{hllines={, ,}}%
\begin{sphinxVerbatim}[commandchars=\\\{\},numbers=left,firstnumber=1,stepnumber=1]
\PYG{o}{*} \PYG{n}{Generate} \PYG{n}{age} \PYG{n}{variable}
\PYG{n}{gen} \PYG{n}{alter}\PYG{o}{=} \PYG{l+m+mi}{2016}\PYG{o}{\PYGZhy{}}\PYG{n}{gebjahr} \PYG{k}{if} \PYG{n}{gebjahr} \PYG{o}{\PYGZgt{}} \PYG{l+m+mi}{0}
\PYG{n}{gen} \PYG{n}{alter\PYGZus{}cat}\PYG{o}{=}\PYG{l+m+mi}{1} \PYG{k}{if} \PYG{n}{alter}\PYG{o}{\PYGZlt{}}\PYG{o}{=}\PYG{l+m+mi}{20}
\PYG{n}{replace} \PYG{n}{alter\PYGZus{}cat}\PYG{o}{=}\PYG{l+m+mi}{2} \PYG{k}{if} \PYG{n}{alter}\PYG{o}{\PYGZgt{}}\PYG{l+m+mi}{20} \PYG{o}{\PYGZam{}} \PYG{n}{alter}\PYG{o}{\PYGZlt{}}\PYG{o}{=}\PYG{l+m+mi}{30}
\PYG{n}{replace} \PYG{n}{alter\PYGZus{}cat}\PYG{o}{=}\PYG{l+m+mi}{3} \PYG{k}{if} \PYG{n}{alter}\PYG{o}{\PYGZgt{}}\PYG{l+m+mi}{30} \PYG{o}{\PYGZam{}} \PYG{n}{alter}\PYG{o}{\PYGZlt{}}\PYG{o}{=}\PYG{l+m+mi}{65}
\PYG{n}{replace} \PYG{n}{alter\PYGZus{}cat}\PYG{o}{=}\PYG{l+m+mi}{4} \PYG{k}{if} \PYG{n}{alter}\PYG{o}{\PYGZgt{}}\PYG{l+m+mi}{65} \PYG{o}{\PYGZam{}} \PYG{n}{alter}\PYG{o}{\PYGZlt{}}\PYG{o}{=}\PYG{l+m+mi}{120}

\PYG{n}{lab} \PYG{n}{var} \PYG{n}{alter} \PYG{l+s+s2}{\PYGZdq{}}\PYG{l+s+s2}{age}\PYG{l+s+s2}{\PYGZdq{}}
\PYG{n}{lab} \PYG{n}{var} \PYG{n}{alter\PYGZus{}cat} \PYG{l+s+s2}{\PYGZdq{}}\PYG{l+s+s2}{age categorized}\PYG{l+s+s2}{\PYGZdq{}}
\PYG{n}{lab} \PYG{k}{def} \PYG{n+nf}{alter\PYGZus{}cat} \PYG{l+m+mi}{1} \PYG{l+s+s2}{\PYGZdq{}}\PYG{l+s+s2}{\PYGZlt{}=20}\PYG{l+s+s2}{\PYGZdq{}} \PYG{l+m+mi}{2} \PYG{l+s+s2}{\PYGZdq{}}\PYG{l+s+s2}{21\PYGZhy{}30}\PYG{l+s+s2}{\PYGZdq{}} \PYG{l+m+mi}{3} \PYG{l+s+s2}{\PYGZdq{}}\PYG{l+s+s2}{31\PYGZhy{}65}\PYG{l+s+s2}{\PYGZdq{}} \PYG{l+m+mi}{4} \PYG{l+s+s2}{\PYGZdq{}}\PYG{l+s+s2}{\PYGZgt{}65}\PYG{l+s+s2}{\PYGZdq{}}
\PYG{n}{lab} \PYG{n}{val} \PYG{n}{alter\PYGZus{}cat} \PYG{n}{alter\PYGZus{}cat} 
\end{sphinxVerbatim}

Categorize federal states variable.

\fvset{hllines={, ,}}%
\begin{sphinxVerbatim}[commandchars=\\\{\},numbers=left,firstnumber=1,stepnumber=1]
\PYG{o}{*} \PYG{n}{Categorize} \PYG{n}{federal} \PYG{n}{states}
\PYG{n}{gen} \PYG{n}{bgbula\PYGZus{}cat}\PYG{o}{=}\PYG{o}{.}
\PYG{o}{*} \PYG{n}{Schleswig}\PYG{o}{\PYGZhy{}}\PYG{n}{Holstein} \PYG{o}{+} \PYG{n}{Hamburg}
\PYG{n}{replace} \PYG{n}{bgbula\PYGZus{}cat}\PYG{o}{=}\PYG{l+m+mi}{1} \PYG{k}{if} \PYG{n}{bgbula}\PYG{o}{==}\PYG{l+m+mi}{1} \PYG{o}{\textbar{}} \PYG{n}{bgbula}\PYG{o}{==}\PYG{l+m+mi}{2}
\PYG{o}{*} \PYG{n}{Lower} \PYG{n}{Saxony} \PYG{o}{+} \PYG{n}{Bremen}
\PYG{n}{replace} \PYG{n}{bgbula\PYGZus{}cat}\PYG{o}{=}\PYG{l+m+mi}{2} \PYG{k}{if} \PYG{n}{bgbula}\PYG{o}{==}\PYG{l+m+mi}{3} \PYG{o}{\textbar{}} \PYG{n}{bgbula}\PYG{o}{==}\PYG{l+m+mi}{4}
\PYG{o}{*} \PYG{n}{Mecklenburg} \PYG{n}{Western} \PYG{n}{Pomerania} \PYG{o}{+} \PYG{n}{Brandenburg}
\PYG{n}{replace} \PYG{n}{bgbula\PYGZus{}cat}\PYG{o}{=}\PYG{l+m+mi}{3} \PYG{k}{if} \PYG{n}{bgbula}\PYG{o}{==}\PYG{l+m+mi}{13} \PYG{o}{\textbar{}} \PYG{n}{bgbula}\PYG{o}{==}\PYG{l+m+mi}{12}
\PYG{o}{*} \PYG{n}{Saarland} \PYG{o}{+} \PYG{n}{Rhineland} \PYG{n}{Palatinate}
\PYG{n}{replace} \PYG{n}{bgbula\PYGZus{}cat}\PYG{o}{=}\PYG{l+m+mi}{4} \PYG{k}{if} \PYG{n}{bgbula}\PYG{o}{==}\PYG{l+m+mi}{7} \PYG{o}{\textbar{}} \PYG{n}{bgbula}\PYG{o}{==}\PYG{l+m+mi}{10}
\PYG{o}{*} \PYG{n}{Northrhine}\PYG{o}{\PYGZhy{}}\PYG{n}{Westphalia}
\PYG{n}{replace} \PYG{n}{bgbula\PYGZus{}cat}\PYG{o}{=}\PYG{l+m+mi}{5} \PYG{k}{if} \PYG{n}{bgbula}\PYG{o}{==}\PYG{l+m+mi}{5}
\PYG{o}{*} \PYG{n}{Hesse}
\PYG{n}{replace} \PYG{n}{bgbula\PYGZus{}cat}\PYG{o}{=}\PYG{l+m+mi}{6} \PYG{k}{if} \PYG{n}{bgbula}\PYG{o}{==}\PYG{l+m+mi}{6}
\PYG{o}{*} \PYG{n}{Baden}\PYG{o}{\PYGZhy{}}\PYG{n}{Württemberg}
\PYG{n}{replace} \PYG{n}{bgbula\PYGZus{}cat}\PYG{o}{=}\PYG{l+m+mi}{7} \PYG{k}{if} \PYG{n}{bgbula}\PYG{o}{==}\PYG{l+m+mi}{8}
\PYG{o}{*} \PYG{n}{Bavaria}
\PYG{n}{replace} \PYG{n}{bgbula\PYGZus{}cat}\PYG{o}{=}\PYG{l+m+mi}{8} \PYG{k}{if} \PYG{n}{bgbula}\PYG{o}{==}\PYG{l+m+mi}{9}
\PYG{o}{*} \PYG{n}{Berlin}
\PYG{n}{replace} \PYG{n}{bgbula\PYGZus{}cat}\PYG{o}{=}\PYG{l+m+mi}{9} \PYG{k}{if} \PYG{n}{bgbula}\PYG{o}{==}\PYG{l+m+mi}{11}
\PYG{o}{*} \PYG{n}{Saxony}
\PYG{n}{replace} \PYG{n}{bgbula\PYGZus{}cat}\PYG{o}{=}\PYG{l+m+mi}{10} \PYG{k}{if} \PYG{n}{bgbula}\PYG{o}{==}\PYG{l+m+mi}{14}
\PYG{o}{*} \PYG{n}{Saxony}\PYG{o}{\PYGZhy{}}\PYG{n}{Anhalt}
\PYG{n}{replace} \PYG{n}{bgbula\PYGZus{}cat}\PYG{o}{=}\PYG{l+m+mi}{11} \PYG{k}{if} \PYG{n}{bgbula}\PYG{o}{==}\PYG{l+m+mi}{15}
\PYG{o}{*} \PYG{n}{Thuringia}
\PYG{n}{replace} \PYG{n}{bgbula\PYGZus{}cat}\PYG{o}{=}\PYG{l+m+mi}{12} \PYG{k}{if} \PYG{n}{bgbula}\PYG{o}{==}\PYG{l+m+mi}{16}

\PYG{n}{lab} \PYG{n}{var} \PYG{n}{bgbula\PYGZus{}cat} \PYG{l+s+s2}{\PYGZdq{}}\PYG{l+s+s2}{Federal states categorized}\PYG{l+s+s2}{\PYGZdq{}}
\PYG{n}{lab} \PYG{k}{def} \PYG{n+nf}{bgbula\PYGZus{}cat} \PYG{l+m+mi}{1} \PYG{l+s+s2}{\PYGZdq{}}\PYG{l+s+s2}{Schleswig\PYGZhy{}Holstein/Hamburg}\PYG{l+s+s2}{\PYGZdq{}} \PYG{l+m+mi}{2} \PYG{l+s+s2}{\PYGZdq{}}\PYG{l+s+s2}{Lower Saxony/Bremen}\PYG{l+s+s2}{\PYGZdq{}} \PYG{l+m+mi}{3} \PYG{l+s+s2}{\PYGZdq{}}\PYG{l+s+s2}{Mecklenburg Western Pomerania/Brandenburg}\PYG{l+s+s2}{\PYGZdq{}} \PYG{o}{/}\PYG{o}{/}\PYG{o}{/} 
\PYG{l+m+mi}{4} \PYG{l+s+s2}{\PYGZdq{}}\PYG{l+s+s2}{Saarland/Rhineland Palatinate}\PYG{l+s+s2}{\PYGZdq{}} \PYG{l+m+mi}{5} \PYG{l+s+s2}{\PYGZdq{}}\PYG{l+s+s2}{Northrhine\PYGZhy{}Westphalia}\PYG{l+s+s2}{\PYGZdq{}} \PYG{l+m+mi}{6} \PYG{l+s+s2}{\PYGZdq{}}\PYG{l+s+s2}{Hesse}\PYG{l+s+s2}{\PYGZdq{}} \PYG{o}{/}\PYG{o}{/}\PYG{o}{/} 
\PYG{l+m+mi}{7} \PYG{l+s+s2}{\PYGZdq{}}\PYG{l+s+s2}{Baden\PYGZhy{}Wuerttenberg}\PYG{l+s+s2}{\PYGZdq{}} \PYG{l+m+mi}{8} \PYG{l+s+s2}{\PYGZdq{}}\PYG{l+s+s2}{Bavaria}\PYG{l+s+s2}{\PYGZdq{}} \PYG{l+m+mi}{9} \PYG{l+s+s2}{\PYGZdq{}}\PYG{l+s+s2}{Berlin}\PYG{l+s+s2}{\PYGZdq{}} \PYG{l+m+mi}{10} \PYG{l+s+s2}{\PYGZdq{}}\PYG{l+s+s2}{Saxony}\PYG{l+s+s2}{\PYGZdq{}} \PYG{l+m+mi}{11} \PYG{l+s+s2}{\PYGZdq{}}\PYG{l+s+s2}{Saxony\PYGZhy{}Anhalt}\PYG{l+s+s2}{\PYGZdq{}} \PYG{l+m+mi}{12} \PYG{l+s+s2}{\PYGZdq{}}\PYG{l+s+s2}{Thuringia}\PYG{l+s+s2}{\PYGZdq{}}
\PYG{n}{lab} \PYG{n}{val} \PYG{n}{bgbula\PYGZus{}cat} \PYG{n}{bgbula\PYGZus{}cat}
\PYG{n}{drop} \PYG{n}{bgbula}
\PYG{n}{rename} \PYG{n}{bgbula\PYGZus{}cat} \PYG{n}{bgbula}
\end{sphinxVerbatim}

Put the variables in your preferred order and save your data set.

\fvset{hllines={, ,}}%
\begin{sphinxVerbatim}[commandchars=\\\{\},numbers=left,firstnumber=1,stepnumber=1]
* Order demography and identifiers first
order persnr hhnr bghhnr syear sex gebjahr alter alter\PYGZus{}cat bgsampreg bgbula ggk /// 
ggk\PYGZus{}cat bgregtyp  

save \PYGZdl{}\PYGZob{}MY\PYGZus{}OUT\PYGZus{}DATA\PYGZcb{}\PYGZbs{}zeit\PYGZus{}online.dta, replace
\end{sphinxVerbatim}

\sphinxstylestrong{b) You want to get an initial overview of regional differences in satisfaction with various aspects in Germany. Use the variable bgsampreg and cross-stabilize the variable with all satisfaction variables to identify differences between East and West Germany, display the absolute and relative frequencies.}

To save the tables, save them in a log file.

\fvset{hllines={, ,}}%
\begin{sphinxVerbatim}[commandchars=\\\{\},numbers=left,firstnumber=1,stepnumber=1]
********************************************************************************
capture log close
log using \PYGZdq{}\PYGZdl{}\PYGZob{}MY\PYGZus{}LOG\PYGZus{}OUT\PYGZcb{}\PYGZbs{}satisfaction.log\PYGZdq{}, replace

* Life satisfaction

local varlist bgp0101 bgp0102 bgp0103 bgp0104 bgp0105 bgp0106 bgp0107 bgp0108 /// 
bgp0109 bgp0110 bgp0111 bgp0112
foreach x of local varlist \PYGZob{}
tab bgsampreg {}`x\PYGZsq{} [aw= bgphrf] , row
\PYGZcb{}
\end{sphinxVerbatim}

\begin{figure}[H]
\centering

\noindent\sphinxincludegraphics{{reg_01}.PNG}
\end{figure}

\begin{figure}[H]
\centering

\noindent\sphinxincludegraphics{{reg_02}.PNG}
\end{figure}

\begin{figure}[H]
\centering

\noindent\sphinxincludegraphics{{reg_03}.PNG}
\end{figure}

To view all tables, look at your generated log file.

\sphinxstylestrong{c) Now take a closer look at satisfaction with various aspects of life with the help of SOEP regional data. Use the community size classes. Create a table showing you satisfaction with different aspects of life and revealing differences by gender, age, community size class and federal state.}

\fvset{hllines={, ,}}%
\begin{sphinxVerbatim}[commandchars=\\\{\},numbers=left,firstnumber=1,stepnumber=1]
foreach x of local varlist \PYGZob{}
* Tabulation of satisfaction by size of community and federal state
table {}`x\PYGZsq{} sex alter\PYGZus{}cat, by(bgbula ggk\PYGZus{}cat) contents(freq) column row stubwidth(20) cellwidth(8) csepwidth(2) nomissing
* Tabulation of satisfaction by size of community
table {}`x\PYGZsq{} sex alter\PYGZus{}cat, by(ggk\PYGZus{}cat) contents(freq) column row stubwidth(20) cellwidth(8) csepwidth(2) nomissing
* Tabulation of satisfaction by federal state
table {}`x\PYGZsq{} sex alter\PYGZus{}cat, by(bgbula) contents(freq) column row stubwidth(20) cellwidth (8) csepwidth(2) nomissing 
\PYGZcb{}
\end{sphinxVerbatim}

\begin{figure}[H]
\centering

\noindent\sphinxincludegraphics{{reg_08}.PNG}
\end{figure}

\begin{figure}[H]
\centering

\noindent\sphinxincludegraphics{{reg_10}.PNG}
\end{figure}

To view all tables, look at your generated log file. As you can see, SOEP regional data can be used to analyze variables at the smallest regional levels.

\sphinxstylestrong{d) Create a table that shows you the political interest differentiated by age, gender and community size class for Bavaria}

\fvset{hllines={, ,}}%
\begin{sphinxVerbatim}[commandchars=\\\{\},numbers=left,firstnumber=1,stepnumber=1]
\PYG{o}{*}\PYG{o}{*}\PYG{o}{*}\PYG{o}{*}\PYG{o}{*}\PYG{o}{*}\PYG{o}{*}\PYG{o}{*}\PYG{o}{*}\PYG{o}{*}\PYG{o}{*}\PYG{o}{*}\PYG{o}{*}\PYG{o}{*}\PYG{o}{*}\PYG{o}{*}\PYG{o}{*}\PYG{o}{*}\PYG{o}{*}\PYG{o}{*}\PYG{o}{*}\PYG{o}{*}\PYG{o}{*}\PYG{o}{*}\PYG{o}{*}\PYG{o}{*}\PYG{o}{*}\PYG{o}{*}\PYG{o}{*}\PYG{o}{*}\PYG{o}{*}\PYG{o}{*}\PYG{o}{*}\PYG{o}{*}\PYG{o}{*}\PYG{o}{*}\PYG{o}{*}\PYG{o}{*}\PYG{o}{*}\PYG{o}{*}\PYG{o}{*}\PYG{o}{*}\PYG{o}{*}\PYG{o}{*}\PYG{o}{*}\PYG{o}{*}\PYG{o}{*}\PYG{o}{*}\PYG{o}{*}\PYG{o}{*}\PYG{o}{*}\PYG{o}{*}\PYG{o}{*}\PYG{o}{*}\PYG{o}{*}\PYG{o}{*}\PYG{o}{*}\PYG{o}{*}\PYG{o}{*}\PYG{o}{*}\PYG{o}{*}\PYG{o}{*}\PYG{o}{*}\PYG{o}{*}\PYG{o}{*}\PYG{o}{*}\PYG{o}{*}\PYG{o}{*}\PYG{o}{*}\PYG{o}{*}\PYG{o}{*}\PYG{o}{*}\PYG{o}{*}\PYG{o}{*}\PYG{o}{*}\PYG{o}{*}\PYG{o}{*}\PYG{o}{*}\PYG{o}{*}\PYG{o}{*}
\PYG{n}{capture} \PYG{n}{log} \PYG{n}{close}
\PYG{n}{log} \PYG{n}{using} \PYG{l+s+s2}{\PYGZdq{}}\PYG{l+s+s2}{\PYGZdl{}}\PYG{l+s+si}{\PYGZob{}MY\PYGZus{}LOG\PYGZus{}OUT\PYGZcb{}}\PYG{l+s+s2}{\PYGZbs{}}\PYG{l+s+s2}{political\PYGZus{}interest.log}\PYG{l+s+s2}{\PYGZdq{}}\PYG{p}{,} \PYG{n}{replace}

\PYG{o}{*} \PYG{n}{Political} \PYG{n}{interest}
\PYG{o}{*} \PYG{n}{Tabulation} \PYG{n}{of} \PYG{n}{political} \PYG{n}{interest} \PYG{n}{by} \PYG{n}{size} \PYG{n}{of} \PYG{n}{community} \PYG{k}{for} \PYG{n}{Bavaria}
\PYG{n}{table} \PYG{n}{bgp143} \PYG{n}{sex} \PYG{n}{alter\PYGZus{}cat} \PYG{k}{if} \PYG{n}{bgbula}\PYG{o}{==}\PYG{l+m+mi}{8}\PYG{p}{,} \PYG{n}{by}\PYG{p}{(}\PYG{n}{ggk\PYGZus{}cat}\PYG{p}{)} \PYG{n}{contents}\PYG{p}{(}\PYG{n}{freq}\PYG{p}{)} \PYG{n}{column} \PYG{n}{row} \PYG{n}{stubwidth}\PYG{p}{(}\PYG{l+m+mi}{20}\PYG{p}{)} \PYG{n}{cellwidth} \PYG{p}{(}\PYG{l+m+mi}{8}\PYG{p}{)} \PYG{n}{csepwidth}\PYG{p}{(}\PYG{l+m+mi}{2}\PYG{p}{)} \PYG{n}{nomissing}
\end{sphinxVerbatim}

\begin{figure}[H]
\centering

\noindent\sphinxincludegraphics{{reg_11}.PNG}
\end{figure}

It becomes clear that the SOEP offers a wide range of possibilities for region-related analyses. It is possible to allocate a multitude of regional indicators at the level of the federal states, the regional planning regions, the districts and the postal codes.


\chapter{Working with SOEP Documentation}
\label{\detokenize{Working with SOEP Documentation/index:working-with-soep-documentation}}\label{\detokenize{Working with SOEP Documentation/index::doc}}

\section{Variable Search with Questionnaires}
\label{\detokenize{Working with SOEP Documentation/index:variable-search-with-questionnaires}}\label{\detokenize{Working with SOEP Documentation/index:quest-search}}
If you come across a variable in the data set whose variable content is unclear, you should always check whether there is a suitable questionnaire for the data set. Under {\hyperref[\detokenize{Principles of Data Structure/index:overview}]{\sphinxcrossref{\DUrole{std,std-ref}{Overview Data Sets}}}} you can see whether the data sets correspond to a survey instrument. The related questionnaires can be found here:



\sphinxstylestrong{Example: During your research project you come across the variable bbh5508 with the German label “Auto: Gründe” (Car: Reasons) and the Englisch label “Reason for No Car in Household}

\begin{figure}[H]
\centering

\noindent\sphinxincludegraphics{{table_variable}.PNG}
\end{figure}

Unfortunately, it is difficult to determine the variables content from the output and also from the label designations. To understand the complete question and also possible filter instructions, you should use the questionnaires.

\sphinxstylestrong{Example Variable:}

bbh5508:  Wave „bb“ (Survey Year 2011); household questionnaire („h“), question number 55, item 8

Open 

The variable “bbh5508” can be found in the questionnaires for 2011. Select the survey year 2011 and download the household questionnaire.

\begin{figure}[H]
\centering

\noindent\sphinxincludegraphics{{documentation}.PNG}
\end{figure}

Search the variable “bbh5508” in the 

Since you are already in the correct questionnaire, you must now search for question 55.

\begin{figure}[H]
\centering

\noindent\sphinxincludegraphics{{question}.PNG}
\end{figure}

To understand which information the variable “bbh5508” contains, you have to deal with the question. For each answer category, respondents should indicate whether or not the shown items apply to the household. If the item does not apply, respondents must answer an additional question about the reasons. Both questions should be understood as separate variables. The variable “bbh5501” indicates whether a TV is present in the household. The reasons why there is no TV in the house can be found in the variable “bbh5502”. The variable “bbh5507” shows whether a car is present in the household and the variable “bbh5508” shows reasons why no car is present in the household. By looking into the questionnaire, the variable is now easier to understand. The variable “bbh5508” only contains people who do not have a car in their household and shows the reasons given.


\section{Variable Search with paneldata.org}
\label{\detokenize{Working with SOEP Documentation/index:variable-search-with-paneldata-org}}\label{\detokenize{Working with SOEP Documentation/index:var-search}}
With paneldata.org it is also possible to search for variables. For example, if you want to find more information about generated variables, a search with paneldata.org is indispensable. For example, the platform offers comprehensive frequency counts, the chronology of the variables searched for, a cross-study variable linkage via concepts, a syntax generator and a topic list for content search in the SOEP.

\sphinxstylestrong{Example Variable:}

bbh5508:  Wave „bb“ (Survey Year 2011); household questionnaire („h“), question number 55, item 8

Open 

\begin{figure}[H]
\centering

\noindent\sphinxincludegraphics{{paneldata}.PNG}
\end{figure}

Please select the study SOEP-Core. The SOEP-Core overview contains important general information about the study, e.g. data access, survey method, questionnaires, thematic diversity, terms for missing codes, all available data sets of the study and metadata-based questionnaires. To search for a variable, a data set or a publication, simply enter the desired search term in the search field.

\begin{figure}[H]
\centering

\noindent\sphinxincludegraphics{{paneldata_2}.PNG}
\end{figure}

In order for the search to be successful, specific information from the user are necessary. The results window displays all results of the search. It can be seen that the variable “bbh5508” originates from the data provided by SOEP-Core and can be found in the data set “bbh” (survey year 2011). If your search is not so specific, you can also search by keywords. We are still interested in the topic “car”.

\begin{figure}[H]
\centering

\noindent\sphinxincludegraphics{{paneldata_3}.PNG}
\end{figure}

To better limit the 1091 results, the filter options on the left should be used. We are looking for variables from the ordered SOEP-Core datasets. In the windows “type” and “study” we select “variable” and “soep-core”.

\begin{figure}[H]
\centering

\noindent\sphinxincludegraphics{{paneldata_4}.PNG}
\end{figure}

Now all variables are displayed, which contain the term “Car” in the SOEP-Core data. The variable search can be further limited by specifying the data set or the survey year. For more information about the different data sets in SOEP-Core visit the chapter {\hyperref[\detokenize{Principles of Data Structure/index:datasets}]{\sphinxcrossref{\DUrole{std,std-ref}{Data Sets SOEP-Core}}}}. To select original data that can be assigned to a question in the questionnaire, select the subtype “org/net”. The specific selection of the analyzing unit allows you to choose whether the variable should provide information on the household level(“h”) or on the individual level (“p”). If you are interested in household-specific variables, select “h” as the “Analysis unit”. If you are explicitly interested in the survey year 2011, the variable search can be limited to five variables.

\begin{figure}[H]
\centering

\noindent\sphinxincludegraphics{{paneldata_5}.PNG}
\end{figure}

There are only five results left, which also shows our searched variable. If you click on the variable “bbh5508” you will get additional information about the variable.

\begin{figure}[H]
\centering

\noindent\sphinxincludegraphics{{paneldata_6}.PNG}
\end{figure}

First you see the weighted absolute frequencies for the variable. It is possible to remove the missing codes from the analysis and/or to display the relative frequencies. Even without opening the data set,  gives you a good overview of the frequencies of a variable.

\begin{figure}[H]
\centering

\noindent\sphinxincludegraphics{{paneldata_7}.PNG}
\end{figure}

In the Related Variables section you will also find the chronology of the variable you are looking for. The sample variable was collected in 2001, 2003, 2005, 2007, 2011, 2013. Below the survey year, the name of the variable in the respective year is displayed and can be clicked to access the respective variable page. At one glance it is possible to see when a variable was measured, how often it was measured and what its name is in the respective survey year

\begin{figure}[H]
\centering

\noindent\sphinxincludegraphics{{paneldata_8}.PNG}
\end{figure}

The field “Label translations” shows the value labels of the variables in German and English. In addition, all missing codes used in SOEP are listed and explained.

\begin{figure}[H]
\centering

\noindent\sphinxincludegraphics{{paneldata_9}.PNG}
\end{figure}

The Label table window shows you the absolute frequencies of the variable at different collection times. This makes it possible to identify initial trends in how response behaviour has changed over a period of time. The assigned value code is output for each possible characteristic value and the absolute frequencies are displayed in parentheses.

In our example output we see that for the variable “th5106” 800 respondents in the wave “t” (2003) state “financial reasons” as the reason for the absence of a car in the household. For our example variable “bbh5508” in the survey year 2011 (wave “bb”) there are already 871 respondents.

Paneldata.org is an excellent way to get an first overview of certain variables.

\begin{figure}[H]
\centering

\noindent\sphinxincludegraphics{{paneldata_10}.PNG}
\end{figure}

\begin{figure}[H]
\centering

\noindent\sphinxincludegraphics{{paneldata_11}.PNG}
\end{figure}

The info box on the right-hand side provides an overview of all relevant information about the variable and the data set. Beside the basic information you will find the information what kind of variable you are looking for under “Conceptual Dataset”. In our example “bbh5508” you can see that variables with a “Conceptual Dataset: org/net” describe original variables that are assigned to a questionnaire. Generated variables are  “Conceptual Dataset: gen”. To get an overview of the different data set types of SOEP-Core, visit the chapter {\hyperref[\detokenize{Principles of Data Structure/index:overview}]{\sphinxcrossref{\DUrole{std,std-ref}{Overview Data Sets}}}}. In addition, the info box under “Transformations: target variables” provides a link or forwarding to the variable in “long” format. For a more detailed understanding of the long format, read the chapter {\hyperref[\detokenize{Principles of Data Structure/index:datasets-long}]{\sphinxcrossref{\DUrole{std,std-ref}{Data Structure in long Format (long)}}}}.

\begin{figure}[H]
\centering

\noindent\sphinxincludegraphics{{paneldata_12}.PNG}
\end{figure}

As soon as you click on the “long” variable, you will get to the variable overview for this variable in long-format. The overview of variables does not differ. It can be seen that our example variable “bbh5508” can also be found in long-format in the data set “hl” with the variable label “hlf0181”.

In addition to searching for keywords or using the various filter settings, you can also find what you are looking for directly in the data set search. Open paneldata.org, click on the study SOEP-Core and select the menu field “data”.

\begin{figure}[H]
\centering

\noindent\sphinxincludegraphics{{paneldata_13}.PNG}
\end{figure}

Now you get to an overview which shows you all data sets contained in SOEP Core.

\begin{figure}[H]
\centering

\noindent\sphinxincludegraphics{{paneldata_14}.PNG}
\end{figure}

Enter the data set you are looking for (“bbh”) in the search field at the top right and click on the data set. You are forwarded to an overview which shows you all variables from the “bbh” data set.

\begin{figure}[H]
\centering

\noindent\sphinxincludegraphics{{paneldata_15}.PNG}
\end{figure}

Now enter the variable you are looking for in the search field at the top right and click on the desired variable. You are then forwarded to the variable overview and receive detailed information about the variable.
Paneldata.org offers the user very different search options to suit the individual search behavior of each user.


\section{Topic Search with paneldata.org}
\label{\detokenize{Working with SOEP Documentation/index:topic-search-with-paneldata-org}}\label{\detokenize{Working with SOEP Documentation/index:topic}}
In order to obtain an overview of the content of the variables provided by the SOEP, the variables on paneldata.org were assigned to different topics. If you are looking for your research variables and do not want to check all data sets or questionnaires, the topic search on paneldata.org could help you.
Open  and select the main study SOEP Core. The upper navigation bar leads you to the Topics area. Click on Topics and have a look at the list of variables.

\begin{figure}[H]
\centering

\noindent\sphinxincludegraphics{{paneldata_16}.PNG}
\end{figure}

Select a topic that corresponds to your research interest and a more specific selection of topics will appear

\begin{figure}[H]
\centering

\noindent\sphinxincludegraphics{{paneldata_17}.PNG}
\end{figure}

For example, if you are interested in different types of satisfaction, select the appropriate topic “attitudes, values, and personality {[}at{]}”. With a little search you will discover the sub-topic “satisfaction{[}sat{]}”.

\begin{figure}[H]
\centering

\noindent\sphinxincludegraphics{{paneldata_18}.PNG}
\end{figure}

Suppose you are interested in health satisfaction. Based on the label, the “pzuf1” concept could be of interest to you. By clicking on the concept “pzuf1” you will get to the concept overview.

\begin{figure}[H]
\centering

\noindent\sphinxincludegraphics{{paneldata_19}.PNG}
\end{figure}

The concept overview displays the study and wave specific variables of the concept. The concept allows you to determine whether the variable you are looking for is also available and comparable across studies.
In the column “Study” you can see in which studies the same variable is linked via the concept. The label of the respective variable is also displayed in the “Label” column. The column “path” shows the wave name of the variable. By clicking on the label you will get to the known overview of variables with all relevant information. The “Object” column in the concept overview shows you the type of information which is displayed.

\begin{figure}[H]
\centering

\noindent\sphinxincludegraphics{{paneldata_21}.PNG}
\end{figure}

In addition to the variables linked via the concept, you can find the relevant questions in the concept overview. Questions are displayed in the “Object” column with question. Without having to open the questionnaire, you can get an overview of the question and determine possible differences. Click on the desired question and you will be taken to the question display.

\begin{figure}[H]
\centering

\noindent\sphinxincludegraphics{{paneldata_22}.PNG}
\end{figure}

\begin{figure}[H]
\centering

\noindent\sphinxincludegraphics{{paneldata_23}.PNG}
\end{figure}

\sphinxstylestrong{Attention:} The variable search via the questionnaires is unavoidable in order to find out the exact wording of the question and the possible filter structure. The question display of  only provides a quick overview.
In the question overview you can navigate through the questionnaire using the “next question” and “previous question” buttons. The “Instrument” section shows the position of the question in the questionnaire, the survey year and links to the metadata-based survey instrument. Click on the survey instrument “Questionnaire 2011”.

\begin{figure}[H]
\centering

\noindent\sphinxincludegraphics{{paneldata_24}.PNG}
\end{figure}

The survey instrument for the survey year 2011 of the SOEP-IS study is now displayed. You can navigate through the questionnaire in this overview. The search field allows you to search for research-relevant terms. Click on the question to access the question display.


\section{Documentation of Generated Data}
\label{\detokenize{Working with SOEP Documentation/index:documentation-of-generated-data}}\label{\detokenize{Working with SOEP Documentation/index:documentation}}
The range of generated variables and data sets from SOEP-Core is very extensive. To make work easier for users, many variables are already generated for the user in the data preparation process and published with SOEP-Core. The large number of generated data sets and variables is comprehensively documented so that the generation process remains transparent for the user. Here you will find an overview of the 

Example: A number of frequently used variables are provided in SOEP as so-called generated variables (e.g. data sets \$PGEN and \$HGEN). These variables are checked for consistency across waves and have a uniform name. Please use the appropriate documentation to answer the following questions:

\sphinxstylestrong{a) In which variable is the highest school leaving degree for the persons surveyed in 2007?}

To search for the variable with the highest school leaving degree, use paneldata.org. Open  and enter school leaving degree in the search field. Then specify your search by adjusting the filter settings as follows:
\begin{itemize}
\item {} 
type: variable

\item {} 
subtype: gen

\item {} 
study: soep-core

\item {} 
analysis unit: p

\item {} 
period: 2007

\end{itemize}

\begin{figure}[H]
\centering

\noindent\sphinxincludegraphics{{paneldata_25}.PNG}
\end{figure}

All variables could contain the information you are looking for. Since almost all variables in the search result come from the generated “xpgen” data set, the documentation for the \$pgen data set should be used.
Open the 

\begin{figure}[H]
\centering

\noindent\sphinxincludegraphics{{docimentation_2}.PNG}
\end{figure}

Now select the documentation of 

\begin{figure}[H]
\centering

\noindent\sphinxincludegraphics{{docimentation_3}.PNG}
\end{figure}

The table of contents on the left shows you a thematic classification of the data set. To find the variable you are looking for, select topic area 10.

\begin{figure}[H]
\centering

\noindent\sphinxincludegraphics{{docimentation_4}.PNG}
\end{figure}

After a few searches you will find the variable you are looking for. Some interesting information can be derived from the documentation. It can be seen that the information from the generated variable has been taken from the CV questionnaire since 1994 and is surveyed once. In addition, the two additional variables \$psbila and \$psbilo are explained in more detail. The documentation describes that the \$psbil variable is updated regularly and also takes into account possible changes in the highest level of education. This is precisely why it is worth using the generated variable to represent the most recent highest school leaving degree of those surveyed.

The variable we are looking for is xpsbil and describes the highest school leaving degree of the persons surveyed from the survey year 2007.

\sphinxstylestrong{b) Which values are given to persons with Upper Secondary Degree (Abitur) in this variable??}

Since you now know the variable you are looking for, you can use the extensive functions of paneldata.org in addition to the information from the documentation. If you search for the variable “xpsbil” in paneldata.org and click on it, the frequency counts are displayed.

\begin{figure}[H]
\centering

\noindent\sphinxincludegraphics{{paneldata_26}.PNG}
\end{figure}

In addition to the absolute and relative frequencies, you can also read the value codes of specific response categories. A translation of the answer categories can be found in the “Label translations” section:

\begin{figure}[H]
\centering

\noindent\sphinxincludegraphics{{paneldata_27}.PNG}
\end{figure}

You can answer the question without opening the data. In the 2007 survey year, the variable “xpsbil” with the value code “4” describes the answer category “Upper Secondary Degree (Abitur)”.


\section{Syntax Generator on paneldata.org}
\label{\detokenize{Working with SOEP Documentation/index:syntax-generator-on-paneldata-org}}\label{\detokenize{Working with SOEP Documentation/index:syntax}}
 allows registered users to collect and save their research-relevant variables in a variable basket. These variables can be simply written into a single data set with the script generator. The script generator helps you with data management and can save valuable working time.

Open 

\begin{figure}[H]
\centering

\noindent\sphinxincludegraphics{{paneldata_28}.PNG}
\end{figure}

Click on the “Register/ log” to log in to paneldata.org.

\begin{figure}[H]
\centering

\noindent\sphinxincludegraphics{{paneldata_29}.PNG}
\end{figure}

If you have already registered, you can login in the “User login” area. As a new user you can register at “Register here”. Once you have logged in successfully, you have access to the variable basket and the syntax generator.

\begin{figure}[H]
\centering

\noindent\sphinxincludegraphics{{paneldata_30}.PNG}
\end{figure}

To access the activated functions, click on the navigation field “Workspace”. You will be taken to your personal workspace on paneldata.org.

\begin{figure}[H]
\centering

\noindent\sphinxincludegraphics{{paneldata_31}.PNG}
\end{figure}

The “Workspace” displays your created variable baskets. If you click on “Create basket”, you can create a new basket.

\begin{figure}[H]
\centering

\noindent\sphinxincludegraphics{{paneldata_32}.PNG}
\end{figure}

When creating the basket, first define the name of the variable basket. The name must be lower case to be accepted by Paneldata. Optionally, you can assign a label and enter a description. You can create a security key via “Security token”. Finally, you select the study that you want to use as a database for your research. Now click on “Create basket” and your newly created variable basket appears in the “Workspace” interface.

\begin{figure}[H]
\centering

\noindent\sphinxincludegraphics{{paneldata_33}.PNG}
\end{figure}

Now search for your relevant variables on paneldata.org and add them to your individual basket. For example, you are interested in the monthly net household income. If you do not know the variable name, you can find the superordinate concept using the topic search. Click on the navigation field “paneldata.org” to get to the main page. Select the study SOEP-Core and click on the navigation field “Topics”.

\begin{figure}[H]
\centering

\noindent\sphinxincludegraphics{{paneldata_34}.PNG}
\end{figure}

Check the different topics for income-relevant concepts and select “income, taxes, and social security”.

\begin{figure}[H]
\centering

\noindent\sphinxincludegraphics{{paneldata_35}.PNG}
\end{figure}

Browse the topic list and you will reach the sub-topic “household income hhi”. There you will find the concept you are looking for under “monthly income moi”. Click on the concept and you will see the history of variables, possible links to other studies and perhaps the question in metadata-based form.

\begin{figure}[H]
\centering

\noindent\sphinxincludegraphics{{paneldata_36}.PNG}
\end{figure}

Select the variable of your desired study SOEP-Core and you will reach the variable overview with important information about the variable. In the variable overview, you should make sure that the variable also meets your requirements.

\begin{figure}[H]
\centering

\noindent\sphinxincludegraphics{{paneldata_37}.PNG}
\end{figure}

When logged in, the Basket area appears in the overview of variables. Your baskets are listed there. If you want to add the variable to a basket, click on “Add to basket”. If the variable is already in the basket and you want to remove it, select “Remove from basket”. If you want to create a new basket within the overview of variables, click on “Create a new basket” to go to basket creation and its variable is automatically placed in the new basket. You can access the basket overview by clicking on the name of your basket in the “Basket” section. Alternatively, you can click on the navigation field “Workspace” and you will also return to the basket overview.

\begin{figure}[H]
\centering

\noindent\sphinxincludegraphics{{paneldata_38}.PNG}
\end{figure}

Click on the basket with your added variable and you will get an overview of all variables in your basket. With “Add all” you add the variables of all survey waves and the shopping cart is highlighted in green. If you are interested in a specific survey period, you can select the wave-specific variables by clicking on the shopping cart. Click on “Remove all” to remove the variable from your basket.

\begin{figure}[H]
\centering

\noindent\sphinxincludegraphics{{paneldata_39}.PNG}
\end{figure}

Once you have filled your basket and selected the desired survey waves, you can merge all variables into one data set. To do this, click on “CREATE A NEW SCRIPT” in the “List of scripts” area.

\begin{figure}[H]
\centering

\noindent\sphinxincludegraphics{{paneldata_40}.PNG}
\end{figure}

In the script generator you can create a script that matches your preferred variables. Specify the name of your script. Select the statistics program you are using. Then enter the path where you have stored your data records in the “Input path”. In the “Output path” you write your desired output path for the created data set.

\begin{figure}[H]
\centering

\noindent\sphinxincludegraphics{{paneldata_41}.PNG}
\end{figure}

In the “Analysis Unit” section, you decide whether all persons are considered individually within the household (“Individual”) or whether you are only interested in the household as a whole (“Household”). With “Sample composition” you can choose between “balanced” and “unbalanced”. If you select “balanced”, you will receive a data set without missing codes. The respondents provided information on all variables. For more information about balanced and unbalanced datasets visit the chapter {\hyperref[\detokenize{Principles of Data Structure/index:analysis}]{\sphinxcrossref{\DUrole{std,std-ref}{Panel Data Analysis}}}}. Under “Age group” you can limit the respondents. When you are satisfied with your settings, click on “Update Script” and your script will be created.

\begin{figure}[H]
\centering

\noindent\sphinxincludegraphics{{paneldata_42}.PNG}
\end{figure}

If you click on the “raw script” button, the script is displayed in text form. Copy it to your statistics software. To name the data set correctly, you should change the name of the data set in the script. Execute the script with your statistics software and you will receive your data set with all your choosen variables.



\renewcommand{\indexname}{Index}
\printindex
\end{document}